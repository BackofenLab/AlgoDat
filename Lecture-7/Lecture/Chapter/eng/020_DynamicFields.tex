\section{Dynamic Fields}

\subsection{Introduction}

\begin{frame}{Dynamic Fields}{Introduction}
  \begin{itemize}
    \item
      The field is created with an initial size
    \item
      The size can be dynamicly modified
    \item
      \textbf{Problem:}
      We need a dynamic structure to store the data
  \end{itemize}
\end{frame}

%-------------------------------------------------------------------------------

\begin{frame}{Dynamic Fields}{Python}
  \textbf{Python:}
  \lstinputlisting[
    language=Python,
    basicstyle=\small,
    tabsize=4,
    style={python-idle-code},
    escapechar={@}
  ]{Lecture/Code/DynamicLists.py}
\end{frame}

%-------------------------------------------------------------------------------

\begin{frame}{Dynamic Fields}{Java}
  \textbf{Java:}
  \lstinputlisting[
    language=Java,
    basicstyle=\small,
    tabsize=4,
    style={java-eclipse-code},
    escapechar={@},
    emph={words},
    emphstyle=\color{java_variable}
  ]{Lecture/Code/DynamicField.java}
\end{frame}

%-------------------------------------------------------------------------------

\begin{frame}{Dynamic Fields}{C++}
  \textbf{C++:}
  \lstinputlisting[
    language=C++,
    basicstyle=\small,
    tabsize=4,
    style={cpp-eclipse-code},
    morekeywords={endl},
    escapechar={@}
  ]{Lecture/Code/DynamicField.cpp}
\end{frame}

%-------------------------------------------------------------------------------

\begin{frame}{Dynamic Fields}{Implementation 1}
  \begin{itemize}
    \item
      We store the data in a fixed-size array with the needed size
    \item
      \textbf{Append:}
      \begin{itemize}
        \item
          Create fixed-size array with the needed size
        \item
          Copy elements from the old to the new array
      \end{itemize}
    \item
      \textbf{Remove:}
      \begin{itemize}
        \item
          Create fixed-size array with the needed size
        \item
          Copy elements from the old to the new array
      \end{itemize}
  \end{itemize}
\end{frame}

%-------------------------------------------------------------------------------

\begin{frame}{Dynamic Fields}{Implementation 1}
  \begin{itemize}
    \item
      We resize the field before each append
    \item
      We choose the size exactly as needed
  \end{itemize}
  \lstinputlisting[
    language=Python,
    basicstyle=\small,
    tabsize=4,
    style={python-idle-code},
    escapechar={@},
%    morekeywords={self},
    emph={DynamicArray, capacity, __init__},
    emphstyle=\color{blue}
  ]{Lecture/Code/DynamicListImpl1_Part1.py}
\end{frame}

%-------------------------------------------------------------------------------

\begin{frame}{Dynamic Fields}{Implementation 1}
  \lstinputlisting[
    language=Python,
    basicstyle=\small,
    tabsize=4,
    style={python-idle-code},
    escapechar={@},
%    morekeywords={self},
    breaklines=false,
    emph={DynamicArray, append},
    emphstyle=\color{blue}
  ]{Lecture/Code/DynamicListImpl1_Part2.py}
\end{frame}

%-------------------------------------------------------------------------------

\begin{frame}{Dynamic Fields}{Implementation 1}
  \begin{figure}
    \caption{Runtime of \textit{DynamicArray}}
    \label{fig:runtime_dynamic_array_impl1}
    %TODO
  \end{figure}
  \begin{itemize}
    \item
      Why is the runtime qubic?
  \end{itemize}
\end{frame}

%-------------------------------------------------------------------------------

\begin{frame}{Dynamic Fields}{Implementation 1}
  \textbf{Runtime:}\\[0.5em]
  \begin{tabularx}{\linewidth}{@{}lcl}
    \def\FSAsize{1}\def\FSAelements{0}%
    \def\FSAcopy{0}\def\FSAdelete{0}\def\FSAinsert{1}%
    \raisebox{-0.25em}{%
\begin{adjustbox}{height=1.5em}%
\begin{tikzpicture}[%
  element/.style={
    draw=black,
    fill={black!20!white}
  }, element_copy/.style={
    draw=black,
    fill={Hell-Blau}
  }, element_delete/.style={
    draw=black,
    fill=white
  }, element_delete_cross/.style={
    draw=red,
    line cap=round,
    line width=0.25em
  }, element_insert/.style={
      draw=black,
      fill={yellow!50!white}
  }, element_free/.style={
    draw=black,
    fill=white
  }, arrow_copy/.style={
    ->,
    line width=0.3em,
    color={Mittel-Gruen}
  }, arrow_insert/.style={
    ->,
    line width=0.3em,
    color={Hell-Gruen}
  }
]%
\foreach \x in {0,...,\FSAelements}{
  \ifnum \x<\FSAelements
    \draw[element]
      (\x, 0.0)
      rectangle
      (\x + 1.0, 1.0);
    \draw (\x + 0.5, 0.5) node {\Huge \x};
  \fi
}
\foreach \x in {0,...,\FSAcopy}{
  \ifnum \x<\FSAcopy
    \draw[element_copy]
      (\x + \FSAelements, 0.0)
      rectangle
      (\x + \FSAelements + 1.0, 1.0);
    \FPeval{\value}{clip(\x+\FSAelements)}
    \draw (\x + \FSAelements + 0.5, 0.5)
      node {\Huge \value};
    \draw[arrow_copy]
      (\x + \FSAelements + 0.5, 1.4) --
      (\x + \FSAelements + 0.5, 0.75);
  \fi
}
\foreach \x in {0,...,\FSAdelete}{
  \ifnum \x<\FSAdelete
  \draw[element_delete]
    (\x + \FSAelements + \FSAcopy, 0.0)
    rectangle
    (\x + \FSAelements + \FSAcopy + 1.0, 1.0);
  \FPeval{\value}{clip(\x+\FSAelements+\FSAcopy)}
  \draw (\x + \FSAelements + \FSAcopy + 0.5, 0.5)
    node {\Huge \value};
  \fi
}
\foreach \x in {0,...,\FSAinsert}{
  \ifnum \x<\FSAinsert
    \draw[element_insert]
      (\x + \FSAelements + \FSAcopy + \FSAdelete, 0.0)
      rectangle
      (\x + \FSAelements + \FSAcopy + \FSAdelete + 1.0, 1.0);
    \FPeval{\value}{clip(\x+\FSAelements+\FSAcopy+\FSAdelete)}
    \draw (\x + \FSAelements + \FSAcopy + \FSAdelete + 0.5, 0.5)
      node {\Huge \value};
    \draw[arrow_insert]
      (\x + \FSAelements + \FSAcopy + \FSAdelete + 0.5, 1.4) --
      (\x + \FSAelements + \FSAcopy + \FSAdelete + 0.5, 0.75);
  \fi
}
\foreach \x in {0,...,\FSAsize}{
  \FPeval{\value}{clip(\FSAelements+\FSAcopy+\FSAdelete+\FSAinsert-2)}
  \ifnum \x>\value
    \FPeval{\value}{clip(\FSAsize-1)}
    \ifnum \x<\value
      \draw[element_free]
        (\x + 1.0, 0.0)
        rectangle
        (\x + 2.0, 1.0);
      \FPeval{\value}{clip(\value+1)}
%      \draw (\x + 1.5, 0.5)
%        node {\Huge \value};
    \fi
  \fi
}
\foreach \x in {0,...,\FSAdelete}{
  \ifnum \x<\FSAdelete
  \draw[element_delete_cross]
    (\x + \FSAelements + \FSAcopy, 1.0) --
    (\x + \FSAelements + \FSAcopy + 1.0, 0.0);
  \draw[element_delete_cross]
    (\x + \FSAelements + \FSAcopy, 0.0) --
    (\x + \FSAelements + \FSAcopy + 1.0, 1.0);
  \fi
}
\end{tikzpicture}%
\end{adjustbox}%
}% &
    $\mathcal{O}(1)$ &
    write 1 element\\
    \def\FSAsize{2}\def\FSAelements{0}%
    \def\FSAcopy{1}\def\FSAdelete{0}\def\FSAinsert{1}%
    \raisebox{-0.25em}{%
\begin{adjustbox}{height=1.5em}%
\begin{tikzpicture}[%
  element/.style={
    draw=black,
    fill={black!20!white}
  }, element_copy/.style={
    draw=black,
    fill={Hell-Blau}
  }, element_delete/.style={
    draw=black,
    fill=white
  }, element_delete_cross/.style={
    draw=red,
    line cap=round,
    line width=0.25em
  }, element_insert/.style={
      draw=black,
      fill={yellow!50!white}
  }, element_free/.style={
    draw=black,
    fill=white
  }, arrow_copy/.style={
    ->,
    line width=0.3em,
    color={Mittel-Gruen}
  }, arrow_insert/.style={
    ->,
    line width=0.3em,
    color={Hell-Gruen}
  }
]%
\foreach \x in {0,...,\FSAelements}{
  \ifnum \x<\FSAelements
    \draw[element]
      (\x, 0.0)
      rectangle
      (\x + 1.0, 1.0);
    \draw (\x + 0.5, 0.5) node {\Huge \x};
  \fi
}
\foreach \x in {0,...,\FSAcopy}{
  \ifnum \x<\FSAcopy
    \draw[element_copy]
      (\x + \FSAelements, 0.0)
      rectangle
      (\x + \FSAelements + 1.0, 1.0);
    \FPeval{\value}{clip(\x+\FSAelements)}
    \draw (\x + \FSAelements + 0.5, 0.5)
      node {\Huge \value};
    \draw[arrow_copy]
      (\x + \FSAelements + 0.5, 1.4) --
      (\x + \FSAelements + 0.5, 0.75);
  \fi
}
\foreach \x in {0,...,\FSAdelete}{
  \ifnum \x<\FSAdelete
  \draw[element_delete]
    (\x + \FSAelements + \FSAcopy, 0.0)
    rectangle
    (\x + \FSAelements + \FSAcopy + 1.0, 1.0);
  \FPeval{\value}{clip(\x+\FSAelements+\FSAcopy)}
  \draw (\x + \FSAelements + \FSAcopy + 0.5, 0.5)
    node {\Huge \value};
  \fi
}
\foreach \x in {0,...,\FSAinsert}{
  \ifnum \x<\FSAinsert
    \draw[element_insert]
      (\x + \FSAelements + \FSAcopy + \FSAdelete, 0.0)
      rectangle
      (\x + \FSAelements + \FSAcopy + \FSAdelete + 1.0, 1.0);
    \FPeval{\value}{clip(\x+\FSAelements+\FSAcopy+\FSAdelete)}
    \draw (\x + \FSAelements + \FSAcopy + \FSAdelete + 0.5, 0.5)
      node {\Huge \value};
    \draw[arrow_insert]
      (\x + \FSAelements + \FSAcopy + \FSAdelete + 0.5, 1.4) --
      (\x + \FSAelements + \FSAcopy + \FSAdelete + 0.5, 0.75);
  \fi
}
\foreach \x in {0,...,\FSAsize}{
  \FPeval{\value}{clip(\FSAelements+\FSAcopy+\FSAdelete+\FSAinsert-2)}
  \ifnum \x>\value
    \FPeval{\value}{clip(\FSAsize-1)}
    \ifnum \x<\value
      \draw[element_free]
        (\x + 1.0, 0.0)
        rectangle
        (\x + 2.0, 1.0);
      \FPeval{\value}{clip(\value+1)}
%      \draw (\x + 1.5, 0.5)
%        node {\Huge \value};
    \fi
  \fi
}
\foreach \x in {0,...,\FSAdelete}{
  \ifnum \x<\FSAdelete
  \draw[element_delete_cross]
    (\x + \FSAelements + \FSAcopy, 1.0) --
    (\x + \FSAelements + \FSAcopy + 1.0, 0.0);
  \draw[element_delete_cross]
    (\x + \FSAelements + \FSAcopy, 0.0) --
    (\x + \FSAelements + \FSAcopy + 1.0, 1.0);
  \fi
}
\end{tikzpicture}%
\end{adjustbox}%
}% &
    $\mathcal{O}(1 + {\color{Mittel-Blau}1})$ &
    write 1 element, {\color{Mittel-Blau}copy 1 element}\\
    \def\FSAsize{3}\def\FSAelements{0}%
    \def\FSAcopy{2}\def\FSAdelete{0}\def\FSAinsert{1}%
    \raisebox{-0.25em}{%
\begin{adjustbox}{height=1.5em}%
\begin{tikzpicture}[%
  element/.style={
    draw=black,
    fill={black!20!white}
  }, element_copy/.style={
    draw=black,
    fill={Hell-Blau}
  }, element_delete/.style={
    draw=black,
    fill=white
  }, element_delete_cross/.style={
    draw=red,
    line cap=round,
    line width=0.25em
  }, element_insert/.style={
      draw=black,
      fill={yellow!50!white}
  }, element_free/.style={
    draw=black,
    fill=white
  }, arrow_copy/.style={
    ->,
    line width=0.3em,
    color={Mittel-Gruen}
  }, arrow_insert/.style={
    ->,
    line width=0.3em,
    color={Hell-Gruen}
  }
]%
\foreach \x in {0,...,\FSAelements}{
  \ifnum \x<\FSAelements
    \draw[element]
      (\x, 0.0)
      rectangle
      (\x + 1.0, 1.0);
    \draw (\x + 0.5, 0.5) node {\Huge \x};
  \fi
}
\foreach \x in {0,...,\FSAcopy}{
  \ifnum \x<\FSAcopy
    \draw[element_copy]
      (\x + \FSAelements, 0.0)
      rectangle
      (\x + \FSAelements + 1.0, 1.0);
    \FPeval{\value}{clip(\x+\FSAelements)}
    \draw (\x + \FSAelements + 0.5, 0.5)
      node {\Huge \value};
    \draw[arrow_copy]
      (\x + \FSAelements + 0.5, 1.4) --
      (\x + \FSAelements + 0.5, 0.75);
  \fi
}
\foreach \x in {0,...,\FSAdelete}{
  \ifnum \x<\FSAdelete
  \draw[element_delete]
    (\x + \FSAelements + \FSAcopy, 0.0)
    rectangle
    (\x + \FSAelements + \FSAcopy + 1.0, 1.0);
  \FPeval{\value}{clip(\x+\FSAelements+\FSAcopy)}
  \draw (\x + \FSAelements + \FSAcopy + 0.5, 0.5)
    node {\Huge \value};
  \fi
}
\foreach \x in {0,...,\FSAinsert}{
  \ifnum \x<\FSAinsert
    \draw[element_insert]
      (\x + \FSAelements + \FSAcopy + \FSAdelete, 0.0)
      rectangle
      (\x + \FSAelements + \FSAcopy + \FSAdelete + 1.0, 1.0);
    \FPeval{\value}{clip(\x+\FSAelements+\FSAcopy+\FSAdelete)}
    \draw (\x + \FSAelements + \FSAcopy + \FSAdelete + 0.5, 0.5)
      node {\Huge \value};
    \draw[arrow_insert]
      (\x + \FSAelements + \FSAcopy + \FSAdelete + 0.5, 1.4) --
      (\x + \FSAelements + \FSAcopy + \FSAdelete + 0.5, 0.75);
  \fi
}
\foreach \x in {0,...,\FSAsize}{
  \FPeval{\value}{clip(\FSAelements+\FSAcopy+\FSAdelete+\FSAinsert-2)}
  \ifnum \x>\value
    \FPeval{\value}{clip(\FSAsize-1)}
    \ifnum \x<\value
      \draw[element_free]
        (\x + 1.0, 0.0)
        rectangle
        (\x + 2.0, 1.0);
      \FPeval{\value}{clip(\value+1)}
%      \draw (\x + 1.5, 0.5)
%        node {\Huge \value};
    \fi
  \fi
}
\foreach \x in {0,...,\FSAdelete}{
  \ifnum \x<\FSAdelete
  \draw[element_delete_cross]
    (\x + \FSAelements + \FSAcopy, 1.0) --
    (\x + \FSAelements + \FSAcopy + 1.0, 0.0);
  \draw[element_delete_cross]
    (\x + \FSAelements + \FSAcopy, 0.0) --
    (\x + \FSAelements + \FSAcopy + 1.0, 1.0);
  \fi
}
\end{tikzpicture}%
\end{adjustbox}%
}% &
    $\mathcal{O}(1 + {\color{Mittel-Blau}2})$ &
    write 1 element, {\color{Mittel-Blau}copy 2 elements}\\
    \def\FSAsize{4}\def\FSAelements{0}%
    \def\FSAcopy{3}\def\FSAdelete{0}\def\FSAinsert{1}%
    \raisebox{-0.25em}{%
\begin{adjustbox}{height=1.5em}%
\begin{tikzpicture}[%
  element/.style={
    draw=black,
    fill={black!20!white}
  }, element_copy/.style={
    draw=black,
    fill={Hell-Blau}
  }, element_delete/.style={
    draw=black,
    fill=white
  }, element_delete_cross/.style={
    draw=red,
    line cap=round,
    line width=0.25em
  }, element_insert/.style={
      draw=black,
      fill={yellow!50!white}
  }, element_free/.style={
    draw=black,
    fill=white
  }, arrow_copy/.style={
    ->,
    line width=0.3em,
    color={Mittel-Gruen}
  }, arrow_insert/.style={
    ->,
    line width=0.3em,
    color={Hell-Gruen}
  }
]%
\foreach \x in {0,...,\FSAelements}{
  \ifnum \x<\FSAelements
    \draw[element]
      (\x, 0.0)
      rectangle
      (\x + 1.0, 1.0);
    \draw (\x + 0.5, 0.5) node {\Huge \x};
  \fi
}
\foreach \x in {0,...,\FSAcopy}{
  \ifnum \x<\FSAcopy
    \draw[element_copy]
      (\x + \FSAelements, 0.0)
      rectangle
      (\x + \FSAelements + 1.0, 1.0);
    \FPeval{\value}{clip(\x+\FSAelements)}
    \draw (\x + \FSAelements + 0.5, 0.5)
      node {\Huge \value};
    \draw[arrow_copy]
      (\x + \FSAelements + 0.5, 1.4) --
      (\x + \FSAelements + 0.5, 0.75);
  \fi
}
\foreach \x in {0,...,\FSAdelete}{
  \ifnum \x<\FSAdelete
  \draw[element_delete]
    (\x + \FSAelements + \FSAcopy, 0.0)
    rectangle
    (\x + \FSAelements + \FSAcopy + 1.0, 1.0);
  \FPeval{\value}{clip(\x+\FSAelements+\FSAcopy)}
  \draw (\x + \FSAelements + \FSAcopy + 0.5, 0.5)
    node {\Huge \value};
  \fi
}
\foreach \x in {0,...,\FSAinsert}{
  \ifnum \x<\FSAinsert
    \draw[element_insert]
      (\x + \FSAelements + \FSAcopy + \FSAdelete, 0.0)
      rectangle
      (\x + \FSAelements + \FSAcopy + \FSAdelete + 1.0, 1.0);
    \FPeval{\value}{clip(\x+\FSAelements+\FSAcopy+\FSAdelete)}
    \draw (\x + \FSAelements + \FSAcopy + \FSAdelete + 0.5, 0.5)
      node {\Huge \value};
    \draw[arrow_insert]
      (\x + \FSAelements + \FSAcopy + \FSAdelete + 0.5, 1.4) --
      (\x + \FSAelements + \FSAcopy + \FSAdelete + 0.5, 0.75);
  \fi
}
\foreach \x in {0,...,\FSAsize}{
  \FPeval{\value}{clip(\FSAelements+\FSAcopy+\FSAdelete+\FSAinsert-2)}
  \ifnum \x>\value
    \FPeval{\value}{clip(\FSAsize-1)}
    \ifnum \x<\value
      \draw[element_free]
        (\x + 1.0, 0.0)
        rectangle
        (\x + 2.0, 1.0);
      \FPeval{\value}{clip(\value+1)}
%      \draw (\x + 1.5, 0.5)
%        node {\Huge \value};
    \fi
  \fi
}
\foreach \x in {0,...,\FSAdelete}{
  \ifnum \x<\FSAdelete
  \draw[element_delete_cross]
    (\x + \FSAelements + \FSAcopy, 1.0) --
    (\x + \FSAelements + \FSAcopy + 1.0, 0.0);
  \draw[element_delete_cross]
    (\x + \FSAelements + \FSAcopy, 0.0) --
    (\x + \FSAelements + \FSAcopy + 1.0, 1.0);
  \fi
}
\end{tikzpicture}%
\end{adjustbox}%
}% &
    $\mathcal{O}(1 + {\color{Mittel-Blau}3})$ &
    write 1 element, {\color{Mittel-Blau}copy 3 elements}\\
    \def\FSAsize{5}\def\FSAelements{0}%
    \def\FSAcopy{4}\def\FSAdelete{0}\def\FSAinsert{1}%
    \raisebox{-0.25em}{%
\begin{adjustbox}{height=1.5em}%
\begin{tikzpicture}[%
  element/.style={
    draw=black,
    fill={black!20!white}
  }, element_copy/.style={
    draw=black,
    fill={Hell-Blau}
  }, element_delete/.style={
    draw=black,
    fill=white
  }, element_delete_cross/.style={
    draw=red,
    line cap=round,
    line width=0.25em
  }, element_insert/.style={
      draw=black,
      fill={yellow!50!white}
  }, element_free/.style={
    draw=black,
    fill=white
  }, arrow_copy/.style={
    ->,
    line width=0.3em,
    color={Mittel-Gruen}
  }, arrow_insert/.style={
    ->,
    line width=0.3em,
    color={Hell-Gruen}
  }
]%
\foreach \x in {0,...,\FSAelements}{
  \ifnum \x<\FSAelements
    \draw[element]
      (\x, 0.0)
      rectangle
      (\x + 1.0, 1.0);
    \draw (\x + 0.5, 0.5) node {\Huge \x};
  \fi
}
\foreach \x in {0,...,\FSAcopy}{
  \ifnum \x<\FSAcopy
    \draw[element_copy]
      (\x + \FSAelements, 0.0)
      rectangle
      (\x + \FSAelements + 1.0, 1.0);
    \FPeval{\value}{clip(\x+\FSAelements)}
    \draw (\x + \FSAelements + 0.5, 0.5)
      node {\Huge \value};
    \draw[arrow_copy]
      (\x + \FSAelements + 0.5, 1.4) --
      (\x + \FSAelements + 0.5, 0.75);
  \fi
}
\foreach \x in {0,...,\FSAdelete}{
  \ifnum \x<\FSAdelete
  \draw[element_delete]
    (\x + \FSAelements + \FSAcopy, 0.0)
    rectangle
    (\x + \FSAelements + \FSAcopy + 1.0, 1.0);
  \FPeval{\value}{clip(\x+\FSAelements+\FSAcopy)}
  \draw (\x + \FSAelements + \FSAcopy + 0.5, 0.5)
    node {\Huge \value};
  \fi
}
\foreach \x in {0,...,\FSAinsert}{
  \ifnum \x<\FSAinsert
    \draw[element_insert]
      (\x + \FSAelements + \FSAcopy + \FSAdelete, 0.0)
      rectangle
      (\x + \FSAelements + \FSAcopy + \FSAdelete + 1.0, 1.0);
    \FPeval{\value}{clip(\x+\FSAelements+\FSAcopy+\FSAdelete)}
    \draw (\x + \FSAelements + \FSAcopy + \FSAdelete + 0.5, 0.5)
      node {\Huge \value};
    \draw[arrow_insert]
      (\x + \FSAelements + \FSAcopy + \FSAdelete + 0.5, 1.4) --
      (\x + \FSAelements + \FSAcopy + \FSAdelete + 0.5, 0.75);
  \fi
}
\foreach \x in {0,...,\FSAsize}{
  \FPeval{\value}{clip(\FSAelements+\FSAcopy+\FSAdelete+\FSAinsert-2)}
  \ifnum \x>\value
    \FPeval{\value}{clip(\FSAsize-1)}
    \ifnum \x<\value
      \draw[element_free]
        (\x + 1.0, 0.0)
        rectangle
        (\x + 2.0, 1.0);
      \FPeval{\value}{clip(\value+1)}
%      \draw (\x + 1.5, 0.5)
%        node {\Huge \value};
    \fi
  \fi
}
\foreach \x in {0,...,\FSAdelete}{
  \ifnum \x<\FSAdelete
  \draw[element_delete_cross]
    (\x + \FSAelements + \FSAcopy, 1.0) --
    (\x + \FSAelements + \FSAcopy + 1.0, 0.0);
  \draw[element_delete_cross]
    (\x + \FSAelements + \FSAcopy, 0.0) --
    (\x + \FSAelements + \FSAcopy + 1.0, 1.0);
  \fi
}
\end{tikzpicture}%
\end{adjustbox}%
}% &
    $\mathcal{O}(1 + {\color{Mittel-Blau}4})$ &
    write 1 element, {\color{Mittel-Blau}copy 4 elements}\\
    \def\FSAsize{6}\def\FSAelements{0}%
    \def\FSAcopy{5}\def\FSAdelete{0}\def\FSAinsert{1}%
    \raisebox{-0.25em}{%
\begin{adjustbox}{height=1.5em}%
\begin{tikzpicture}[%
  element/.style={
    draw=black,
    fill={black!20!white}
  }, element_copy/.style={
    draw=black,
    fill={Hell-Blau}
  }, element_delete/.style={
    draw=black,
    fill=white
  }, element_delete_cross/.style={
    draw=red,
    line cap=round,
    line width=0.25em
  }, element_insert/.style={
      draw=black,
      fill={yellow!50!white}
  }, element_free/.style={
    draw=black,
    fill=white
  }, arrow_copy/.style={
    ->,
    line width=0.3em,
    color={Mittel-Gruen}
  }, arrow_insert/.style={
    ->,
    line width=0.3em,
    color={Hell-Gruen}
  }
]%
\foreach \x in {0,...,\FSAelements}{
  \ifnum \x<\FSAelements
    \draw[element]
      (\x, 0.0)
      rectangle
      (\x + 1.0, 1.0);
    \draw (\x + 0.5, 0.5) node {\Huge \x};
  \fi
}
\foreach \x in {0,...,\FSAcopy}{
  \ifnum \x<\FSAcopy
    \draw[element_copy]
      (\x + \FSAelements, 0.0)
      rectangle
      (\x + \FSAelements + 1.0, 1.0);
    \FPeval{\value}{clip(\x+\FSAelements)}
    \draw (\x + \FSAelements + 0.5, 0.5)
      node {\Huge \value};
    \draw[arrow_copy]
      (\x + \FSAelements + 0.5, 1.4) --
      (\x + \FSAelements + 0.5, 0.75);
  \fi
}
\foreach \x in {0,...,\FSAdelete}{
  \ifnum \x<\FSAdelete
  \draw[element_delete]
    (\x + \FSAelements + \FSAcopy, 0.0)
    rectangle
    (\x + \FSAelements + \FSAcopy + 1.0, 1.0);
  \FPeval{\value}{clip(\x+\FSAelements+\FSAcopy)}
  \draw (\x + \FSAelements + \FSAcopy + 0.5, 0.5)
    node {\Huge \value};
  \fi
}
\foreach \x in {0,...,\FSAinsert}{
  \ifnum \x<\FSAinsert
    \draw[element_insert]
      (\x + \FSAelements + \FSAcopy + \FSAdelete, 0.0)
      rectangle
      (\x + \FSAelements + \FSAcopy + \FSAdelete + 1.0, 1.0);
    \FPeval{\value}{clip(\x+\FSAelements+\FSAcopy+\FSAdelete)}
    \draw (\x + \FSAelements + \FSAcopy + \FSAdelete + 0.5, 0.5)
      node {\Huge \value};
    \draw[arrow_insert]
      (\x + \FSAelements + \FSAcopy + \FSAdelete + 0.5, 1.4) --
      (\x + \FSAelements + \FSAcopy + \FSAdelete + 0.5, 0.75);
  \fi
}
\foreach \x in {0,...,\FSAsize}{
  \FPeval{\value}{clip(\FSAelements+\FSAcopy+\FSAdelete+\FSAinsert-2)}
  \ifnum \x>\value
    \FPeval{\value}{clip(\FSAsize-1)}
    \ifnum \x<\value
      \draw[element_free]
        (\x + 1.0, 0.0)
        rectangle
        (\x + 2.0, 1.0);
      \FPeval{\value}{clip(\value+1)}
%      \draw (\x + 1.5, 0.5)
%        node {\Huge \value};
    \fi
  \fi
}
\foreach \x in {0,...,\FSAdelete}{
  \ifnum \x<\FSAdelete
  \draw[element_delete_cross]
    (\x + \FSAelements + \FSAcopy, 1.0) --
    (\x + \FSAelements + \FSAcopy + 1.0, 0.0);
  \draw[element_delete_cross]
    (\x + \FSAelements + \FSAcopy, 0.0) --
    (\x + \FSAelements + \FSAcopy + 1.0, 1.0);
  \fi
}
\end{tikzpicture}%
\end{adjustbox}%
}% &
    $\mathcal{O}(1 + {\color{Mittel-Blau}5})$ &
    write 1 element, {\color{Mittel-Blau}copy 5 elements}\\
    \hspace*{1.5em}\dots & \dots & \hspace*{1.5em}\dots
  \end{tabularx}
\end{frame}

%-------------------------------------------------------------------------------

\begin{frame}{Dynamic Fields}{Implementation 1}
  \textbf{Analysis:}
  \begin{itemize}
    \item
      Let $T(n)$ be the runtime of $n$ sequential append operations
    \item
      Let $T_i$ be the runtime of each $i$-th operation
      \begin{itemize}
        \item
          We have to copy $i-1$ elements
        \item
          The runtime of each copy operation
          $T_i = C_1 \cdot (i - 1) \in \mathcal{O}(i)$ is linear
      \end{itemize}
  \end{itemize}
  \begin{align*}
    T(n) &= \sum_{i=1}^n T_i
      = \sum_{i=1}^n \left(C_1 \cdot (i-1) \right)
      = C_1 \cdot \sum_{i=1}^n (i-1)\\
      &= C_1 \cdot \dfrac{n^2-n}{2} \in \mathcal{O}(n^2)
  \end{align*}
\end{frame}

%-------------------------------------------------------------------------------

\begin{frame}{Dynamic Fields}{Implementation 2}
  \begin{itemize}
    \item
      \textbf{Idea:}
      Better resize stratergy
    \item
      We allocate more space than needed
    \item
      We over-allocate a constant amount of elements
      \begin{itemize}
        \item
          Amount: $C = 3$ or $C = 100$
      \end{itemize}
  \end{itemize}
\end{frame}

%-------------------------------------------------------------------------------

\begin{frame}{Dynamic Fields}{Implementation 2}
  \lstinputlisting[
    language=Python,
    basicstyle=\small,
    tabsize=4,
    style={python-idle-code},
    escapechar={@},
    %    morekeywords={self},
    breaklines=false,
    emph={DynamicArray, append},
    emphstyle=\color{blue}
  ]{Lecture/Code/DynamicListImpl2_Part2.py}
\end{frame}

%-------------------------------------------------------------------------------

\begin{frame}{Dynamic Fields}{Implementation 2}
  \begin{figure}
    \caption{Runtime of \textit{DynamicArray}}
    \label{fig:runtime_dynamic_array_impl2}
    %TODO
  \end{figure}
  \begin{itemize}
    \item
      Why is the runtime still qubic?
  \end{itemize}
\end{frame}

%-------------------------------------------------------------------------------

\begin{frame}{Dynamic Fields}{Implementation 2}
  \textbf{Runtime for $C = 3$:}\\[0.5em]
  \begin{tabularx}{\linewidth}{@{}lcl}
    \def\FSAsize{3}\def\FSAelements{0}%
    \def\FSAcopy{0}\def\FSAdelete{0}\def\FSAinsert{1}%
    \raisebox{-0.25em}{%
\begin{adjustbox}{height=1.5em}%
\begin{tikzpicture}[%
  element/.style={
    draw=black,
    fill={black!20!white}
  }, element_copy/.style={
    draw=black,
    fill={Hell-Blau}
  }, element_delete/.style={
    draw=black,
    fill=white
  }, element_delete_cross/.style={
    draw=red,
    line cap=round,
    line width=0.25em
  }, element_insert/.style={
      draw=black,
      fill={yellow!50!white}
  }, element_free/.style={
    draw=black,
    fill=white
  }, arrow_copy/.style={
    ->,
    line width=0.3em,
    color={Mittel-Gruen}
  }, arrow_insert/.style={
    ->,
    line width=0.3em,
    color={Hell-Gruen}
  }
]%
\foreach \x in {0,...,\FSAelements}{
  \ifnum \x<\FSAelements
    \draw[element]
      (\x, 0.0)
      rectangle
      (\x + 1.0, 1.0);
    \draw (\x + 0.5, 0.5) node {\Huge \x};
  \fi
}
\foreach \x in {0,...,\FSAcopy}{
  \ifnum \x<\FSAcopy
    \draw[element_copy]
      (\x + \FSAelements, 0.0)
      rectangle
      (\x + \FSAelements + 1.0, 1.0);
    \FPeval{\value}{clip(\x+\FSAelements)}
    \draw (\x + \FSAelements + 0.5, 0.5)
      node {\Huge \value};
    \draw[arrow_copy]
      (\x + \FSAelements + 0.5, 1.4) --
      (\x + \FSAelements + 0.5, 0.75);
  \fi
}
\foreach \x in {0,...,\FSAdelete}{
  \ifnum \x<\FSAdelete
  \draw[element_delete]
    (\x + \FSAelements + \FSAcopy, 0.0)
    rectangle
    (\x + \FSAelements + \FSAcopy + 1.0, 1.0);
  \FPeval{\value}{clip(\x+\FSAelements+\FSAcopy)}
  \draw (\x + \FSAelements + \FSAcopy + 0.5, 0.5)
    node {\Huge \value};
  \fi
}
\foreach \x in {0,...,\FSAinsert}{
  \ifnum \x<\FSAinsert
    \draw[element_insert]
      (\x + \FSAelements + \FSAcopy + \FSAdelete, 0.0)
      rectangle
      (\x + \FSAelements + \FSAcopy + \FSAdelete + 1.0, 1.0);
    \FPeval{\value}{clip(\x+\FSAelements+\FSAcopy+\FSAdelete)}
    \draw (\x + \FSAelements + \FSAcopy + \FSAdelete + 0.5, 0.5)
      node {\Huge \value};
    \draw[arrow_insert]
      (\x + \FSAelements + \FSAcopy + \FSAdelete + 0.5, 1.4) --
      (\x + \FSAelements + \FSAcopy + \FSAdelete + 0.5, 0.75);
  \fi
}
\foreach \x in {0,...,\FSAsize}{
  \FPeval{\value}{clip(\FSAelements+\FSAcopy+\FSAdelete+\FSAinsert-2)}
  \ifnum \x>\value
    \FPeval{\value}{clip(\FSAsize-1)}
    \ifnum \x<\value
      \draw[element_free]
        (\x + 1.0, 0.0)
        rectangle
        (\x + 2.0, 1.0);
      \FPeval{\value}{clip(\value+1)}
%      \draw (\x + 1.5, 0.5)
%        node {\Huge \value};
    \fi
  \fi
}
\foreach \x in {0,...,\FSAdelete}{
  \ifnum \x<\FSAdelete
  \draw[element_delete_cross]
    (\x + \FSAelements + \FSAcopy, 1.0) --
    (\x + \FSAelements + \FSAcopy + 1.0, 0.0);
  \draw[element_delete_cross]
    (\x + \FSAelements + \FSAcopy, 0.0) --
    (\x + \FSAelements + \FSAcopy + 1.0, 1.0);
  \fi
}
\end{tikzpicture}%
\end{adjustbox}%
}% &
    $\mathcal{O}(1)$ &
    write 1 element\\
    \def\FSAsize{3}\def\FSAelements{1}%
    \def\FSAcopy{0}\def\FSAdelete{0}\def\FSAinsert{1}%
    \raisebox{-0.25em}{%
\begin{adjustbox}{height=1.5em}%
\begin{tikzpicture}[%
  element/.style={
    draw=black,
    fill={black!20!white}
  }, element_copy/.style={
    draw=black,
    fill={Hell-Blau}
  }, element_delete/.style={
    draw=black,
    fill=white
  }, element_delete_cross/.style={
    draw=red,
    line cap=round,
    line width=0.25em
  }, element_insert/.style={
      draw=black,
      fill={yellow!50!white}
  }, element_free/.style={
    draw=black,
    fill=white
  }, arrow_copy/.style={
    ->,
    line width=0.3em,
    color={Mittel-Gruen}
  }, arrow_insert/.style={
    ->,
    line width=0.3em,
    color={Hell-Gruen}
  }
]%
\foreach \x in {0,...,\FSAelements}{
  \ifnum \x<\FSAelements
    \draw[element]
      (\x, 0.0)
      rectangle
      (\x + 1.0, 1.0);
    \draw (\x + 0.5, 0.5) node {\Huge \x};
  \fi
}
\foreach \x in {0,...,\FSAcopy}{
  \ifnum \x<\FSAcopy
    \draw[element_copy]
      (\x + \FSAelements, 0.0)
      rectangle
      (\x + \FSAelements + 1.0, 1.0);
    \FPeval{\value}{clip(\x+\FSAelements)}
    \draw (\x + \FSAelements + 0.5, 0.5)
      node {\Huge \value};
    \draw[arrow_copy]
      (\x + \FSAelements + 0.5, 1.4) --
      (\x + \FSAelements + 0.5, 0.75);
  \fi
}
\foreach \x in {0,...,\FSAdelete}{
  \ifnum \x<\FSAdelete
  \draw[element_delete]
    (\x + \FSAelements + \FSAcopy, 0.0)
    rectangle
    (\x + \FSAelements + \FSAcopy + 1.0, 1.0);
  \FPeval{\value}{clip(\x+\FSAelements+\FSAcopy)}
  \draw (\x + \FSAelements + \FSAcopy + 0.5, 0.5)
    node {\Huge \value};
  \fi
}
\foreach \x in {0,...,\FSAinsert}{
  \ifnum \x<\FSAinsert
    \draw[element_insert]
      (\x + \FSAelements + \FSAcopy + \FSAdelete, 0.0)
      rectangle
      (\x + \FSAelements + \FSAcopy + \FSAdelete + 1.0, 1.0);
    \FPeval{\value}{clip(\x+\FSAelements+\FSAcopy+\FSAdelete)}
    \draw (\x + \FSAelements + \FSAcopy + \FSAdelete + 0.5, 0.5)
      node {\Huge \value};
    \draw[arrow_insert]
      (\x + \FSAelements + \FSAcopy + \FSAdelete + 0.5, 1.4) --
      (\x + \FSAelements + \FSAcopy + \FSAdelete + 0.5, 0.75);
  \fi
}
\foreach \x in {0,...,\FSAsize}{
  \FPeval{\value}{clip(\FSAelements+\FSAcopy+\FSAdelete+\FSAinsert-2)}
  \ifnum \x>\value
    \FPeval{\value}{clip(\FSAsize-1)}
    \ifnum \x<\value
      \draw[element_free]
        (\x + 1.0, 0.0)
        rectangle
        (\x + 2.0, 1.0);
      \FPeval{\value}{clip(\value+1)}
%      \draw (\x + 1.5, 0.5)
%        node {\Huge \value};
    \fi
  \fi
}
\foreach \x in {0,...,\FSAdelete}{
  \ifnum \x<\FSAdelete
  \draw[element_delete_cross]
    (\x + \FSAelements + \FSAcopy, 1.0) --
    (\x + \FSAelements + \FSAcopy + 1.0, 0.0);
  \draw[element_delete_cross]
    (\x + \FSAelements + \FSAcopy, 0.0) --
    (\x + \FSAelements + \FSAcopy + 1.0, 1.0);
  \fi
}
\end{tikzpicture}%
\end{adjustbox}%
}% &
    $\mathcal{O}(1)$ &
    write 1 element\\
    \def\FSAsize{3}\def\FSAelements{2}%
    \def\FSAcopy{0}\def\FSAdelete{0}\def\FSAinsert{1}%
    \raisebox{-0.25em}{%
\begin{adjustbox}{height=1.5em}%
\begin{tikzpicture}[%
  element/.style={
    draw=black,
    fill={black!20!white}
  }, element_copy/.style={
    draw=black,
    fill={Hell-Blau}
  }, element_delete/.style={
    draw=black,
    fill=white
  }, element_delete_cross/.style={
    draw=red,
    line cap=round,
    line width=0.25em
  }, element_insert/.style={
      draw=black,
      fill={yellow!50!white}
  }, element_free/.style={
    draw=black,
    fill=white
  }, arrow_copy/.style={
    ->,
    line width=0.3em,
    color={Mittel-Gruen}
  }, arrow_insert/.style={
    ->,
    line width=0.3em,
    color={Hell-Gruen}
  }
]%
\foreach \x in {0,...,\FSAelements}{
  \ifnum \x<\FSAelements
    \draw[element]
      (\x, 0.0)
      rectangle
      (\x + 1.0, 1.0);
    \draw (\x + 0.5, 0.5) node {\Huge \x};
  \fi
}
\foreach \x in {0,...,\FSAcopy}{
  \ifnum \x<\FSAcopy
    \draw[element_copy]
      (\x + \FSAelements, 0.0)
      rectangle
      (\x + \FSAelements + 1.0, 1.0);
    \FPeval{\value}{clip(\x+\FSAelements)}
    \draw (\x + \FSAelements + 0.5, 0.5)
      node {\Huge \value};
    \draw[arrow_copy]
      (\x + \FSAelements + 0.5, 1.4) --
      (\x + \FSAelements + 0.5, 0.75);
  \fi
}
\foreach \x in {0,...,\FSAdelete}{
  \ifnum \x<\FSAdelete
  \draw[element_delete]
    (\x + \FSAelements + \FSAcopy, 0.0)
    rectangle
    (\x + \FSAelements + \FSAcopy + 1.0, 1.0);
  \FPeval{\value}{clip(\x+\FSAelements+\FSAcopy)}
  \draw (\x + \FSAelements + \FSAcopy + 0.5, 0.5)
    node {\Huge \value};
  \fi
}
\foreach \x in {0,...,\FSAinsert}{
  \ifnum \x<\FSAinsert
    \draw[element_insert]
      (\x + \FSAelements + \FSAcopy + \FSAdelete, 0.0)
      rectangle
      (\x + \FSAelements + \FSAcopy + \FSAdelete + 1.0, 1.0);
    \FPeval{\value}{clip(\x+\FSAelements+\FSAcopy+\FSAdelete)}
    \draw (\x + \FSAelements + \FSAcopy + \FSAdelete + 0.5, 0.5)
      node {\Huge \value};
    \draw[arrow_insert]
      (\x + \FSAelements + \FSAcopy + \FSAdelete + 0.5, 1.4) --
      (\x + \FSAelements + \FSAcopy + \FSAdelete + 0.5, 0.75);
  \fi
}
\foreach \x in {0,...,\FSAsize}{
  \FPeval{\value}{clip(\FSAelements+\FSAcopy+\FSAdelete+\FSAinsert-2)}
  \ifnum \x>\value
    \FPeval{\value}{clip(\FSAsize-1)}
    \ifnum \x<\value
      \draw[element_free]
        (\x + 1.0, 0.0)
        rectangle
        (\x + 2.0, 1.0);
      \FPeval{\value}{clip(\value+1)}
%      \draw (\x + 1.5, 0.5)
%        node {\Huge \value};
    \fi
  \fi
}
\foreach \x in {0,...,\FSAdelete}{
  \ifnum \x<\FSAdelete
  \draw[element_delete_cross]
    (\x + \FSAelements + \FSAcopy, 1.0) --
    (\x + \FSAelements + \FSAcopy + 1.0, 0.0);
  \draw[element_delete_cross]
    (\x + \FSAelements + \FSAcopy, 0.0) --
    (\x + \FSAelements + \FSAcopy + 1.0, 1.0);
  \fi
}
\end{tikzpicture}%
\end{adjustbox}%
}% &
    $\mathcal{O}(1)$ &
    write 1 element\\
    \def\FSAsize{6}\def\FSAelements{0}%
    \def\FSAcopy{3}\def\FSAdelete{0}\def\FSAinsert{1}%
    \raisebox{-0.25em}{%
\begin{adjustbox}{height=1.5em}%
\begin{tikzpicture}[%
  element/.style={
    draw=black,
    fill={black!20!white}
  }, element_copy/.style={
    draw=black,
    fill={Hell-Blau}
  }, element_delete/.style={
    draw=black,
    fill=white
  }, element_delete_cross/.style={
    draw=red,
    line cap=round,
    line width=0.25em
  }, element_insert/.style={
      draw=black,
      fill={yellow!50!white}
  }, element_free/.style={
    draw=black,
    fill=white
  }, arrow_copy/.style={
    ->,
    line width=0.3em,
    color={Mittel-Gruen}
  }, arrow_insert/.style={
    ->,
    line width=0.3em,
    color={Hell-Gruen}
  }
]%
\foreach \x in {0,...,\FSAelements}{
  \ifnum \x<\FSAelements
    \draw[element]
      (\x, 0.0)
      rectangle
      (\x + 1.0, 1.0);
    \draw (\x + 0.5, 0.5) node {\Huge \x};
  \fi
}
\foreach \x in {0,...,\FSAcopy}{
  \ifnum \x<\FSAcopy
    \draw[element_copy]
      (\x + \FSAelements, 0.0)
      rectangle
      (\x + \FSAelements + 1.0, 1.0);
    \FPeval{\value}{clip(\x+\FSAelements)}
    \draw (\x + \FSAelements + 0.5, 0.5)
      node {\Huge \value};
    \draw[arrow_copy]
      (\x + \FSAelements + 0.5, 1.4) --
      (\x + \FSAelements + 0.5, 0.75);
  \fi
}
\foreach \x in {0,...,\FSAdelete}{
  \ifnum \x<\FSAdelete
  \draw[element_delete]
    (\x + \FSAelements + \FSAcopy, 0.0)
    rectangle
    (\x + \FSAelements + \FSAcopy + 1.0, 1.0);
  \FPeval{\value}{clip(\x+\FSAelements+\FSAcopy)}
  \draw (\x + \FSAelements + \FSAcopy + 0.5, 0.5)
    node {\Huge \value};
  \fi
}
\foreach \x in {0,...,\FSAinsert}{
  \ifnum \x<\FSAinsert
    \draw[element_insert]
      (\x + \FSAelements + \FSAcopy + \FSAdelete, 0.0)
      rectangle
      (\x + \FSAelements + \FSAcopy + \FSAdelete + 1.0, 1.0);
    \FPeval{\value}{clip(\x+\FSAelements+\FSAcopy+\FSAdelete)}
    \draw (\x + \FSAelements + \FSAcopy + \FSAdelete + 0.5, 0.5)
      node {\Huge \value};
    \draw[arrow_insert]
      (\x + \FSAelements + \FSAcopy + \FSAdelete + 0.5, 1.4) --
      (\x + \FSAelements + \FSAcopy + \FSAdelete + 0.5, 0.75);
  \fi
}
\foreach \x in {0,...,\FSAsize}{
  \FPeval{\value}{clip(\FSAelements+\FSAcopy+\FSAdelete+\FSAinsert-2)}
  \ifnum \x>\value
    \FPeval{\value}{clip(\FSAsize-1)}
    \ifnum \x<\value
      \draw[element_free]
        (\x + 1.0, 0.0)
        rectangle
        (\x + 2.0, 1.0);
      \FPeval{\value}{clip(\value+1)}
%      \draw (\x + 1.5, 0.5)
%        node {\Huge \value};
    \fi
  \fi
}
\foreach \x in {0,...,\FSAdelete}{
  \ifnum \x<\FSAdelete
  \draw[element_delete_cross]
    (\x + \FSAelements + \FSAcopy, 1.0) --
    (\x + \FSAelements + \FSAcopy + 1.0, 0.0);
  \draw[element_delete_cross]
    (\x + \FSAelements + \FSAcopy, 0.0) --
    (\x + \FSAelements + \FSAcopy + 1.0, 1.0);
  \fi
}
\end{tikzpicture}%
\end{adjustbox}%
}% &
    $\mathcal{O}(1 + {\color{Mittel-Blau}3})$ &
    write 1 element, {\color{Mittel-Blau}copy 3 elements}\\
    \def\FSAsize{6}\def\FSAelements{4}%
    \def\FSAcopy{0}\def\FSAdelete{0}\def\FSAinsert{1}%
    \raisebox{-0.25em}{%
\begin{adjustbox}{height=1.5em}%
\begin{tikzpicture}[%
  element/.style={
    draw=black,
    fill={black!20!white}
  }, element_copy/.style={
    draw=black,
    fill={Hell-Blau}
  }, element_delete/.style={
    draw=black,
    fill=white
  }, element_delete_cross/.style={
    draw=red,
    line cap=round,
    line width=0.25em
  }, element_insert/.style={
      draw=black,
      fill={yellow!50!white}
  }, element_free/.style={
    draw=black,
    fill=white
  }, arrow_copy/.style={
    ->,
    line width=0.3em,
    color={Mittel-Gruen}
  }, arrow_insert/.style={
    ->,
    line width=0.3em,
    color={Hell-Gruen}
  }
]%
\foreach \x in {0,...,\FSAelements}{
  \ifnum \x<\FSAelements
    \draw[element]
      (\x, 0.0)
      rectangle
      (\x + 1.0, 1.0);
    \draw (\x + 0.5, 0.5) node {\Huge \x};
  \fi
}
\foreach \x in {0,...,\FSAcopy}{
  \ifnum \x<\FSAcopy
    \draw[element_copy]
      (\x + \FSAelements, 0.0)
      rectangle
      (\x + \FSAelements + 1.0, 1.0);
    \FPeval{\value}{clip(\x+\FSAelements)}
    \draw (\x + \FSAelements + 0.5, 0.5)
      node {\Huge \value};
    \draw[arrow_copy]
      (\x + \FSAelements + 0.5, 1.4) --
      (\x + \FSAelements + 0.5, 0.75);
  \fi
}
\foreach \x in {0,...,\FSAdelete}{
  \ifnum \x<\FSAdelete
  \draw[element_delete]
    (\x + \FSAelements + \FSAcopy, 0.0)
    rectangle
    (\x + \FSAelements + \FSAcopy + 1.0, 1.0);
  \FPeval{\value}{clip(\x+\FSAelements+\FSAcopy)}
  \draw (\x + \FSAelements + \FSAcopy + 0.5, 0.5)
    node {\Huge \value};
  \fi
}
\foreach \x in {0,...,\FSAinsert}{
  \ifnum \x<\FSAinsert
    \draw[element_insert]
      (\x + \FSAelements + \FSAcopy + \FSAdelete, 0.0)
      rectangle
      (\x + \FSAelements + \FSAcopy + \FSAdelete + 1.0, 1.0);
    \FPeval{\value}{clip(\x+\FSAelements+\FSAcopy+\FSAdelete)}
    \draw (\x + \FSAelements + \FSAcopy + \FSAdelete + 0.5, 0.5)
      node {\Huge \value};
    \draw[arrow_insert]
      (\x + \FSAelements + \FSAcopy + \FSAdelete + 0.5, 1.4) --
      (\x + \FSAelements + \FSAcopy + \FSAdelete + 0.5, 0.75);
  \fi
}
\foreach \x in {0,...,\FSAsize}{
  \FPeval{\value}{clip(\FSAelements+\FSAcopy+\FSAdelete+\FSAinsert-2)}
  \ifnum \x>\value
    \FPeval{\value}{clip(\FSAsize-1)}
    \ifnum \x<\value
      \draw[element_free]
        (\x + 1.0, 0.0)
        rectangle
        (\x + 2.0, 1.0);
      \FPeval{\value}{clip(\value+1)}
%      \draw (\x + 1.5, 0.5)
%        node {\Huge \value};
    \fi
  \fi
}
\foreach \x in {0,...,\FSAdelete}{
  \ifnum \x<\FSAdelete
  \draw[element_delete_cross]
    (\x + \FSAelements + \FSAcopy, 1.0) --
    (\x + \FSAelements + \FSAcopy + 1.0, 0.0);
  \draw[element_delete_cross]
    (\x + \FSAelements + \FSAcopy, 0.0) --
    (\x + \FSAelements + \FSAcopy + 1.0, 1.0);
  \fi
}
\end{tikzpicture}%
\end{adjustbox}%
}% &
    $\mathcal{O}(1)$ &
    write 1 element\\
    \def\FSAsize{6}\def\FSAelements{5}%
    \def\FSAcopy{0}\def\FSAdelete{0}\def\FSAinsert{1}%
    \raisebox{-0.25em}{%
\begin{adjustbox}{height=1.5em}%
\begin{tikzpicture}[%
  element/.style={
    draw=black,
    fill={black!20!white}
  }, element_copy/.style={
    draw=black,
    fill={Hell-Blau}
  }, element_delete/.style={
    draw=black,
    fill=white
  }, element_delete_cross/.style={
    draw=red,
    line cap=round,
    line width=0.25em
  }, element_insert/.style={
      draw=black,
      fill={yellow!50!white}
  }, element_free/.style={
    draw=black,
    fill=white
  }, arrow_copy/.style={
    ->,
    line width=0.3em,
    color={Mittel-Gruen}
  }, arrow_insert/.style={
    ->,
    line width=0.3em,
    color={Hell-Gruen}
  }
]%
\foreach \x in {0,...,\FSAelements}{
  \ifnum \x<\FSAelements
    \draw[element]
      (\x, 0.0)
      rectangle
      (\x + 1.0, 1.0);
    \draw (\x + 0.5, 0.5) node {\Huge \x};
  \fi
}
\foreach \x in {0,...,\FSAcopy}{
  \ifnum \x<\FSAcopy
    \draw[element_copy]
      (\x + \FSAelements, 0.0)
      rectangle
      (\x + \FSAelements + 1.0, 1.0);
    \FPeval{\value}{clip(\x+\FSAelements)}
    \draw (\x + \FSAelements + 0.5, 0.5)
      node {\Huge \value};
    \draw[arrow_copy]
      (\x + \FSAelements + 0.5, 1.4) --
      (\x + \FSAelements + 0.5, 0.75);
  \fi
}
\foreach \x in {0,...,\FSAdelete}{
  \ifnum \x<\FSAdelete
  \draw[element_delete]
    (\x + \FSAelements + \FSAcopy, 0.0)
    rectangle
    (\x + \FSAelements + \FSAcopy + 1.0, 1.0);
  \FPeval{\value}{clip(\x+\FSAelements+\FSAcopy)}
  \draw (\x + \FSAelements + \FSAcopy + 0.5, 0.5)
    node {\Huge \value};
  \fi
}
\foreach \x in {0,...,\FSAinsert}{
  \ifnum \x<\FSAinsert
    \draw[element_insert]
      (\x + \FSAelements + \FSAcopy + \FSAdelete, 0.0)
      rectangle
      (\x + \FSAelements + \FSAcopy + \FSAdelete + 1.0, 1.0);
    \FPeval{\value}{clip(\x+\FSAelements+\FSAcopy+\FSAdelete)}
    \draw (\x + \FSAelements + \FSAcopy + \FSAdelete + 0.5, 0.5)
      node {\Huge \value};
    \draw[arrow_insert]
      (\x + \FSAelements + \FSAcopy + \FSAdelete + 0.5, 1.4) --
      (\x + \FSAelements + \FSAcopy + \FSAdelete + 0.5, 0.75);
  \fi
}
\foreach \x in {0,...,\FSAsize}{
  \FPeval{\value}{clip(\FSAelements+\FSAcopy+\FSAdelete+\FSAinsert-2)}
  \ifnum \x>\value
    \FPeval{\value}{clip(\FSAsize-1)}
    \ifnum \x<\value
      \draw[element_free]
        (\x + 1.0, 0.0)
        rectangle
        (\x + 2.0, 1.0);
      \FPeval{\value}{clip(\value+1)}
%      \draw (\x + 1.5, 0.5)
%        node {\Huge \value};
    \fi
  \fi
}
\foreach \x in {0,...,\FSAdelete}{
  \ifnum \x<\FSAdelete
  \draw[element_delete_cross]
    (\x + \FSAelements + \FSAcopy, 1.0) --
    (\x + \FSAelements + \FSAcopy + 1.0, 0.0);
  \draw[element_delete_cross]
    (\x + \FSAelements + \FSAcopy, 0.0) --
    (\x + \FSAelements + \FSAcopy + 1.0, 1.0);
  \fi
}
\end{tikzpicture}%
\end{adjustbox}%
}% &
    $\mathcal{O}(1)$ &
    write 1 element\\
    \def\FSAsize{9}\def\FSAelements{0}%
    \def\FSAcopy{6}\def\FSAdelete{0}\def\FSAinsert{1}%
    \raisebox{-0.25em}{%
\begin{adjustbox}{height=1.5em}%
\begin{tikzpicture}[%
  element/.style={
    draw=black,
    fill={black!20!white}
  }, element_copy/.style={
    draw=black,
    fill={Hell-Blau}
  }, element_delete/.style={
    draw=black,
    fill=white
  }, element_delete_cross/.style={
    draw=red,
    line cap=round,
    line width=0.25em
  }, element_insert/.style={
      draw=black,
      fill={yellow!50!white}
  }, element_free/.style={
    draw=black,
    fill=white
  }, arrow_copy/.style={
    ->,
    line width=0.3em,
    color={Mittel-Gruen}
  }, arrow_insert/.style={
    ->,
    line width=0.3em,
    color={Hell-Gruen}
  }
]%
\foreach \x in {0,...,\FSAelements}{
  \ifnum \x<\FSAelements
    \draw[element]
      (\x, 0.0)
      rectangle
      (\x + 1.0, 1.0);
    \draw (\x + 0.5, 0.5) node {\Huge \x};
  \fi
}
\foreach \x in {0,...,\FSAcopy}{
  \ifnum \x<\FSAcopy
    \draw[element_copy]
      (\x + \FSAelements, 0.0)
      rectangle
      (\x + \FSAelements + 1.0, 1.0);
    \FPeval{\value}{clip(\x+\FSAelements)}
    \draw (\x + \FSAelements + 0.5, 0.5)
      node {\Huge \value};
    \draw[arrow_copy]
      (\x + \FSAelements + 0.5, 1.4) --
      (\x + \FSAelements + 0.5, 0.75);
  \fi
}
\foreach \x in {0,...,\FSAdelete}{
  \ifnum \x<\FSAdelete
  \draw[element_delete]
    (\x + \FSAelements + \FSAcopy, 0.0)
    rectangle
    (\x + \FSAelements + \FSAcopy + 1.0, 1.0);
  \FPeval{\value}{clip(\x+\FSAelements+\FSAcopy)}
  \draw (\x + \FSAelements + \FSAcopy + 0.5, 0.5)
    node {\Huge \value};
  \fi
}
\foreach \x in {0,...,\FSAinsert}{
  \ifnum \x<\FSAinsert
    \draw[element_insert]
      (\x + \FSAelements + \FSAcopy + \FSAdelete, 0.0)
      rectangle
      (\x + \FSAelements + \FSAcopy + \FSAdelete + 1.0, 1.0);
    \FPeval{\value}{clip(\x+\FSAelements+\FSAcopy+\FSAdelete)}
    \draw (\x + \FSAelements + \FSAcopy + \FSAdelete + 0.5, 0.5)
      node {\Huge \value};
    \draw[arrow_insert]
      (\x + \FSAelements + \FSAcopy + \FSAdelete + 0.5, 1.4) --
      (\x + \FSAelements + \FSAcopy + \FSAdelete + 0.5, 0.75);
  \fi
}
\foreach \x in {0,...,\FSAsize}{
  \FPeval{\value}{clip(\FSAelements+\FSAcopy+\FSAdelete+\FSAinsert-2)}
  \ifnum \x>\value
    \FPeval{\value}{clip(\FSAsize-1)}
    \ifnum \x<\value
      \draw[element_free]
        (\x + 1.0, 0.0)
        rectangle
        (\x + 2.0, 1.0);
      \FPeval{\value}{clip(\value+1)}
%      \draw (\x + 1.5, 0.5)
%        node {\Huge \value};
    \fi
  \fi
}
\foreach \x in {0,...,\FSAdelete}{
  \ifnum \x<\FSAdelete
  \draw[element_delete_cross]
    (\x + \FSAelements + \FSAcopy, 1.0) --
    (\x + \FSAelements + \FSAcopy + 1.0, 0.0);
  \draw[element_delete_cross]
    (\x + \FSAelements + \FSAcopy, 0.0) --
    (\x + \FSAelements + \FSAcopy + 1.0, 1.0);
  \fi
}
\end{tikzpicture}%
\end{adjustbox}%
}% &
    $\mathcal{O}(1 + {\color{Mittel-Blau}6})$ &
    write 1 element, {\color{Mittel-Blau}copy 6 elements}\\
    \hspace*{1.5em}\dots & \dots & \hspace*{1.5em}\dots
  \end{tabularx}
\end{frame}

%-------------------------------------------------------------------------------

\begin{frame}{Dynamic Fields}{Implementation 2}
  \textbf{Analysis:}
  \begin{itemize}
    \item
      $\frac{C - 1}{C} = \frac{2}{3}$ of all operations are $\in \mathcal{O}(1)$
    \item
      Every $C$ steps we have a copy operation
    \begin{itemize}
      \item
        We have to copy $\frac{i- 1}{C}$ elements
      \item
        The runtime of each copy operation is linear\\
        $\Rightarrow T_i =
          \underbrace{
            C_2\vphantom{\begin{cases}C_1\\0\end{cases}}
          }_\text{insert}
          + \underbrace{\begin{cases}
            C_1 \cdot \frac{i - 1}{C} & \text{if } i \mod C = 1\\
            0 & \text{else}
            \end{cases}}_\text{copy}
          \in \mathcal{O}(i)$
    \end{itemize}
  \end{itemize}
  \begin{align*}
    T(n)
      &= \sum_{i=1}^{n} T_i
      = \sum_{i=1}^{n/C} \left(C_1 \cdot C \cdot (i-1) \right)
        + \sum_{i=1}^n C_2\\
      &= \sum_{i=1}^{n/C} \left(C_1 \cdot C \cdot (i-1) \right)
        + \underbrace{\dots}_{\mathcal{O}(n)}
  \end{align*}
\end{frame}

%-------------------------------------------------------------------------------

\begin{frame}{Dynamic Fields}{Implementation 2}
  \textbf{Analysis:}
  \begin{itemize}
    \item
      The factor of $n^2$ is getting smaller
    \item
      The runtime still remains $T(n) \in \mathcal{O}(n^2)$
  \end{itemize}
  \begin{align*}%
    T(n)
      &= \sum_{i=1}^{n/C} \left(C_1 \cdot C \cdot (i-1) \right)
        + \underbrace{\dots}_{\mathcal{O}(n)}\\
    {} &\approx C_1 \cdot C \cdot \sum_{i=1}^{n/C} (i-1)
    = C_1 \cdot C \cdot \dfrac{\frac{n^2}{C^2}-\frac{n}{C}}{2}\\
    {} &= C_1 \cdot \dfrac{\frac{n^2}{C}-n}{2}
    \in \mathcal{O}(n^2)
  \end{align*}
\end{frame}

%-------------------------------------------------------------------------------

\begin{frame}{Dynamic Fields}{Implementation 3}
  \begin{itemize}
    \item
      \textbf{Idea:}
      Multiply the size with a constant $C > 1$
      \begin{itemize}
        \item
          Amount: $C = 2$ or $C = \frac{3}{2}$
      \end{itemize}
    \item
      The distance between reallocations is increasing
  \end{itemize}
\end{frame}

%-------------------------------------------------------------------------------

\begin{frame}{Dynamic Fields}{Implementation 3}
  \lstinputlisting[
    language=Python,
    basicstyle=\small,
    tabsize=4,
    style={python-idle-code},
    escapechar={@},
%    morekeywords={self},
    breaklines=false,
    emph={DynamicArray, append},
    emphstyle=\color{blue}
  ]{Lecture/Code/DynamicListImpl3_Part2.py}
\end{frame}

%-------------------------------------------------------------------------------

\begin{frame}{Dynamic Fields}{Implementation 3}
  \begin{figure}
    \caption{Runtime of \textit{DynamicArray}}
    \label{fig:runtime_dynamic_array_impl3}
    %TODO
  \end{figure}
  \begin{itemize}
    \item
      Now the runtime is linear with some bumps. Why?
  \end{itemize}
\end{frame}

%-------------------------------------------------------------------------------

\begin{frame}{Dynamic Fields}{Implementation 2}
  \textbf{Runtime for $C = 2$ (Double the size):}\\[0.5em]
  \begin{tabularx}{\linewidth}{@{}lcl}
    \def\FSAsize{1}\def\FSAelements{0}%
    \def\FSAcopy{0}\def\FSAdelete{0}\def\FSAinsert{1}%
    \raisebox{-0.25em}{%
\begin{adjustbox}{height=1.5em}%
\begin{tikzpicture}[%
  element/.style={
    draw=black,
    fill={black!20!white}
  }, element_copy/.style={
    draw=black,
    fill={Hell-Blau}
  }, element_delete/.style={
    draw=black,
    fill=white
  }, element_delete_cross/.style={
    draw=red,
    line cap=round,
    line width=0.25em
  }, element_insert/.style={
      draw=black,
      fill={yellow!50!white}
  }, element_free/.style={
    draw=black,
    fill=white
  }, arrow_copy/.style={
    ->,
    line width=0.3em,
    color={Mittel-Gruen}
  }, arrow_insert/.style={
    ->,
    line width=0.3em,
    color={Hell-Gruen}
  }
]%
\foreach \x in {0,...,\FSAelements}{
  \ifnum \x<\FSAelements
    \draw[element]
      (\x, 0.0)
      rectangle
      (\x + 1.0, 1.0);
    \draw (\x + 0.5, 0.5) node {\Huge \x};
  \fi
}
\foreach \x in {0,...,\FSAcopy}{
  \ifnum \x<\FSAcopy
    \draw[element_copy]
      (\x + \FSAelements, 0.0)
      rectangle
      (\x + \FSAelements + 1.0, 1.0);
    \FPeval{\value}{clip(\x+\FSAelements)}
    \draw (\x + \FSAelements + 0.5, 0.5)
      node {\Huge \value};
    \draw[arrow_copy]
      (\x + \FSAelements + 0.5, 1.4) --
      (\x + \FSAelements + 0.5, 0.75);
  \fi
}
\foreach \x in {0,...,\FSAdelete}{
  \ifnum \x<\FSAdelete
  \draw[element_delete]
    (\x + \FSAelements + \FSAcopy, 0.0)
    rectangle
    (\x + \FSAelements + \FSAcopy + 1.0, 1.0);
  \FPeval{\value}{clip(\x+\FSAelements+\FSAcopy)}
  \draw (\x + \FSAelements + \FSAcopy + 0.5, 0.5)
    node {\Huge \value};
  \fi
}
\foreach \x in {0,...,\FSAinsert}{
  \ifnum \x<\FSAinsert
    \draw[element_insert]
      (\x + \FSAelements + \FSAcopy + \FSAdelete, 0.0)
      rectangle
      (\x + \FSAelements + \FSAcopy + \FSAdelete + 1.0, 1.0);
    \FPeval{\value}{clip(\x+\FSAelements+\FSAcopy+\FSAdelete)}
    \draw (\x + \FSAelements + \FSAcopy + \FSAdelete + 0.5, 0.5)
      node {\Huge \value};
    \draw[arrow_insert]
      (\x + \FSAelements + \FSAcopy + \FSAdelete + 0.5, 1.4) --
      (\x + \FSAelements + \FSAcopy + \FSAdelete + 0.5, 0.75);
  \fi
}
\foreach \x in {0,...,\FSAsize}{
  \FPeval{\value}{clip(\FSAelements+\FSAcopy+\FSAdelete+\FSAinsert-2)}
  \ifnum \x>\value
    \FPeval{\value}{clip(\FSAsize-1)}
    \ifnum \x<\value
      \draw[element_free]
        (\x + 1.0, 0.0)
        rectangle
        (\x + 2.0, 1.0);
      \FPeval{\value}{clip(\value+1)}
%      \draw (\x + 1.5, 0.5)
%        node {\Huge \value};
    \fi
  \fi
}
\foreach \x in {0,...,\FSAdelete}{
  \ifnum \x<\FSAdelete
  \draw[element_delete_cross]
    (\x + \FSAelements + \FSAcopy, 1.0) --
    (\x + \FSAelements + \FSAcopy + 1.0, 0.0);
  \draw[element_delete_cross]
    (\x + \FSAelements + \FSAcopy, 0.0) --
    (\x + \FSAelements + \FSAcopy + 1.0, 1.0);
  \fi
}
\end{tikzpicture}%
\end{adjustbox}%
}% &
    $\mathcal{O}(1)$ &
    write 1\\
    \def\FSAsize{2}\def\FSAelements{0}%
    \def\FSAcopy{1}\def\FSAdelete{0}\def\FSAinsert{1}%
    \raisebox{-0.25em}{%
\begin{adjustbox}{height=1.5em}%
\begin{tikzpicture}[%
  element/.style={
    draw=black,
    fill={black!20!white}
  }, element_copy/.style={
    draw=black,
    fill={Hell-Blau}
  }, element_delete/.style={
    draw=black,
    fill=white
  }, element_delete_cross/.style={
    draw=red,
    line cap=round,
    line width=0.25em
  }, element_insert/.style={
      draw=black,
      fill={yellow!50!white}
  }, element_free/.style={
    draw=black,
    fill=white
  }, arrow_copy/.style={
    ->,
    line width=0.3em,
    color={Mittel-Gruen}
  }, arrow_insert/.style={
    ->,
    line width=0.3em,
    color={Hell-Gruen}
  }
]%
\foreach \x in {0,...,\FSAelements}{
  \ifnum \x<\FSAelements
    \draw[element]
      (\x, 0.0)
      rectangle
      (\x + 1.0, 1.0);
    \draw (\x + 0.5, 0.5) node {\Huge \x};
  \fi
}
\foreach \x in {0,...,\FSAcopy}{
  \ifnum \x<\FSAcopy
    \draw[element_copy]
      (\x + \FSAelements, 0.0)
      rectangle
      (\x + \FSAelements + 1.0, 1.0);
    \FPeval{\value}{clip(\x+\FSAelements)}
    \draw (\x + \FSAelements + 0.5, 0.5)
      node {\Huge \value};
    \draw[arrow_copy]
      (\x + \FSAelements + 0.5, 1.4) --
      (\x + \FSAelements + 0.5, 0.75);
  \fi
}
\foreach \x in {0,...,\FSAdelete}{
  \ifnum \x<\FSAdelete
  \draw[element_delete]
    (\x + \FSAelements + \FSAcopy, 0.0)
    rectangle
    (\x + \FSAelements + \FSAcopy + 1.0, 1.0);
  \FPeval{\value}{clip(\x+\FSAelements+\FSAcopy)}
  \draw (\x + \FSAelements + \FSAcopy + 0.5, 0.5)
    node {\Huge \value};
  \fi
}
\foreach \x in {0,...,\FSAinsert}{
  \ifnum \x<\FSAinsert
    \draw[element_insert]
      (\x + \FSAelements + \FSAcopy + \FSAdelete, 0.0)
      rectangle
      (\x + \FSAelements + \FSAcopy + \FSAdelete + 1.0, 1.0);
    \FPeval{\value}{clip(\x+\FSAelements+\FSAcopy+\FSAdelete)}
    \draw (\x + \FSAelements + \FSAcopy + \FSAdelete + 0.5, 0.5)
      node {\Huge \value};
    \draw[arrow_insert]
      (\x + \FSAelements + \FSAcopy + \FSAdelete + 0.5, 1.4) --
      (\x + \FSAelements + \FSAcopy + \FSAdelete + 0.5, 0.75);
  \fi
}
\foreach \x in {0,...,\FSAsize}{
  \FPeval{\value}{clip(\FSAelements+\FSAcopy+\FSAdelete+\FSAinsert-2)}
  \ifnum \x>\value
    \FPeval{\value}{clip(\FSAsize-1)}
    \ifnum \x<\value
      \draw[element_free]
        (\x + 1.0, 0.0)
        rectangle
        (\x + 2.0, 1.0);
      \FPeval{\value}{clip(\value+1)}
%      \draw (\x + 1.5, 0.5)
%        node {\Huge \value};
    \fi
  \fi
}
\foreach \x in {0,...,\FSAdelete}{
  \ifnum \x<\FSAdelete
  \draw[element_delete_cross]
    (\x + \FSAelements + \FSAcopy, 1.0) --
    (\x + \FSAelements + \FSAcopy + 1.0, 0.0);
  \draw[element_delete_cross]
    (\x + \FSAelements + \FSAcopy, 0.0) --
    (\x + \FSAelements + \FSAcopy + 1.0, 1.0);
  \fi
}
\end{tikzpicture}%
\end{adjustbox}%
}% &
    $\mathcal{O}(1 + {\color{Mittel-Blau}1})$ &
    write 1, {\color{Mittel-Blau}copy 1 element}\\
    \def\FSAsize{4}\def\FSAelements{0}%
    \def\FSAcopy{2}\def\FSAdelete{0}\def\FSAinsert{1}%
    \raisebox{-0.25em}{%
\begin{adjustbox}{height=1.5em}%
\begin{tikzpicture}[%
  element/.style={
    draw=black,
    fill={black!20!white}
  }, element_copy/.style={
    draw=black,
    fill={Hell-Blau}
  }, element_delete/.style={
    draw=black,
    fill=white
  }, element_delete_cross/.style={
    draw=red,
    line cap=round,
    line width=0.25em
  }, element_insert/.style={
      draw=black,
      fill={yellow!50!white}
  }, element_free/.style={
    draw=black,
    fill=white
  }, arrow_copy/.style={
    ->,
    line width=0.3em,
    color={Mittel-Gruen}
  }, arrow_insert/.style={
    ->,
    line width=0.3em,
    color={Hell-Gruen}
  }
]%
\foreach \x in {0,...,\FSAelements}{
  \ifnum \x<\FSAelements
    \draw[element]
      (\x, 0.0)
      rectangle
      (\x + 1.0, 1.0);
    \draw (\x + 0.5, 0.5) node {\Huge \x};
  \fi
}
\foreach \x in {0,...,\FSAcopy}{
  \ifnum \x<\FSAcopy
    \draw[element_copy]
      (\x + \FSAelements, 0.0)
      rectangle
      (\x + \FSAelements + 1.0, 1.0);
    \FPeval{\value}{clip(\x+\FSAelements)}
    \draw (\x + \FSAelements + 0.5, 0.5)
      node {\Huge \value};
    \draw[arrow_copy]
      (\x + \FSAelements + 0.5, 1.4) --
      (\x + \FSAelements + 0.5, 0.75);
  \fi
}
\foreach \x in {0,...,\FSAdelete}{
  \ifnum \x<\FSAdelete
  \draw[element_delete]
    (\x + \FSAelements + \FSAcopy, 0.0)
    rectangle
    (\x + \FSAelements + \FSAcopy + 1.0, 1.0);
  \FPeval{\value}{clip(\x+\FSAelements+\FSAcopy)}
  \draw (\x + \FSAelements + \FSAcopy + 0.5, 0.5)
    node {\Huge \value};
  \fi
}
\foreach \x in {0,...,\FSAinsert}{
  \ifnum \x<\FSAinsert
    \draw[element_insert]
      (\x + \FSAelements + \FSAcopy + \FSAdelete, 0.0)
      rectangle
      (\x + \FSAelements + \FSAcopy + \FSAdelete + 1.0, 1.0);
    \FPeval{\value}{clip(\x+\FSAelements+\FSAcopy+\FSAdelete)}
    \draw (\x + \FSAelements + \FSAcopy + \FSAdelete + 0.5, 0.5)
      node {\Huge \value};
    \draw[arrow_insert]
      (\x + \FSAelements + \FSAcopy + \FSAdelete + 0.5, 1.4) --
      (\x + \FSAelements + \FSAcopy + \FSAdelete + 0.5, 0.75);
  \fi
}
\foreach \x in {0,...,\FSAsize}{
  \FPeval{\value}{clip(\FSAelements+\FSAcopy+\FSAdelete+\FSAinsert-2)}
  \ifnum \x>\value
    \FPeval{\value}{clip(\FSAsize-1)}
    \ifnum \x<\value
      \draw[element_free]
        (\x + 1.0, 0.0)
        rectangle
        (\x + 2.0, 1.0);
      \FPeval{\value}{clip(\value+1)}
%      \draw (\x + 1.5, 0.5)
%        node {\Huge \value};
    \fi
  \fi
}
\foreach \x in {0,...,\FSAdelete}{
  \ifnum \x<\FSAdelete
  \draw[element_delete_cross]
    (\x + \FSAelements + \FSAcopy, 1.0) --
    (\x + \FSAelements + \FSAcopy + 1.0, 0.0);
  \draw[element_delete_cross]
    (\x + \FSAelements + \FSAcopy, 0.0) --
    (\x + \FSAelements + \FSAcopy + 1.0, 1.0);
  \fi
}
\end{tikzpicture}%
\end{adjustbox}%
}% &
    $\mathcal{O}(1 + {\color{Mittel-Blau}2})$ &
    write 1, {\color{Mittel-Blau}copy 2 elements}\\
    \def\FSAsize{4}\def\FSAelements{3}%
    \def\FSAcopy{0}\def\FSAdelete{0}\def\FSAinsert{1}%
    \raisebox{-0.25em}{%
\begin{adjustbox}{height=1.5em}%
\begin{tikzpicture}[%
  element/.style={
    draw=black,
    fill={black!20!white}
  }, element_copy/.style={
    draw=black,
    fill={Hell-Blau}
  }, element_delete/.style={
    draw=black,
    fill=white
  }, element_delete_cross/.style={
    draw=red,
    line cap=round,
    line width=0.25em
  }, element_insert/.style={
      draw=black,
      fill={yellow!50!white}
  }, element_free/.style={
    draw=black,
    fill=white
  }, arrow_copy/.style={
    ->,
    line width=0.3em,
    color={Mittel-Gruen}
  }, arrow_insert/.style={
    ->,
    line width=0.3em,
    color={Hell-Gruen}
  }
]%
\foreach \x in {0,...,\FSAelements}{
  \ifnum \x<\FSAelements
    \draw[element]
      (\x, 0.0)
      rectangle
      (\x + 1.0, 1.0);
    \draw (\x + 0.5, 0.5) node {\Huge \x};
  \fi
}
\foreach \x in {0,...,\FSAcopy}{
  \ifnum \x<\FSAcopy
    \draw[element_copy]
      (\x + \FSAelements, 0.0)
      rectangle
      (\x + \FSAelements + 1.0, 1.0);
    \FPeval{\value}{clip(\x+\FSAelements)}
    \draw (\x + \FSAelements + 0.5, 0.5)
      node {\Huge \value};
    \draw[arrow_copy]
      (\x + \FSAelements + 0.5, 1.4) --
      (\x + \FSAelements + 0.5, 0.75);
  \fi
}
\foreach \x in {0,...,\FSAdelete}{
  \ifnum \x<\FSAdelete
  \draw[element_delete]
    (\x + \FSAelements + \FSAcopy, 0.0)
    rectangle
    (\x + \FSAelements + \FSAcopy + 1.0, 1.0);
  \FPeval{\value}{clip(\x+\FSAelements+\FSAcopy)}
  \draw (\x + \FSAelements + \FSAcopy + 0.5, 0.5)
    node {\Huge \value};
  \fi
}
\foreach \x in {0,...,\FSAinsert}{
  \ifnum \x<\FSAinsert
    \draw[element_insert]
      (\x + \FSAelements + \FSAcopy + \FSAdelete, 0.0)
      rectangle
      (\x + \FSAelements + \FSAcopy + \FSAdelete + 1.0, 1.0);
    \FPeval{\value}{clip(\x+\FSAelements+\FSAcopy+\FSAdelete)}
    \draw (\x + \FSAelements + \FSAcopy + \FSAdelete + 0.5, 0.5)
      node {\Huge \value};
    \draw[arrow_insert]
      (\x + \FSAelements + \FSAcopy + \FSAdelete + 0.5, 1.4) --
      (\x + \FSAelements + \FSAcopy + \FSAdelete + 0.5, 0.75);
  \fi
}
\foreach \x in {0,...,\FSAsize}{
  \FPeval{\value}{clip(\FSAelements+\FSAcopy+\FSAdelete+\FSAinsert-2)}
  \ifnum \x>\value
    \FPeval{\value}{clip(\FSAsize-1)}
    \ifnum \x<\value
      \draw[element_free]
        (\x + 1.0, 0.0)
        rectangle
        (\x + 2.0, 1.0);
      \FPeval{\value}{clip(\value+1)}
%      \draw (\x + 1.5, 0.5)
%        node {\Huge \value};
    \fi
  \fi
}
\foreach \x in {0,...,\FSAdelete}{
  \ifnum \x<\FSAdelete
  \draw[element_delete_cross]
    (\x + \FSAelements + \FSAcopy, 1.0) --
    (\x + \FSAelements + \FSAcopy + 1.0, 0.0);
  \draw[element_delete_cross]
    (\x + \FSAelements + \FSAcopy, 0.0) --
    (\x + \FSAelements + \FSAcopy + 1.0, 1.0);
  \fi
}
\end{tikzpicture}%
\end{adjustbox}%
}% &
    $\mathcal{O}(1)$ &
    write 1\\
    \def\FSAsize{8}\def\FSAelements{0}%
    \def\FSAcopy{4}\def\FSAdelete{0}\def\FSAinsert{1}%
    \raisebox{-0.25em}{%
\begin{adjustbox}{height=1.5em}%
\begin{tikzpicture}[%
  element/.style={
    draw=black,
    fill={black!20!white}
  }, element_copy/.style={
    draw=black,
    fill={Hell-Blau}
  }, element_delete/.style={
    draw=black,
    fill=white
  }, element_delete_cross/.style={
    draw=red,
    line cap=round,
    line width=0.25em
  }, element_insert/.style={
      draw=black,
      fill={yellow!50!white}
  }, element_free/.style={
    draw=black,
    fill=white
  }, arrow_copy/.style={
    ->,
    line width=0.3em,
    color={Mittel-Gruen}
  }, arrow_insert/.style={
    ->,
    line width=0.3em,
    color={Hell-Gruen}
  }
]%
\foreach \x in {0,...,\FSAelements}{
  \ifnum \x<\FSAelements
    \draw[element]
      (\x, 0.0)
      rectangle
      (\x + 1.0, 1.0);
    \draw (\x + 0.5, 0.5) node {\Huge \x};
  \fi
}
\foreach \x in {0,...,\FSAcopy}{
  \ifnum \x<\FSAcopy
    \draw[element_copy]
      (\x + \FSAelements, 0.0)
      rectangle
      (\x + \FSAelements + 1.0, 1.0);
    \FPeval{\value}{clip(\x+\FSAelements)}
    \draw (\x + \FSAelements + 0.5, 0.5)
      node {\Huge \value};
    \draw[arrow_copy]
      (\x + \FSAelements + 0.5, 1.4) --
      (\x + \FSAelements + 0.5, 0.75);
  \fi
}
\foreach \x in {0,...,\FSAdelete}{
  \ifnum \x<\FSAdelete
  \draw[element_delete]
    (\x + \FSAelements + \FSAcopy, 0.0)
    rectangle
    (\x + \FSAelements + \FSAcopy + 1.0, 1.0);
  \FPeval{\value}{clip(\x+\FSAelements+\FSAcopy)}
  \draw (\x + \FSAelements + \FSAcopy + 0.5, 0.5)
    node {\Huge \value};
  \fi
}
\foreach \x in {0,...,\FSAinsert}{
  \ifnum \x<\FSAinsert
    \draw[element_insert]
      (\x + \FSAelements + \FSAcopy + \FSAdelete, 0.0)
      rectangle
      (\x + \FSAelements + \FSAcopy + \FSAdelete + 1.0, 1.0);
    \FPeval{\value}{clip(\x+\FSAelements+\FSAcopy+\FSAdelete)}
    \draw (\x + \FSAelements + \FSAcopy + \FSAdelete + 0.5, 0.5)
      node {\Huge \value};
    \draw[arrow_insert]
      (\x + \FSAelements + \FSAcopy + \FSAdelete + 0.5, 1.4) --
      (\x + \FSAelements + \FSAcopy + \FSAdelete + 0.5, 0.75);
  \fi
}
\foreach \x in {0,...,\FSAsize}{
  \FPeval{\value}{clip(\FSAelements+\FSAcopy+\FSAdelete+\FSAinsert-2)}
  \ifnum \x>\value
    \FPeval{\value}{clip(\FSAsize-1)}
    \ifnum \x<\value
      \draw[element_free]
        (\x + 1.0, 0.0)
        rectangle
        (\x + 2.0, 1.0);
      \FPeval{\value}{clip(\value+1)}
%      \draw (\x + 1.5, 0.5)
%        node {\Huge \value};
    \fi
  \fi
}
\foreach \x in {0,...,\FSAdelete}{
  \ifnum \x<\FSAdelete
  \draw[element_delete_cross]
    (\x + \FSAelements + \FSAcopy, 1.0) --
    (\x + \FSAelements + \FSAcopy + 1.0, 0.0);
  \draw[element_delete_cross]
    (\x + \FSAelements + \FSAcopy, 0.0) --
    (\x + \FSAelements + \FSAcopy + 1.0, 1.0);
  \fi
}
\end{tikzpicture}%
\end{adjustbox}%
}% &
    $\mathcal{O}(1 + {\color{Mittel-Blau}4})$ &
    write 1, {\color{Mittel-Blau}copy 4 elements}\\
    \def\FSAsize{8}\def\FSAelements{5}%
    \def\FSAcopy{0}\def\FSAdelete{0}\def\FSAinsert{1}%
    \raisebox{-0.25em}{%
\begin{adjustbox}{height=1.5em}%
\begin{tikzpicture}[%
  element/.style={
    draw=black,
    fill={black!20!white}
  }, element_copy/.style={
    draw=black,
    fill={Hell-Blau}
  }, element_delete/.style={
    draw=black,
    fill=white
  }, element_delete_cross/.style={
    draw=red,
    line cap=round,
    line width=0.25em
  }, element_insert/.style={
      draw=black,
      fill={yellow!50!white}
  }, element_free/.style={
    draw=black,
    fill=white
  }, arrow_copy/.style={
    ->,
    line width=0.3em,
    color={Mittel-Gruen}
  }, arrow_insert/.style={
    ->,
    line width=0.3em,
    color={Hell-Gruen}
  }
]%
\foreach \x in {0,...,\FSAelements}{
  \ifnum \x<\FSAelements
    \draw[element]
      (\x, 0.0)
      rectangle
      (\x + 1.0, 1.0);
    \draw (\x + 0.5, 0.5) node {\Huge \x};
  \fi
}
\foreach \x in {0,...,\FSAcopy}{
  \ifnum \x<\FSAcopy
    \draw[element_copy]
      (\x + \FSAelements, 0.0)
      rectangle
      (\x + \FSAelements + 1.0, 1.0);
    \FPeval{\value}{clip(\x+\FSAelements)}
    \draw (\x + \FSAelements + 0.5, 0.5)
      node {\Huge \value};
    \draw[arrow_copy]
      (\x + \FSAelements + 0.5, 1.4) --
      (\x + \FSAelements + 0.5, 0.75);
  \fi
}
\foreach \x in {0,...,\FSAdelete}{
  \ifnum \x<\FSAdelete
  \draw[element_delete]
    (\x + \FSAelements + \FSAcopy, 0.0)
    rectangle
    (\x + \FSAelements + \FSAcopy + 1.0, 1.0);
  \FPeval{\value}{clip(\x+\FSAelements+\FSAcopy)}
  \draw (\x + \FSAelements + \FSAcopy + 0.5, 0.5)
    node {\Huge \value};
  \fi
}
\foreach \x in {0,...,\FSAinsert}{
  \ifnum \x<\FSAinsert
    \draw[element_insert]
      (\x + \FSAelements + \FSAcopy + \FSAdelete, 0.0)
      rectangle
      (\x + \FSAelements + \FSAcopy + \FSAdelete + 1.0, 1.0);
    \FPeval{\value}{clip(\x+\FSAelements+\FSAcopy+\FSAdelete)}
    \draw (\x + \FSAelements + \FSAcopy + \FSAdelete + 0.5, 0.5)
      node {\Huge \value};
    \draw[arrow_insert]
      (\x + \FSAelements + \FSAcopy + \FSAdelete + 0.5, 1.4) --
      (\x + \FSAelements + \FSAcopy + \FSAdelete + 0.5, 0.75);
  \fi
}
\foreach \x in {0,...,\FSAsize}{
  \FPeval{\value}{clip(\FSAelements+\FSAcopy+\FSAdelete+\FSAinsert-2)}
  \ifnum \x>\value
    \FPeval{\value}{clip(\FSAsize-1)}
    \ifnum \x<\value
      \draw[element_free]
        (\x + 1.0, 0.0)
        rectangle
        (\x + 2.0, 1.0);
      \FPeval{\value}{clip(\value+1)}
%      \draw (\x + 1.5, 0.5)
%        node {\Huge \value};
    \fi
  \fi
}
\foreach \x in {0,...,\FSAdelete}{
  \ifnum \x<\FSAdelete
  \draw[element_delete_cross]
    (\x + \FSAelements + \FSAcopy, 1.0) --
    (\x + \FSAelements + \FSAcopy + 1.0, 0.0);
  \draw[element_delete_cross]
    (\x + \FSAelements + \FSAcopy, 0.0) --
    (\x + \FSAelements + \FSAcopy + 1.0, 1.0);
  \fi
}
\end{tikzpicture}%
\end{adjustbox}%
}% &
    $\mathcal{O}(1)$ &
    write 1\\
    \def\FSAsize{8}\def\FSAelements{6}%
    \def\FSAcopy{0}\def\FSAdelete{0}\def\FSAinsert{1}%
    \raisebox{-0.25em}{%
\begin{adjustbox}{height=1.5em}%
\begin{tikzpicture}[%
  element/.style={
    draw=black,
    fill={black!20!white}
  }, element_copy/.style={
    draw=black,
    fill={Hell-Blau}
  }, element_delete/.style={
    draw=black,
    fill=white
  }, element_delete_cross/.style={
    draw=red,
    line cap=round,
    line width=0.25em
  }, element_insert/.style={
      draw=black,
      fill={yellow!50!white}
  }, element_free/.style={
    draw=black,
    fill=white
  }, arrow_copy/.style={
    ->,
    line width=0.3em,
    color={Mittel-Gruen}
  }, arrow_insert/.style={
    ->,
    line width=0.3em,
    color={Hell-Gruen}
  }
]%
\foreach \x in {0,...,\FSAelements}{
  \ifnum \x<\FSAelements
    \draw[element]
      (\x, 0.0)
      rectangle
      (\x + 1.0, 1.0);
    \draw (\x + 0.5, 0.5) node {\Huge \x};
  \fi
}
\foreach \x in {0,...,\FSAcopy}{
  \ifnum \x<\FSAcopy
    \draw[element_copy]
      (\x + \FSAelements, 0.0)
      rectangle
      (\x + \FSAelements + 1.0, 1.0);
    \FPeval{\value}{clip(\x+\FSAelements)}
    \draw (\x + \FSAelements + 0.5, 0.5)
      node {\Huge \value};
    \draw[arrow_copy]
      (\x + \FSAelements + 0.5, 1.4) --
      (\x + \FSAelements + 0.5, 0.75);
  \fi
}
\foreach \x in {0,...,\FSAdelete}{
  \ifnum \x<\FSAdelete
  \draw[element_delete]
    (\x + \FSAelements + \FSAcopy, 0.0)
    rectangle
    (\x + \FSAelements + \FSAcopy + 1.0, 1.0);
  \FPeval{\value}{clip(\x+\FSAelements+\FSAcopy)}
  \draw (\x + \FSAelements + \FSAcopy + 0.5, 0.5)
    node {\Huge \value};
  \fi
}
\foreach \x in {0,...,\FSAinsert}{
  \ifnum \x<\FSAinsert
    \draw[element_insert]
      (\x + \FSAelements + \FSAcopy + \FSAdelete, 0.0)
      rectangle
      (\x + \FSAelements + \FSAcopy + \FSAdelete + 1.0, 1.0);
    \FPeval{\value}{clip(\x+\FSAelements+\FSAcopy+\FSAdelete)}
    \draw (\x + \FSAelements + \FSAcopy + \FSAdelete + 0.5, 0.5)
      node {\Huge \value};
    \draw[arrow_insert]
      (\x + \FSAelements + \FSAcopy + \FSAdelete + 0.5, 1.4) --
      (\x + \FSAelements + \FSAcopy + \FSAdelete + 0.5, 0.75);
  \fi
}
\foreach \x in {0,...,\FSAsize}{
  \FPeval{\value}{clip(\FSAelements+\FSAcopy+\FSAdelete+\FSAinsert-2)}
  \ifnum \x>\value
    \FPeval{\value}{clip(\FSAsize-1)}
    \ifnum \x<\value
      \draw[element_free]
        (\x + 1.0, 0.0)
        rectangle
        (\x + 2.0, 1.0);
      \FPeval{\value}{clip(\value+1)}
%      \draw (\x + 1.5, 0.5)
%        node {\Huge \value};
    \fi
  \fi
}
\foreach \x in {0,...,\FSAdelete}{
  \ifnum \x<\FSAdelete
  \draw[element_delete_cross]
    (\x + \FSAelements + \FSAcopy, 1.0) --
    (\x + \FSAelements + \FSAcopy + 1.0, 0.0);
  \draw[element_delete_cross]
    (\x + \FSAelements + \FSAcopy, 0.0) --
    (\x + \FSAelements + \FSAcopy + 1.0, 1.0);
  \fi
}
\end{tikzpicture}%
\end{adjustbox}%
}% &
    $\mathcal{O}(1)$ &
    write 1\\
    \def\FSAsize{8}\def\FSAelements{7}%
    \def\FSAcopy{0}\def\FSAdelete{0}\def\FSAinsert{1}%
    \raisebox{-0.25em}{%
\begin{adjustbox}{height=1.5em}%
\begin{tikzpicture}[%
  element/.style={
    draw=black,
    fill={black!20!white}
  }, element_copy/.style={
    draw=black,
    fill={Hell-Blau}
  }, element_delete/.style={
    draw=black,
    fill=white
  }, element_delete_cross/.style={
    draw=red,
    line cap=round,
    line width=0.25em
  }, element_insert/.style={
      draw=black,
      fill={yellow!50!white}
  }, element_free/.style={
    draw=black,
    fill=white
  }, arrow_copy/.style={
    ->,
    line width=0.3em,
    color={Mittel-Gruen}
  }, arrow_insert/.style={
    ->,
    line width=0.3em,
    color={Hell-Gruen}
  }
]%
\foreach \x in {0,...,\FSAelements}{
  \ifnum \x<\FSAelements
    \draw[element]
      (\x, 0.0)
      rectangle
      (\x + 1.0, 1.0);
    \draw (\x + 0.5, 0.5) node {\Huge \x};
  \fi
}
\foreach \x in {0,...,\FSAcopy}{
  \ifnum \x<\FSAcopy
    \draw[element_copy]
      (\x + \FSAelements, 0.0)
      rectangle
      (\x + \FSAelements + 1.0, 1.0);
    \FPeval{\value}{clip(\x+\FSAelements)}
    \draw (\x + \FSAelements + 0.5, 0.5)
      node {\Huge \value};
    \draw[arrow_copy]
      (\x + \FSAelements + 0.5, 1.4) --
      (\x + \FSAelements + 0.5, 0.75);
  \fi
}
\foreach \x in {0,...,\FSAdelete}{
  \ifnum \x<\FSAdelete
  \draw[element_delete]
    (\x + \FSAelements + \FSAcopy, 0.0)
    rectangle
    (\x + \FSAelements + \FSAcopy + 1.0, 1.0);
  \FPeval{\value}{clip(\x+\FSAelements+\FSAcopy)}
  \draw (\x + \FSAelements + \FSAcopy + 0.5, 0.5)
    node {\Huge \value};
  \fi
}
\foreach \x in {0,...,\FSAinsert}{
  \ifnum \x<\FSAinsert
    \draw[element_insert]
      (\x + \FSAelements + \FSAcopy + \FSAdelete, 0.0)
      rectangle
      (\x + \FSAelements + \FSAcopy + \FSAdelete + 1.0, 1.0);
    \FPeval{\value}{clip(\x+\FSAelements+\FSAcopy+\FSAdelete)}
    \draw (\x + \FSAelements + \FSAcopy + \FSAdelete + 0.5, 0.5)
      node {\Huge \value};
    \draw[arrow_insert]
      (\x + \FSAelements + \FSAcopy + \FSAdelete + 0.5, 1.4) --
      (\x + \FSAelements + \FSAcopy + \FSAdelete + 0.5, 0.75);
  \fi
}
\foreach \x in {0,...,\FSAsize}{
  \FPeval{\value}{clip(\FSAelements+\FSAcopy+\FSAdelete+\FSAinsert-2)}
  \ifnum \x>\value
    \FPeval{\value}{clip(\FSAsize-1)}
    \ifnum \x<\value
      \draw[element_free]
        (\x + 1.0, 0.0)
        rectangle
        (\x + 2.0, 1.0);
      \FPeval{\value}{clip(\value+1)}
%      \draw (\x + 1.5, 0.5)
%        node {\Huge \value};
    \fi
  \fi
}
\foreach \x in {0,...,\FSAdelete}{
  \ifnum \x<\FSAdelete
  \draw[element_delete_cross]
    (\x + \FSAelements + \FSAcopy, 1.0) --
    (\x + \FSAelements + \FSAcopy + 1.0, 0.0);
  \draw[element_delete_cross]
    (\x + \FSAelements + \FSAcopy, 0.0) --
    (\x + \FSAelements + \FSAcopy + 1.0, 1.0);
  \fi
}
\end{tikzpicture}%
\end{adjustbox}%
}% &
    $\mathcal{O}(1)$ &
    write 1\\
    \def\FSAsize{10}\def\FSAelements{0}%
    \def\FSAcopy{8}\def\FSAdelete{0}\def\FSAinsert{1}%
    \raisebox{-0.25em}{%
\begin{adjustbox}{height=1.5em}%
\begin{tikzpicture}[%
  element/.style={
    draw=black,
    fill={black!20!white}
  }, element_copy/.style={
    draw=black,
    fill={Hell-Blau}
  }, element_delete/.style={
    draw=black,
    fill=white
  }, element_delete_cross/.style={
    draw=red,
    line cap=round,
    line width=0.25em
  }, element_insert/.style={
      draw=black,
      fill={yellow!50!white}
  }, element_free/.style={
    draw=black,
    fill=white
  }, arrow_copy/.style={
    ->,
    line width=0.3em,
    color={Mittel-Gruen}
  }, arrow_insert/.style={
    ->,
    line width=0.3em,
    color={Hell-Gruen}
  }
]%
\foreach \x in {0,...,\FSAelements}{
  \ifnum \x<\FSAelements
    \draw[element]
      (\x, 0.0)
      rectangle
      (\x + 1.0, 1.0);
    \draw (\x + 0.5, 0.5) node {\Huge \x};
  \fi
}
\foreach \x in {0,...,\FSAcopy}{
  \ifnum \x<\FSAcopy
    \draw[element_copy]
      (\x + \FSAelements, 0.0)
      rectangle
      (\x + \FSAelements + 1.0, 1.0);
    \FPeval{\value}{clip(\x+\FSAelements)}
    \draw (\x + \FSAelements + 0.5, 0.5)
      node {\Huge \value};
    \draw[arrow_copy]
      (\x + \FSAelements + 0.5, 1.4) --
      (\x + \FSAelements + 0.5, 0.75);
  \fi
}
\foreach \x in {0,...,\FSAdelete}{
  \ifnum \x<\FSAdelete
  \draw[element_delete]
    (\x + \FSAelements + \FSAcopy, 0.0)
    rectangle
    (\x + \FSAelements + \FSAcopy + 1.0, 1.0);
  \FPeval{\value}{clip(\x+\FSAelements+\FSAcopy)}
  \draw (\x + \FSAelements + \FSAcopy + 0.5, 0.5)
    node {\Huge \value};
  \fi
}
\foreach \x in {0,...,\FSAinsert}{
  \ifnum \x<\FSAinsert
    \draw[element_insert]
      (\x + \FSAelements + \FSAcopy + \FSAdelete, 0.0)
      rectangle
      (\x + \FSAelements + \FSAcopy + \FSAdelete + 1.0, 1.0);
    \FPeval{\value}{clip(\x+\FSAelements+\FSAcopy+\FSAdelete)}
    \draw (\x + \FSAelements + \FSAcopy + \FSAdelete + 0.5, 0.5)
      node {\Huge \value};
    \draw[arrow_insert]
      (\x + \FSAelements + \FSAcopy + \FSAdelete + 0.5, 1.4) --
      (\x + \FSAelements + \FSAcopy + \FSAdelete + 0.5, 0.75);
  \fi
}
\foreach \x in {0,...,\FSAsize}{
  \FPeval{\value}{clip(\FSAelements+\FSAcopy+\FSAdelete+\FSAinsert-2)}
  \ifnum \x>\value
    \FPeval{\value}{clip(\FSAsize-1)}
    \ifnum \x<\value
      \draw[element_free]
        (\x + 1.0, 0.0)
        rectangle
        (\x + 2.0, 1.0);
      \FPeval{\value}{clip(\value+1)}
%      \draw (\x + 1.5, 0.5)
%        node {\Huge \value};
    \fi
  \fi
}
\foreach \x in {0,...,\FSAdelete}{
  \ifnum \x<\FSAdelete
  \draw[element_delete_cross]
    (\x + \FSAelements + \FSAcopy, 1.0) --
    (\x + \FSAelements + \FSAcopy + 1.0, 0.0);
  \draw[element_delete_cross]
    (\x + \FSAelements + \FSAcopy, 0.0) --
    (\x + \FSAelements + \FSAcopy + 1.0, 1.0);
  \fi
}
\end{tikzpicture}%
\end{adjustbox}%
}%%
    \hspace*{0.25em}$\cdots$ &
    $\mathcal{O}(1 + {\color{Mittel-Blau}8})$ &
    write 1, {\color{Mittel-Blau}copy 8 elements}\\
    \hspace*{1.5em}\dots & \dots & \hspace*{1.5em}\dots
  \end{tabularx}
\end{frame}

%-------------------------------------------------------------------------------

\begin{frame}{Dynamic Fields}{Implementation 3}
  \textbf{Analysis:}
  \begin{itemize}
    \item
      The number of $\mathcal{O}(1)$ operations is increasing
    \item
      We need to copy the array every $C^i$ steps $(1, 2, 4, 8,\dots)$\\[0.5em]
      $\Rightarrow T_\text{copy} = C_1 \cdot C^{(i-1)}$\\
      $\Rightarrow T_i
        = \underbrace{
            C_2\vphantom{\begin{cases}C_1\\0\end{cases}}
          }_\text{insert}
          + \underbrace{\begin{cases}
            C_1 \cdot C^{(i-1)} & \text{if } log_C(i-1) \in \mathbb{N}\\
            0 & \text{else}
          \end{cases}}_\text{copy}$
  \end{itemize}
\end{frame}

%-------------------------------------------------------------------------------

\begin{frame}{Dynamic Fields}{Implementation 3}
  \textbf{Analysis:}
  Assuming $C = 2$
  \begin{align*}
    T_i
      &= \underbrace{
        C_2\vphantom{\begin{cases}C_1\\0\end{cases}}
      }_\text{insert}
      + \underbrace{\begin{cases}
        C_1 \cdot 2^{(i-1)} & \text{if } log_2(i-1) \in \mathbb{N}\\
        0 & \text{else}
        \end{cases}}_\text{copy}\\
    \Rightarrow k &= \mathrm{floor}(\log_2(n-1))
      = \left\lfloor\log_2(n-1)\right\rfloor\\
    T(n) &= C_2 \cdot \sum_{i=1}^{n} 1
        + C_1 \cdot \underbrace{\sum_{i=0}^{k} 2^i}_{2^{(k+1)}-1}
      = C_2 \cdot n + C_1 \cdot (2^{(k+1)}-1)
  \end{align*}
\end{frame}

%-------------------------------------------------------------------------------

\begin{frame}{Dynamic Fields}{Implementation 3}
  \textbf{Analysis:}
  Assuming $C = 2, \, k = \left\lfloor\log_2(n-1)\right\rfloor$
  \begin{align*}
    T(n) &= C_2 \cdot n + C_1 \cdot (2^{(k+1)}-1)\\
    {} &\leq C_2 \cdot n + C_1 \cdot 2^{(k+1)}
      = C_2 \cdot n + 2 \, C_1 \cdot 2^{k}\\
    {} &\leq C_2 \cdot n + 2 \, C_1 \cdot (n - 1)\\
    {} &\leq C_2 \cdot n + 2 \, C_1 \cdot n
      = \underbrace{(C_2 + 2 C_1)}_{C_3} \cdot n\\
    \Rightarrow T(n) &\leq C_3 \cdot n \in \mathcal{O}(n)
  \end{align*}
\end{frame}

%-------------------------------------------------------------------------------

\begin{frame}{Dynamic Fields}{Shrinking}
  \textbf{How do we shrink the array?}
  \begin{itemize}
    \item
      \textbf{Idea:}
      Shrink the array by a constant factor
      $C_\text{shrink} = \frac{1}{2}$ or $C_\text{shrink} = \frac{1}{8}$
    \item
      If we \textit{append} directly after \textit{shrinking} we have to
      resize the array
      \begin{itemize}
        \item
          We only shrink the array by $\frac{1}{2}$ of the possible amount
        \item
          We have enough space to insert or delete more elements
       \end{itemize}
  \end{itemize}
\end{frame}

%-------------------------------------------------------------------------------

\begin{frame}{Dynamic Fields}{Shrinking}
  \textbf{Analysis:}
  \begin{itemize}
    \item
      \textbf{Difficult:}
      We have a random number of \textit{append} / \textit{remove} operations
    \item
      We can not exactly predict when resizing is happening
    \item
      We do an amortized runtime analysis \cite{wikipedia_amortized_analysis}
  \end{itemize}
\end{frame}