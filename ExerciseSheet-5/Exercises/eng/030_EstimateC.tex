\exercise{2} \points{5}\\
Write a function \textit{estimate\_c\_for\_single\_set}.
Given a set of keys $S$, the method calculates the mean bucket size for 1000
random hash functions and from this calculates the best possible value of $c$
and returns $c$.
For the calculation of $c$ use the formula from the lecture
$E(\|S_i\|) \; \leq \; 1 + c \cdot \frac{\|S\|}{m}$. The function receives a list of keys and the hash function object.

\exercise{3} \points{5}\\

Write a function \emph{estimate\_c\_for\_multiple\_sets} which randomly generates a given number $n$ of key sets with 
a given size $k$ (no duplicates inside one set of keys!) and calculates for each of the $n$ key sets the best 
possible $c$ with the function \textit{estimate\_c\_for\_single\_set} from exercise 2. Based on these the function then should remember the mean, minimum 
and maximum $c$ value and finally return the three values. The function receives $n$, $k$ and the hash function object. Implement an additional function \textit{create\_random\_universe\_subset()} for the random key list generation. It receives $k$ and the universe size $u$ from the hash function and returns the subset list.

