\exercise{2} \points{7}\\
Expand your implementation such that different tests can be selected by 
using a command line argument to set which test to execute. 
You can e.g. use Python's \textit{argparse} module for that.
Make the program produce an output which represents the accumulated runtime
for a sequence of $n$ operations.
You can plot your results e.g. with \textit{gnuplot} (see description below).

Implement the following tests:
\begin{itemize}
  \item[Test 1:]
    A sequence of 10 million \textit{append} operations starting with an empty
    array.
  \item[Test 2:]
    A sequence of 10 million \textit{remove} operations starting with an array
    containing 10 million elements.
  \item[Test 3:]
    A sequence of 10 million operations starting with an array containing 1
    million elements.
    The operations should start with \textit{append} and then alternate after
    each reallocation between \textit{append} and \textit{remove}.
  \item[Test 4:]
    Same as Test 3, but this time starting with \textit{remove}.
\end{itemize}
Generate runtime plots for all tests and commit them inside a subdirectory
\textit{non-code} as part of your solution into the SVN directory.
You can e.g. use \textit{gnuplot} for plotting the runtime.
Make your program generate a text file (e.g. "runtime.txt") or just redirect
the standard output of your program into the text file on the command line:

\texttt{python3 dynamic\_int\_array.py -{}-test 1 > runtime.txt}

The text file should have the following format:\\
\textit{<input size 1> <TAB> <runtime 1>}\\
\textit{<input size 2> <TAB> <runtime 2>}\\
...\\
\textit{<input size n> <TAB> <runtime n>}

Based on this file you can generate a plot with the following command:\\
\texttt{gnuplot -e "plot 'runtime.txt'; pause -1;"}

A window will subsequently pop up, containing the desired plot.
You can then store the plot e.g. in JPEG format.
For details on the used parameters please check gnuplot’s man pages by
executing \enquote{\texttt{man gnuplot}} or have a look here:
\url{http://www.gnuplot.info/docs\_4.6/gnuplot.pdf}

