\section{Recursion Equation}



\begin{frame}{Recursion Equation}{Equation}
	\begin{itemize}
		\item
			describe the runtime for recursion:
			
		TODO: Formula -> look at slide 30 \vspace{2em}
		
		\item
			$n_0$ is normally small, $f_0(n_0) \; \in \; \Theta(1)$
		\item
			normally a > 1 and b > 1 
		\item
			dependent on the way of solving is $f_0$ going to be ignored
		\item
			T is only defined for whole numbers of $\dfrac{n}{b}$, which is also 
			often ignored in the solution 
	\end{itemize}
\end{frame}

%---------------------------------------------------------------------------------------

\begin{frame}{Recursion Equation}{Substitution Method}
	\begin{itemize}
		\item
			Guess solution and prove it with induction
		\item
			Example:
			
			TODO: Formula -> look at slide 31 \vspace{2em}
			
		\item
			Assumption:  $T(n) = n + n \cdot \log_2 n$
	\end{itemize}
\end{frame}

%---------------------------------------------------------------------------------------

\begin{frame}{Recursion Equation}{Substitution Method}
	\begin{itemize}
		\item
			Induction basis (for n = 1): $T(1) = 1 + 1 \cdot \log_2 1 = 1$
		\item
			Induction step (from $\dfrac{n}{2}$ to n):
			\begin{align*}
				T(n) & = 2 \cdot T\left(\dfrac{n}{2}\right) + n \\
					{} & = 2 \cdot \left(\dfrac{n}{2} + \dfrac{n}{2} \cdot \log_2 
						 \dfrac{n}{2}\right) + n \\
					{} & = 2 \cdot \left(\dfrac{n}{2} + \dfrac{n}{2} \cdot \left(\log_2 
					n - 1\right)\right) + 
						 n\\
					{} & = n + n \cdot \log_2 n - n + n\\
					{} & = n + n \cdot \log_2 n
			\end{align*}
	\end{itemize}
\end{frame}

%---------------------------------------------------------------------------------------

\begin{frame}{Recursion Equation}{Substitution Method}
	\begin{itemize}
		\item
			Alternative assumption
		\item
		Example:
		
		TODO: Formula -> look at slide 33 \vspace{2em}
		
		\item
		Assumption: $T(n) \; \in \; \mathcal{O}(n \log n)$
	\end{itemize}
\end{frame}

%---------------------------------------------------------------------------------------

\begin{frame}{Recursion Equation}{Substitution Method}
	\begin{itemize}
		\item
			Solution: (find c > 0 with) $T(n) = c \cdot n \cdot \log_2 n$
		\item
		Induction step (from $\dfrac{n}{2}$ to n):
		\begin{align*}
		T(n) & = 2 \cdot T\left(\dfrac{n}{2}\right) + n \\
		{} & \leq 2 \cdot \left(c \cdot \dfrac{n}{2} \cdot \log_2 
		\dfrac{n}{2}\right) + n \\
		{} & = c \cdot n \cdot \log_2 n - c \cdot n \cdot \log_2 2 + n\\
		{} & = c \cdot n \cdot \log_2 n - c \cdot n + n\\
		{} & \leq c \cdot n \cdot \log_2 n
		\end{align*}
	\end{itemize}
\end{frame}

%-------------------------------------------------------------------------------

\begin{frame}{Recursion Equation}{Recursion Tree Metode}
	\begin{itemize}
		\item
			Can be used to make assumptions
		\item
			Example:\\
			$T(n) = 3 \cdot T\left(\dfrac{n}{4}\right) + \Theta(n^2) \leq 3 \cdot 3 
			\cdot T\left(\dfrac{n}{4}\right) + c \cdot n^2$
	\end{itemize}
	
	TODO: Graphics -> look at slide 34 \vspace{2em}
	
	\end{frame}

%-------------------------------------------------------------------------------

\begin{frame}{Recursion Equation}{Recursion Tree Metode}
	
	TODO: Graphics -> look at slide 35 \vspace{2em}
	
\end{frame}

%-------------------------------------------------------------------------------

\begin{frame}{Recursion Equation}{Recursion Tree Metode Costs}
	\textbf{Costs (connecting the partial solution) on levels excepted the last 
	one}
	\begin{itemize}
		\item
			the size of a partial problems on level i: $\left(\dfrac{1}{4}\right)i 
			\cdot n$
		\item
			Costs of a partial problems on level i: $c \cdot 
			\left(\left(\dfrac{1}{4}\right)i \cdot n\right)^2$
		\item
			Number of partial problems on level i: $3^{i}$
		\item
			Costs on level i: $3^{i} \cdot c \cdot 
			\left(\left(\dfrac{1}{4}\right)^{i} \cdot n 
			\right)^2 = \left(\dfrac{3}{16}\right)^{i} \cdot c \cdot n^2$
	\end{itemize}
\end{frame}

%-------------------------------------------------------------------------------

\begin{frame}{Recursion Equation}{Recursion Tree Metode Costs}
	\textbf{Costs (connecting the partial solution) on the last level}
	\begin{itemize}
		\item
			The size of a partial problems on the last level: 1
		\item
			Costs of a partial problems on the last level: 
			$\left(\dfrac{1}{4}\right)i \cdot n = 1 \rightarrow n = 4i \rightarrow i 
			= \log_4 n$
		\item
			Number of partial problems on the last level: $3^{\log_4 n} = n^{\log_4 
			3}$
		\item
			Costs on the last level: $d \cdot n^{\log_4 3}$
	\end{itemize}
\end{frame}

%-------------------------------------------------------------------------------

\begin{frame}{Recursion Equation}{Logarithm}
	Number of subproblems in the last level (leafs in the tree):
	\begin{align*}
		3^{\log_4 n} & = 3^{\log_4 (\log_3 n)} && \text{use } \log a^b = b \cdot 
		\log a\\
		{} & = 3^{(\log_3 n) \cdot \log_4 3}  && \text{use }x^{a \cdot b} = 
		(x^a)^b\\
		{} & = 3^{(\log_3 n)^{\log_4 3}}\\
		{} & = n^{\log_4 3}
	\end{align*}
\end{frame}

%-------------------------------------------------------------------------------

\begin{frame}{Recursion Equation}{Overall costs}
	\begin{itemize}
		\item
			Costs for the level i: $\left(\dfrac{3}{16}\right)^i \cdot c \cdot n^2$
		\item
			Costs for the last level: $d \cdot n^{\log_4 3}$
		\item
			Overall:\\
			$\underbrace{\sum\limits_{i = 0}^{(\log_4 n) - 1} 
			\left(\dfrac{3}{16}\right)^i}_{\text{TODO}} \cdot 
			c \cdot n^2 
			+ \underbrace{d \cdot n^{\log_4 3}}_{\text{TODO}} \;\; \in 
			\mathcal{O}(n^2)$
		\item
			Here: costs for connecting the subproblems dominate
	\end{itemize}
\end{frame}

%-------------------------------------------------------------------------------

\begin{frame}{Recursion Equation}{Geometric Series}
	\begin{itemize}
		\item
			\textbf{Geometric progression:}\\
			Quotient of two neighbored progression parts is constant
		\item
			\textbf{Geometric series:}\\
			The series (cumulative sum) of a geometric progression.\\
			For $\mid q \mid < 1: \;\;\; \displaystyle \sum\limits^{\infty}_{k=0} a_0 
			\cdot q^k = \dfrac{a_0}{1 - q}\hspace{4em} \rightarrow$ constant
	\end{itemize}
\end{frame}

%-------------------------------------------------------------------------------

\begin{frame}{Recursion Equation}{proof of O($n^2$)}
	\begin{itemize}
		\item
			Given: $T(n) = 3T\left(\dfrac{n}{4}\right) + \Theta (n^2) \; \leq \; 
			3T \left( \dfrac{n}{4}\right) + c \cdot n^2$
		\item
			Guess: $T(n) \; \in \; O(n^2)$, so it exists a k > 0 with T(n) < $k \cdot 
			n^2$
		\item
			Substitution method:
			\begin{align*}
				T(n) & \leq 3 \cdot T \left( \dfrac{n}{4}\right)  + c \cdot n^2\\
				{} & \leq 3 \cdot k \left( \dfrac{n}{4}\right)^2  + c \cdot n^2\\
				{} & = \dfrac{3}{16} \cdot k \cdot n^2  + c \cdot n^2\\
				{} & \leq k \cdot n^2 \hspace{6em}(\text{for } k \geq \dfrac{16}{13} 
				\cdot c)
			\end{align*}
	\end{itemize}
\end{frame}

%-------------------------------------------------------------------------------

\begin{frame}{Recursion Equation}{Mastertheorem}
	\begin{itemize}
		\item
			The approach to the solution for the recursion equation of the form:\\
			$T(n) = a \cdot T\left(\dfrac{n}{b}\right) + f(n)$ with constants $a \geq 
			1$ and $b > 1$ \\\vspace{3em}
		\item
			T(n) is the runtime of an algorithm
			\begin{itemize}
				\item
					which has a problem of size n which is divided in a subproblems.
				\item
					Each of this a subproblems is solved recursively with the runtime 
					$T\left(\dfrac{n}{b}\right)$.
				\item
					It takes f(n) steps to reconnect all a subproblems.
			\end{itemize} 
	\end{itemize}
\end{frame}

%-------------------------------------------------------------------------------

\begin{frame}{Recursion Equation}{Phyton implementation (to check)}
	
	TODO: Code -> look at slide 18 \vspace{2em}
	
\end{frame}

%-------------------------------------------------------------------------------

\begin{frame}{Recursion Equation}{Implementation lmax}
	
	TODO: Code -> look at slide 19 \vspace{2em}
	
\end{frame}

%-------------------------------------------------------------------------------

\begin{frame}{Recursion Equation}{Phyton implementation (to check)}
	
	TODO: Code -> look at slide 20 \vspace{2em}
	
\end{frame}

%-------------------------------------------------------------------------------

\begin{frame}{Recursion Equation}{Illustration lmax}
		
		TODO: Table -> look at slide 21 \vspace{2em}
		
	\begin{itemize}
		\item
			lmax and sum will be initialized with X(u)
		\item
			we go through X from u to o
		\item
			Sum refreshes
		\item
			lmax refreshes , if sum > lmax
	\end{itemize}
\end{frame}

%-------------------------------------------------------------------------------

\begin{frame}{Recursion Equation}{Illustraiting maxSubArray}
	
	TODO: Graphics -> look at slide 21 \vspace{2em}
	
\end{frame}

%-------------------------------------------------------------------------------

\begin{frame}{Recursion Equation}{Runtime}
	
	\textbf{maxSubArray(X, u, o)} \vspace{2em}
	
	if (u = o) then return X(u);\hspace{4em} $\mathcal{O}(1)$\\\vspace{0.5em}
	m = $\dfrac{u+o}{2}$;\hspace{10.7em} $\mathcal{O}(1)$\\\vspace{0.5em}
	A = maxSubArray (X, u, m);\hspace{3.1em} T($\dfrac{n}{2})$\\\vspace{0.5em}
	B = maxSubArray (X, m+1, o);\hspace{1.9em} T($\dfrac{n}{2})$\\\vspace{0.5em}
	$C_1$ = rmax (X, u, m);\hspace{6.2em} $\mathcal{O}(n)$\\\vspace{0.5em}
	$C_2$ = lmax (X, m+1, o);\hspace{5.2em} $\mathcal{O}(n)$\\\vspace{0.5em}
	return max (A, B, $C_1$ + $C_2$);\hspace{3.3em} $\mathcal{O}(1)$
\end{frame}

%-------------------------------------------------------------------------------

\begin{frame}{Recursion Equation}{Number of steps T(n)}
	
	TODO: Grapic -> look at slide 24 \vspace{2em}
	
	so it existing constants a and b with\\\vspace{2em}
	
	TODO: Grapic -> look at slide 24 \vspace{2em}
	
	We define c:= max(a,b), then\\\vspace{2em}
	
	TODO: Grapic -> look at slide 24
	
\end{frame}

%-------------------------------------------------------------------------------

\begin{frame}{Recursion Equation}{Illustration of T(n)}

TODO: Graphics -> look at slide 25 \vspace{2em}

\end{frame}

%-------------------------------------------------------------------------------

\begin{frame}{Recursion Equation}{Illustration of T(n)}

TODO: Graphics -> look at slide 26 \vspace{2em}

\end{frame}

%-------------------------------------------------------------------------------

\begin{frame}{Recursion Equation}{Illustration of T(n)}
	\begin{itemize}
		\item
			Upper level i = 0, lower level at 2i = n $\rightarrow$ i = $\log_2 n$, so 
			overall $\log_2 n + 1$ levels
		\item
			In each level has cost of $c \cdot n$ (the costs for merging of 
			subproblems and the solving of trivial problems are here the same)
	\end{itemize}
\end{frame}

%-------------------------------------------------------------------------------

\begin{frame}{Recursion Equation}{Summaray Maximum subtotal}
	\begin{itemize}
		\item
			Direct solution is $\mathcal{O}(n^3)$
		\item
			A better solution with incremental refreshment of the subtotals was 
			$\mathcal{O}(n^2)$
		\item
			Divide and concur approach results in $\mathcal{O}(n \cdot \log n)$
		\item
			There is an approach in $\mathcal{O}(n)$, if you assume that all 
			subtotals are positive
	\end{itemize}
\end{frame}