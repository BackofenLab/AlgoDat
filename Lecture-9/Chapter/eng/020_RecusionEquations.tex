\section{Recursion Equations}

\begin{frame}{Recursion Equations}{Recursion Equation}
  \textbf{Recursion equation:}
  \begin{itemize}
    \item
      Describes the runtime for recursive functions:
  \end{itemize}
  \vspace{2em}
  \begin{displaymath}
    {\color{Mittel-Blau}T(n)} =
    \begin{cases}
      \hfill
      \smash{
        \overbrace{f_0(n)}^\text{%
          \footnotesize%
          trivial case for $n_0$
        }
      }
      \vphantom{f_0(n)} & n = n_0\\[0.5em]
      \smash{
        \underbrace{
          a \cdot {\color{Mittel-Blau}T\left(\frac{n}{b}\right)}
        }_\text{
          \footnotesize%
          \begin{tabular}{c}
            \text{solving of {\color{Mittel-Blau}$a$}}\\
            \text{subproblems}\\
            \text{with reduced}\\
            \text{input size {\color{Mittel-Blau}$\frac{n}{b}$}}
          \end{tabular}
        }
      } \hspace{0.5em} + \hspace{1.5em}
      \smash{
        \underbrace{
          f(n)
          \vphantom{\left(\frac{n}{b}\right)}
        }_\text{
          \makebox[0pt][c]{
            \footnotesize%
            \begin{tabular}{c}
              slicing and\\
              splicing of\\
              subsolutions
            \end{tabular}
          }
        }
      }
      \vphantom{\left(\frac{n}{b}\right)}
      &n > n_0
    \end{cases}
  \end{displaymath}
  \vspace{4em}
\end{frame}

%-------------------------------------------------------------------------------

\begin{frame}{Recursion Equations}{Recursion Equation}
  \textbf{Recursion equation:}
  \begin{itemize}
    \item
    Describes the runtime for recursive functions:
    \begin{displaymath}
    {\color{Mittel-Blau}T(n)} = \begin{cases}
      f_0(n) & n = n_0\\
      a \cdot {\color{Mittel-Blau}T\left(\frac{n}{b}\right)} + f(n) & n > n_0
    \end{cases}
    \end{displaymath}
    \item
      $n_0$ is normally small, $f_0(n_0) \; \in \; \Theta(1)$
    \item
      Normally $a > 1$ and $b > 1$
    \item
      Dependent on the strategy of solving {\color{Mittel-Blau}$T(n)$}
      $f_0$ is ignored
    \item
      {\color{Mittel-Blau}$T(n)$} is only defined for whole numbers of
      {\color{Mittel-Blau}$\frac{n}{b}$} which is often ignored in benefit of
      a simpler solution
  \end{itemize}
\end{frame}

%-------------------------------------------------------------------------------

\subsection{Substitution Method}

\begin{frame}{Recursion Equations}{Substitution Method}
  \textbf{Substitution Method:}
  \begin{itemize}
    \item
      Guess the solution and prove it with induction
    \item
      Example:
      \begin{displaymath}
        \color{Mittel-Blau}
        T(n) = \begin{cases}
          \hfill 1 & n = 1\\
          2 \cdot T\left(\frac{n}{2}\right) + n & n > 1
        \end{cases}
      \end{displaymath}
    \item
      Assumption:  {\color{Mittel-Blau}$T(n) = n + n \cdot \log_2 n$}
  \end{itemize}
\end{frame}

%---------------------------------------------------------------------------------------

\begin{frame}{Recursion Equations}{Substitution Method}
  \textbf{Induction:}
  \begin{itemize}
    \item
      Induction basis (for {\color{Mittel-Blau}$n = 1$}):
      ${\color{Mittel-Blau}T(1)} = 1 + 1 \cdot \log_2 1 = 1$
    \item
      Induction step (from {\color{Mittel-Blau}$\frac{n}{2}$}
      to {\color{Mittel-Blau}$n$}):
      \begin{align*}
        {\color{Mittel-Blau}T(n)} & =
          2 \cdot {\color{Mittel-Blau}T\left(\dfrac{n}{2}\right)} + n\\
          {} & \stackrel{IA}{=} 2 \cdot \left(
            \dfrac{n}{2} + \dfrac{n}{2} \cdot \log_2 \dfrac{n}{2}
          \right) + n \\
          {} & = 2 \cdot \left(
            \dfrac{n}{2} + \dfrac{n}{2} \cdot \left(\log_2 n - 1\right)
          \right) + n\\
          {} & = n + n \, \log_2 n - n + n\\
          {} & = n + n \, \log_2 n
      \end{align*}
  \end{itemize}
\end{frame}

%---------------------------------------------------------------------------------------

\begin{frame}{Recursion Equations}{Substitution Method}
  \textbf{Substitution Method:}
  \begin{itemize}
    \item
      Example:
      \begin{displaymath}
        \color{Mittel-Blau}
        T(n) = \begin{cases}
          \hfill 1 & n = 1\\
          2 \cdot T\left(\frac{n}{2}\right) + n & n > 0
        \end{cases}
      \end{displaymath}
    \item
      Alternative assumption:
      {\color{Mittel-Blau}$T(n) \, \in \, \mathcal{O}(n \, \log n)$}
    \item
      Solution: Find {\color{Mittel-Blau}$c > 0$} with
      {\color{Mittel-Blau}$T(n) \leq c \cdot n \, \log_2 n$}
  \end{itemize}
\end{frame}

%---------------------------------------------------------------------------------------

\begin{frame}{Recursion Equations}{Substitution Method}
  \textbf{Induction:}
  \begin{itemize}
    \item
      Solution: Find {\color{Mittel-Blau}$c > 0$} with
      {\color{Mittel-Blau}$T(n) \leq c \cdot n \, \log_2 n$}
    \item
      Induction step (from {\color{Mittel-Blau}$\frac{n}{2}$} to
      {\color{Mittel-Blau}$n$}):
      \begin{align*}
        {\color{Mittel-Blau}T(n)} & =
          2 \cdot {\color{Mittel-Blau}T\left(\dfrac{n}{2}\right)} + n\\
        {} & \leq 2 \cdot \left(
          c \cdot \dfrac{n}{2} \, \log_2 \dfrac{n}{2}\right) + n\\
        {} & = c \cdot n \, \log_2 n - c \cdot n \, \log_2 2 + n\\
        {} & = c \cdot n \, \log_2 n - c \cdot n + n\\
        {} & \leq c \cdot n \, \log_2 n,
          \hspace{0.5em} {\color{Mittel-Blau}c \geq 1}
      \end{align*}
  \end{itemize}
\end{frame}

%-------------------------------------------------------------------------------

\subsection{Recursion Tree Method}

\begin{frame}{Recursion Equations}{Recursion Tree Method}
  \textbf{Recursion tree method:}
  \begin{itemize}
    \item
      Can be used to make assumptions about the runtime
    \item
      Example:\\
      \begin{displaymath}
        {\color{Mittel-Blau}T(n)}
        = 3 \cdot {\color{Mittel-Blau}T\left(\dfrac{n}{4}\right)} + \Theta(n^2)
        \leq 3 \cdot {\color{Mittel-Blau}T\left(\dfrac{n}{4}\right)} + c \cdot 
        n^2
      \end{displaymath}
  \end{itemize}
\end{frame}

%-------------------------------------------------------------------------------

\begin{frame}{Recursion Equations}{Recursion Tree Method}
  \begin{figure}
    \begin{adjustbox}{width=\linewidth}
      \begin{tikzpicture}[
  op/.style={
    draw=black,
    color=black,
    fill=white,
    font=\Large,
    thick,
    ellipse,
    minimum size=3em
  }, sum/.style={
    font=\Large,
    text=Mittel-Blau
  }, connection/.style={
    ultra thick,
    color=Mittel-Gruen
  }, arrow/.style={
    ultra thick,
    color=Mittel-Blau
  }, arrow_alt/.style={
    ultra thick,
    color=Hell-Blau
  }, label/.style={
    font=\Large,
    text=Mittel-Blau
  }, label_alt/.style={
    font=\Large,
    text=Hell-Blau
  }
]%
% Bounding boxes
\only<1>{
  \draw[white] (-10, -4) rectangle (10, 4);
  \node[sum, anchor=north west] at (-9.5, 3.5) {%
    $T\left(n\right) = 3 \cdot T\left(\frac{n}{4}\right) + c \cdot n^2$
  };
}
\only<2>{
  \draw[white] (-10, -5) rectangle (10, 3);
  \node[sum, anchor=north west] at (-9.5, 2.5) {%
    $T\left(n\right) = 3 \cdot T\left(\frac{n}{4}\right) + c \cdot n^2$
  };
}
\only<3>{
  \draw[white] (-10, -6) rectangle (10, 2);
  \node[sum, anchor=north west] at (-9.5, 1.5) {%
    $T\left(n\right) = 3 \cdot T\left(\frac{n}{4}\right) + c \cdot n^2$
  };
}

% Main Problem with first-level subproblems
\node[op] (root) at (0, 0) {%
  \only<1>{$T(n)$}%
  \only<2->{$c \cdot n^2$}%
};

\only<2->{
  \node[op] (left) at (-6.75, -2) {%
    \only<2>{$T\left(\frac{n}{4}\right)$}%
    \only<3->{$c \cdot \left(\frac{n}{4}\right)^2$}%
  };
  \node[op] (center) at (0, -2) {%
    \only<2>{$T\left(\frac{n}{4}\right)$}%
    \only<3->{$c \cdot \left(\frac{n}{4}\right)^2$}%
  };
  \node[op] (right) at (6.75, -2) {%
    \only<2>{$T\left(\frac{n}{4}\right)$}%
    \only<3->{$c \cdot \left(\frac{n}{4}\right)^2$}%
  };

  \draw[-, connection] (root) -- (left);
  \draw[-, connection] (root) -- (center);
  \draw[-, connection] (root) -- (right);
}
\only<2>{
  \node[sum] at (0, -3.5) {%
    $T(n) = 3 \cdot T\left(\frac{n}{4}\right) + c \cdot n^2$%
  };
}

\only<3->{
  \node[op] (left_left) at (-9, -4) {$T\left(\frac{n}{4}\right)$};
  \node[op] (left_center) at (-6.75, -4) {$T\left(\frac{n}{4}\right)$};
  \node[op] (left_right) at (-4.5, -4) {$T\left(\frac{n}{4}\right)$};
  \node[op] (center_left) at (-2.25, -4) {$T\left(\frac{n}{4}\right)$};
  \node[op] (center_center) at (0, -4) {$T\left(\frac{n}{4}\right)$};
  \node[op] (center_right) at (2.25, -4) {$T\left(\frac{n}{4}\right)$};
  \node[op] (right_left) at (4.5, -4) {$T\left(\frac{n}{4}\right)$};
  \node[op] (right_center) at (6.75, -4) {$T\left(\frac{n}{4}\right)$};
  \node[op] (right_right) at (9, -4) {$T\left(\frac{n}{4}\right)$};

  \draw[-, connection] (left) -- (left_left);
  \draw[-, connection] (left) -- (left_center);
  \draw[-, connection] (left) -- (left_right);
  \draw[-, connection] (center) -- (center_left);
  \draw[-, connection] (center) -- (center_center);
  \draw[-, connection] (center) -- (center_right);
  \draw[-, connection] (right) -- (right_left);
  \draw[-, connection] (right) -- (right_center);
  \draw[-, connection] (right) -- (right_right);

  \only<3>{
    \node[sum] at (0, -5.5) {%
      \begin{math}%
      T\left(n\right) = 12 \cdot T\left(\frac{n}{16}\right)
        + 3 c \cdot \left(\frac{n}{4}\right)^2 + c \cdot n^2
      \end{math}%
    };
  }
}
\end{tikzpicture}
    \end{adjustbox}
    \caption{Recursion tree of example}
    \label{fig:recursion_equations:example_recursion_tree}
  \end{figure}
\end{frame}

%-------------------------------------------------------------------------------

\begin{frame}{Recursion Equations}{Recursion Tree Method}
  \begin{figure}
    \begin{adjustbox}{width=\linewidth}
      \begin{tikzpicture}[
  op/.style={
    draw=black,
    color=black,
    fill=white,
    font=\Large,
    thick,
    ellipse,
    minimum size=3em
  }, dots/.style={
    font=\Large,
    minimum height=3em
  }, sum/.style={
    font=\Large,
    text=Mittel-Blau
  }, connection/.style={
    ultra thick,
    color=Mittel-Gruen,
    draw=Mittel-Gruen
  }, arrow/.style={
    ultra thick,
    color=Mittel-Blau
  }, arrow_alt/.style={
    ultra thick,
    color=Hell-Blau
  }, label/.style={
    font=\Large,
    text=Mittel-Blau
  }, label_alt/.style={
    font=\Large,
    text=Hell-Blau
  }
]%
% Bounding box
\draw[white] (-2.25, -9) rectangle (19, 1);

% Main Problem with first-level subproblems
\node[op] (root) at (0, 0) {$c \cdot n^2$};

\node[dots] (left) at (-2, -2) {...};
\node[op] (center) at (0, -2) {$c \cdot \left(\frac{n}{4}\right)^2$};
\node[op] (right) at (11.25, -2) {$c \cdot \left(\frac{n}{4}\right)^2$};

\draw[-, connection] (root) -- (left);
\draw[-, connection] (root) -- (center);
\draw[-, connection] (root) -- (right);

\node[dots] (center_left) at (-2, -4) {...};
\node[op] (center_center) at (0, -4) {$c \cdot \left(\frac{n}{16}\right)^2$};
\node[op] (center_right) at (3.75, -4) {$c \cdot \left(\frac{n}{16}\right)^2$};
\node[op] (right_left) at (7.5, -4) {$c \cdot \left(\frac{n}{16}\right)^2$};
\node[op] (right_center) at (11.25, -4) {$c \cdot \left(\frac{n}{16}\right)^2$};
\node[op] (right_right) at (15, -4) {$c \cdot \left(\frac{n}{16}\right)^2$};

\draw[-, connection] (center) -- (center_left);
\draw[-, connection] (center) -- (center_center);
\draw[-, connection] (center) -- (center_right);
\draw[-, connection] (right) -- (right_left);
\draw[-, connection] (right) -- (right_center);
\draw[-, connection] (right) -- (right_right);

\draw[connection] (-1.25, -6) node[op] (center_center_l) {$\cdots$} to
  ++(0, -2) node[op] {$d$};
\draw[connection] (0, -6) node[op] (center_center_c) {$\cdots$} to
  ++(0, -2) node[op] {$d$};
\draw[connection] (1.25, -6) node[op] (center_center_r) {$\cdots$} to
  ++(0, -2) node[op] {$d$};
\draw[connection] (2.5, -6) node[op] (center_right_l) {$\cdots$} to
  ++(0, -2) node[op] {$d$};
\draw[connection] (3.75, -6) node[op] (center_right_c) {$\cdots$} to
  ++(0, -2) node[op] {$d$};
\draw[connection] (5, -6) node[op] (center_right_r) {$\cdots$} to
  ++(0, -2) node[op] {$d$};
\draw[connection] (6.25, -6) node[op] (right_left_l) {$\cdots$} to
  ++(0, -2) node[op] {$d$};
\draw[connection] (7.5, -6) node[op] (right_left_c) {$\cdots$} to
  ++(0, -2) node[op] {$d$};
\draw[connection] (8.75, -6) node[op] (right_left_r) {$\cdots$} to
  ++(0, -2) node[op] {$d$};
\draw[connection] (10, -6) node[op] (right_center_l) {$\cdots$} to
  ++(0, -2) node[op] {$d$};
\draw[connection] (11.25, -6) node[op] (right_center_c) {$\cdots$} to
  ++(0, -2) node[op] {$d$};
\draw[connection] (12.5, -6) node[op] (right_center_r) {$\cdots$} to
  ++(0, -2) node[op] {$d$};
\draw[connection] (13.75, -6) node[op] (right_right_l) {$\cdots$} to
  ++(0, -2) node[op] {$d$};
\draw[connection] (15, -6) node[op] (right_right_c) {$\cdots$} to
  ++(0, -2) node[op] {$d$};
\draw[connection] (16.25, -6) node[op] (right_right_r) {$\cdots$} to
  ++(0, -2) node[op] {$d$};

\draw[-, connection] (center_center) -- (center_center_l);
\draw[-, connection] (center_center) -- (center_center_c);
\draw[-, connection] (center_center) -- (center_center_r);
\draw[-, connection] (center_right) -- (center_right_l);
\draw[-, connection] (center_right) -- (center_right_c);
\draw[-, connection] (center_right) -- (center_right_r);
\draw[-, connection] (right_left) -- (right_left_l);
\draw[-, connection] (right_left) -- (right_left_c);
\draw[-, connection] (right_left) -- (right_left_r);
\draw[-, connection] (right_center) -- (right_center_l);
\draw[-, connection] (right_center) -- (right_center_c);
\draw[-, connection] (right_center) -- (right_center_r);
\draw[-, connection] (right_right) -- (right_right_l);
\draw[-, connection] (right_right) -- (right_right_c);
\draw[-, connection] (right_right) -- (right_right_r);

\node[sum, anchor=east] at (19, 0) {level 0};
\node[sum, anchor=east] at (19, -2) {level 1};
\node[sum, anchor=east] at (19, -4) {level 2};
\node[sum, anchor=east] at (19, -6) {level $i$};
\node[sum, anchor=east, align=center] at (19, -8) {last\\level};
\end{tikzpicture}
    \end{adjustbox}
    \caption{Levels of the recursion tree}
    \label{fig:recursion_equations:example_recursion_tree_layer}
  \end{figure}
\end{frame}
%-------------------------------------------------------------------------------

\begin{frame}{Recursion Equations}{Recursion Tree Method Costs}
  \textbf{Costs of connecting the partial solutions:}\\
  \hspace{1.5em}(excludes the last layer)
  \begin{itemize}
    \item
      Size of partial problems on {\color{Mittel-Blau}level $i$}:
      \begin{math}
        \mathcolor{Mittel-Blau}{
          \mathrm{s}_i(n) = \left(\frac{1}{4}\right)^i \cdot n
        }
      \end{math}
    \item
      Costs of partial problem on {\color{Mittel-Blau}level $i$}:
      \begin{displaymath}
        \mathcolor{Mittel-Blau}{
          T_{i_p}(n) =
          c \cdot \left(\left(\tfrac{1}{4}\right)^i \cdot n\right)^2
        }
      \end{displaymath}
    \item
      Number of partial problems on {\color{Mittel-Blau}level $i$}:
      {\color{Mittel-Blau}$n_{i} = 3^{i}$}
    \item
      Costs on {\color{Mittel-Blau}level $i$}:
      \begin{displaymath}
        \color{Mittel-Blau}
        T_i(n) = 3^{i} \cdot c \cdot \left(
          \left(\tfrac{1}{4}\right)^{i} \cdot n
        \right)^2
        = \left(\tfrac{3}{16}\right)^{i} \cdot c \cdot n^2
      \end{displaymath}
  \end{itemize}
\end{frame}

%-------------------------------------------------------------------------------

\begin{frame}{Recursion Equations}{Recursion Tree Method Costs}
  \textbf{Costs of solving partial solutions:} (only the last layer)
  \begin{itemize}
    \item
      Size of partial problems on the {\color{Mittel-Blau}last level}:
      {\color{Mittel-Blau}$\mathrm{s}_{i+1}(n) = 1$}
    \item
      Costs of partial problem on the {\color{Mittel-Blau}last level}:
      {\color{Mittel-Blau}$T_{i+1_p}(n) = d$}
    \item
      With this the depth of the tree is:
      \begin{displaymath}
        \left(\tfrac{1}{4}\right)^i \cdot n = 1
        \hspace{1.5em}\Rightarrow n = 4^i
        \hspace{1.5em}\Rightarrow\color{Mittel-Blau} i = \log_4 n
      \end{displaymath}
      \vspace{-1em}
    \item
      Number of partial problems on the {\color{Mittel-Blau}last level}:
      \begin{displaymath}
        \mathcolor{Mittel-Blau}{n_{i+1} = 3^{\log_4 n} = n^{\log_4 3}}
      \end{displaymath}
    \item
      Costs on the {\color{Mittel-Blau}last level}:
      {\color{Mittel-Blau}$T_{i+1}(n) = d \cdot n^{\log_4 3}$}
  \end{itemize}
\end{frame}

%-------------------------------------------------------------------------------

\begin{frame}{Recursion Equations}{Logarithm}
  Number of partial problems in the {\color{Mittel-Blau}last level}
  (leaves in the tree):
  \begin{align*}
    3^{\log_4 n} & = 3^{\log_4 (\log_3 n)} &&
      \text{use } \log a^b = b \cdot \log a\\
    {} & = 3^{(\log_3 n) \cdot \log_4 3} &&
      \text{use }x^{a \cdot b} = (x^a)^b\\
    {} & = 3^{(\log_3 n)^{\log_4 3}}\\
    {} & = n^{\log_4 3}
  \end{align*}
  {\color{gray}(This term will recur in the master theorem)}
\end{frame}

%-------------------------------------------------------------------------------

\begin{frame}{Recursion Equations}{Total costs}
  \textbf{Total costs:}
  \begin{itemize}
    \item
      Costs of {\color{Mittel-Blau}level i}:
      $T_i(n) = \left(\frac{3}{16}\right)^i \cdot c \cdot n^2$
    \item
      Costs of {\color{Mittel-Blau}last level}:
      $T_{i+1}(n) = d \cdot n^{\log_4 3}$
  \end{itemize}
  \vspace{0.5em}
  \begin{displaymath}
    \mathclap{T(n) =}\underbrace{
      \sum\limits_{i = 0}^{(\log_4 n) - 1} \left(\tfrac{3}{16}\right)^i
      \mathrlap{\cdot c \cdot n^2}
    }_{
      \text{
        \footnotesize
        \begin{tabular}{c}
          geometric series,\\
          constant\\
          \color{gray}
          $\left(\begin{array}{c}
            \text{even with}\\
            \text{infinite elements}
          \end{array}\right)$
        \end{tabular}
      }
    } \hspace{1.25em} + \hspace{0.5em} \underbrace{
      d \cdot n^{\log_4 3}
      \vphantom{\sum\limits_{i = 0}^{(\log_4 n) - 1}}
    }_{
      \text{
        \makebox[0pt][c]{
          \footnotesize
          \begin{tabular}{c}
            {\color{Mittel-Blau}$\log_4 3 < 1$},\\
            grows a lot\\
            slower than {\color{Mittel-Blau}$n^2$}
          \end{tabular}
        }
      }
    } ~ \in \mathcal{O}(n^2)
  \end{displaymath}
  \begin{itemize}
    \item
      Here: The costs of connecting the partial problems dominate
  \end{itemize}
\end{frame}

%-------------------------------------------------------------------------------

\begin{frame}{Recursion Equations}{Geometric Series}
  \begin{itemize}
    \item
      \textbf{Geometric progression:}\\
      Quotient of two neighbored progression parts is constant
      %TODO: Progression example
    \item
      \textbf{Geometric series:}\\
      The series (cumulative sum) of a geometric progression\\
      For $\mid q \mid < 1$:
      \begin{displaymath}
        \sum\limits^{\infty}_{k=0} a_0 \cdot q^k = \dfrac{a_0}{1 - q}
        \hspace{1.5em}\Rightarrow\text{constant}
      \end{displaymath}
  \end{itemize}
\end{frame}

%-------------------------------------------------------------------------------

\begin{frame}{Recursion Equations}{Proof of $\mathcal{O}(n^2)$}
  \textbf{Proof of $\mathcal{O}(n^2)$:}
  \begin{itemize}
    \item
      We know:
      \begin{align*}
        T(n) &= 3T\left(\frac{n}{4}\right) + \Theta(n^2)\\
        {} &\leq 3T \left(\frac{n}{4}\right) + c \cdot n^2
      \end{align*}
    \item
      Presumption:
      {\color{Mittel-Blau}$T(n) \, \in \, O(n^2)$},
      so there exists a {\color{Mittel-Blau}$k > 0$} with
      \begin{displaymath}
        \color{Mittel-Blau}
        T(n) \leq k \cdot n^2
      \end{displaymath}
  \end{itemize}
\end{frame}

%-------------------------------------------------------------------------------

\begin{frame}{Recursion Equations}{Proof of $\mathcal{O}(n^2)$}
  \textbf{Proof of $\mathcal{O}(n^2)$:}
  \begin{itemize}
    \item
      Presumption:
      {\color{Mittel-Blau}$T(n) \, \in \, O(n^2)$},
      so there exists a {\color{Mittel-Blau}$k > 0$} with
      \begin{displaymath}
        \mathcolor{Mittel-Blau}{T(n) < k \cdot n^2}
      \end{displaymath}
    \item
      Substitution method:
      \begin{align*}
        T(n) & \leq 3 \cdot T \left( \frac{n}{4}\right)  + c \cdot n^2\\
        {} & \leq 3 \, k \cdot \left( \frac{n}{4}\right)^2  + c \cdot n^2\\
        {} & = \frac{3}{16} \, k \cdot n^2  + c \cdot n^2\\
        {} & \leq k \cdot n^2
        \hspace{6em}\text{for } k \geq \frac{16}{13} \, c
      \end{align*}
  \end{itemize}
\end{frame}

%-------------------------------------------------------------------------------

\subsection{Mastertheorem}

\begin{frame}{Recursion Equations}{Mastertheorem}
  \textbf{Mastertheorem:}
  \begin{itemize}
    \item
      Approach to solve for a recursion equation of the form:
      \begin{displaymath}
        \mathcolor{Mittel-Blau}{
          T(n) = a \cdot T\left(\frac{n}{b}\right) + f(n),%
          \hspace{1.5em}a \geq 1, b > 1
        }
      \end{displaymath}
    \item
      {\color{Mittel-Blau}$T(n)$} is the runtime of an algorithm $\ldots$
      \begin{itemize}
        \item
          $\ldots$ which divides a {\color{Mittel-Blau}problem of size $n$}
          in {\color{Mittel-Blau}$a$ partial problems}
        \item
          $\ldots$ which solves each partial problem recursively\newline
          \hphantom{$\ldots$} with a
          {\color{Mittel-Blau}runtime of $T\left(\tfrac{n}{b}\right)$}
        \item
          $\ldots$ which takes {\color{Mittel-Blau}$f(n)$} steps to
          merge all partial solutions
      \end{itemize} 
  \end{itemize}
\end{frame}

%-------------------------------------------------------------------------------

\begin{frame}{Recursion Equations}{Mastertheorem (Simple Form)}
  \textbf{Mastertheorem:}
  \begin{itemize}
    \item
      We have seen that $\ldots$
      \begin{itemize}
        \item
          $\ldots$ the runtime of {\color{Mittel-Blau}connecting the solutions}
          dominates
        \item
          Or the runtime of {\color{Mittel-Blau}solving the problems} dominates
        \item
          Or both have {\color{Mittel-Blau}equal scaling runtime}
      \end{itemize}
    \item
      \textbf{Simple form:}
      Special case with runtime of connecting the solutions
      {\color{Mittel-Blau}$f(n) \in \mathcal{O}(n)$}
%TODO
%    \item
%      This can be shown quite easily.
%      See \cite{}
  \end{itemize}
\end{frame}

%-------------------------------------------------------------------------------

\begin{frame}{Recursion Equations}{Mastertheorem (Simple Form)}
  \textbf{Simple form:}
  \begin{displaymath}
    \mathcolor{Mittel-Blau}{
      T(n) = a \cdot T\left(\frac{n}{b}\right) +
      \underbrace{c \cdot n}_{
        \text{
          \clap{
            \footnotesize\color{black}
            \begin{tabular}{c}
              was any $f(n)$\\
              in general form
            \end{tabular}
          }
        }
      },
      \hspace{1.5em}a \geq 1, b > 1, c > 0
    }
  \end{displaymath}
   \begin{itemize}
     \item
       This yields a runtime of:
  \end{itemize}
  \vspace{2em}
  \begin{displaymath}
    T(n) = \begin{cases}
      \Theta(\mathcolor{Mittel-Blau}{
        \smash{\overbrace{n^{\log_b a}}^{
          \text{\clap{Number of leaves}}
        }}
      }) & \text{if } a > b\\
      \Theta(n \, \log n) & \text{if } a = b\\
      \Theta(n) & \text{if } a < b
    \end{cases}
  \end{displaymath}
\end{frame}

%-------------------------------------------------------------------------------

\begin{frame}{Recursion Equations}{Mastertheorem (Simple Form)}
  \begin{figure}[!h]
    \begin{adjustbox}{width=\linewidth}
      \def\AlgoREDivide{3}% 3
      \def\AlgoRESize{0.5}% 1/2
      \def\AlgoREScale{4.444}% 15 / (3/2)^3
      \begin{tikzpicture}[
  arrow/.style={
    >=stealth,
    ultra thick,
    color=black
  }, connection/.style={
    thick,
    color=black
  }
]%
\coordinate (root) at (0, 0);

\foreach \n in {1,...,\AlgoREDivide} {
  %Second layer
  \coordinate (node\n) at
    ({((\n-0.5)-0.5*\AlgoREDivide)*\AlgoRESize*\AlgoREScale}, -0.75);

  \foreach \m in {1,...,\AlgoREDivide} {
    % Third layer
    \coordinate (node\n_\m) at
      ({((((\n-1)*\AlgoREDivide+\m)-0.5)-0.5*pow(\AlgoREDivide, 2))
      *\AlgoREScale*pow(\AlgoRESize, 2)}, -1.5);
    
    \foreach \l in {1,...,\AlgoREDivide} {
      % Fourth layer
      \coordinate (node\n_\m_\l) at
        ({((((\m-1 + (\n-1)*\AlgoREDivide)*\AlgoREDivide+\l)-0.5)
        -0.5*pow(\AlgoREDivide, 3))*\AlgoREScale*pow(\AlgoRESize, 3)}, -2.25);
      
      \draw[-, connection] (node\n_\m_\l) -- (node\n_\m);
      \draw[<->, arrow]
        ($(node\n_\m_\l) + ({-0.5*\AlgoREScale*pow(\AlgoRESize, 3)}, 0)$)
        -- ++({\AlgoREScale*pow(\AlgoRESize, 3)}, 0);
    }

    \draw[-, connection] (node\n_\m) -- (node\n);
    \draw[<->, arrow]
      ($(node\n_\m) + ({-0.5*\AlgoREScale*pow(\AlgoRESize, 2)}, 0)$)
      -- ++({\AlgoREScale*pow(\AlgoRESize, 2)}, 0);
  }

  \draw[-, connection] (node\n) -- (root);
  \draw[<->, arrow]
  ($(node\n) + (-0.5*\AlgoREScale*\AlgoRESize, 0)$)
    -- ++(\AlgoREScale*\AlgoRESize, 0);
}

\draw[<->, arrow] ($(root) + (-0.5*\AlgoREScale, 0)$) -- ++(\AlgoREScale, 0);
\end{tikzpicture}
    \end{adjustbox}
    \caption{Simple recursion equation with {\color{Mittel-Blau}$a = 3, b = 2$}}
    \label{fig:recursion_equations:mastertheorem_tree_3_2}
  \end{figure}
  \textbf{Case 1:}
  \begin{itemize}
    \item
      Three partial problems with $\frac{1}{2}$ the size
    \item
      Solving the partial problems dominates (last layer, leaves)
  \end{itemize}
\end{frame}

%-------------------------------------------------------------------------------

\begin{frame}{Recursion Equations}{Mastertheorem (Simple Form)}
  \begin{figure}[!h]
    \begin{adjustbox}{width=\linewidth}
      \def\AlgoREDivide{2}% 2
      \def\AlgoRESize{0.5}% 1/2
      \def\AlgoREScale{15}% 15
      \begin{tikzpicture}[
  arrow/.style={
    >=stealth,
    ultra thick,
    color=black
  }, connection/.style={
    thick,
    color=black
  }
]%
\coordinate (root) at (0, 0);

\foreach \n in {1,...,\AlgoREDivide} {
  %Second layer
  \coordinate (node\n) at
    ({((\n-0.5)-0.5*\AlgoREDivide)*\AlgoRESize*\AlgoREScale}, -0.75);

  \foreach \m in {1,...,\AlgoREDivide} {
    % Third layer
    \coordinate (node\n_\m) at
      ({((((\n-1)*\AlgoREDivide+\m)-0.5)-0.5*pow(\AlgoREDivide, 2))
      *\AlgoREScale*pow(\AlgoRESize, 2)}, -1.5);
    
    \foreach \l in {1,...,\AlgoREDivide} {
      % Fourth layer
      \coordinate (node\n_\m_\l) at
        ({((((\m-1 + (\n-1)*\AlgoREDivide)*\AlgoREDivide+\l)-0.5)
        -0.5*pow(\AlgoREDivide, 3))*\AlgoREScale*pow(\AlgoRESize, 3)}, -2.25);
      
      \draw[-, connection] (node\n_\m_\l) -- (node\n_\m);
      \draw[<->, arrow]
        ($(node\n_\m_\l) + ({-0.5*\AlgoREScale*pow(\AlgoRESize, 3)}, 0)$)
        -- ++({\AlgoREScale*pow(\AlgoRESize, 3)}, 0);
    }

    \draw[-, connection] (node\n_\m) -- (node\n);
    \draw[<->, arrow]
      ($(node\n_\m) + ({-0.5*\AlgoREScale*pow(\AlgoRESize, 2)}, 0)$)
      -- ++({\AlgoREScale*pow(\AlgoRESize, 2)}, 0);
  }

  \draw[-, connection] (node\n) -- (root);
  \draw[<->, arrow]
  ($(node\n) + (-0.5*\AlgoREScale*\AlgoRESize, 0)$)
    -- ++(\AlgoREScale*\AlgoRESize, 0);
}

\draw[<->, arrow] ($(root) + (-0.5*\AlgoREScale, 0)$) -- ++(\AlgoREScale, 0);
\end{tikzpicture}
    \end{adjustbox}
    \caption{Simple recursion equation with {\color{Mittel-Blau}$a = 2, b = 2$}}
    \label{fig:recursion_equations:mastertheorem_tree_2_2}
  \end{figure}
  \textbf{Case 2:}
  \begin{itemize}
    \item
      Two partial problems with $\frac{1}{2}$ the size
    \item
      Each layer has equal costs
  \end{itemize}
\end{frame}

%-------------------------------------------------------------------------------

%\begin{frame}{Recursion Equations}{Mastertheorem (Simple Form)}
%  \begin{figure}[!h]
%    \begin{adjustbox}{width=\linewidth}
%      \def\AlgoREDivide{2}% 2
%      \def\AlgoRESize{0.25}% 1/4
%      \def\AlgoREScale{15}% 15
%      \begin{tikzpicture}[
  arrow/.style={
    >=stealth,
    ultra thick,
    color=black
  }, connection/.style={
    thick,
    color=black
  }
]%
\coordinate (root) at (0, 0);

\foreach \n in {1,...,\AlgoREDivide} {
  %Second layer
  \coordinate (node\n) at
    ({((\n-0.5)-0.5*\AlgoREDivide)*\AlgoRESize*\AlgoREScale}, -0.75);

  \foreach \m in {1,...,\AlgoREDivide} {
    % Third layer
    \coordinate (node\n_\m) at
      ({((((\n-1)*\AlgoREDivide+\m)-0.5)-0.5*pow(\AlgoREDivide, 2))
      *\AlgoREScale*pow(\AlgoRESize, 2)}, -1.5);
    
    \foreach \l in {1,...,\AlgoREDivide} {
      % Fourth layer
      \coordinate (node\n_\m_\l) at
        ({((((\m-1 + (\n-1)*\AlgoREDivide)*\AlgoREDivide+\l)-0.5)
        -0.5*pow(\AlgoREDivide, 3))*\AlgoREScale*pow(\AlgoRESize, 3)}, -2.25);
      
      \draw[-, connection] (node\n_\m_\l) -- (node\n_\m);
      \draw[<->, arrow]
        ($(node\n_\m_\l) + ({-0.5*\AlgoREScale*pow(\AlgoRESize, 3)}, 0)$)
        -- ++({\AlgoREScale*pow(\AlgoRESize, 3)}, 0);
    }

    \draw[-, connection] (node\n_\m) -- (node\n);
    \draw[<->, arrow]
      ($(node\n_\m) + ({-0.5*\AlgoREScale*pow(\AlgoRESize, 2)}, 0)$)
      -- ++({\AlgoREScale*pow(\AlgoRESize, 2)}, 0);
  }

  \draw[-, connection] (node\n) -- (root);
  \draw[<->, arrow]
  ($(node\n) + (-0.5*\AlgoREScale*\AlgoRESize, 0)$)
    -- ++(\AlgoREScale*\AlgoRESize, 0);
}

\draw[<->, arrow] ($(root) + (-0.5*\AlgoREScale, 0)$) -- ++(\AlgoREScale, 0);
\end{tikzpicture}
%    \end{adjustbox}
%    \caption{Simple recursion equation with
%      {\color{Mittel-Blau}$a = 2, b = 4$}}
%    \label{fig:recursion_equations:mastertheorem_tree_2_4}
%  \end{figure}
%  \textbf{Case 3:}
%  \begin{itemize}
%    \item
%      Two partial problems with $\frac{1}{4}$ the size
%    \item
%      Connecting all partial solutions dominates (first layer, root)
%  \end{itemize}
%\end{frame}

%-------------------------------------------------------------------------------

\begin{frame}{Recursion Equations}{Mastertheorem (Simple Form)}
  \begin{figure}[!h]
    \begin{adjustbox}{width=\linewidth}
      \def\AlgoREDivide{2}% 2
      \def\AlgoRESize{0.33}% 1/3
      \def\AlgoREScale{15}% 15
      \begin{tikzpicture}[
  arrow/.style={
    >=stealth,
    ultra thick,
    color=black
  }, connection/.style={
    thick,
    color=black
  }
]%
\coordinate (root) at (0, 0);

\foreach \n in {1,...,\AlgoREDivide} {
  %Second layer
  \coordinate (node\n) at
    ({((\n-0.5)-0.5*\AlgoREDivide)*\AlgoRESize*\AlgoREScale}, -0.75);

  \foreach \m in {1,...,\AlgoREDivide} {
    % Third layer
    \coordinate (node\n_\m) at
      ({((((\n-1)*\AlgoREDivide+\m)-0.5)-0.5*pow(\AlgoREDivide, 2))
      *\AlgoREScale*pow(\AlgoRESize, 2)}, -1.5);
    
    \foreach \l in {1,...,\AlgoREDivide} {
      % Fourth layer
      \coordinate (node\n_\m_\l) at
        ({((((\m-1 + (\n-1)*\AlgoREDivide)*\AlgoREDivide+\l)-0.5)
        -0.5*pow(\AlgoREDivide, 3))*\AlgoREScale*pow(\AlgoRESize, 3)}, -2.25);
      
      \draw[-, connection] (node\n_\m_\l) -- (node\n_\m);
      \draw[<->, arrow]
        ($(node\n_\m_\l) + ({-0.5*\AlgoREScale*pow(\AlgoRESize, 3)}, 0)$)
        -- ++({\AlgoREScale*pow(\AlgoRESize, 3)}, 0);
    }

    \draw[-, connection] (node\n_\m) -- (node\n);
    \draw[<->, arrow]
      ($(node\n_\m) + ({-0.5*\AlgoREScale*pow(\AlgoRESize, 2)}, 0)$)
      -- ++({\AlgoREScale*pow(\AlgoRESize, 2)}, 0);
  }

  \draw[-, connection] (node\n) -- (root);
  \draw[<->, arrow]
  ($(node\n) + (-0.5*\AlgoREScale*\AlgoRESize, 0)$)
    -- ++(\AlgoREScale*\AlgoRESize, 0);
}

\draw[<->, arrow] ($(root) + (-0.5*\AlgoREScale, 0)$) -- ++(\AlgoREScale, 0);
\end{tikzpicture}
    \end{adjustbox}
    \caption{Simple recursion equation with {\color{Mittel-Blau}$a = 2, b = 3$}}
    \label{fig:recursion_equations:mastertheorem_tree_2_3}
  \end{figure}
  \textbf{Case 3:}
  \begin{itemize}
    \item
      Two partial problems with $\frac{1}{3}$ the size
    \item
      Connecting all partial solutions dominates (first layer, root)
  \end{itemize}
\end{frame}

%-------------------------------------------------------------------------------

\begin{frame}{Recursion Equation}{Illustration lmax}
    
    TODO: Table -> look at slide 21 \vspace{2em}
    
  \begin{itemize}
    \item
      lmax and sum will be initialized with X(u)
    \item
      we go through X from u to o
    \item
      Sum refreshes
    \item
      lmax refreshes , if sum > lmax
  \end{itemize}
\end{frame}

%-------------------------------------------------------------------------------

\begin{frame}{Recursion Equation}{Illustraiting maxSubArray}
  
  TODO: Graphics -> look at slide 21 \vspace{2em}
  
\end{frame}

%-------------------------------------------------------------------------------

\begin{frame}{Recursion Equation}{Runtime}
  
  \textbf{maxSubArray(X, u, o)} \vspace{2em}
  
  if (u = o) then return X(u);\hspace{4em} $\mathcal{O}(1)$\\\vspace{0.5em}
  m = $\dfrac{u+o}{2}$;\hspace{10.7em} $\mathcal{O}(1)$\\\vspace{0.5em}
  A = maxSubArray (X, u, m);\hspace{3.1em} T($\dfrac{n}{2})$\\\vspace{0.5em}
  B = maxSubArray (X, m+1, o);\hspace{1.9em} T($\dfrac{n}{2})$\\\vspace{0.5em}
  $C_1$ = rmax (X, u, m);\hspace{6.2em} $\mathcal{O}(n)$\\\vspace{0.5em}
  $C_2$ = lmax (X, m+1, o);\hspace{5.2em} $\mathcal{O}(n)$\\\vspace{0.5em}
  return max (A, B, $C_1$ + $C_2$);\hspace{3.3em} $\mathcal{O}(1)$
\end{frame}

%-------------------------------------------------------------------------------

\begin{frame}{Recursion Equation}{Number of steps T(n)}
  
  TODO: Grapic -> look at slide 24 \vspace{2em}
  
  so it existing constants a and b with\\\vspace{2em}
  
  TODO: Grapic -> look at slide 24 \vspace{2em}
  
  We define c:= max(a,b), then\\\vspace{2em}
  
  TODO: Grapic -> look at slide 24
  
\end{frame}

%-------------------------------------------------------------------------------

\begin{frame}{Recursion Equation}{Illustration of T(n)}

TODO: Graphics -> look at slide 25 \vspace{2em}

\end{frame}

%-------------------------------------------------------------------------------

\begin{frame}{Recursion Equation}{Illustration of T(n)}

TODO: Graphics -> look at slide 26 \vspace{2em}

\end{frame}

%-------------------------------------------------------------------------------

\begin{frame}{Recursion Equation}{Illustration of T(n)}
  \begin{itemize}
    \item
      Upper level i = 0, lower level at 2i = n $\rightarrow$ i = $\log_2 
      n$, so 
      overall $\log_2 n + 1$ levels
    \item
      In each level has cost of $c \cdot n$ (the costs for merging of 
      subproblems and the solving of trivial problems are here the same)
  \end{itemize}
\end{frame}

%-------------------------------------------------------------------------------

\begin{frame}{Recursion Equation}{Summaray Maximum subtotal}
  \begin{itemize}
    \item
      Direct solution is $\mathcal{O}(n^3)$
    \item
      A better solution with incremental refreshment of the subtotals was 
      $\mathcal{O}(n^2)$
    \item
      Divide and concur approach results in $\mathcal{O}(n \cdot \log n)$
    \item
      There is an approach in $\mathcal{O}(n)$, if you assume that all 
      subtotals are positive
  \end{itemize}
\end{frame}