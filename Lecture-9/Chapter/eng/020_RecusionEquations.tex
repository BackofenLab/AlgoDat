%%% ===================================================================
%%% RCS Header:
%%% $RCSfile: local.el,v $
%%% $Revision: 1.27 $
%%% $Date: 2014/11/07 15:36:00 $
%%% $Author: backofen $
%%% $Locker:  $
%%% ===================================================================

\section{Recursion Equations}

\begin{frame}{Recursion Equations}{Recursion Equation}
  \textbf{Recursion equation:}
  \begin{itemize}
    \item
      Describes the runtime for recursive functions:
  \end{itemize}
  \vspace{2em}
  \begin{displaymath}
    {\color{Mittel-Blau}T(n)} =
    \begin{cases}
      \hfill
      \smash{
        \overbrace{f_0(n)}^\text{%
          \footnotesize%
          trivial case for $n_0$
        }
      }
      \vphantom{f_0(n)} & n = n_0\\[0.5em]
      \smash{
        \underbrace{
          a \cdot {\color{Mittel-Blau}T\left(\frac{n}{b}\right)}
        }_\text{
          \footnotesize%
          \begin{tabular}{c}
            \text{solving of {\color{Mittel-Blau}$a$}}\\
            \text{subproblems}\\
            \text{with reduced}\\
            \text{input size {\color{Mittel-Blau}$\frac{n}{b}$}}
          \end{tabular}
        }
      } \hspace{0.5em} + \hspace{1.5em}
      \smash{
        \underbrace{
          f(n)
          \vphantom{\left(\frac{n}{b}\right)}
        }_\text{
          \makebox[0pt][c]{
            \footnotesize%
            \begin{tabular}{c}
              slicing and\\
              splicing of\\
              subsolutions
            \end{tabular}
          }
        }
      }
      \vphantom{\left(\frac{n}{b}\right)}
      &n > n_0
    \end{cases}
  \end{displaymath}
  \vspace{4em}
\end{frame}

%-------------------------------------------------------------------------------

\begin{frame}{Recursion Equations}{Recursion Equation}
  \textbf{Recursion equation:}
  \begin{itemize}
    \item<2->
    Describes the runtime for recursive functions:
    \begin{displaymath}
    {\color{Mittel-Blau}T(n)} = \begin{cases}
      f_0(n) & n = n_0\\
      a \cdot {\color{Mittel-Blau}T\left(\frac{n}{b}\right)} + f(n) & n > n_0
    \end{cases}
    \end{displaymath}
    \item<3->
      {\color{Mittel-Blau}$n_0$} is normally small, {\color{Mittel-Blau}$f_0(n_0) \; \in \; \Theta(1)$}
    \item<4->
      Normally {\color{Mittel-Blau}$a > 1$} and {\color{Mittel-Blau}$b > 1$}
    \item<5->
      Dependent on the strategy of solving {\color{Mittel-Blau}$T(n)$}
      {\color{Mittel-Blau}$f_0$} is ignored
    \item<6->
      {\color{Mittel-Blau}$T(n)$} is only defined for integers of
      {\color{Mittel-Blau}$\frac{n}{b}$} which is often ignored in benefit of
      a simpler solution
  \end{itemize}
\end{frame}

%-------------------------------------------------------------------------------

\subsection{Substitution Method}

\begin{frame}{Recursion Equations}{Substitution Method}
  \textbf{Substitution Method:}
  \begin{itemize}
    \item<2->
      Guess the solution and prove it with induction
    \item<3->
      Example:        \color{Mittel-Blau}
      \begin{displaymath}
        T(n) = \begin{cases}
          \hfill 1 & n = 1\\
          2 \cdot T\left(\frac{n}{2}\right) + n & n > 1
        \end{cases}
      \end{displaymath}        \color{black}
    \item<4->
      Assumption:  {\color{Mittel-Blau}$T(n) = n + n \cdot \log_2 n$}
  \end{itemize}
\end{frame}

%---------------------------------------------------------------------------------------

\begin{frame}{Recursion Equations}{Substitution Method}
  \textbf{Induction:}
  \begin{itemize}
    \item<2->
      Induction basis (for {\color{Mittel-Blau}$n = 1$}):
      ${\color{Mittel-Blau}T(1)} = 1 + 1 \cdot \log_2 1 = 1$
    \item<3->
      Induction step (from {\color{Mittel-Blau}$\frac{n}{2}$}
      to {\color{Mittel-Blau}$n$}):
      \begin{align*}
        \onslide<4->{{\color{Mittel-Blau}T(n)} & =
          2 \cdot {\color{Mittel-Blau}T\left(\dfrac{n}{2}\right)} + n\\}
        \onslide<5->{{} & \stackrel{IA}{=} 2 \cdot \left(
            \dfrac{n}{2} + \dfrac{n}{2} \cdot \log_2 \dfrac{n}{2}
          \right) + n \\}
        \onslide<6->{{} & = 2 \cdot \left(
            \dfrac{n}{2} + \dfrac{n}{2} \cdot \left(\log_2 n - 1\right)
          \right) + n}\\
        \onslide<7->{{} & = n + n \, \log_2 n - n + n\\}
        \onslide<8->{{} & = n + n \, \log_2 n}
      \end{align*}
  \end{itemize}
\end{frame}

%---------------------------------------------------------------------------------------

\begin{frame}{Recursion Equations}{Substitution Method}
  \textbf{Substitution Method:}
  \begin{itemize}
    \item<2->
      Alternative assumption
    \item<3->
      Example:
      \begin{displaymath}
        \color{Mittel-Blau}
        T(n) = \begin{cases}
          \hfill 1 & n = 1\\
          2 \cdot T\left(\frac{n}{2}\right) + n & n > 0
        \end{cases}
      \end{displaymath}
    \item<4->
      Assumption:
      {\color{Mittel-Blau}$T(n) \, \in \, O(n \, \log n)$}
    \item<5->
      Solution: Find {\color{Mittel-Blau}$c > 0$} with
      {\color{Mittel-Blau}$T(n) \leq c \cdot n \, \log_2 n$}
  \end{itemize}
\end{frame}

%---------------------------------------------------------------------------------------

\begin{frame}{Recursion Equations}{Substitution Method}
  \textbf{Induction:}
  \begin{itemize}
    \item<2->
      Solution: Find {\color{Mittel-Blau}$c > 0$} with
      {\color{Mittel-Blau}$T(n) \leq c \cdot n \, \log_2 n$}
    \item<3->
      Induction step (from {\color{Mittel-Blau}$\frac{n}{2}$} to
      {\color{Mittel-Blau}$n$}):
      \begin{align*}
        \onslide<4->{{\color{Mittel-Blau}T(n)} & =
          2 \cdot {\color{Mittel-Blau}T\left(\dfrac{n}{2}\right)} + n\\}
        \onslide<5->{{} & \leq 2 \cdot \left(
          c \cdot \dfrac{n}{2} \, \log_2 \dfrac{n}{2}\right) + n\\}
        \onslide<6->{{} & = c \cdot n \, \log_2 n - c \cdot n \, \log_2 2 + n\\}
        \onslide<7->{{} & = c \cdot n \, \log_2 n - c \cdot n + n\\}
        \onslide<8->{{} & \leq c \cdot n \, \log_2 n,
          \hspace{0.5em} {\color{Mittel-Blau}c \geq 1}}
      \end{align*}
  \end{itemize}
\end{frame}

%-------------------------------------------------------------------------------

\subsection{Recursion Tree Method}

\begin{frame}{Recursion Equations}{Recursion Tree Method}
  \textbf{Recursion tree method:}
  \begin{itemize}
    \item<2->
      Can be used to make assumptions about the runtime
    \item<3->
      Example:\\
      \begin{displaymath}
        {\color{Mittel-Blau}T(n)}
        = 3 \cdot {\color{Mittel-Blau}T\left(\dfrac{n}{4}\right)} + \Theta(n^2)
        \leq 3 \cdot {\color{Mittel-Blau}T\left(\dfrac{n}{4}\right)} + c \cdot 
        n^2
      \end{displaymath}
  \end{itemize}
\end{frame}

%-------------------------------------------------------------------------------

\begin{frame}{Recursion Equations}{Recursion Tree Method}
  \begin{figure}
    \begin{adjustbox}{width=\linewidth}
      \begin{tikzpicture}[
  op/.style={
    draw=black,
    color=black,
    fill=white,
    font=\Large,
    thick,
    ellipse,
    minimum size=3em
  }, sum/.style={
    font=\Large,
    text=Mittel-Blau
  }, connection/.style={
    ultra thick,
    color=Mittel-Gruen
  }, arrow/.style={
    ultra thick,
    color=Mittel-Blau
  }, arrow_alt/.style={
    ultra thick,
    color=Hell-Blau
  }, label/.style={
    font=\Large,
    text=Mittel-Blau
  }, label_alt/.style={
    font=\Large,
    text=Hell-Blau
  }
]%
% Bounding boxes
\only<1>{
  \draw[white] (-10, -4) rectangle (10, 4);
  \node[sum, anchor=north west] at (-9.5, 3.5) {%
    $T\left(n\right) = 3 \cdot T\left(\frac{n}{4}\right) + c \cdot n^2$
  };
}
\only<2>{
  \draw[white] (-10, -5) rectangle (10, 3);
  \node[sum, anchor=north west] at (-9.5, 2.5) {%
    $T\left(n\right) = 3 \cdot T\left(\frac{n}{4}\right) + c \cdot n^2$
  };
}
\only<3>{
  \draw[white] (-10, -6) rectangle (10, 2);
  \node[sum, anchor=north west] at (-9.5, 1.5) {%
    $T\left(n\right) = 3 \cdot T\left(\frac{n}{4}\right) + c \cdot n^2$
  };
}

% Main Problem with first-level subproblems
\node[op] (root) at (0, 0) {%
  \only<1>{$T(n)$}%
  \only<2->{$c \cdot n^2$}%
};

\only<2->{
  \node[op] (left) at (-6.75, -2) {%
    \only<2>{$T\left(\frac{n}{4}\right)$}%
    \only<3->{$c \cdot \left(\frac{n}{4}\right)^2$}%
  };
  \node[op] (center) at (0, -2) {%
    \only<2>{$T\left(\frac{n}{4}\right)$}%
    \only<3->{$c \cdot \left(\frac{n}{4}\right)^2$}%
  };
  \node[op] (right) at (6.75, -2) {%
    \only<2>{$T\left(\frac{n}{4}\right)$}%
    \only<3->{$c \cdot \left(\frac{n}{4}\right)^2$}%
  };

  \draw[-, connection] (root) -- (left);
  \draw[-, connection] (root) -- (center);
  \draw[-, connection] (root) -- (right);
}
\only<2>{
  \node[sum] at (0, -3.5) {%
    $T(n) = 3 \cdot T\left(\frac{n}{4}\right) + c \cdot n^2$%
  };
}

\only<3->{
  \node[op] (left_left) at (-9, -4) {$T\left(\frac{n}{4}\right)$};
  \node[op] (left_center) at (-6.75, -4) {$T\left(\frac{n}{4}\right)$};
  \node[op] (left_right) at (-4.5, -4) {$T\left(\frac{n}{4}\right)$};
  \node[op] (center_left) at (-2.25, -4) {$T\left(\frac{n}{4}\right)$};
  \node[op] (center_center) at (0, -4) {$T\left(\frac{n}{4}\right)$};
  \node[op] (center_right) at (2.25, -4) {$T\left(\frac{n}{4}\right)$};
  \node[op] (right_left) at (4.5, -4) {$T\left(\frac{n}{4}\right)$};
  \node[op] (right_center) at (6.75, -4) {$T\left(\frac{n}{4}\right)$};
  \node[op] (right_right) at (9, -4) {$T\left(\frac{n}{4}\right)$};

  \draw[-, connection] (left) -- (left_left);
  \draw[-, connection] (left) -- (left_center);
  \draw[-, connection] (left) -- (left_right);
  \draw[-, connection] (center) -- (center_left);
  \draw[-, connection] (center) -- (center_center);
  \draw[-, connection] (center) -- (center_right);
  \draw[-, connection] (right) -- (right_left);
  \draw[-, connection] (right) -- (right_center);
  \draw[-, connection] (right) -- (right_right);

  \only<3>{
    \node[sum] at (0, -5.5) {%
      \begin{math}%
      T\left(n\right) = 12 \cdot T\left(\frac{n}{16}\right)
        + 3 c \cdot \left(\frac{n}{4}\right)^2 + c \cdot n^2
      \end{math}%
    };
  }
}
\end{tikzpicture}
    \end{adjustbox}
    \caption{Recursion tree of example}
    \label{fig:recursion_equations:example_recursion_tree}
  \end{figure}
\end{frame}

%-------------------------------------------------------------------------------

\begin{frame}{Recursion Equations}{Recursion Tree Method}
  \begin{figure}
    \begin{adjustbox}{width=\linewidth}
      \begin{tikzpicture}[
  op/.style={
    draw=black,
    color=black,
    fill=white,
    font=\Large,
    thick,
    ellipse,
    minimum size=3em
  }, dots/.style={
    font=\Large,
    minimum height=3em
  }, sum/.style={
    font=\Large,
    text=Mittel-Blau
  }, connection/.style={
    ultra thick,
    color=Mittel-Gruen,
    draw=Mittel-Gruen
  }, arrow/.style={
    ultra thick,
    color=Mittel-Blau
  }, arrow_alt/.style={
    ultra thick,
    color=Hell-Blau
  }, label/.style={
    font=\Large,
    text=Mittel-Blau
  }, label_alt/.style={
    font=\Large,
    text=Hell-Blau
  }
]%
% Bounding box
\draw[white] (-2.25, -9) rectangle (19, 1);

% Main Problem with first-level subproblems
\node[op] (root) at (0, 0) {$c \cdot n^2$};

\node[dots] (left) at (-2, -2) {...};
\node[op] (center) at (0, -2) {$c \cdot \left(\frac{n}{4}\right)^2$};
\node[op] (right) at (11.25, -2) {$c \cdot \left(\frac{n}{4}\right)^2$};

\draw[-, connection] (root) -- (left);
\draw[-, connection] (root) -- (center);
\draw[-, connection] (root) -- (right);

\node[dots] (center_left) at (-2, -4) {...};
\node[op] (center_center) at (0, -4) {$c \cdot \left(\frac{n}{16}\right)^2$};
\node[op] (center_right) at (3.75, -4) {$c \cdot \left(\frac{n}{16}\right)^2$};
\node[op] (right_left) at (7.5, -4) {$c \cdot \left(\frac{n}{16}\right)^2$};
\node[op] (right_center) at (11.25, -4) {$c \cdot \left(\frac{n}{16}\right)^2$};
\node[op] (right_right) at (15, -4) {$c \cdot \left(\frac{n}{16}\right)^2$};

\draw[-, connection] (center) -- (center_left);
\draw[-, connection] (center) -- (center_center);
\draw[-, connection] (center) -- (center_right);
\draw[-, connection] (right) -- (right_left);
\draw[-, connection] (right) -- (right_center);
\draw[-, connection] (right) -- (right_right);

\draw[connection] (-1.25, -6) node[op] (center_center_l) {$\cdots$} to
  ++(0, -2) node[op] {$d$};
\draw[connection] (0, -6) node[op] (center_center_c) {$\cdots$} to
  ++(0, -2) node[op] {$d$};
\draw[connection] (1.25, -6) node[op] (center_center_r) {$\cdots$} to
  ++(0, -2) node[op] {$d$};
\draw[connection] (2.5, -6) node[op] (center_right_l) {$\cdots$} to
  ++(0, -2) node[op] {$d$};
\draw[connection] (3.75, -6) node[op] (center_right_c) {$\cdots$} to
  ++(0, -2) node[op] {$d$};
\draw[connection] (5, -6) node[op] (center_right_r) {$\cdots$} to
  ++(0, -2) node[op] {$d$};
\draw[connection] (6.25, -6) node[op] (right_left_l) {$\cdots$} to
  ++(0, -2) node[op] {$d$};
\draw[connection] (7.5, -6) node[op] (right_left_c) {$\cdots$} to
  ++(0, -2) node[op] {$d$};
\draw[connection] (8.75, -6) node[op] (right_left_r) {$\cdots$} to
  ++(0, -2) node[op] {$d$};
\draw[connection] (10, -6) node[op] (right_center_l) {$\cdots$} to
  ++(0, -2) node[op] {$d$};
\draw[connection] (11.25, -6) node[op] (right_center_c) {$\cdots$} to
  ++(0, -2) node[op] {$d$};
\draw[connection] (12.5, -6) node[op] (right_center_r) {$\cdots$} to
  ++(0, -2) node[op] {$d$};
\draw[connection] (13.75, -6) node[op] (right_right_l) {$\cdots$} to
  ++(0, -2) node[op] {$d$};
\draw[connection] (15, -6) node[op] (right_right_c) {$\cdots$} to
  ++(0, -2) node[op] {$d$};
\draw[connection] (16.25, -6) node[op] (right_right_r) {$\cdots$} to
  ++(0, -2) node[op] {$d$};

\draw[-, connection] (center_center) -- (center_center_l);
\draw[-, connection] (center_center) -- (center_center_c);
\draw[-, connection] (center_center) -- (center_center_r);
\draw[-, connection] (center_right) -- (center_right_l);
\draw[-, connection] (center_right) -- (center_right_c);
\draw[-, connection] (center_right) -- (center_right_r);
\draw[-, connection] (right_left) -- (right_left_l);
\draw[-, connection] (right_left) -- (right_left_c);
\draw[-, connection] (right_left) -- (right_left_r);
\draw[-, connection] (right_center) -- (right_center_l);
\draw[-, connection] (right_center) -- (right_center_c);
\draw[-, connection] (right_center) -- (right_center_r);
\draw[-, connection] (right_right) -- (right_right_l);
\draw[-, connection] (right_right) -- (right_right_c);
\draw[-, connection] (right_right) -- (right_right_r);

\node[sum, anchor=east] at (19, 0) {level 0};
\node[sum, anchor=east] at (19, -2) {level 1};
\node[sum, anchor=east] at (19, -4) {level 2};
\node[sum, anchor=east] at (19, -6) {level $i$};
\node[sum, anchor=east, align=center] at (19, -8) {last\\level};
\end{tikzpicture}
    \end{adjustbox}
    \caption{Levels of the recursion tree}
    \label{fig:recursion_equations:example_recursion_tree_layer}
  \end{figure}
\end{frame}
%-------------------------------------------------------------------------------

\begin{frame}{Recursion Equations}{Recursion Tree Method Costs}
  \textbf{Costs of connecting the partial solutions:}\\
  \hspace{1.5em}(excludes the last layer)
  \begin{itemize}
    \item<2->
      Size of partial problems on {\color{Mittel-Blau}level $i$}:
      \begin{math}
        \color{Mittel-Blau}{
          \mathrm{s}_i(n) = \left(\frac{1}{4}\right)^i \cdot n
        }
      \end{math}
    \item<3->
      Costs of partial problem on {\color{Mittel-Blau}level $i$}:
      \begin{displaymath}
        \color{Mittel-Blau}{
          T_{i_p}(n) =
          c \cdot \left(\left(\tfrac{1}{4}\right)^i \cdot n\right)^2
        }
      \end{displaymath}
    \item<4->
      Number of partial problems on {\color{Mittel-Blau}level $i$}:
      {\color{Mittel-Blau}$n_{i} = 3^{i}$}
    \item<5->
      Costs on {\color{Mittel-Blau}level $i$}:
        \color{Mittel-Blau}
      \begin{displaymath}
        T_i(n) = 3^{i} \cdot c \cdot \left(
          \left(\tfrac{1}{4}\right)^{i} \cdot n
        \right)^2
        = \left(\tfrac{3}{16}\right)^{i} \cdot c \cdot n^2
      \end{displaymath}
        \color{black}
  \end{itemize}
\end{frame}

%-------------------------------------------------------------------------------

\begin{frame}{Recursion Equations}{Recursion Tree Method Costs}
  \textbf{Costs of solving partial solutions:} (only the last layer)
  \begin{itemize}
    \item<2->
      Size of partial problems on the {\color{Mittel-Blau}last level}:
      {\color{Mittel-Blau}$\mathrm{s}_{i+1}(n) = 1$}
    \item<3->
      Costs of partial problem on the {\color{Mittel-Blau}last level}:
      {\color{Mittel-Blau}$T_{i+1_p}(n) = d$}
    \item<4->
      With this the depth of the tree is:
      \begin{displaymath}
        \left(\tfrac{1}{4}\right)^i \cdot n = 1
        \hspace{1.5em}\Rightarrow n = 4^i
        \hspace{1.5em}\Rightarrow\color{Mittel-Blau} i = \log_4 n
      \end{displaymath}
      \vspace{-1em}
    \item<5->
      Number of partial problems on the {\color{Mittel-Blau}last level}:
      \begin{displaymath}
        \color{Mittel-Blau}{n_{i+1} = 3^{\log_4 n}    \onslide<6-> \quad=
          n^{\log_4 3}}  \quad \leftarrow \text{next slide}
      \end{displaymath}
    \item<7->
      Costs on the {\color{Mittel-Blau}last level}:
      {\color{Mittel-Blau}$T_{i+1}(n) = d \cdot n^{\log_4 3}$}
  \end{itemize}
\end{frame}

%-------------------------------------------------------------------------------

\begin{frame}{Fun with logarithm}{Logarithm}
  \begin{itemize}
  \item transforming $\color{Mittel-Blau}3^{\log_4 n}$ uses general log rules
\color{Mittel-Blau}
    \begin{align*}
      \log_4 n & =  \log_4 \left(3^{\log_3 n} \right) && \qquad  \text{\color{black}uses } n= 3^{\log_3 n}\\
\onslide<2->{      & = {\log_3 n \cdot \log_4 3} &&\qquad \text{\color{black}uses } \log a^b =
      b \cdot \log a }
    \end{align*}
  \item<3-> \color{black}
 this proves the general log rule $\color{Mittel-Blau}\log_b c = \log_a c \cdot
    \log_b a$
  \item<4->\color{black}
 now the whole expression:
\color{Mittel-Blau}
    \begin{align*}
      3^{\log_4 n}
      & = 3^{\log_3 n \cdot \log_4 3}       &&\qquad  \text{\color{black}uses reformulation above} \\ 
\onslide<5->{      & = {\left(3^{\log_3 n}\right)}^{\log_4 3}      &&\qquad  \text{\color{black}uses }x^{a \cdot b} = (x^a)^b}\\
\onslide<6->{      & = n^{\log_4 3}}
    \end{align*}
  \item<7-> 
    {\color{gray}This term will recur in the master theorem}
  \end{itemize}
\end{frame}

%-------------------------------------------------------------------------------

\begin{frame}{Recursion Equations}{Total costs}
  \textbf{Total costs:}
  \begin{itemize}
    \item<2->
      Costs of {\color{Mittel-Blau}level i}:
      $T_i(n) = \left(\frac{3}{16}\right)^i \cdot c \cdot n^2$
    \item<3->
      Costs of {\color{Mittel-Blau}last level}:
      $T_{i+1}(n) = d \cdot n^{\log_4 3}$
  \end{itemize}
  \vspace{0.5em}
  \onslide<4->
  \begin{displaymath}
    \mathclap{T(n) =}\underbrace{
      \sum\limits_{i = 0}^{(\log_4 n) - 1} \left(\tfrac{3}{16}\right)^i
      \mathrlap{\cdot c \cdot n^2}
    }_{
      \text{
        \footnotesize
        \begin{tabular}{c}
          geometric series,\\
          constant\\
          \color{gray}
          $\left(\begin{array}{c}
            \text{even with}\\
            \text{infinite elements}
          \end{array}\right)$
        \end{tabular}
      }
    } \hspace{1.25em} + \hspace{0.5em} \underbrace{
      d \cdot n^{\log_4 3}
      \vphantom{\sum\limits_{i = 0}^{(\log_4 n) - 1}}
    }_{
      \text{
        \makebox[0pt][c]{
          \footnotesize
          \begin{tabular}{c}
            {\color{Mittel-Blau}$\log_4 3 < 1$},\\
            grows a lot\\
            slower than {\color{Mittel-Blau}$n^2$}
          \end{tabular}
        }
      }
    } ~ \in O(n^2)
  \end{displaymath}
  \begin{itemize}
    \item<5->
      Here: The costs of connecting the partial problems dominate
  \end{itemize}
\end{frame}

%-------------------------------------------------------------------------------

\begin{frame}{Recursion Equations}{Geometric Series}
  \begin{itemize}
    \item<2->
      \textbf{Geometric progression:}\\
      Quotient of two neighbored progression parts is constant
      %TODO: Progression example
    \item<3->
      \textbf{Geometric series:}\\
      The series (cumulative sum) of a geometric progression\\
    \item<4->  
      For $\mid q \mid < 1$:
      \begin{displaymath}
        \sum\limits^{\infty}_{k=0} a_0 \cdot q^k = \dfrac{a_0}{1 - q}
        \hspace{1.5em}\Rightarrow\text{constant}
      \end{displaymath}
    \item<5->
      Therefore constant
  \end{itemize}
\end{frame}

%-------------------------------------------------------------------------------

\begin{frame}{Recursion Equations}{Proof of $O(n^2)$}
  \textbf{Proof of $O(n^2)$:}
  \begin{itemize}
    \item<2->
      We know:
      \begin{align*}
        T(n) &= 3T\left(\frac{n}{4}\right) + \Theta(n^2)\\
        {} &\leq 3T \left(\frac{n}{4}\right) + c \cdot n^2
      \end{align*}
    \item<3->
      Presumption:
      {\color{Mittel-Blau}$T(n) \, \in \, O(n^2)$},
      so there exists a {\color{Mittel-Blau}$k > 0$} with
      \begin{displaymath}
        \color{Mittel-Blau}
        T(n) \leq k \cdot n^2
      \end{displaymath}
  \end{itemize}
\end{frame}

%-------------------------------------------------------------------------------

\begin{frame}{Recursion Equations}{Proof of $O(n^2)$}
  \textbf{Proof of $O(n^2)$:}
  \begin{itemize}
    \item<2->
      Presumption:
      {\color{Mittel-Blau}$T(n) \, \in \, O(n^2)$},
      so there exists a {\color{Mittel-Blau}$k > 0$} with
      \begin{displaymath}
        \color{Mittel-Blau}{T(n) < k \cdot n^2}
      \end{displaymath}
    \item<3->
      Substitution method:
      \begin{align*}
        \onslide<4->{T(n) & \leq 3 \cdot T \left( \frac{n}{4}\right)  + c \cdot n^2\\}
        \onslide<5->{{} & \leq 3 \, k \cdot \left( \frac{n}{4}\right)^2  + c \cdot n^2\\}
        \onslide<6->{{} & = \frac{3}{16} \, k \cdot n^2  + c \cdot n^2\\}
        \onslide<7->{{} & \leq k \cdot n^2
        \hspace{6em}\text{for } k \geq \frac{16}{13} \, c}
      \end{align*}
  \end{itemize}
\end{frame}

%-------------------------------------------------------------------------------

\subsection{Master theorem}

\begin{frame}{Recursion Equations}{Master theorem}
  \textbf{Master theorem:}
  \begin{itemize}
    \item<2->
      Approach to solve for a recursion equation of the form:
      \begin{displaymath}
        \color{Mittel-Blau}{
          T(n) = a \cdot T\left(\frac{n}{b}\right) + f(n),%
          \hspace{1.5em}a \geq 1, b > 1
        }
      \end{displaymath}
    \item<3->
      {\color{Mittel-Blau}$T(n)$} is the runtime of an algorithm $\ldots$
      \begin{itemize}
        \item<4->
          $\ldots$ which divides a {\color{Mittel-Blau}problem of size $n$}
          in {\color{Mittel-Blau}$a$ partial problems}
        \item<5->
          $\ldots$ which solves each partial problem recursively\newline
          \hphantom{$\ldots$} with a
          {\color{Mittel-Blau}runtime of $T\left(\tfrac{n}{b}\right)$}
        \item<6->
          $\ldots$ which takes {\color{Mittel-Blau}$f(n)$} steps to
          merge all partial solutions
      \end{itemize} 
  \end{itemize}
\end{frame}

%-------------------------------------------------------------------------------

\subsubsection{Master theorem (Simple Form)}

\begin{frame}{Recursion Equations}{Master theorem (Simple Form)}
  \textbf{Master theorem:}
  \begin{itemize}
    \item<2->
      In the examples we have seen that $\ldots$
      \begin{itemize}
        \item<3->
          Either the runtime of {\color{Mittel-Blau}connecting the solutions}
          dominates
        \item<4->
          Or the runtime of {\color{Mittel-Blau}solving the problems} dominates
        \item<5->
          Or both have {\color{Mittel-Blau}equal influence on runtime}
      \end{itemize}
    \item<6->
      \textbf{Simple form:}
      Special case with runtime of connecting the solutions
      {\color{Mittel-Blau}$f(n) \in O(n)$}
%TODO
%    \item
%      This can be shown quite easily.
%      See \cite{}
  \end{itemize}
\end{frame}

%-------------------------------------------------------------------------------

\begin{frame}{Recursion Equations}{Master theorem (Simple Form)}
  \textbf{Simple form:}
  \onslide<2->
  \begin{displaymath}
    \color{Mittel-Blau}{
      T(n) = a \cdot T\left(\frac{n}{b}\right) +
      \underbrace{c \cdot n}_{
        \text{
          \clap{
            \footnotesize\color{black}
            \begin{tabular}{c}
              was any $f(n)$\\
              in general form
            \end{tabular}
          }
        }
      },
      \hspace{1.5em}a \geq 1, b > 1, c > 0
    }
  \end{displaymath}
  \begin{itemize}
     \item<3->
       This yields a runtime of:
  \end{itemize}
  \onslide<4->
  \vspace{2em}
  \begin{displaymath}
    T(n) = \begin{cases}
      \Theta(\color{Mittel-Blau}{
        \smash{\overbrace{n^{\log_b a}}^{
          \text{\clap{Number of leaves}}
        }}
      }) & \text{if } a > b\\
      \Theta(n \, \log n) & \text{if } a = b\\
      \Theta(n) & \text{if } a < b
    \end{cases}
  \end{displaymath}
\end{frame}

%-------------------------------------------------------------------------------

\begin{frame}{Recursion Equations}{Master theorem (Simple Form)}
  \begin{figure}[!h]
    \begin{adjustbox}{width=\linewidth}
      \def\AlgoREDivide{3}% 3
      \def\AlgoRESize{0.5}% 1/2
      \def\AlgoREScale{4.444}% 15 / (3/2)^3
      \begin{tikzpicture}[
  arrow/.style={
    >=stealth,
    ultra thick,
    color=black
  }, connection/.style={
    thick,
    color=black
  }
]%
\coordinate (root) at (0, 0);

\foreach \n in {1,...,\AlgoREDivide} {
  %Second layer
  \coordinate (node\n) at
    ({((\n-0.5)-0.5*\AlgoREDivide)*\AlgoRESize*\AlgoREScale}, -0.75);

  \foreach \m in {1,...,\AlgoREDivide} {
    % Third layer
    \coordinate (node\n_\m) at
      ({((((\n-1)*\AlgoREDivide+\m)-0.5)-0.5*pow(\AlgoREDivide, 2))
      *\AlgoREScale*pow(\AlgoRESize, 2)}, -1.5);
    
    \foreach \l in {1,...,\AlgoREDivide} {
      % Fourth layer
      \coordinate (node\n_\m_\l) at
        ({((((\m-1 + (\n-1)*\AlgoREDivide)*\AlgoREDivide+\l)-0.5)
        -0.5*pow(\AlgoREDivide, 3))*\AlgoREScale*pow(\AlgoRESize, 3)}, -2.25);
      
      \draw[-, connection] (node\n_\m_\l) -- (node\n_\m);
      \draw[<->, arrow]
        ($(node\n_\m_\l) + ({-0.5*\AlgoREScale*pow(\AlgoRESize, 3)}, 0)$)
        -- ++({\AlgoREScale*pow(\AlgoRESize, 3)}, 0);
    }

    \draw[-, connection] (node\n_\m) -- (node\n);
    \draw[<->, arrow]
      ($(node\n_\m) + ({-0.5*\AlgoREScale*pow(\AlgoRESize, 2)}, 0)$)
      -- ++({\AlgoREScale*pow(\AlgoRESize, 2)}, 0);
  }

  \draw[-, connection] (node\n) -- (root);
  \draw[<->, arrow]
  ($(node\n) + (-0.5*\AlgoREScale*\AlgoRESize, 0)$)
    -- ++(\AlgoREScale*\AlgoRESize, 0);
}

\draw[<->, arrow] ($(root) + (-0.5*\AlgoREScale, 0)$) -- ++(\AlgoREScale, 0);
\end{tikzpicture}
    \end{adjustbox}
    \caption{Simple recursion equation with {\color{Mittel-Blau}$a = 3, b = 2$}}
    \label{fig:recursion_equations:master_theorem_tree_3_2}
  \end{figure}
  \onslide<2->
  \textbf{Case 1:} {\color{Mittel-Blau}$a > b$}
  \begin{itemize}
    \item<3->
      Three partial problems with $\frac{1}{2}$ the size
    \item<4->
      Solving the partial problems dominates (last layer, leaves)
    \item<5->
      Runtime of {\color{Mittel-Blau}$\Theta(n^{\log_b a})$}
  \end{itemize}
\end{frame}

%-------------------------------------------------------------------------------

\begin{frame}{Recursion Equations}{Master theorem (Simple Form)}
  \begin{figure}[!h]
    \begin{adjustbox}{width=\linewidth}
      \def\AlgoREDivide{2}% 2
      \def\AlgoRESize{0.5}% 1/2
      \def\AlgoREScale{15}% 15
      \begin{tikzpicture}[
  arrow/.style={
    >=stealth,
    ultra thick,
    color=black
  }, connection/.style={
    thick,
    color=black
  }
]%
\coordinate (root) at (0, 0);

\foreach \n in {1,...,\AlgoREDivide} {
  %Second layer
  \coordinate (node\n) at
    ({((\n-0.5)-0.5*\AlgoREDivide)*\AlgoRESize*\AlgoREScale}, -0.75);

  \foreach \m in {1,...,\AlgoREDivide} {
    % Third layer
    \coordinate (node\n_\m) at
      ({((((\n-1)*\AlgoREDivide+\m)-0.5)-0.5*pow(\AlgoREDivide, 2))
      *\AlgoREScale*pow(\AlgoRESize, 2)}, -1.5);
    
    \foreach \l in {1,...,\AlgoREDivide} {
      % Fourth layer
      \coordinate (node\n_\m_\l) at
        ({((((\m-1 + (\n-1)*\AlgoREDivide)*\AlgoREDivide+\l)-0.5)
        -0.5*pow(\AlgoREDivide, 3))*\AlgoREScale*pow(\AlgoRESize, 3)}, -2.25);
      
      \draw[-, connection] (node\n_\m_\l) -- (node\n_\m);
      \draw[<->, arrow]
        ($(node\n_\m_\l) + ({-0.5*\AlgoREScale*pow(\AlgoRESize, 3)}, 0)$)
        -- ++({\AlgoREScale*pow(\AlgoRESize, 3)}, 0);
    }

    \draw[-, connection] (node\n_\m) -- (node\n);
    \draw[<->, arrow]
      ($(node\n_\m) + ({-0.5*\AlgoREScale*pow(\AlgoRESize, 2)}, 0)$)
      -- ++({\AlgoREScale*pow(\AlgoRESize, 2)}, 0);
  }

  \draw[-, connection] (node\n) -- (root);
  \draw[<->, arrow]
  ($(node\n) + (-0.5*\AlgoREScale*\AlgoRESize, 0)$)
    -- ++(\AlgoREScale*\AlgoRESize, 0);
}

\draw[<->, arrow] ($(root) + (-0.5*\AlgoREScale, 0)$) -- ++(\AlgoREScale, 0);
\end{tikzpicture}
    \end{adjustbox}
    \caption{Simple recursion equation with {\color{Mittel-Blau}$a = 2, b = 2$}}
    \label{fig:recursion_equations:master_theorem_tree_2_2}
  \end{figure}
  \onslide<2->
  \textbf{Case 2:} {\color{Mittel-Blau}$a = b$}
  \begin{itemize}
    \item<3->
      Two partial problems with $\frac{1}{2}$ the size
    \item<4->
      Each layer has equal costs, $\log n$ layers
    \item<5->
      Runtime of {\color{Mittel-Blau}$\Theta(n \, \log n)$}
  \end{itemize}
\end{frame}

%-------------------------------------------------------------------------------

%\begin{frame}{Recursion Equations}{Master theorem (Simple Form)}
%  \begin{figure}[!h]
%    \begin{adjustbox}{width=\linewidth}
%      \def\AlgoREDivide{2}% 2
%      \def\AlgoRESize{0.25}% 1/4
%      \def\AlgoREScale{15}% 15
%      \begin{tikzpicture}[
  arrow/.style={
    >=stealth,
    ultra thick,
    color=black
  }, connection/.style={
    thick,
    color=black
  }
]%
\coordinate (root) at (0, 0);

\foreach \n in {1,...,\AlgoREDivide} {
  %Second layer
  \coordinate (node\n) at
    ({((\n-0.5)-0.5*\AlgoREDivide)*\AlgoRESize*\AlgoREScale}, -0.75);

  \foreach \m in {1,...,\AlgoREDivide} {
    % Third layer
    \coordinate (node\n_\m) at
      ({((((\n-1)*\AlgoREDivide+\m)-0.5)-0.5*pow(\AlgoREDivide, 2))
      *\AlgoREScale*pow(\AlgoRESize, 2)}, -1.5);
    
    \foreach \l in {1,...,\AlgoREDivide} {
      % Fourth layer
      \coordinate (node\n_\m_\l) at
        ({((((\m-1 + (\n-1)*\AlgoREDivide)*\AlgoREDivide+\l)-0.5)
        -0.5*pow(\AlgoREDivide, 3))*\AlgoREScale*pow(\AlgoRESize, 3)}, -2.25);
      
      \draw[-, connection] (node\n_\m_\l) -- (node\n_\m);
      \draw[<->, arrow]
        ($(node\n_\m_\l) + ({-0.5*\AlgoREScale*pow(\AlgoRESize, 3)}, 0)$)
        -- ++({\AlgoREScale*pow(\AlgoRESize, 3)}, 0);
    }

    \draw[-, connection] (node\n_\m) -- (node\n);
    \draw[<->, arrow]
      ($(node\n_\m) + ({-0.5*\AlgoREScale*pow(\AlgoRESize, 2)}, 0)$)
      -- ++({\AlgoREScale*pow(\AlgoRESize, 2)}, 0);
  }

  \draw[-, connection] (node\n) -- (root);
  \draw[<->, arrow]
  ($(node\n) + (-0.5*\AlgoREScale*\AlgoRESize, 0)$)
    -- ++(\AlgoREScale*\AlgoRESize, 0);
}

\draw[<->, arrow] ($(root) + (-0.5*\AlgoREScale, 0)$) -- ++(\AlgoREScale, 0);
\end{tikzpicture}
%    \end{adjustbox}
%    \caption{Simple recursion equation with
%      {\color{Mittel-Blau}$a = 2, b = 4$}}
%    \label{fig:recursion_equations:master_theorem_tree_2_4}
%  \end{figure}
%  \textbf{Case 3:} {\color{Mittel-Blau}$a < b$}
%  \begin{itemize}
%    \item
%      Two partial problems with $\frac{1}{4}$ the size
%    \item
%      Connecting all partial solutions dominates (first layer, root)
%    \item
%      Runtime of {\color{Mittel-Blau}$\Theta(n)$}
%  \end{itemize}
%\end{frame}

%-------------------------------------------------------------------------------

\begin{frame}{Recursion Equations}{Master theorem (Simple Form)}
  \begin{figure}[!h]
    \begin{adjustbox}{width=\linewidth}
      \def\AlgoREDivide{2}% 2
      \def\AlgoRESize{0.33}% 1/3
      \def\AlgoREScale{15}% 15
      \begin{tikzpicture}[
  arrow/.style={
    >=stealth,
    ultra thick,
    color=black
  }, connection/.style={
    thick,
    color=black
  }
]%
\coordinate (root) at (0, 0);

\foreach \n in {1,...,\AlgoREDivide} {
  %Second layer
  \coordinate (node\n) at
    ({((\n-0.5)-0.5*\AlgoREDivide)*\AlgoRESize*\AlgoREScale}, -0.75);

  \foreach \m in {1,...,\AlgoREDivide} {
    % Third layer
    \coordinate (node\n_\m) at
      ({((((\n-1)*\AlgoREDivide+\m)-0.5)-0.5*pow(\AlgoREDivide, 2))
      *\AlgoREScale*pow(\AlgoRESize, 2)}, -1.5);
    
    \foreach \l in {1,...,\AlgoREDivide} {
      % Fourth layer
      \coordinate (node\n_\m_\l) at
        ({((((\m-1 + (\n-1)*\AlgoREDivide)*\AlgoREDivide+\l)-0.5)
        -0.5*pow(\AlgoREDivide, 3))*\AlgoREScale*pow(\AlgoRESize, 3)}, -2.25);
      
      \draw[-, connection] (node\n_\m_\l) -- (node\n_\m);
      \draw[<->, arrow]
        ($(node\n_\m_\l) + ({-0.5*\AlgoREScale*pow(\AlgoRESize, 3)}, 0)$)
        -- ++({\AlgoREScale*pow(\AlgoRESize, 3)}, 0);
    }

    \draw[-, connection] (node\n_\m) -- (node\n);
    \draw[<->, arrow]
      ($(node\n_\m) + ({-0.5*\AlgoREScale*pow(\AlgoRESize, 2)}, 0)$)
      -- ++({\AlgoREScale*pow(\AlgoRESize, 2)}, 0);
  }

  \draw[-, connection] (node\n) -- (root);
  \draw[<->, arrow]
  ($(node\n) + (-0.5*\AlgoREScale*\AlgoRESize, 0)$)
    -- ++(\AlgoREScale*\AlgoRESize, 0);
}

\draw[<->, arrow] ($(root) + (-0.5*\AlgoREScale, 0)$) -- ++(\AlgoREScale, 0);
\end{tikzpicture}
    \end{adjustbox}
    \caption{Simple recursion equation with {\color{Mittel-Blau}$a = 2, b = 3$}}
    \label{fig:recursion_equations:master_theorem_tree_2_3}
  \end{figure}
  \onslide<2->
  \textbf{Case 3:} {\color{Mittel-Blau}$a < b$}
  \begin{itemize}
    \item<3->
      Two partial problems with $\frac{1}{3}$ the size
    \item<4->
      Connecting all partial solutions dominates (first layer, root)
    \item<5->
      Runtime of {\color{Mittel-Blau}$\Theta(n)$}
  \end{itemize}
\end{frame}

%-------------------------------------------------------------------------------

\begin{frame}{Recursion Equations}{Master theorem (Simple Form)}
  \textbf{For a recursion equation like}
  \begin{displaymath}
    \color{Mittel-Blau}{%
      T(n) = a \cdot T\left(\frac{n}{b}\right) + c \cdot n,%
      \hspace{1.5em}a \geq 1, b > 1, c > 0%
    }%
  \end{displaymath}
  \vspace{-1.0em}
  \begin{itemize}
    \item<2->
      $\ldots$ yields to a runtime of:
  \end{itemize}
  \onslide<3->
  \begin{displaymath}
    T(n) = \begin{cases}
      \Theta(n^{\log_b a}) & \text{if } a > b\\
      \Theta(n \, \log_b n) & \text{if } a = b\\
      \Theta(n) & \text{if } a < b
    \end{cases}
  \end{displaymath}
  \begin{itemize}
    \item<4->
      Proof with \textit{geometric series}:
      Number of operations per layer grows / shrinks by constant factor
      $\tfrac{a}{b}$
  \end{itemize}
\end{frame}

%-------------------------------------------------------------------------------

\subsubsection{Master theorem (General Form)}

\begin{frame}<beamer>{\LectureToC}
  \tableofcontents[currentsection,
    currentsubsection,
    subsubsectionstyle=show/shaded/shaded/shaded]
\end{frame}

\begin{frame}{Recursion Equations}{Master theorem (General Form)}
  \textbf{Master theorem (general form):}
  \onslide<2->
  \begin{displaymath}
  \color{Mittel-Blau}{%
    T(n) = a \cdot T\left(\frac{n}{b}\right) + f(n),%
    \hspace{1.5em}a \geq 1, b > 1%
  }%
  \end{displaymath}
  \vspace{-1.0em}
  \begin{itemize}
    \item<3->
      \textbf{Case 1:}
      {\color{Mittel-Gruen}$T(n) \in \Theta(n^{\log_b a})$}
      \hfill if {\color{Mittel-Blau}
        $f(n) \in \mathcal{O}(n^{\log_b a-\varepsilon}), ~ \varepsilon > 0$
      }\\
      {\color{gray}Solving the partial problems dominates\\
        (last layer, leaves)}
      \vspace{1.5em}
    \item<4->
      \textbf{Case 2:}
      {\color{Mittel-Gruen}$T(n) \in \Theta(n^{\log_b a} \log n)$}
      \hspace{1.5em} if {\color{Mittel-Blau}
        $f(n) \in \Theta(n^{\log_b a})$
      }\\
      {\color{gray}Each layer has equal costs, $\log_b n$ layers}
  \end{itemize}
\end{frame}

%-------------------------------------------------------------------------------

\begin{frame}{Recursion Equations}{Master theorem (General Form)}
  \textbf{Master theorem (general form):}
  \begin{itemize}
    \item<2->
      \textbf{Case 3:}
      {\color{Mittel-Gruen}$T(n) \in \Theta(f(n))$}
      \hfill if {\color{Mittel-Blau}
        $f(n) \in \Omega(n^{\log_b a+\varepsilon}), ~ \varepsilon > 0$
      }\\
      {\color{gray}Connecting all partial solutions dominates\\
        (first layer, root)}\\[1.0em]
      {\color{Mittel-Blau}Regularity condition:}
      \begin{align*}
        a \cdot f\left(\frac{n}{b}\right) \leq c \cdot f(n), \hspace{1.5em} &
        0 \leq c \leq 1,\\
        {} & n > n_0
      \end{align*}
  \end{itemize}
\end{frame}

%-------------------------------------------------------------------------------

\begin{frame}{Recursion Equations}{Master theorem (General Form) - Case 1}
  \textbf{Case 1 - Example:}
  \onslide<2->
  {\color{Mittel-Gruen}$T(n) \in \Theta(n^{\log_b a})$}
  \hfill if {\color{Mittel-Blau}
    $f(n) \in O(n^{\log_b a-\varepsilon}), ~ \varepsilon > 0$
   }\\
   {\color{gray}Solving the partial problems dominates (last layer, leaves)}
   \vspace{1.0em}
   \begin{itemize}
     \item<3->
       $T(n) = 8 \cdot T(\tfrac{n}{2}) + 1000 \cdot n^2$
       \vspace{-0.5em}
       \begin{displaymath}
         a = 8, ~ b = 2, ~ f(n) = 1000 \cdot n^2, ~
         \smash{\underbrace{
           \log_b a = \log_2 8 = 3
         }_{\text{$n^3$ leaves}}}
       \end{displaymath}
       {\color{Mittel-Blau}\begin{math}
          f(n) \in \mathcal{O}(n^{3-\varepsilon})$
          $\Rightarrow$
          $T(n) \in \Theta(n^3)
       \end{math}}
      \vspace{1.5em}
    \item<4->
      $T(n) = 9 \cdot T(\tfrac{n}{3}) + 17 \cdot n$
      \vspace{-0.5em}
      \begin{displaymath}
        a = 9, ~ b = 3, ~ f(n) = 17 \cdot n, ~
        \smash{\underbrace{
          \log_b a = \log_3 9 = 2
        }_{\text{$n^2$ leaves}}}
      \end{displaymath}
      {\color{Mittel-Blau}\begin{math}
        f(n) \in \mathcal{O}(n^{2-\varepsilon})$
        $\Rightarrow$
        $T(n) \in \Theta(n^2)
      \end{math}}
   \end{itemize}
\end{frame}

%-------------------------------------------------------------------------------

\begin{frame}{Recursion Equations}{Master theorem (General Form) - Case 2}
  \textbf{Case 2:}
  \onslide<2->
  {\color{Mittel-Gruen}$T(n) \in \Theta(n^{\log_b a} \log n)$}
  \hspace{1.5em} if {\color{Mittel-Blau}
    $f(n) \in \Theta(n^{\log_b a})$
  }\\
  {\color{gray}Each layer has equal costs, $\log n$ layers}
  \vspace{1.0em}
  \begin{itemize}
    \item<3->
      $T(n) = 2 \cdot T(\tfrac{n}{2}) + 10 \cdot n$
      \vspace{-0.5em}
      \begin{displaymath}
        a = 2, ~ b = 2, ~ f(n) = 10 \cdot n, ~
        \smash{\underbrace{
          \log_b a = \log_2 2 = 1
        }_{\text{$n^1$ leaves}}}
      \end{displaymath}
      {\color{Mittel-Blau}\begin{math}
        f(n) \in \Theta(n^{\log_2 2})$
        $\Rightarrow$
        $T(n) \in \Theta(n \, \log n)
      \end{math}}
      \vspace{1.0em}
    \item<4->
      $T(n) = T(\tfrac{2\,n}{3}) + 1$
      \vspace{-0.5em}
      \begin{displaymath}
        a = 1, ~ b = \tfrac{2}{3}, ~ f(n) = 1, ~
        \smash{\underbrace{
          \log_b a = \log_{\nicefrac{3}{2}} 1 = 0
        }_{\text{$n^0$ leaves $=$ 1 leaf}}}
      \end{displaymath}
      \hfill\\[1.0em]
      {\color{Mittel-Blau}\begin{math}
        f(n) \in \Theta(n^{\log_{\nicefrac{3}{2}} 1})$
        $\Rightarrow$
        $T(n) \in \Theta(n^0 \log n) = \Theta(\log n)
      \end{math}}
  \end{itemize}
\end{frame}

%-------------------------------------------------------------------------------

\begin{frame}{Recursion Equations}{Master theorem (General Form) - Case 3}
  \textbf{Case 3:}
  \onslide<2->
  {\color{Mittel-Gruen}$T(n) \in \Theta(f(n))$}
  \hfill if {\color{Mittel-Blau}
    $f(n) \in \Omega(n^{\log_b a+\varepsilon}), ~ \varepsilon > 0$
  }\\
  {\color{gray}Connecting all partial solutions dominates (first layer, root)}
  \vspace{1.0em}
  \begin{itemize}
    \item<3->
      $T(n) = 2 \cdot T(\tfrac{n}{2}) + n^2$
      \vspace{-0.5em}
      \begin{displaymath}
        a = 2, ~ b = 2, ~ f(n) = n^2, ~
        \smash{\underbrace{
          \log_b a = \log_2 2 = 1
        }_{\text{$n^1$ leaves}}}
      \end{displaymath}
      {\color{Mittel-Blau}$f(n) \in \Omega(n^{1+\varepsilon})$}\\[1.0em]
      Check if {\color{Mittel-Blau}regularity condition} also holds:\\
      \begin{displaymath}
        2 \cdot \left(\frac{n}{2}\right)^2 \leq c \cdot n^2
        \hspace{1.5em} \Rightarrow \frac{1}{2} \cdot n^2 \leq c \cdot n^2
        \hspace{1.5em} \Rightarrow c \geq \frac{1}{2}
      \end{displaymath}
      {\color{Mittel-Blau}$\Rightarrow T(n) \in \Theta(n^2)$}
  \end{itemize}
\end{frame}

%-------------------------------------------------------------------------------

\begin{frame}{Recursion Equations}{Master theorem (General Form)}
  \textbf{Master theorem:}
  \begin{itemize}
    \item<2->
      Not always applicable:
      {\color{Mittel-Blau}$T(n) = 2 \cdot T(\frac{n}{2}) + n \, \log n$}
  \end{itemize}
  \onslide<3->
  \vspace{1.0em}
  \begin{displaymath}
    a = 2, ~ b = 2, ~ f(n) = n \, \log n, ~
    \underbrace{\log_b a = \log_2 2 = 1}_{\text{$n^1$ leaves}}
  \end{displaymath}
  \begin{itemize}
    \item<4->
      \textbf{Case 1:}
      $f(n) \notin O(n^{1-\varepsilon})$
    \item<5->
      \textbf{Case 2:}
      $f(n) \notin \Theta(n^1)$
    \item<6->
      \textbf{Case 3:}
      $f(n) \notin \Omega(n^{1+\varepsilon})$
  \end{itemize}
  \onslide<7->
  \begin{center}
    {\color{Mittel-Blau}$n \, \log n$} is \textit{asymptotically} larger than
    {\color{Mittel-Blau}$n$},\\ but not \textit{polynominal} larger
  \end{center}
\end{frame}

%-------------------------------------------------------------------------------

\begin{frame}{Recursion Equations}{Master theorem - Summary}
  \textbf{Master theorem:}
  \onslide<2->
  \begin{displaymath}
    \color{Mittel-Blau}{
      T(n) = a \cdot T(\frac{n}{b}) + f(n)
    }
  \end{displaymath}
  \vspace{-1.0em}
  \begin{itemize}
    \item<3->
      Three cases depending on the dominance of the terms
    \item<4->
      \textbf{Case 1:}
      Solving the partial problems is \textit{polynominal} bigger than merging
      all solutions\\
      {\color{Mittel-Blau}$T(n) \in \Theta(n^{\log_b a})$}, \hspace{3.5em}
      {\color{gray}$T(n) \in \Theta(\text{number of leaves})$}
      \vspace{0.5em}
   \item<5->
     \textbf{Case 2:}
     Each layer has equal costs\\
     {\color{Mittel-Blau}$T(n) \in \Theta(n^{\log_b a} \log n)$}, \hspace{1.5em}
     {\color{gray}$\log n$ layers}
     \vspace{0.5em}
   \item<6->
     \textbf{Case 3:}
     Connecting all partial solutions is \textit{polynominal} bigger than
     solving all partial porblems\\
     {\color{Mittel-Blau}$T(n) \in \Theta(f(n))$}
  \end{itemize}
\end{frame}

%%% ===================================================================
%%% This should be at the END of the file !!!!!!
%%%
%%% Local Variables: 
%%% mode: latex 
%%% TeX-master: "~/TeX/TeXinput/Scripts/Algo-Data-EMS/Rolf-2016/AlgoDat/Lecture-9/Lecture.tex" 
%%% End: 
%%% ===================================================================
