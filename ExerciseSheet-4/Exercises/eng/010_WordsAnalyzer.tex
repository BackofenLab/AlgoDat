\exercise{1} \points{12}\\
  The aim of this exercise is to compare the effective runtime of two different methods to count the number of word occurrences. 
  The first method will use a sorted list of words for counting their occurences while the second method will use a hashmap for the counting part.
  To accomplish this you will use a given function to read in words from a text file and then compute the most common word as well as the 5 most common words in this file for both methods.
  Use the template on our website (\textit{compare\_count\_methods.py}) with the predefined function \textit{read\_word\_list()} to read in the data file \textit{data.encrypted.txt}. 
  The read-in words are returned as a list and are ready to be processed by you. 
  Use the remaining functions in the template and implement the needed functionality.

\textbf{ATTENTION}:\\
    \textbf{DO NOT UPLOAD THE \textit{data.encrypted.txt} FILE INTO THE SVN REPOSITORY!}

Hints:
\begin{itemize}
  \item
    As mentioned on Exercise Sheet 2, Python uses call-by-reference when passing objects to functions. 
    Since we want each function to work on the same input list when comparing runtimes, you therefore need to make a copy of the list before passing it to a function. 
    In our case we want to sort the list, for which you can use the built-in function \textit{yourlist.sort()}. 
    After sorting the list, each distinct word will be grouped together.
    Use this to your advantage and count the size of each group to determine the occurrence count of each word.
    Compute the most common as well as the 5 most common words (see given functions for more details) and print them to the console.
  \item
    When counting words with a hashmap (we use Python's dictionary data type for this), insert a new entry with a count of 1 into the map for each new word. For each repeating occurence just increment the counter.
    After counting, you can convert the dictionary into a list of tuples (word, count) using \textit{list(yourmap.items())}. For writing tests, you can use our \textit{test.encrypted.txt} file.
\end{itemize}

