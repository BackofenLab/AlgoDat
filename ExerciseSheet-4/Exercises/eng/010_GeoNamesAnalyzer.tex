\exercise{1} \points{15}\\
Implement a \textit{geo\_names\_analyzer.py} file with the following functionalities:
\begin{enumerate}
  \item
    A function \textit{read\_info\_from\_file} to get the described information from
    the \textit{GeoNames.org} tab-separated table file \textit{allCountries.txt}
    which you can find on our website.
    All lines in the file with a $P$ in column 7 are localities (cities, villages .. ).
    You can find the locality names in column 2.
    Only consider localities with > 0 inhabitants (column 15).
    You can find the country code in column 9.
  \item
    A function \textit{compute\_names\_by\_sorting} which
    calculates the most-frequent world-wide locality names via sorting (using lists only).
  \item
    A function \textit{compute\_names\_by\_map} which does the
    same, but instead of sorting lists via the usage of an associative array (use Python's \textit{dictionary} class).
    Write some unit tests for both functions.
    Do not use the \textit{allCountries.txt} file for testing,
    but instead generate a small table yourself (see lecture) or use the table for Austria (\textit{AT.zip}) 
    from the website.
    
    \textbf{ATTENTION}: Do not upload the \textit{allCountries.txt} file into the SVN directory!
\end{enumerate}

Hints:
\begin{itemize}
\item Read in the file and process it line by line, do not store the file in a data structure
\item If you want to you can also read the file line by line without unpacking it to save space (using the \textit{zipfile} module)
\end{itemize}

