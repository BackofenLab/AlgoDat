\begin{frame}<beamer>{\LectureToC}
  \tableofcontents[currentsection, currentsubsection,
    subsubsectionstyle=show/show/shaded
  ]
\end{frame}

\subsubsection{Daphne}

\begin{frame}{Daphne}
  \textbf{Daphne:}
  \begin{itemize}
    \item
      Provides the following information:
      \begin{itemize}
        \item
          Name / contact information of your tutor
        \item
          Download of / info needed for exercise sheets
        \item
          Collected points of all exercise sheets
        \item
          Links to:
          \begin{enumerate}
            \item
              Coding standards
            \item
              Build system
            \item
              The other systems
          \end{enumerate}
      \end{itemize}
    \item
      Link: {\color{MainA}\href{\LectureDaphneLink}{Daphne}}
  \end{itemize}
\end{frame}

%-------------------------------------------------------------------------------

\subsubsection{Forum}
\begin{frame}{Forum}
  \textbf{Forum:}
  \begin{itemize}
    \item
      Please don't hesitate to ask if something is unclear
    \item
      Ask in the forum and not separate.
      Others might also be interested in the answer
    \item
      {\color{MainA}I}, {\color{MainA}Claudis Korzen} or one of the
      {\color{MainA}tutors} will reply as fast as possible
    \item
      Link: {\color{MainA}\href{\LectureForumLink}{Forum}}
  \end{itemize}
\end{frame}

%-------------------------------------------------------------------------------

\codeslide{python}{
\subsubsection{Checkstyle}
\begin{frame}{Checkstyle}{flake8}
  \textbf{Checkstyle (flake8):}
  \begin{itemize}
    \item
      Installation: \textbf{python3} \texttt{-m pip install flake8}
    \item
      Check file: \textbf{python3} \texttt{-m flake8 path/to/files/*.py}
    \item
      Link: {\color{MainA}\href{\LectureCheckstyleLink}{flake8}}
  \end{itemize}
\end{frame}
}

%-------------------------------------------------------------------------------

%\codeslide{python}{
%\subsubsection{Checkstyle}
%\begin{frame}{Checkstyle}
%  \textbf{Checkstyle (pep8):}
%  \begin{itemize}
%    \item
%      Installation: \textbf{pip} \texttt{install pep8}
%    \item
%      Update: \textbf{pip} \texttt{install --upgrade pep8}
%    \item
%      Check: \textbf{pep8} \texttt{--show-source --show-pep8 <file>}
%    \item
%      Link: {\color{MainA}\href{\LectureCheckstyleLink}{pep8}}
%  \end{itemize}
%\end{frame}
%}

%-------------------------------------------------------------------------------

\subsubsection{Unit Tests}
\begin{frame}{Unit Tests}
  \textbf{Why unit tests?}
  \begin{enumerate}
    \item
      A non-trivial method without an unit test is probably wrong
    \item<2- |handout:1>
      Simplifies debugging
    \item<3- |handout:1>
      We and you can automatic check corectness of code
  \end{enumerate}
  \vspace{1em}
  \only<4- |handout:1>{\textbf{What is a good unit test?}}
  \begin{itemize}
    \item<5- |handout:1>
      Unit test checks desired output for a given input
    \item<6- |handout:1>
      At least one \textbf{typical} input
    \item<7- |handout:1>
      At least one \textbf{critical} case\\
      {\color{Hint}E.g. double occurrence of a value in sorting}
  \end{itemize}
\end{frame}

%-------------------------------------------------------------------------------

\codeslide{python}{
\begin{frame}{Unit Tests}{doctest}
  \textbf{Testing (doctest):}
  \vspace{-1.0em}
  \begin{columns}
    \begin{column}[t]{0.5\linewidth}
      \lstinputlisting[
        language=Python,
        style={python-idle-code},
        basicstyle=\small,
        tabsize=4,
        breaklines=false,
        emph={subOne},
        emphstyle=\color{blue}
      ]{Code/DocTest.py}
    \end{column}
    \begin{column}[t]{0.5\linewidth}
      \begin{itemize}
        \item
          Tests are contained in docstrings
        \item<2- |handout:1>
          Module \texttt{doctest} runs them
        \item<3- |handout:1>
          Run check with:\\
          \textbf{python3} \textit{-m doctest path/to/files/*.py -v}
      \end{itemize}
    \end{column}
  \end{columns}
\end{frame}
}

%-------------------------------------------------------------------------------

%\subsubsection{Version management}
%\begin{frame}{Version management}{git}
%  \begin{itemize}
%    \item
%      Initialize directory: \textbf{git} \texttt{clone <URL> .}
%    \item
%      Add files/folders: \textbf{git} \texttt{add <file> --all}
%    \item
%      Create snapshot: \textbf{git} \texttt{commit -m "<Your Message>"}
%    \item
%      Upload: \textbf{git} \texttt{push origin master}
%    \item
%      Link: {\color{MainA}\href{\LectureGitLink}{Git}}
%  \end{itemize}
%\end{frame}

%-------------------------------------------------------------------------------

\subsubsection{Version management}
\begin{frame}{Version management}{Subversion}
  \textbf{Version management (subversion):}
  \begin{itemize}
    \item
      Initialize / update directory: \textbf{svn} \texttt{checkout <URL>}
    \item
      Add files / folders: \textbf{svn} \texttt{add <file> -{}-all}
    \item
      Create snapshot: \textbf{svn} \texttt{commit -m "<Your Message>"}
    \item
      Link: {\color{MainA}\href{\LectureSubversionLink}{Subversion}}
  \end{itemize}
\end{frame}

%-------------------------------------------------------------------------------

\subsubsection{Jenkins}
\begin{frame}{Jenkins}
  \textbf{Jenkins:}
  \begin{itemize}
    \item
      Provides our build system
    \item
      You can check if your uploded code runs
      \begin{itemize}
        \item
          Especially whether all \textbf{unit test} pass
        \item
          And if \textbf{checkstyle} \texttt{(flake8)} is statisfied
      \end{itemize}
    \item
      Will be shown in the {\color{MainA}first exercise}
    \item
      Link: {\color{MainA}\href{\LectureJenkinsLink}{Jenkins}}
  \end{itemize}
\end{frame}