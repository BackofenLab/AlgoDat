\subsection{Heapsort}

\begin{frame}{Heapsort - Algorithm 1 / 10}
  \textbf{Heapsort:}
  \begin{itemize}
    \item
      The principle stays the same
    \item
      Better structure for finding the smallest element quicker
  \end{itemize}
  \vspace{1em}
  \only<2- |handout:1>{\textbf{Binary heap:}}
  \begin{itemize}
    \item<2- |handout:1>
      Preferably a complete binary tree
    \item<2- |handout:1>
      \textbf{Heap property:} Each child is {\color{MainA}smaller} (larger) than the parent
      element
  \end{itemize}
\end{frame}

%-------------------------------------------------------------------------------

\begin{frame}{Heapsort - Algorithm 2 / 10}
  \textbf{Min heap:}
  \begin{itemize}
    \item<1- |handout:1>
      \textbf{Heap property:} Each child is {\color{MainA}smaller}
      (larger) than the parent element
  \item<2- |handout:1>
    A valid heap fulfills the property at each node
  \end{itemize}
  \vspace{-1em}
  \begin{columns}%
    \begin{column}[b]{0.45\textwidth}%
      \begin{figure}[!h]%
        \begin{adjustbox}{height=0.75\linewidth}%
          \begin{tikzpicture}[
  node/.style={
    color=black
  },
  path/.style={
    ->,
    line width=0.25em,
    color=Mittel-Blau
  },
  level/.style={
    sibling distance = 8.0em/#1,
    level distance = 4.0em
  }
]

\node [node] (root) {4}
child [path] {
  node [node] {8}
  child {node [node] {17}}
  child {node [node] {9}}
}
child [path] {
  node [node] {5}
  child {node [node] {11}}
  child {node [node] {7}}
};
\end{tikzpicture}
        \end{adjustbox}
        \caption{Valid min heap}
        \label{fig:minheap_valid}
      \end{figure}
    \end{column}%
    \hspace*{0.1em}%
    \begin{column}[b]{0.45\textwidth}%
      \begin{figure}[!h]%
        \begin{adjustbox}{height=0.75\linewidth}%
          \begin{tikzpicture}[
  node/.style={
    color=black
  },
  path/.style={
    ->,
    line width=0.25em,
    color=Mittel-Blau
  },
  path_error/.style={
    <-,
    line width=0.25em,
    color=red!70!gray
  },
  level/.style={
    sibling distance = 8.0em/#1,
    level distance = 4.0em
  }
]

\node [node] (root) {17}
child [path_error] {
  node [node] {15}
  child[path_error] {node [node] {8}}
  child[path] {node [node] {42}}
}
child [path] {
  node [node] {22}
  child {node [node] {44}}
  child {node [node] {23}}
};
\end{tikzpicture}
        \end{adjustbox}
        \caption{Invalid min heap}
        \label{fig:minheap_invalid}
      \end{figure}
    \end{column}
  \end{columns}
\end{frame}

%-------------------------------------------------------------------------------

\begin{frame}{Heapsort - Algorithm 3 / 10}
  \textbf{How to save the heap?}\\[0.25em]
  \begin{itemize}
    \item
      We number all nodes from top to bottom and left to right starting at
      {\color{MainA}0}
      \begin{itemize}
        \item
          The children of node {\color{MainA}$i$} are
          {\color{MainA}$2i + 1$} and {\color{MainA}$2i + 2$}
        \vspace*{0.5em}
        \item
          The parent node of node {\color{MainA}$i$} is
          {\color{MainA}$\mathrm{floor}\left(\frac{i-1}{2}\right)$}
      \end{itemize}
  \end{itemize}%
  \vspace{-2em}%
  \begin{columns}%
    \begin{column}{0.5\textwidth}
      \begin{figure}[!h]%
        \begin{adjustbox}{width=\linewidth}
          \begin{tikzpicture}[
  node/.style={
    color=black
  },
  path/.style={
    ->,
    line width=0.25em,
    color=Mittel-Blau
  },
  lleft/.style={
    color=Mittel-Gruen,
    shift={(-0.5em,-0.75em)}
  },
  lright/.style={
    color=Mittel-Gruen!80!black,
    shift={(0.5em,-0.75em)}
  },
  level/.style={
    sibling distance = 8.0em/#1,
    level distance = 4.0em
  }
]

\node [node, label={[lleft]:0}] (root) {4}
child [path] {
  node [node, label={[lleft]:1}] {8}
  child {node [node, label={[lleft]:3}] {17}}
  child {node [node, label={[lright]:4}] {9}}
}
child [path] {
  node [node, label={[lright]:2}] {5}
  child {node [node, label={[lleft]:5}] {11}}
  child {node [node, label={[lright]:6}] {7}}
};
\end{tikzpicture}%
        \end{adjustbox}
        \vspace*{-0.5em}%
        \caption{Min heap}%
        \label{fig:minheap_numbered}%
      \end{figure}%
    \end{column}
    \begin{column}{0.5\textwidth}
      \begin{table}[!h]
        \caption{Elements can be stored in array}
        \label{tab:minheap_numbered}
        \begin{tabular}{ccccccc}
          \only<2- |handout:1>{\color{MainB}0}&
          \only<3- |handout:1>{\color{MainB}1}&
          \only<4- |handout:1>{\color{MainB}2}&
          \only<5- |handout:1>{\color{MainB}3}&
          \only<6- |handout:1>{\color{MainB}4}&
          \only<7- |handout:1>{\color{MainB}5}&
          \only<8- |handout:1>{\color{MainB}6}\\
          \hline
          \multicolumn{1}{|c}{\only<2- |handout:1>{2}}&%
          \multicolumn{1}{|c}{\only<3- |handout:1>{3}}&%
          \multicolumn{1}{|c}{\only<4- |handout:1>{4}}&%
          \multicolumn{1}{|c}{\only<5- |handout:1>{11}}&%
          \multicolumn{1}{|c}{\only<6- |handout:1>{7}}&%
          \multicolumn{1}{|c}{\only<7- |handout:1>{5}}&%
          \multicolumn{1}{|c|}{\only<8- |handout:1>{8}}\\
          \hline
        \end{tabular}
      \end{table}
    \end{column}
  \end{columns}
\end{frame}

%-------------------------------------------------------------------------------

\begin{frame}{Heapsort - Algorithm 4 / 10}
  \textbf{Repairing after taking the smallest element:} \texttt{heap.pop()}
  \begin{itemize}
    \item<2- |handout:1>
      Remove the smallest element (root node)
    \item<3- |handout:1>
      Replace the root with the last node
    \item<4- |handout:1>
      {\color{MainA}Sift} the new root node down until the
      {\color{MainA}heap property} is satisfied
  \end{itemize}
  \only<5- |handout:1>{
    \begin{figure}[!h]%
      \begin{columns}%
        \begin{column}{0.3\textwidth}%
          \begin{adjustbox}{width=\linewidth}
            \begin{tikzpicture}[
  node/.style={
    color=black
  },
  path/.style={
    ->,
    line width=0.25em,
    color=MainA
  },
  path_error/.style={
    <-,
    line width=0.25em,
    color=red!70!gray
  },
  level/.style={
    sibling distance = 6.0em/#1,
    level distance = 4.0em
  }
]

\node [node, color=Mittel-Gruen] (root) {\textbf{17}}
child [path_error] {
  node [node] {8}
  child[path] {node [node] {10}}
  child[path] {node [node] {9}}
}
child [path] {
  node [node] {22}
  child {node [node] {25}}
  child {node [node] {29}}
};
\end{tikzpicture}%
          \end{adjustbox}%
        \end{column}%
        \hspace*{0.05em}%
        \begin{column}{0.3\textwidth}<7- |handout:1>%
          \begin{adjustbox}{width=\linewidth}
            \begin{adjustbox}{width=\linewidth}
  \begin{tikzpicture}[
    node/.style={
      color=black
    },
    path/.style={
      ->,
      line width=0.25em,
      color=Mittel-Blau
    },
    path_error/.style={
      <-,
      line width=0.25em,
      color=red!70!gray
    },
    level/.style={
      sibling distance = 6.0em/#1,
      level distance = 4.0em
    }
  ]
  
  \node [node] (root) {8}
  child [path] {
    node [node, color=Mittel-Gruen] {\textbf{17}}
    child[path_error] {node [node] {10}}
    child[path_error] {node [node] {9}}
  }
  child [path] {
    node [node] {22}
    child {node [node] {25}}
    child {node [node] {29}}
  };
  \end{tikzpicture}
\end{adjustbox}%
          \end{adjustbox}
        \end{column}%
        \hspace*{0.05em}%
        \begin{column}{0.3\textwidth}<9- |handout:1>%
          \begin{adjustbox}{width=\linewidth}
            \begin{tikzpicture}[
  node/.style={
    color=black
  },
  path/.style={
    ->,
    line width=0.25em,
    color=Mittel-Blau
  },
  path_error/.style={
    <-,
    line width=0.25em,
    color=\currentred%red!70!gray
  },
  level/.style={
    sibling distance = 6.0em/#1,
    level distance = 4.0em
  }
]

\node [node] (root) {8}
child [path] {
  node [node] {9}
  child[path] {node [node, color=Mittel-Gruen] {\textbf{17}}}
  child[path] {node [node] {10}}
}
child [path] {
  node [node] {22}
  child {node [node] {25}}
  child {node [node] {29}}
};
\end{tikzpicture}%
          \end{adjustbox}%
        \end{column}%
      \end{columns}%
      \caption{Repairing a min heap via sifting}%
      \label{fig:minheap_repair}%
    \end{figure}
  }
\end{frame}

%-------------------------------------------------------------------------------

\begin{frame}{Heapsort - Algorithm 5 / 10}
  \textbf{Heapsort:}
  \begin{itemize}
    \item
      Organize the {\color{MainA}$n$} elements as heap
    \item
      While the heap still contains elements
      \begin{itemize}
        \item
          Take the smallest element
        \item
          Move the last node to the root
        \item
          Repair the heap as described
      \end{itemize}
    \item<2- |handout:1>
      Output: {\color{MainB}2}%
      \only<9- |handout:1>{, {\color{MainB}3}, {\color{MainB}\ldots}}
  \end{itemize}
  \vspace*{-0.5em}
  \only<2- |handout:1>{
    \begin{center}
      \begin{figure}[!h]%
        \begin{columns}%
          \begin{column}{0.33\textwidth}%
            \begin{centering}
              \begin{adjustbox}{height=8em}
                \begin{tikzpicture}[
  node/.style={
    color=black
  },
  path/.style={
    ->,
    line width=0.25em,
    color=Mittel-Blau
  },
  path_error/.style={
    <-,
    line width=0.25em,
    color=red!70!gray
  },
  level/.style={
    sibling distance = 7.0em/#1,
    level distance = 4.0em
  }
]%
\node [node, color=Mittel-Gruen] (root) {\textbf{4}}
child [path] {
  node [node] {8}
  child[path] {node [node] {17}}
  child[path] {node [node] {9}}
}
child [path] {
  node [node] {5}
  child {node [node] {11}}
  child {node [node] (last) {\textbf{7}}}
};
\draw[path, color={Mittel-Gruen}, ->] (root) to ($(root) + (0, 1.0)$);
\onslide<2->\draw[path, color=black, ->] (last) to [out=45, in=0, looseness=1.0] (root);
\end{tikzpicture}%
              \end{adjustbox}%
            \end{centering}
          \end{column}%
          \begin{column}{0.33\textwidth}<4- |handout:1>%
            \begin{centering}
              \begin{adjustbox}{height=8em}
                \only<2-4 |handout:0>{%
  \colorlet{AlgoHeapErrPath}{MainA}%
}%
\only<5- |handout:1>{%
  \colorlet{AlgoHeapErrPath}{SecA}%
}%
\begin{tikzpicture}[
  node/.style={
    color=black
  },
  path/.style={
    ->,
    line width=0.25em,
    color=MainA
  },
  path_error/.style={
    <-,
    line width=0.25em,
    color=AlgoHeapErrPath
  },
  level/.style={
    sibling distance = 7.0em/#1,
    level distance = 4.0em
  }
]%
\node [node] (root) {\textbf{7}}
child [path] {
  node [node] {8}
  child[path] {node [node] {17}}
  child[path] {node [node] {9}}
}
child [path_error] {
  node [node] {5}
  child [path] {node [node] {11}}
};
\draw[path, color=white, ->] (root) to ($(root) + (0, 1.0)$);
\end{tikzpicture}%
              \end{adjustbox}%
            \end{centering}
          \end{column}%
          \begin{column}{0.33\textwidth}<8- |handout:1>%
            \begin{centering}
              \begin{adjustbox}{height=8em}
                \begin{tikzpicture}[
  node/.style={
    color=black
  },
  path/.style={
    ->,
    line width=0.25em,
    color=Mittel-Blau
  },
  path_error/.style={
    <-,
    line width=0.25em,
    color=red!70!gray
  },
  level/.style={
    sibling distance = 7.0em/#1,
    level distance = 4.0em
  }
]%
\node [node, color=Mittel-Gruen] (root) {\textbf{5}}
child [path] {
  node [node] {8}
  child[path] {node [node] {17}}
  child[path] {node [node] {9}}
}
child [path] {
  node [node] {7}
  child [path] {node [node] (last) {\textbf{11}}}
};
\draw[path, color={Mittel-Gruen}, ->] (root) to ($(root) + (0, 1.0)$);
\onslide<6->\draw[path, color=black, ->] (last) to [out=45, in=0, looseness=1.25] (root);
\end{tikzpicture}%
              \end{adjustbox}%
            \end{centering}
          \end{column}%
        \end{columns}%
        \caption{One iteration of Heapsort}%
        \label{fig:heapsort_repair}%
      \end{figure}
    \end{center}
  }
\end{frame}

%-------------------------------------------------------------------------------

\begin{frame}{Heapsort - Algorithm 6 / 10}
  \textbf{Creating a heap:}
  \begin{itemize}
    \item
      This operation is called {\color{MainA}heapify}
    \item<2- |handout:1>
      The {\color{MainA}$n$} elements are already stored in an array
    \item<3- |handout:1>
      Interpret the array as binary heap where the {\color{MainA}heap property} is not yet satisfied
    \item<4- |handout:1>
      We repair the heap from bottom up (in layers) with {\color{MainA}sifting}
  \end{itemize}
\end{frame}

%-------------------------------------------------------------------------------

\begin{frame}{Heapsort - Algorithm 7 / 10}
  \vspace{-1.0em}
  \begin{table}[!h]%
    \caption{Input in array}%
    \label{tab:heapify_numbers}%
    \begin{tabular}{ccccccc}
      {\color{MainB}0}&
      {\color{MainB}1}&
      {\color{MainB}2}&
      {\color{MainB}3}&
      {\color{MainB}4}&
      {\color{MainB}5}&
      {\color{MainB}6}\\
      \hline
      \multicolumn{1}{|c}{11}&%
      \multicolumn{1}{|c}{7}&%
      \multicolumn{1}{|c}{8}&%
      \multicolumn{1}{|c}{3}&%
      \multicolumn{1}{|c}{2}&%
      \multicolumn{1}{|c}{5}&%
      \multicolumn{1}{|c|}{4}\\
      \hline
    \end{tabular}
  \end{table}
  \vspace*{-0.5em}
  \begin{centering}
    \begin{figure}[!h]%
      \begin{columns}%
        \begin{column}{0.425\textwidth}%
          \begin{adjustbox}{width=\linewidth}%
            \pgfdeclarelayer{background}
\pgfsetlayers{background,main}
\begin{tikzpicture}[
  node/.style={
    color=black
  },
  path/.style={
    ->,
    line width=0.25em,
    color=Mittel-Blau
  },
  path_error/.style={
    <-,
    line width=0.25em,
    color=red!70!gray
  },
  lleft/.style={
    color=Mittel-Gruen,
    shift={(-0.5em,-0.5em)}
  },
  lright/.style={
    color=Mittel-Gruen!80!black,
    shift={(0.5em,-0.5em)}
  },
  level/.style={
    sibling distance = 7.0em/#1,
    level distance = 4.0em
  }
]%
%\begin{pgfonlayer}{main}
  \node [node, label={[lleft]above:0}] {11}
  child [path] {
    node [node, label={[lleft]above:1}] (left) {\textbf{7}}
    child[path_error] {node [node, label={[lleft]above:3}] {3}}
    child[path_error] {node [node, label={[lright]above:4}] {2}}
  }
  child [path] {
    node [node, label={[lright]above:2}] (right) {\textbf{8}}
    child[path_error] {node [node, label={[lleft]above:5}] {5}}
    child[path_error] {node [node, label={[lright]above:6}] {4}}
  };
%\end{pgfonlayer}

% \begin{pgfonlayer}{background}
%   \fill [color={Hell-Blau!40}] ($(left) - (1.0, 1.80) - (0.25, 0.025)$)
%     rectangle ($(right) + (1.0, 0.2) + (0.25, 0.025)$);
% \end{pgfonlayer}
\end{tikzpicture}%
          \end{adjustbox}%
        \end{column}%
        \begin{column}{0.425\textwidth}<2- |handout:1>%
          \begin{adjustbox}{width=\linewidth}%
              \pgfdeclarelayer{background}
\pgfsetlayers{background,main}
\begin{tikzpicture}[
  node/.style={
    color=black
  },
  path/.style={
    ->,
    line width=0.25em,
    color=Mittel-Blau
  },
  path_error/.style={
    <-,
    line width=0.25em,
    color=red!70!gray
  },
  lleft/.style={
    color=Mittel-Gruen,
    shift={(-0.5em,-0.5em)}
  },
  lright/.style={
    color=Mittel-Gruen!80!black,
    shift={(0.5em,-0.5em)}
  },
  level/.style={
    sibling distance = 7.0em/#1,
    level distance = 4.0em
  }
]%
\begin{pgfonlayer}{main}
\node [node, label={[lleft]:\vphantom{0}}] {11}
child [path] {
  node [node] (left) {2}
  child[path] {node [node] {3}}
  child[path] {node [node] {\textbf{7}}}
}
child [path] {
  node [node] (right) {4}
  child[path] {node [node] {5}}
  child[path] {node [node] {\textbf{8}}}
};
\end{pgfonlayer}

\begin{pgfonlayer}{background}
  \fill [color={Hell-Blau!40}] ($(left) - (1.0, 1.80) - (0.25, 0.025)$)
    rectangle ($(right) + (1.0, 0.2) + (0.25, 0.025)$);
\end{pgfonlayer}
\end{tikzpicture}%
          \end{adjustbox}%
        \end{column}%
      \end{columns}%
      \caption{Heapify lower layer}%
      \label{fig:heapify_lower}%
    \end{figure}
  \end{centering}
\end{frame}

%-------------------------------------------------------------------------------

\begin{frame}{Heapsort - Algorithm 8 / 10}
  \begin{centering}
   \begin{figure}[!h]%
     \begin{columns}%
       \begin{column}{0.425\textwidth}%
         \begin{adjustbox}{width=\linewidth}%
           \pgfdeclarelayer{background}
\pgfsetlayers{background,main}
\begin{tikzpicture}[
  node/.style={
    color=black
  },
  path/.style={
    ->,
    line width=0.25em,
    color=MainA
  },
  path_error/.style={
    <-,
    line width=0.25em,
    color=SecA
  },
  level/.style={
    sibling distance = 7.0em/#1,
    level distance = 4.0em
  }
]%
\begin{pgfonlayer}{main}
  \node [node] {\textbf{11}}
  child [path_error] {
    node [node] (left) {2}
    child[path] {node [node] {3}}
    child[path] {node [node] {7}}
  }
  child [path_error] {
    node [node] (right) {4}
    child[path] {node [node] {5}}
    child[path] {node [node] {8}}
  };
\end{pgfonlayer}

\begin{pgfonlayer}{background}
  \fill [color={MainALight!40}] ($(left) - (1.0, 0.2) - (0.25, 0.025)$)
    rectangle ($(right) + (1.0, 1.80) + (0.25, 0.025)$);
\end{pgfonlayer}
\end{tikzpicture}%
         \end{adjustbox}%
       \end{column}%
       \begin{column}{0.425\textwidth}<2- |handout:1>%
          \begin{adjustbox}{width=\linewidth}%
            \begin{adjustbox}{width=\textwidth}
  
  \pgfdeclarelayer{background}
  \pgfsetlayers{background,main}
  \begin{tikzpicture}[
    node/.style={
      color=black
    },
    path/.style={
      ->,
      line width=0.25em,
      color=Mittel-Blau
    },
    path_error/.style={
      <-,
      line width=0.25em,
      color=red!70!gray
    },
    level/.style={
      sibling distance = 7.0em/#1,
      level distance = 4.0em
    }
  ]
  
  \begin{pgfonlayer}{main}
  \node [node] {2}
  child [path] {
    node [node] (left) {\textbf{11}}
    child[path_error] {node [node] {3}}
    child[path_error] {node [node] {7}}
  }
  child [path] {
    node [node] (right) {4}
    child[path] {node [node] {5}}
    child[path] {node [node] {8}}
  };
  \end{pgfonlayer}
  
  \begin{pgfonlayer}{background}
    \fill [color={Hell-Blau!40}] ($(left) - (1.0, 0.2) - (0.25, 0.025)$)
      rectangle ($(right) + (1.0, 1.80) + (0.25, 0.025)$);
  \end{pgfonlayer}
  \end{tikzpicture}
\end{adjustbox}%
          \end{adjustbox}%
        \end{column}%
      \end{columns}%
      \caption{Heapify upper layer}%
      \label{fig:heapify_upper}%
    \end{figure}
  \end{centering}
\end{frame}

%-------------------------------------------------------------------------------

\begin{frame}{Heapsort - Algorithm 9 / 10}
  \begin{centering}
    \begin{figure}[!h]
      \begin{adjustbox}{width=0.425\linewidth}%
        \begin{tikzpicture}[
  node/.style={
    color=black
  },
  path/.style={
    ->,
    line width=0.25em,
    color=Mittel-Blau
  },
  level/.style={
    sibling distance = 7.0em/#1,
    level distance = 4.0em
  }
]%
\node [node] {2}
child [path] {
  node [node] (left) {3}
  child[path] {node [node] {\textbf{11}}}
  child[path] {node [node] {7}}
}
child [path] {
  node [node] (right) {4}
  child[path] {node [node] {5}}
  child[path] {node [node] {8}}
};
\end{tikzpicture}%
      \end{adjustbox}%
      \caption{Resulting heap}%
      \label{fig:heapify_upper_final}%
    \end{figure}
  \end{centering}
\end{frame}

%-------------------------------------------------------------------------------

\begin{frame}{Heapsort - Algorithm 10 / 10}
  \textbf{Finding the minimum is intuitive:}
  \begin{itemize}
    \item
      \textbf{Minsort:} Iterate through all non-sorted elements
    \item
      \textbf{Heapsort:} Finding the minimum is trivial (concept)
      \begin{center}
        \textit{Just take the root of the heap}
      \end{center}
  \end{itemize}
  \vspace*{1.5em}
  \only<2- |handout:1>{
    \textbf{Removing the minimum in Heapsort:}
    \begin{itemize}
      \item
        Repair the heap and restore the {\color{MainA}heap property}
        \begin{itemize}
          \item
            We don't have to repair the whole heap
        \end{itemize}
      \item
        More of this in the next lecture
    \end{itemize}
  }
\end{frame}
