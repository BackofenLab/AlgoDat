\subsection{L'H\^{o}pital / l'Hospital}

\begin{frame}{Limits with L'H\^{o}pital}
  \begin{itemize}
    \item
      \textbf{Intuitive:}\\
      \begin{displaymath}
        \lim\limits_{n \rightarrow \infty} 2 + \dfrac{1}{n}
        = 2  + \dfrac{1}{\infty} = 2
      \end{displaymath}
      \vspace{0em}\\
    \item<2- |handout:1>
      \textbf{With L'H\^{o}pital:}
      \begin{itemize}
        \item
          Let $f, \, g : \mathbb{N} \rightarrow \mathbb{R}$
        \item
          If
          \begin{math}
            \lim\limits_{n \to \infty} f(n)
              = \lim\limits_{n \to \infty} g(n)
              = \infty / 0
          \end{math}
      \end{itemize}
      \begin{displaymath}
        \Rightarrow \lim\limits_{n \rightarrow \infty} \dfrac{f(n)}{g(n)}
          = \lim\limits_{n \rightarrow \infty} \dfrac{f'(n)}{g'(n)}
      \end{displaymath}
    \item<3- |handout:1>
      \textbf{Holy inspiration}
      \begin{center}
        you need a doctoral degree for that
      \end{center}
  \end{itemize}
\end{frame}

%-------------------------------------------------------------------------------

\begin{frame}{Limits with L'H\^{o}pital}
  \textbf{The Limit can not be determined in the way of an Engineer:}
  \begin{displaymath}
    \lim_{n \to \infty} \dfrac{\ln (n)}{n}
      = \dfrac{\lim_{n \to \infty}\; \ln (n)}{\lim\limits_{n \to \infty}\; n}
    \hspace{1em} \stackrel{\text{plugging in}}{\longrightarrow} \hspace{1em}
      \dfrac{\infty}{\infty}
  \end{displaymath}
  \textbf{Determine the limit with using L'H\^{o}pital:}
  \begin{displaymath}
    \lim\limits_{n \rightarrow \infty} \dfrac{f(n)}{g(n)}
      = \lim\limits_{n \rightarrow \infty} \dfrac{f'(n)}{g'(n)} = 0
  \end{displaymath}
\end{frame}

%-------------------------------------------------------------------------------

\begin{frame}{Limits with L'H\^{o}pital}
  \begin{block}{\textbf{Using L'H\^{o}pital:}}
    Numerator: \; $\textbf{f(n)}\!: n \mapsto \ln (n)$\\
    Denominator: $\textbf{g(n)}\!: n \mapsto n$\\
    \hspace{1.5em}
    $\Rightarrow f'(n) = \dfrac{1}{n}$ \; (derivation from Numerator)\\
    \hspace{1.5em}
    $\Rightarrow g'(n) = 1$\; (derivation from Denominator)\\
    \begin{displaymath}
      \lim\limits_{n \rightarrow \infty} \dfrac{f'(n)}{g'(n)}
        = \lim\limits_{n \rightarrow \infty} \dfrac{1}{n} = 0
      \hspace{0.5em} \Rightarrow \hspace{0.5em}
      \lim\limits_{n \rightarrow \infty} \dfrac{f(n)}{g(n)}
        = \lim\limits_{n \rightarrow \infty} \dfrac{\ln (n)}{n} = 0
    \end{displaymath}
  \end{block}
\end{frame}

%-------------------------------------------------------------------------------

\begin{frame}{Limits with L'H\^{o}pital}
  \textbf{What can we take for granted without proofing?}
  \begin{itemize}
    \item
      Only things that are trivial
    \item
      It is always better to proof it
  \end{itemize}
  \textbf{Examples:}
  \begin{eqnarray*}
    \lim\limits_{n \rightarrow \infty} \dfrac{1}{n} &= \; 0
      &\hspace{2em} \text{is trivial}\\
    \lim\limits_{n \rightarrow \infty} \dfrac{1}{n^2} &= \; 0
      &\hspace{2em} \text{is trivial}\\
    \lim\limits_{n \rightarrow \infty} \dfrac{\log (n)}{n} &= \; 0
      &\hspace{2em} \text{use L'Hopital}
  \end{eqnarray*}
\end{frame}

%%% ==========================================
%%% This should be at the END of the file !!!!!!
%%%
%%% Local Variables: 
%%% mode: latex 
%%% TeX-master: "~/TeX/TeXinput/Scripts/Algo-Data-EMS/Rolf-2016/AlgoDat/Lecture-3/Lecture.tex" 
%%% End: 
%%% ==========================================
