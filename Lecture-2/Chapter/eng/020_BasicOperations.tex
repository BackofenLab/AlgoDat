\section{Basic Operations}

\begin{frame}{Basic Operations}
  \textbf{Uncomplete list of basic operations:}
  \begin{center}
    \begin{itemize}
      \item
        Arithmetic operation, for example: \textit{a + b}
      \item
        Allocation of variables, for example: \textit{x = y}
      \item
        Function call, for example: \textit{Sorter.minSort(array)}
    \end{itemize}
  \end{center}
\end{frame}

%-------------------------------------------------------------------------------

\begin{frame}{Basic Operations}
  \begin{tabularx}{\textwidth}{@{}XcXcX@{}}
    \cellcolor{MainA} {\color{white}\textbf{Intuitive:}} &
    {} &
    \cellcolor{MainA} {\color{white}\textbf{Better:}} &
    {} &
    \cellcolor{MainA} {\color{white}\textbf{Best:}}\\[0.5em]
    \rule{0pt}{1.25em}\cellcolor{MainALight}lines of code &
    {} &
    \cellcolor{MainALight}lines of machine code &
    {} &
    \cellcolor{MainALight}process cycles
  \end{tabularx}\\[1.5em]
  \begin{alertblock}{\textbf{Important:}}
    The actual runtime has to be roughly proportional
    to the number of operations.
  \end{alertblock}
\end{frame}