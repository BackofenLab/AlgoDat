\toclesssection{Runtime analysis}

\subsection{Minsort}

%-------------------------------------------------------------------------------

\begin{frame}{Runtime analysis - Minsort 1/10}
 How long does the program run?
 \begin{itemize}
    \item
      In the last lecture we had a schematic.
    \item
      \textbf{Observation:} It is going to be \enquote{disproportional} slower
      the more numbers are being sorted.
    \item
      How can we say more precisely what is happening?
  \end{itemize}
\end{frame}

%-------------------------------------------------------------------------------

\begin{frame}{Runtime analysis - Minsort 2/10}
  How can we analyse the runtime?
  \begin{itemize}
    \item Ideally we have a formula which provides the runtime of the
      program for an specific input
    \item
      \textbf{Problem:} The runtime is also depending on many other
        influences, especially:
      \begin{itemize}
        \item
          Which kind of computer is the code executed on.
        \item What is running in the background.
        \item Which compiler is used to compile the code.
      \end{itemize}
    \item
      \textbf{Abstraction 1:} Analyse the number of basic operations,
        rather than analysing the runtime.
  \end{itemize}
\end{frame}

%-------------------------------------------------------------------------------

\begin{frame}{Runtime analysis - Minsort 3/10}
  How many operations does \textit{Minsort} need?
  \begin{itemize}
    \item
      \textbf{Abstraction 2:} We calculate the upper (lower) bound,
      rather than counting the operations exactly.

      \textbf{Reason}: Runtime is unknown, but we know:
      \begin{itemize}
        \item {\color{Hell-Gruen}upper bound}
        \item {\color{Hell-Gruen}lower bound}
      \end{itemize}
    \item
      \textbf{Observation:} The number of operations needed is only
      depending on the size {\color{Mittel-Blau}$n$} of the array and not on the content!
  \end{itemize}
\end{frame}

%-------------------------------------------------------------------------------

\begin{frame}{Runtime analysis - Minsort 4/10}
  \textbf{We declare:}
  \begin{itemize}
    \item Runtime of opertations: $T(n)$
    \item Number of Elements: $n$
    \item Constants: $C_1$ ({\color{Hell-Gruen}lower bound}),
      $C_2$ ({\color{Hell-Gruen}upper bound})
    \begin{center}
      $C_{1} \cdot n^2
      \leq T(n)
      \leq C_{2} \cdot n^2$
    \end{center}
    \item Number of operations in round $i$: $T_i$
  \end{itemize}
  \begin{figure}[!h]
    \begin{adjustbox}{width=0.5\linewidth}
      \def\AlgoMinsort#1#2#3{
  \draw[fill=#3] (#1 + 0.1, 0.0) rectangle (#1 + 0.9, #2/2);
  \draw (#1 + 0.5, -0.5) node {\huge #2};
}
\begin{tikzpicture}
%Generate MinSort pattern
\foreach[count=\x] \h/\c in {
  1/MainBLight,%
  2/MainBLight,%
  3/MainBLight,%
  12/MainALight,%
  7/MainALight,%
  4/MainA,%
  6/MainALight,%
  10/MainALight,%
  8/MainALight,%
  15/MainALight,%
  14/MainALight,%
  5/MainALight,%
  11/MainALight,%
  9/MainALight,%
  13/MainALight%
} {
  \AlgoMinsort{\x}{\h}{\c}
}
\end{tikzpicture}%
    \end{adjustbox}%
    \caption{\textit{Minsort} at the fourth iteration}%
    \label{fig:minsort_def}%
  \end{figure}
\end{frame}

%-------------------------------------------------------------------------------

\begin{frame}{Runtime analysis - Minsort 5/10}
  \begin{columns}
    \begin{column}{0.55\textwidth}
      \begin{figure}[!h]%
        \begin{adjustbox}{width=\linewidth}%
          \def\MinSortDrawNumbers{0}
\begin{tikzpicture}
%Generate MinSort pattern
\foreach[count=\x] \h/\c in {
  1/MainBLight,%
  2/MainBLight,%
  3/MainBLight,%
  12/MainALight,%
  7/MainALight,%
  4/MainA,%
  6/MainALight,%
  10/MainALight,%
  8/MainALight,%
  15/MainALight,%
  14/MainALight,%
  5/MainALight,%
  11/MainALight,%
  9/MainALight,%
  13/MainALight%
} {
  \draw[fill=\c] (\x + 0.1, 0.0) rectangle (\x + 0.9, \h/2);
  \ifnum \MinSortDrawNumbers>0
    \draw (\x + 0.5, -0.5) node {\huge \h};
  \fi
}

%Draw brace
\draw (4.0, 7.5)
  [
    line width=0.25em,
    color=black,
    decoration={
        brace,
        raise=0.75em,
        amplitude=15pt
    },
    decorate
  ] --
  node [
    color=black,
    pos=0.5,
    yshift=4em,
    font=\Huge
  ] {$n-3$ elements left}
  (16.0, 7.5);
\end{tikzpicture}%
        \end{adjustbox}%
        \caption{\textit{Minsort} at the fourth iteration}%
        \label{fig:minsort_brace}%
      \end{figure}
    \end{column}
    \begin{column}{0.40\textwidth}
      Compares at each iteration:
      \begin{align*}
        T_1  \leq &~ n\\
        T_2  \leq &~ (n-1)\\
        {}  \vdots~ &~ {} \\
        T_{n-1}  \leq &~ 2\\
        T_n  \leq &~ 1
      \end{align*}
    \end{column}
  \end{columns}
  \[
    T(n)
      = C \cdot \left(T_1 + \cdots + T_n\right)
      \leq C \cdot \sum \limits^n_{i=1} i
  \]
\end{frame}

%-------------------------------------------------------------------------------

\codeslide{python}{
\begin{frame}{Runtime analysis - Minsort 6/10}
  \textbf{Alternative: Analyse the Code:}
  \lstinputlisting[
    language=Python,
    style={python-idle-code},
    basicstyle=\small,
    tabsize=4,
    emph={minsort},
    emphstyle=\color{blue}
  ]{Code/MinSort.py}
\end{frame}
}
%TODO: MinSort Java, C++

%-------------------------------------------------------------------------------

\begin{frame}{Runtime analysis - Minsort 7/10}
  \begin{center}
    \begin{math}
      \begin{array}{l}
        \displaystyle
        \max T(n) = \underbrace{
          \sum^{n - 2}_{i = 0}
          \left(
            \hspace*{1.5em}
            \overbrace{
              C_\text{s}
              \vphantom{
                \sum^{n - 1}_{j = i + 1}
              }
            }^\text{swap}
            \hspace*{1.5em} +
            \underbrace{
              \overbrace{
                \hspace*{0.5em}
                \sum^{n - 1}_{j = i + 1} C_\text{c}
                \hspace*{0.5em}
              }^\text{compare}
            }_{
            \lstinline[
              language=Python,
              style={python-idle-code},
              basicstyle=\small
              ]|range(i+1, len(elements))|
            }
            \hspace*{0.5em}
          \right)
        }_{
        \lstinline[
          language=Python,
          style={python-idle-code},
          basicstyle=\small
        ]|range(0, len(elements)-1)|
      }\\[5em]
      \displaystyle\hspace*{2.5em}
        = \sum^{n - 2}_{i = 0} \left(C_\text{s} + (n - i) C_\text{c}\right)\\
      \displaystyle\hspace*{2.5em}
        = \color{Mittel-Blau}
        \sum^{n - 1}_{i = 1} \left(C_\text{s} + (n - i + 1) C_\text{c}\right)
      \end{array}
    \end{math}
  \end{center}
\end{frame}

%-------------------------------------------------------------------------------

\begin{frame}{Runtime analysis - Minsort 7/10}
  \begin{center}
    \begin{math}
      T(n) \leq
      \left\lbrace
      \begin{array}{l}
      \displaystyle\hspace*{1.5em}\color{Mittel-Blau}
        \sum^{n - 1}_{i = 1} \left(C_\text{s} + (n - i + 1) C_\text{c}\right)\\
      \displaystyle\hspace*{1.5em}
        = \sum^{n - 1}_{i = 1} \left(C_\text{s} + C_\text{c}\right)
        + \sum^{n - 1}_{i = 1} \left(n - i\right) C_\text{c}\\
      \displaystyle\hspace{1.5em}
        = \left(n - 1\right) \cdot \left(C_\text{s} + C_\text{c}\right)
        + C_\text{c} \sum_{i = 1}^{n - 1} i\\
     % \displaystyle\hspace*{1.5em}
     %   \leq n \left(C_\text{s} + C_\text{c}\right) + C_\text{c} \sum^n_{i=1} i
      \end{array}
      \right.
    \end{math}
  \end{center}
\end{frame}

%-------------------------------------------------------------------------------

\begin{frame}{Runtime analysis - Minsort 8/10}
  \begin{block}{Proof of the statement ({\color{Mittel-Gruen}upper bound}):}
    We assume: Swap, Compare $\in \mathcal{O}(1)$\\
    $\hspace{1.5em}\Rightarrow C_\text{s} = C_\text{c} = 1$
    \vspace*{0.5em}
    \begin{center}
      $\displaystyle
      T(n)
      \;\leq\;
        2 n \, + \, \underbrace{
          \sum \limits^{n - 1}_{i = 1} i
          \,=\, \frac{n^2 - n}{2}
        }_\text{Small Gauss sum}
        \, + \,2  n
      \;\leq\; \frac{n^2 + 3 n}{2}
      \;\leq\; \frac{n^2 + n^2}{2}
      \;\leq\; \frac{2 n^2}{2}
      \,=\, n^2
      $
    \end{center}
    $\Rightarrow T(n) \leq C_2 \cdot n^2, \, C_2 = 1$
  \end{block}
\end{frame}

%-------------------------------------------------------------------------------

\begin{frame}{Runtime analysis - Minsort 9/10}
  \begin{block}{Proof of $C_1 \cdot n^2 \leq T(n)$
      ({\color{Mittel-Gruen}lower bound}):}
    We assume: Compare $\in \mathcal{O}(1)$\\
    $\hspace{1.5em}\Rightarrow C_\text{c} = 1, \; C_\text{s} = 0$\\
    \vspace*{0.5em}
    Analogous for the {\color{Mittel-Gruen}lower bound}
    (no value is swapped), there exists a $C_1$, for that
    applys:\\[0.5em]
    \begin{centering}
      \begin{math}
        \begin{array}{@{}rcl@{}}
          T(n) & \geq &
          \displaystyle
          n - 1 + \sum^{n - 1}_{i=1} i
          = \frac{n^2 - n}{2} + n - 1
          = \frac{n^2 + n}{2} - 1
        \end{array}
      \end{math}
    \end{centering}
  \end{block}
\end{frame}

%-------------------------------------------------------------------------------

\begin{frame}{Runtime analysis - Minsort 10/10}
  \begin{block}{Proof of $C_1 \cdot n^2 \leq T(n)$:}
    \vspace*{0.5em}
    \begin{center}
      $\displaystyle
        T(n) \; \geq \;
          = \frac{n^2 + n}{2} - 1
      $\\
    \end{center}
    $T(n)$ is always bigger than / equal to $\frac{1}{2} \cdot n^2$
    for $n \geq 2$:\\[1.0em]
    $\hspace*{1.5em}\Rightarrow T(n) \geq C_1 \cdot n^2$
  \end{block}
\end{frame}