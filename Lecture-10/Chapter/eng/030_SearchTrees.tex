\section{Binary Search Trees}

\begin{frame}{Binary Search Trees}{Introduction}
  \textbf{Runtime of a search tree:}
  \begin{itemize}
    \item<2->
      {\color{Mittel-Blau}\texttt{next}} and
      {\color{Mittel-Blau}\texttt{previous}} in $O(1)$\\[0.5em]
      \onslide<3->{
        Pointers corresponding to linked list
      }
      \vspace{1em}
  \item<4->
    {\color{Mittel-Blau}\texttt{insert}} and
    {\color{Mittel-Blau}\texttt{remove}} in $O(log\,n)$
    \vspace{1em}
  \item<5->
    {\color{Mittel-Blau}\texttt{lookup}} in $O(log\,n)$\\[0.5em]
    \onslide<6->{
      The structure helps searching efficiently
    }
  \end{itemize}
\end{frame}

%-------------------------------------------------------------------------------

\begin{frame}{Binary Search Trees}{Introduction}
  \textbf{Idea:}
  \begin{itemize}
    \item<2->
      We define a total order for the search tree
    \item<3->
      All nodes of the left subtree have {\color{Mittel-Blau}smaller keys}
      than the current node
    \item<4->
      All nodes of the right subtree have {\color{Mittel-Blau}bigger keys}
      than the current node
  \end{itemize}
\end{frame}

%-------------------------------------------------------------------------------

\begin{frame}{Binary Search Trees}{Introduction}
  \begin{itemize}
    \item
      Edge direction indicates ordering
  \end{itemize}
  \begin{figure}
    \begin{adjustbox}{height=0.35\linewidth}
\begin{tikzpicture}[
  node/.style={
    circle,
    draw=black,
    color=black,
    line width=0.1em,
    minimum size=2.0em,
    inner sep=0em
  }, path/.style={
    line width=0.25em,
    color=Mittel-Blau
  }, lleft/.style={
    color=Mittel-Gruen,
    shift={(-0.5em,-0.75em)}
  }, lright/.style={
    color=Mittel-Gruen!80!black,
    shift={(0.5em,-0.75em)}
  }, level/.style={
    sibling distance = 16.0em/#1,
    level distance = 4.0em
  }, level 3/.style={
    sibling distance = 4.0em,
  }]
\node (0, 0) [node] (root) {8}
child [<-, path] {
  node [node] {4}
  child [<-, path] {
    node [node] {2}
    child [<-, path] {node [node] {1}}
    child [->, path] {node [node] {3}}
  }
  child [->, path] {
    node [node] {6}
    child [<-, path] {node [node] {5}}
    child [->, path] {node [node] {7}}
  }
}
child [->, path] {
  node [node] {12}
  child [<-, path] {
    node [node] {10}
    child [<-, path] {node [node] {9}}
    child [->, path] {node [node] {11}}
  }
  child [->, path] {
    node [node] {14}
    child [<-, path] {node [node] {13}}
    child [->, path] {node [node] {15}}
  }
};
\end{tikzpicture}
\end{adjustbox}
    \caption{A binary search tree}
    \label{fig:binary_search_trees:binary_tree_example1}
  \end{figure}
\end{frame}

%-------------------------------------------------------------------------------

\begin{frame}{Binary Search Trees}{Introduction}
  \begin{figure}
    \begin{adjustbox}{height=0.35\linewidth}
\begin{tikzpicture}[
  node/.style={
    circle,
    draw=black,
    color=black,
    line width=0.1em,
    minimum size=2.0em,
    inner sep=0em
  }, path/.style={
    line width=0.25em,
    color=Mittel-Blau
  }, lleft/.style={
    color=Mittel-Gruen,
    shift={(-0.5em,-0.75em)}
  }, lright/.style={
    color=Mittel-Gruen!80!black,
    shift={(0.5em,-0.75em)}
  }, level/.style={
    sibling distance = 16.0em/#1,
    level distance = 4.0em
  }, level 3/.style={
    sibling distance = 4.0em,
  }]
\node (0, 0) [node] (root) {8}
child [<-, path] {
  node [node] {4}
  child [<-, path] {
    node [node] {2}
    child [<-, path] {node [node] {1}}
    child [->, path] {node [node] {3}}
  }
  child [->, path] {
    node [node] {6}
    child [<-, path] {node [node] {5}}
    child [->, path] {node [node] {7}}
  }
}
child [->, path] {
  node [node] {12}
};
\end{tikzpicture}
\end{adjustbox}
    \caption{Another binary search tree}
    \label{fig:binary_search_trees:binary_tree_example2}
  \end{figure}
\end{frame}

%-------------------------------------------------------------------------------

\begin{frame}{Binary Search Trees}{Introduction}
  \begin{figure}
    \begin{adjustbox}{height=0.35\linewidth}
\begin{tikzpicture}[
  node/.style={
    circle,
    draw=black,
    color=black,
    line width=0.1em,
    minimum size=2.0em,
    inner sep=0em
  }, path/.style={
    line width=0.25em,
    color=Mittel-Blau
  }, path_error/.style={
    line width=0.25em,
    color=red
  }, lleft/.style={
    color=Mittel-Gruen,
    shift={(-0.5em,-0.75em)}
  }, lright/.style={
    color=Mittel-Gruen!80!black,
    shift={(0.5em,-0.75em)}
  }, level/.style={
    sibling distance = 16.0em/#1,
    level distance = 4.0em
  }, level 3/.style={
    sibling distance = 4.0em,
  }]
\node (0, 0) [node] (root) {8}
child [<-, path] {
  node [node] {4}
  child [<-, path] {
    node [node] {2}
    child [<-, path] {node [node] {1}}
    child [->, path] {node [node] {3}}
  }
  child [->, path] {
    node [node] {6}
    child [->, path_error] {node [node] {9}}
    child [->, path] {node [node] {7}}
  }
}
child [->, path] {
  node [node] {12}
};
\end{tikzpicture}
\end{adjustbox}
    \caption{\textbf{Not} a binary search tree}
    \label{fig:binary_search_trees:binary_tree_example3}
  \end{figure}
\end{frame}

%-------------------------------------------------------------------------------

\begin{frame}{Binary Search Trees}{Implementation}
  \textbf{Implementation:}
  \begin{itemize}
    \item<2->
      For the heap we had all elements stored in an array
    \item<2->
      Here we link all nodes through pointer / references, like linked lists
    \item<3->
      Each node has a pointer / reference to its children
      (\texttt{\color{Mittel-Blau}leftChild} /
      \texttt{\color{Mittel-Blau}rightChild})
    \item<4->
      \texttt{\color{Mittel-Blau}None} for missing children
  \end{itemize}
  \onslide<5->
  \vspace{-1.0em}
  \begin{figure}
    \begin{adjustbox}{height=0.35\linewidth}
\begin{tikzpicture}[
  node/.style={
    circle,
    draw=black,
    color=black,
    line width=0.1em,
    minimum size=2.0em,
    inner sep=0em
  }, node_null/.style={
    draw=none,
    color=Mittel-Blau,
    font=\Large
  }, path/.style={
    line width=0.25em,
    color=Mittel-Blau
  }, path_error/.style={
    line width=0.25em,
    color=red
  }, lleft/.style={
    color=Mittel-Gruen,
    shift={(-0.5em,-0.75em)}
  }, lright/.style={
    color=Mittel-Gruen!80!black,
    shift={(0.5em,-0.75em)}
  }, level/.style={
    sibling distance = 20.0em/#1,
    level distance = 4.0em
  }, level 4/.style={
    sibling distance = 3.5em,
  }]
\node (0, 0) [node] (root) {12}
child [->, path] {
  node [node] (child_left) {7}
  child [->, path] {
    node [node] (child_left_left) {3}
    child [->, path] { node [node_null] {\texttt{None}} }
    child [->, path] {
      node [node] (child_left_left_right) {5}
      child [->, path] { node [node_null] {\texttt{None}} }
      child [->, path] { node [node_null] {\texttt{None}} }
    }
  }
  child [->, path] {
    node [node] (child_left_right) {10}
    child [->, path] { node [node_null] {\texttt{None}} }
    child [->, path] { node [node_null] {\texttt{None}} }
  }
}
child [->, path] {
  node [node] (child_right) {18}
  child [->, path] {
    node [node] (child_right_left) {15}
    child [->, path] {
      node [node] (child_right_left_left) {14}
      child [->, path] { node [node_null] {\texttt{None}} }
      child [->, path] { node [node_null] {\texttt{None}} }
    }
    child [->, path] {
      node [node] (child_right_left_right) {16}
      child [->, path] { node [node_null] {\texttt{None}} }
      child [->, path] { node [node_null] {\texttt{None}} }
    }
  }
  child [->, path] { node [node_null] {\texttt{None}} }
};
\end{tikzpicture}
\end{adjustbox}
    \label{fig:binary_search_trees:binary_tree_impl1}
  \end{figure}
\end{frame}

%-------------------------------------------------------------------------------

\begin{frame}{Binary Search Trees}{Implementation}
  \textbf{Implementation:}
  \begin{itemize}
    \item<2->
      We create a sorted doubly linked list of all elements
    \item<3->
      This enables an efficient implementation of
      (\texttt{\color{Mittel-Blau}next} /
      \texttt{\color{Mittel-Blau}previous})
  \end{itemize}
  \onslide<4->
  \begin{figure}
    \begin{adjustbox}{height=0.35\linewidth}
\begin{tikzpicture}[
  node/.style={
    circle,
    draw=black,
    color=black,
    line width=0.1em,
    minimum size=2.0em,
    inner sep=0em
  }, node_null/.style={
    draw=none,
    color=Mittel-Blau,
    font=\Large
  }, path/.style={
    line width=0.25em,
    color=Mittel-Blau
  }, link/.style={
    <->,
    dotted,
    line width=0.25em,
    color=Mittel-Gruen
  }, lleft/.style={
    color=Mittel-Gruen,
    shift={(-0.5em,-0.75em)}
  }, lright/.style={
    color=Mittel-Gruen!80!black,
    shift={(0.5em,-0.75em)}
  }, level/.style={
    sibling distance = 20.0em/#1,
    level distance = 4.0em
  }, level 4/.style={
    sibling distance = 3.5em,
  }]
\node (0, 0) [node] (root) {12}
child [->, path] {
  node [node] (child_left) {7}
  child [->, path] {
    node [node] (child_left_left) {3}
    child [->, path] { node [node_null] {\texttt{None}} }
    child [->, path] {
      node [node] (child_left_left_right) {5}
      child [->, path] { node [node_null] {\texttt{None}} }
      child [->, path] { node [node_null] {\texttt{None}} }
    }
  }
  child [->, path] {
    node [node] (child_left_right) {10}
    child [->, path] { node [node_null] {\texttt{None}} }
    child [->, path] { node [node_null] {\texttt{None}} }
  }
}
child [->, path] {
  node [node] (child_right) {18}
  child [->, path] {
    node [node] (child_right_left) {15}
    child [->, path] {
      node [node] (child_right_left_left) {14}
      child [->, path] { node [node_null] {\texttt{None}} }
      child [->, path] { node [node_null] {\texttt{None}} }
    }
    child [->, path] {
      node [node] (child_right_left_right) {16}
      child [->, path] { node [node_null] {\texttt{None}} }
      child [->, path] { node [node_null] {\texttt{None}} }
    }
  }
  child [->, path] { node [node_null] {\texttt{None}} }
};
\draw node[left=of child_left_left] (head) {\Large\texttt{head}};
\draw node[right=of child_right] (tail) {\Large\texttt{tail}};

\draw[link] (head) -- (child_left_left);
\draw[link] (child_left_left) -- (child_left_left_right);
\draw[link] (child_left_left_right) -- (child_left);
\draw[link] (child_left) -- (child_left_right);
\draw[link] (child_left_right) -- (root);
\draw[link] (root) -- (child_right_left_left);
\draw[link] (child_right_left_left) -- (child_right_left);
\draw[link] (child_right_left) -- (child_right_left_right);
\draw[link] (child_right_left_right) -- (child_right);
\draw[link] (child_right) -- (tail);
\end{tikzpicture}
\end{adjustbox}
    \caption{Binary search tree with links}
    \label{fig:binary_search_trees:binary_tree_impl2}
  \end{figure}
\end{frame}

%-------------------------------------------------------------------------------

\begin{frame}{Binary Search Trees}{Implementation - Lookup}
  \textbf{Lookup:}
  \begin{itemize}
    \item<2->
      Definition:\\
      \enquote{
        Search the element with the given key.
        If no element is found return the element with the next (bigger) key.
      }
    \item<3->
      We search from the root downwards:
      \begin{itemize}
        \item<4->
          Compare the searched key with the key of the node
        \item<5->
          Go to the left / right until the child is
          \texttt{\color{Mittel-Blau}None} or the key is found
        \item<6->
          If the key is not found return the next bigger one
      \end{itemize}
  \end{itemize}
\end{frame}

%-------------------------------------------------------------------------------

\begin{frame}{Binary Search Trees}{Implementation - Lookup}
  \textbf{For each node applies the total order:}\\
  \onslide<2->
  \hspace{1.5em}keys of left subtree
  < \texttt{\color{Mittel-Blau}node.key} < keys of right
  subtree
  \onslide<3->
  \begin{columns}
    \begin{column}{0.8\linewidth}
      \begin{figure}
        \begin{adjustbox}{width=\linewidth}
          \begin{adjustbox}{height=0.35\linewidth}
\begin{tikzpicture}[
  node/.style={
    circle,
    draw=black,
    color=black,
    line width=0.1em,
    minimum size=2.0em,
    inner sep=0em
  }, node_null/.style={
    draw=none,
    color=Mittel-Blau,
    font=\Large
  }, path/.style={
    line width=0.25em,
    color=Mittel-Blau
  }, path_error/.style={
    line width=0.25em,
    color=red
  }, lleft/.style={
    color=Mittel-Gruen,
    shift={(-0.5em,-0.75em)}
  }, lright/.style={
    color=Mittel-Gruen!80!black,
    shift={(0.5em,-0.75em)}
  }, level/.style={
    sibling distance = 20.0em/#1,
    level distance = 4.0em
  }, level 4/.style={
    sibling distance = 3.5em,
  }]
\node (0, 0) [node] (root) {12}
child [->, path] {
  node [node] (child_left) {7}
  child [->, path] {
    node [node] (child_left_left) {3}
    child [->, path] { node [node_null] {\texttt{None}} }
    child [->, path] {
      node [node] (child_left_left_right) {5}
      child [->, path] { node [node_null] {\texttt{None}} }
      child [->, path] { node [node_null] {\texttt{None}} }
    }
  }
  child [->, path] {
    node [node] (child_left_right) {10}
    child [->, path] { node [node_null] {\texttt{None}} }
    child [->, path] { node [node_null] {\texttt{None}} }
  }
}
child [->, path] {
  node [node] (child_right) {18}
  child [->, path] {
    node [node] (child_right_left) {15}
    child [->, path] {
      node [node] (child_right_left_left) {14}
      child [->, path] { node [node_null] {\texttt{None}} }
      child [->, path] { node [node_null] {\texttt{None}} }
    }
    child [->, path] {
      node [node] (child_right_left_right) {16}
      child [->, path] { node [node_null] {\texttt{None}} }
      child [->, path] { node [node_null] {\texttt{None}} }
    }
  }
  child [->, path] { node [node_null] {\texttt{None}} }
};
\end{tikzpicture}
\end{adjustbox}
        \end{adjustbox}
        \caption{Binary search tree with total order
          \enquote{\color{Mittel-Blau}<}}
        \label{fig:binary_search_trees:binary_tree_lookup}
      \end{figure}
    \end{column}
    \begin{column}{0.2\linewidth}
      \textbf{Examples:}\\[1em]
      \onslide<4->$\rlap\text\texttt{\color{red}lookup(14)}$\\[0.5em]
      \onslide<5->$\rlap\text\texttt{\color{Mittel-Gruen!50!blue}lookup(6)}$
        \\[0.5em]
      \onslide<6->$\rlap\text\texttt{\color{yellow!50!orange}lookup(19)}$
    \end{column}
  \end{columns}
\end{frame}

%-------------------------------------------------------------------------------

\begin{frame}{Binary Search Trees}{Implementation - Insert}
  \textbf{Insert:}
  \begin{itemize}
    \item<2->
      We search for the key in our search tree
    \item<3->
      If a node is found we replace the value with the new one
    \item<4->
      Else we insert a new node
    \item<5->If the key was not present we get a \texttt{\color{Mittel-Blau}None} entry
    \item<6->We insert the node there
  \end{itemize}
  \onslide<7->
  \vspace{-1em}
  \begin{figure}
    \begin{adjustbox}{height=0.35\linewidth}
\begin{tikzpicture}[
  node/.style={
    circle,
    draw=black,
    color=black,
    line width=0.1em,
    minimum size=2.0em,
    inner sep=0em
  }, node_null/.style={
    draw=none,
    color=Mittel-Blau,
    font=\Large
  }, path/.style={
    line width=0.25em,
    color=Mittel-Blau
  }, path_error/.style={
    line width=0.25em,
    color=red
  }, lleft/.style={
    color=Mittel-Gruen,
    shift={(-0.5em,-0.75em)}
  }, lright/.style={
    color=Mittel-Gruen!80!black,
    shift={(0.5em,-0.75em)}
  }, level/.style={
    sibling distance = 20.0em/#1,
    level distance = 4.0em
  }, level 4/.style={
    sibling distance = 3.5em,
  }]
\node (0, 0) [node] (root) {12}
child [->, path] {
  node [node] (child_left) {7}
  child [->, path] {
    node [node] (child_left_left) {3}
    child [->, path] { node [node_null] {\texttt{None}} }
    child [->, path] {
      node [node] (child_left_left_right) {5}
      child [->, path] { node [node_null] {\texttt{None}} }
      child [->, path] { node [node_null] {\texttt{None}} }
    }
  }
  child [->, path] {
    node [node] (child_left_right) {10}
    child [->, path] { node [node_null] {\texttt{None}} }
    child [->, path] { node [node_null] {\texttt{None}} }
  }
}
child [->, path] {
  node [node] (child_right) {18}
  child [->, path] {
    node [node] (child_right_left) {15}
    child [->, path] {
      node [node] (child_right_left_left) {14}
      child [->, path] { node [node_null] {\texttt{None}} }
      child [->, path] { node [node_null] {\texttt{None}} }
    }
    child [->, path] {
      node [node] (child_right_left_right) {16}
      child [->, path] { node [node_null] {\texttt{None}} }
      child [->, path] { node [node_null] {\texttt{None}} }
    }
  }
  child [->, path] { node [node_null] {\texttt{None}} }
};
\end{tikzpicture}
\end{adjustbox}
    \caption{Binary search tree with total order
      \enquote{\color{Mittel-Blau}<}}
    \label{fig:binary_search_trees:binary_tree_insert}
  \end{figure}
\end{frame}

%-------------------------------------------------------------------------------

\begin{frame}{Binary Search Trees}{Implementation - Remove}
  \textbf{Remove:} Case 1: The node \enquote{5} has no children\\
  \begin{itemize}
    \item<2->
      Find {\color{Mittel-Blau}parent} of node \enquote{5} (\enquote{6})
    \item<3->
      Set left / right child of node \enquote{6} to
      \texttt{\color{Mittel-Blau}None} depending on position of node \enquote{5}
  \end{itemize}
  \onslide<4->
  \begin{figure}
    \begin{adjustbox}{height=0.35\linewidth}
\begin{tikzpicture}[
  node/.style={
    circle,
    draw=black,
    color=black,
    line width=0.1em,
    minimum size=2.0em,
    inner sep=0em
  }, path/.style={
    line width=0.25em,
    color=Mittel-Blau
  }, lleft/.style={
    color=Mittel-Gruen,
    shift={(-0.5em,-0.75em)}
  }, lright/.style={
    color=Mittel-Gruen!80!black,
    shift={(0.5em,-0.75em)}
  }, level/.style={
    sibling distance = 16.0em/#1,
    level distance = 4.0em
  }, level 3/.style={
    sibling distance = 4.0em,
  }]
\node (0, 0) [node] (root) {8}
child [<-, path] {
  node [node] {4}
  child [<-, path] {
    node [node] {2}
    child [<-, path] {node [node] {1}}
    child [->, path] {node [node] {3}}
  }
  child [->, path] {
    node [node] {6}
    child [<-, path] {node [node] {5}}
    child [->, path] {node [node] {7}}
  }
}
child [->, path] {
  node [node] {12}
  child [->, path] {node [node] {14}}
};
\end{tikzpicture}
\end{adjustbox}
    \caption{Binary search tree with total order
      \enquote{\color{Mittel-Blau}<}}
    \label{fig:binary_search_trees:binary_tree_remove_no_child}
  \end{figure}
\end{frame}

%-------------------------------------------------------------------------------

\begin{frame}{Binary Search Trees}{Implementation - Remove}
  \textbf{Remove:} Case 1: The node \enquote{5} has no children\\
  \begin{itemize}
    \item
      Find {\color{Mittel-Blau}parent} of node \enquote{5} (\enquote{6})
    \item
      Set left / right child of node \enquote{6} to
      \texttt{\color{Mittel-Blau}None} depending on position of node \enquote{5}
  \end{itemize}
  \begin{figure}
    \begin{adjustbox}{height=0.35\linewidth}
\begin{tikzpicture}[
  node/.style={
    circle,
    draw=black,
    color=black,
    line width=0.1em,
    minimum size=2.0em,
    inner sep=0em
  }, node_null/.style={
    draw=none,
    color=Mittel-Blau,
    font=\Large
  }, path/.style={
    line width=0.25em,
    color=Mittel-Blau
  }, lleft/.style={
    color=Mittel-Gruen,
    shift={(-0.5em,-0.75em)}
  }, lright/.style={
    color=Mittel-Gruen!80!black,
    shift={(0.5em,-0.75em)}
  }, level/.style={
    sibling distance = 16.0em/#1,
    level distance = 4.0em
  }, level 3/.style={
    sibling distance = 4.0em,
  }]
\node (0, 0) [node] (root) {8}
child [<-, path] {
  node [node] {4}
  child [<-, path] {
    node [node] {2}
    child [<-, path] { node [node] {1} }
    child [->, path] { node [node] {3} }
  }
  child [->, path] {
    node [node] {6}
    child [color=white] { }
    child [->, path] { node [node] {7} }
  }
}
child [->, path] {
  node [node] {12}
  child [->, path] { node [node] {14} }
};
\end{tikzpicture}
\end{adjustbox}
    \caption{Binary search tree after deleting node \enquote{5}}
    \label{fig:binary_search_trees:binary_tree_remove_no_child_result}
  \end{figure}
\end{frame}

%-------------------------------------------------------------------------------

\begin{frame}{Binary Search Trees}{Implementation - Remove}
  \textbf{Remove:} Case 2: The node \enquote{12} has one child\\
  \begin{itemize}
    \item<2->
      Find the {\color{Mittel-Blau}child} of node \enquote{12} (\enquote{14})
    \item<3->
      Find the {\color{Mittel-Blau}parent} of node \enquote{12} (\enquote{8})
    \item<4->
      Set left / right {\color{Mittel-Blau}child} of node \enquote{8} to
      \enquote{14} depending on position of node \enquote{12}
      (skip node \enquote{14})
  \end{itemize}
  \onslide<5->
  \vspace{-0.5em}
  \begin{figure}
    \begin{adjustbox}{height=0.35\linewidth}
\begin{tikzpicture}[
  node/.style={
    circle,
    draw=black,
    color=black,
    line width=0.1em,
    minimum size=2.0em,
    inner sep=0em
  }, path/.style={
    line width=0.25em,
    color=Mittel-Blau
  }, lleft/.style={
    color=Mittel-Gruen,
    shift={(-0.5em,-0.75em)}
  }, lright/.style={
    color=Mittel-Gruen!80!black,
    shift={(0.5em,-0.75em)}
  }, level/.style={
    sibling distance = 16.0em/#1,
    level distance = 4.0em
  }, level 3/.style={
    sibling distance = 4.0em,
  }]
\node (0, 0) [node] (root) {8}
child [<-, path] {
  node [node] {4}
  child [<-, path] {
    node [node] {2}
    child [<-, path] {node [node] {1}}
    child [->, path] {node [node] {3}}
  }
  child [->, path] {
    node [node] {6}
    child [<-, path] {node [node] {5}}
    child [->, path] {node [node] {7}}
  }
}
child [->, path] {
  node [node] {12}
  child [->, path] {node [node] {14}}
};
\end{tikzpicture}
\end{adjustbox}
    \vspace{-0.75em}
    \caption{Binary search tree with total order
      \enquote{\color{Mittel-Blau}<}}
    \label{fig:binary_search_trees:binary_tree_remove_one_child}
  \end{figure}
\end{frame}

%-------------------------------------------------------------------------------

\begin{frame}{Binary Search Trees}{Implementation - Remove}
  \textbf{Remove:} Case 2: The node \enquote{12} has one child\\
  \begin{itemize}
    \item
      Find the {\color{Mittel-Blau}child} of node \enquote{12} (\enquote{14})
    \item
      Find the {\color{Mittel-Blau}parent} of node \enquote{12} (\enquote{8})
    \item
      Set left / right {\color{Mittel-Blau}child} of node \enquote{8} to
      \enquote{14} depending on position of node \enquote{12}
      (skip node \enquote{14})
  \end{itemize}
  \vspace{-0.5em}
  \begin{figure}
    \begin{adjustbox}{height=0.35\linewidth}
\begin{tikzpicture}[
  node/.style={
    circle,
    draw=black,
    color=black,
    line width=0.1em,
    minimum size=2.0em,
    inner sep=0em
  }, path/.style={
    line width=0.25em,
    color=Mittel-Blau
  }, lleft/.style={
    color=Mittel-Gruen,
    shift={(-0.5em,-0.75em)}
  }, lright/.style={
    color=Mittel-Gruen!80!black,
    shift={(0.5em,-0.75em)}
  }, level/.style={
    sibling distance = 16.0em/#1,
    level distance = 4.0em
  }, level 3/.style={
    sibling distance = 4.0em,
  }]
\node (0, 0) [node] (root) {8}
child [->, path] {
  node [node] {4}
  child [->, path] {
    node [node] {2}
    child [->, path] {node [node] {1}}
    child [->, path] {node [node] {3}}
  }
  child [->, path] {
    node [node] {6}
    child [->, path] {node [node] {5}}
    child [->, path] {node [node] {7}}
  }
}
child [->, path] {
  node [node] {14}
};
\end{tikzpicture}
\end{adjustbox}
    \vspace{-0.75em}
    \caption{Binary search tree after delting node \enquote{12}}
    \label{fig:binary_search_trees:binary_tree_remove_one_child_result}
  \end{figure}
\end{frame}

%-------------------------------------------------------------------------------

\begin{frame}{Binary Search Trees}{Implementation - Remove}
  \textbf{Remove:} Case 3: The node \enquote{4} has two children\\
  \begin{itemize}
    \item<2->
      Find the {\color{Mittel-Blau}successor} of node \enquote{4} (\enquote{5})
    \item<3->
      Replace the value of node \enquote{4} with the value of node \enquote{5}
    \item<4->
      Delete node \enquote{5} (the {\color{Mittel-Blau}successor} of node
      \enquote{4}) with remove-case 1 or 2
    \item<5->
      There is no left node because we are deleting the
      {\color{Mittel-Blau}predecessor}
  \end{itemize}
  \onslide<6->
  \vspace{-1.5em}
  \begin{figure}
    \begin{adjustbox}{height=0.35\linewidth}
\begin{tikzpicture}[
  node/.style={
    circle,
    draw=black,
    color=black,
    line width=0.1em,
    minimum size=2.0em,
    inner sep=0em
  }, path/.style={
    line width=0.25em,
    color=Mittel-Blau
  }, lleft/.style={
    color=Mittel-Gruen,
    shift={(-0.5em,-0.75em)}
  }, lright/.style={
    color=Mittel-Gruen!80!black,
    shift={(0.5em,-0.75em)}
  }, level/.style={
    sibling distance = 16.0em/#1,
    level distance = 4.0em
  }, level 3/.style={
    sibling distance = 4.0em,
  }]
\node (0, 0) [node] (root) {8}
child [<-, path] {
  node [node] {4}
  child [<-, path] {
    node [node] {2}
    child [<-, path] {node [node] {1}}
    child [->, path] {node [node] {3}}
  }
  child [->, path] {
    node [node] {6}
    child [<-, path] {node [node] {5}}
    child [->, path] {node [node] {7}}
  }
}
child [->, path] {
  node [node] {12}
  child [->, path] {node [node] {14}}
};
\end{tikzpicture}
\end{adjustbox}
    \label{fig:binary_search_trees:binary_tree_remove_two_children}
  \end{figure}
\end{frame}

%-------------------------------------------------------------------------------

\begin{frame}{Binary Search Trees}{Implementation - Remove}
  \textbf{Remove:} Case 3: The node \enquote{4} has two children\\
  \begin{itemize}
    \item
      Find the {\color{Mittel-Blau}successor} of node \enquote{4} (\enquote{5})
    \item
      Replace the value of node \enquote{4} with the value of node \enquote{5}
    \item
      Delete node \enquote{5} (the {\color{Mittel-Blau}successor} of node
      \enquote{4}) with remove-case 1 or 2
    \item
      There is no left node because we are deleting the
      {\color{Mittel-Blau}predecessor}
  \end{itemize}
  \onslide
  \vspace{-1.5em}
  \begin{figure}
    \begin{adjustbox}{height=0.35\linewidth}
\begin{tikzpicture}[
  node/.style={
    circle,
    draw=black,
    color=black,
    line width=0.1em,
    minimum size=2.0em,
    inner sep=0em
  }, path/.style={
    line width=0.25em,
    color=Mittel-Blau
  }, lleft/.style={
    color=Mittel-Gruen,
    shift={(-0.5em,-0.75em)}
  }, lright/.style={
    color=Mittel-Gruen!80!black,
    shift={(0.5em,-0.75em)}
  }, level/.style={
    sibling distance = 16.0em/#1,
    level distance = 4.0em
  }, level 3/.style={
    sibling distance = 4.0em,
  }]
\node (0, 0) [node] (root) {8}
child [->, path] {
  node [node] {5}
  child [->, path] {
    node [node] {2}
    child [->, path] {node [node] {1}}
    child [->, path] {node [node] {3}}
  }
  child [->, path] {
    node [node] {6}
    child [->, path] {node [node] {7}}
  }
}
child [->, path] {
  node [node] {12}
  child [->, path] {node [node] {14}}
};
\end{tikzpicture}
\end{adjustbox}
    \label{fig:binary_search_trees:binary_tree_remove_two_children_result}
  \end{figure}
\end{frame}

%-------------------------------------------------------------------------------

\begin{frame}{Binary Search Trees}{Runtime Complexity}
  \textbf{How long takes \texttt{\color{Mittel-Blau}insert} and
    \texttt{\color{Mittel-Blau}lookup}?}
  \begin{itemize}
    \item<2->
      Up to $\Theta(d)$, with $d$ being the
      {\color{Mittel-Blau}depth of the tree}\\
      (The longest path from the root to a leaf)
    \item<3->
      {\color{Mittel-Blau}Best case} with ${\color{Mittel-Blau}d = \log n}$
      the runtime is ${\color{Mittel-Blau}\Theta(\log n)}$
    \item<4->
      {\color{Mittel-Blau}Worst case} with ${\color{Mittel-Blau}d = n}$
      the runtime is ${\color{Mittel-Blau}\Theta(n)}$
    \item<5->
      If we \textbf{always} want to have a runtime of $\Theta(\log n)$ then
      we have to {\color{Mittel-Blau}rebalance} the tree
  \end{itemize}
  \onslide<6->
  \vspace{-2.0em}
  \begin{columns}%
    \begin{column}[b]{0.4\textwidth}%
      \begin{figure}%
        \begin{adjustbox}{height=0.6\linewidth}
\begin{tikzpicture}[
  node/.style={
    circle,
    draw=black,
    color=black,
    line width=0.1em,
    minimum size=2.0em,
    inner sep=0em
  }, node_null/.style={
    draw=none,
    color=Mittel-Blau,
    font=\Large
  }, path/.style={
    line width=0.25em,
    color=Mittel-Blau
  }, level/.style={
    sibling distance = 8.0em/#1,
    level distance = 4.0em
  }]
\node (0, 0) [node] (root) {8}
child [<-, path] {
  node [node] {7}
  child [<-, path] {
    node [node] {3}
  }
  child [->, path] { node [node_null] {\texttt{None}} }
}
child [->, path] { node [node_null] {\texttt{None}} };
\end{tikzpicture}
\end{adjustbox}%
        \caption{Degenerated binary tree ${\color{Mittel-Blau}d = n}$}%
      \end{figure}%
    \end{column}%
    \begin{column}[b]{0.6\textwidth}%
      \onslide<7->
      \begin{figure}%
        \begin{adjustbox}{height=0.4\linewidth}
\begin{tikzpicture}[
  node/.style={
    circle,
    draw=black,
    color=black,
    line width=0.1em,
    minimum size=2.0em,
    inner sep=0em
  }, path/.style={
    line width=0.25em,
    color=Mittel-Blau
  }, level/.style={
    sibling distance = 16.0em/#1,
    level distance = 4.0em
  }, level 3/.style={
    sibling distance = 4.0em,
  }]
\node (0, 0) [node] (root) {12}
child [<-, path] {
  node [node] {7}
  child [<-, path] { node [node] {3} }
  child [->, path] { node [node] {10} }
}
child [->, path] {
  node [node] {18}
  child [<-, path] { node [node] {15} }
  child [->, path] { node [node] {22} }
};
\end{tikzpicture}
\end{adjustbox}%
        \caption{Complete binary tree ${\color{Mittel-Blau}d = \log n}$}%
        \vspace{0.85em}%
      \end{figure}%
    \end{column}%
  \end{columns}%
\end{frame}
