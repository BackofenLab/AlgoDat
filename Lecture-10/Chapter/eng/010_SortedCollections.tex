\section{Sorted Sequences}

\begin{frame}{Sorted Sequences}{Introduction}
  \textbf{Structure:}
  \begin{itemize}
    \item<2->
      We have a set of {\color{Mittel-Blau}keys} mapped to
      {\color{Mittel-Blau}values}
    \item<3->
      We have a ordering {\color{Mittel-Blau}<} applied to the keys
    \item<4->
      We need the following operations:
      \begin{itemize}
      \item<5->{\color{Mittel-Blau}\texttt{insert(key, value)}}: Insert the given pair
      \item<6->{\color{Mittel-Blau}\texttt{remove(key)}}: Remove the pair with the given {\color{Mittel-Blau}key}
      \item<7->{\color{Mittel-Blau}\texttt{lookup(key)}}:  Find the element with the given {\color{Mittel-Blau}key},
        if it is not available find the element with the smallest key  {\color{Mittel-Blau}>key}
      \item<8->{\color{Mittel-Blau}\texttt{next()/previous()}}: Returns the element with the next bigger/smaller {\color{Mittel-Blau}key}. This enables iteration over all elements.    
      \end{itemize}  
  \end{itemize}
\end{frame}

%-------------------------------------------------------------------------------

\begin{frame}{Sorted Sequences}{Introduction}
  \textbf{Application examples:}
  \begin{itemize}
    \item<2->
      Example: Database for books, products or apartments
    \item<3->
      Large number of records (data sets / tuples)
    \item<4->
      Typical query: Return all apartments with a monthly rent between
      400\euro\, and 600\euro
      \begin{itemize}
        \item<5->
          This is called a {\color{Mittel-Blau}range query}
        \item<6->
          We can implement this with a combination of
          {\color{Mittel-Blau}\texttt{lookup(key)}} and
          {\color{Mittel-Blau}\texttt{next()}}
        \item<7->
          It's not essential if an apartments exists with \textbf{exactly}
          400\euro\, monthly rent
      \end{itemize}
    \item<8->
      We do not want to sort all elements every time on an 
      {\color{Mittel-Blau}insert} operation
     \item<9->
      How could we implement this?
  \end{itemize}
\end{frame}

%-------------------------------------------------------------------------------

\begin{frame}{Sorted Sequences}{Implementation 1 (not good) - Static Array}
  \textbf{Static array:}
    \begin{figure}[!b]
    \begin{tabular}{|c|c|c|c|c|c|c|c|c|c|c|c|}
      \hline
      3 & 5 & 9 & 14 & 18 & 21 & 26 & 40 & 41 & 42 & 43 & 46\\
      \hline
    \end{tabular}
    \label{fig:sorted_collections:impl_static_array}
  \end{figure}
  \begin{itemize}
    \item<2->
      {\color{Mittel-Blau}\texttt{lookup}} in time \texttt{$O(\log n)$}
      \begin{itemize}
        \item<3-> with \textbf{binary search}
        \item<4->
          Example: {\color{Mittel-Blau}\texttt{lookup(41)}}
      \end{itemize}
    \item<5->
      {\color{Mittel-Blau}\texttt{next}} /
      {\color{Mittel-Blau}\texttt{previous}} in time $O(1)$
      \begin{itemize}
        \item<6-> They are next to each other
      \end{itemize}
    \item<7->
      {\color{Mittel-Blau}\texttt{insert}} and
      {\color{Mittel-Blau}\texttt{remove}} up to $\Theta(n)$\\
      \begin{itemize}
        \item<8-> We have to copy up to $n$ elements
      \end{itemize}
  \end{itemize}
\end{frame}

%-------------------------------------------------------------------------------

\begin{frame}{Sorted Sequences}{Implementation 2 (bad) - Hash Table}
  \textbf{Hash map:}
  \begin{itemize}
    \item<2->
      {\color{Mittel-Blau}\texttt{insert}} and
      {\color{Mittel-Blau}\texttt{remove}} in $O(1)$\\
      \begin{itemize}
        \item<3-> If the hash table is big enough and we use a good hash function
      \end{itemize}  
    \item<4->
      {\color{Mittel-Blau}\texttt{lookup}} in time $O(1)$ 
      \begin{itemize}
        \item<5-> if element with \textbf{exactly}
      this key exists, otherwise we get {\color{Mittel-Blau}None} as result
      \end{itemize}    
    \item<5->
      {\color{Mittel-Blau}\texttt{next}} /
      {\color{Mittel-Blau}\texttt{previous}} in time up to $\Theta(n)$\\
       \begin{itemize}
        \item<6->
          The order of the elements is independent of the order of the keys
       \end{itemize}     
  \end{itemize}
\end{frame}

%-------------------------------------------------------------------------------

\begin{frame}{Sorted Sequences}{Implementation 3 (good?) - Linked List}
  \textbf{Linked list:}
  \begin{itemize}
  \item<2-> Runtimes for doubly linked lists:
    \begin{itemize}
    \item<3->
      {\color{Mittel-Blau}\texttt{next}} /
      {\color{Mittel-Blau}\texttt{previous}} in time $O(1)$\\
      %The elements are linked like a chain
    \item<4->
      {\color{Mittel-Blau}\texttt{insert}} and
      {\color{Mittel-Blau}\texttt{remove}} in $O(1)$
    \item<5->
      {\color{Mittel-Blau}\texttt{lookup}} in time $\Theta(n)$\\
      %We have to iterate over the elements in the list
    \end{itemize}
  \item<6-> Not yet what we want, but structure is related
    to binary search trees
  \item<7->Lets have a closer look
  \end{itemize}
  %% \begin{figure}
  %%   \begin{adjustbox}{width=\linewidth}%
\begin{tikzpicture}[
  element/.style={
    draw={Mittel-Blau},
    fill={white},
    line width=0.075em
  }, element_link/.style={
    draw={Mittel-Blau},
    fill={Hell-Blau},
    line width=0.075em
  }, arrow/.style={
    draw={Mittel-Blau},
    line width=0.125em
  }
]%
\foreach \x in {5,4,...,-1}{
  \ifnum \x>-1
    \ifnum \x=4 \else
      \draw[element_link] (1.5*\x, 0.5) rectangle (1.5*\x + 1.0, 0.75);
      \draw[element_link] (1.5*\x, 1.0) rectangle (1.5*\x + 1.0, 0.75);
      \draw[element] (1.5*\x, 0.0) rectangle (1.5*\x + 1.0, 0.5);
      
      \ifnum \x<4
        \draw (1.5*\x + 0.5, 0.25) node {\x};
      \else
        \draw (1.5*\x + 0.5, 0.25) node {$n$};
      \fi
    \fi
  \fi
  
  \draw[->, arrow] (1.5*\x + 1.0, 0.875) -- (1.5*\x + 1.5, 0.875);
  \draw[<-, arrow] (1.5*\x + 1.0, 0.625) -- (1.5*\x + 1.5, 0.625);
}

\draw (6.5, 0.5) node {\footnotesize ...};
\end{tikzpicture}%
\end{adjustbox}%
  %%   \caption{Doubly linked list}
  %%   \label{fig:sorted_collections:impl_linked_list}
  %% \end{figure}
\end{frame}
