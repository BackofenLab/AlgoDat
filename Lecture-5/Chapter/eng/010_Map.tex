\section{Associative Arrays}

\toclesssubsection{Introduction}

\begin{frame}{Associative Arrays}{How do we build a Map?}
  \textbf{Reminder:}
  \begin{itemize}
  \item An associative array is like a normal array, only that
    the indices are not {\color{MainA}$0, 1, 2, \ldots$}, but
    different, e.g. telephone numbers
  \end{itemize}
  \onslide<2->\textbf{Problem:}
  \begin{itemize}
  \onslide<2->\item Quickly find a element with a specific key
  \onslide<3->\item Naive solution: Store pairs of key and value
    in a normal field
  \onslide<4->\item For {\color{MainA}n} keys searching requires {\color{MainA} $\Theta(n)$} time
  \onslide<5->\item With a {\color{MainA}hash map} this just requires {\color{MainA} $\Theta(1)$}
    in the best case, ... regardless how many elements are in the map!
  \end{itemize}
\end{frame}

%% \begin{frame}{Associative Arrays}{How do we build a Map?}
%%   \textbf{Structure:}
%%   \begin{itemize}
%%       \item
%%         Mapping {\color{MainA}keys to values}
%%       \begin{itemize}
%%         \item
%%           Key universe: $\mathbb{U}$
%%         \item
%%           Used keys: $\mathbb{S} \subseteq \mathbb{U}$
%%         \item
%%          The keys / values can have any data type\\
%%       \end{itemize}
%%       \lstinline[
%%         language=Python,
%%         style={python-idle-code},
%%         basicstyle=\small,
%%         mathescape
%%       ]|[key${}_1$: value${}_1$, key${}_2$: value${}_2$, $\dotsc$]|
%%     \item
%%       We are interested in the following operations:\\[0.5em]
%%       \begin{tabularx}{\textwidth}{ll}%
%%         \lstinline[
%%           language=Python,
%%           style={python-idle-code},
%%           basicstyle=\small,
%%           emph={insert},
%%           emphstyle=\color{MainA}
%%         ]|insert(key, value)| & Insert an element with a key and value\\
%%         \lstinline[
%%           language=Python,
%%           style={python-idle-code},
%%           basicstyle=\small,
%%           emph={lookup},
%%           emphstyle=\color{MainA}
%%         ]|lookup(key)| & Retireve the mapped element with a key\\
%%         \lstinline[
%%           language=Python,
%%           style={python-idle-code},
%%           basicstyle=\small,
%%           emph={erase},
%%           emphstyle=\color{MainA}
%%         ]|erase(key)| & Delete the mapping from key to value
%%       \end{tabularx}
%%       \hfill\\
%%   \end{itemize}
%%   \textbf{Task:}
%%   \begin{itemize}
%%     \item
%%       Searching / storing an elemenent the fastest way
%%   \end{itemize}
%% \end{frame}

%--------------------------------------------------------------------------------

%% \begin{frame}{Associative Arrays}{How do we build a Map?}
%%   \textbf{Special case:}
%%   \begin{itemize}
%%     \item
%%       Keys $\mathbb{U}$ are natural numbers from $0$ to 
%%       $n - 1$
%%     \item
%%       Runtime:\\[0.5em]
%%       \begin{tabularx}{\textwidth}{lp{1em}l}%
%%         \lstinline[
%%           language=Python,
%%           style={python-idle-code},
%%           basicstyle=\small,
%%           emph={insert},
%%           emphstyle=\color{MainA}
%%         ]|insert(key, value)| & {} & $\mathcal{O}(1)$\\
%%         \lstinline[
%%           language=Python,
%%           style={python-idle-code},
%%           basicstyle=\small,
%%           emph={lookup},
%%           emphstyle=\color{MainA}
%%         ]|lookup(key)| & {} & $\mathcal{O}(1)$\\
%%         \lstinline[
%%           language=Python,
%%           style={python-idle-code},
%%           basicstyle=\small,
%%           emph={erase},
%%           emphstyle=\color{MainA}
%%         ]|erase(key)| & {} & $\mathcal{O}(1)$
%%       \end{tabularx}
%%     \item
%%       Memory usage: $\mathcal{O}(n)$
%% \end{itemize}
%%   \begin{table}[!b]
%%     \caption{Integer map with $n = 10$}
%%     \label{tab:maps:map_example_introduction_integer}
%%     \begin{tabularx}{\textwidth}{l|cccccccccc}
%%       A & 0 & 1 & 2 & 3 & 4 & 5 & 6 & 7 & 8 & 9\\
%%       \midrule
%%       Value &
%%       {} &
%%       value${}_1$ &
%%       {} &
%%       value${}_3$ &
%%       {} &
%%       {} &
%%       {} &
%%       value${}_7$ &
%%       {} &
%%       {}
%%     \end{tabularx}
%%   \end{table}
%% \end{frame}

%% %--------------------------------------------------------------------------------

%% \begin{frame}{Associative Arrays}{How do we build a Map?}
%%   \textbf{Normal case:}
%%   \begin{itemize}
%%     \item
%%       We have a lot less assigned keys $\mathbb{S}$ than possible
%%       ($\mathbb{S} \subset \mathbb{U}$)
%%     \item
%%       We don't have numbers as keys e.g. names / words
%%   \end{itemize}
%%   \textbf{Naive way:}
%%   \begin{itemize}
%%     \item
%%       Store all key value pairs in a normal field (array)
%%     \item
%%       Runtime:\\[0.5em]
%%       \begin{tabularx}{\textwidth}{lp{1em}l}%
%%         \lstinline[
%%           language=Python,
%%           style={python-idle-code},
%%           basicstyle=\small,
%%           emph={insert},
%%           emphstyle=\color{MainA}
%%         ]|insert(key, value)| & {} & $\mathcal{O}(1)$\\
%%         \lstinline[
%%           language=Python,
%%           style={python-idle-code},
%%           basicstyle=\small,
%%           emph={lookup},
%%           emphstyle=\color{MainA}
%%         ]|lookup(key)| & {} & $\mathcal{O}(n)$\\
%%         \lstinline[
%%           language=Python,
%%           style={python-idle-code},
%%           basicstyle=\small,
%%           emph={erase},
%%           emphstyle=\color{MainA}
%%         ]|erase(key)| & {} & $\mathcal{O}(n)$
%%       \end{tabularx}
%%     \item
%%       Memory usage: $\mathcal{O}(n)$
%%   \end{itemize}
%%   \vspace*{-1.0em}
%%   \begin{table}[!b]
%%     \caption{Map T: names mapped to age}%
%%     \label{tab:maps:map_example_introduction}%
%%     \vspace*{-0.5em}%
%%     \begin{tabularx}{0.775\textwidth}{l|cccc}
%%       T & 0 & 1 & 2\\
%%       \midrule
%%       Value &
%%       \lstinline[
%%         language=Python,
%%         style={python-idle-code},
%%         basicstyle=\small
%%       ]|("Karl", 20)| &
%%       \lstinline[
%%         language=Python,
%%         style={python-idle-code},
%%         basicstyle=\small
%%       ]|("Bob", 15)| &
%%       \lstinline[
%%         language=Python,
%%         style={python-idle-code},
%%         basicstyle=\small
%%       ]|("Peter", 18)|
%%     \end{tabularx}
%%   \end{table}
%% \end{frame}
