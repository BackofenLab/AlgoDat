\section{Universal Hashing}

\toclesssubsection{Introduction}

\begin{frame}{Universal Hashing}{Thought Experiment}
  \begin{itemize}
    \item
      I create a hash function
    \item
      Find a set of keys so that it results in a degenrated hash table
    \item
      Find a solution to avoid that problem
  \end{itemize}
\end{frame}

%-------------------------------------------------------------------------------

\begin{frame}{Universal Hashing}{Idea}
  \textbf{Solution:}
  \begin{itemize}
    \item
      We use a set of hash functions
    \item
      We choose a random hash function so that the expected result is an equal
      distribution
    \item
      This is called {\color{Mittel-Blau}universal hashing}
  \end{itemize}
  \begin{columns}
    \begin{column}{0.33\linewidth}
      \begin{figure}[!h]
        \begin{adjustbox}{height=0.8\linewidth}
          \begin{adjustbox}{height=0.8\linewidth}
\tikzstyle{body}=[
  draw=none,
  fill={Mittel-Blau!40!white}
]%
\tikzstyle{arrow}=[
  ->,
  draw={Mittel-Blau},
  line width=0.5em
]%
\begin{tikzpicture}
\draw[body] (0, 0) -- (0, 10) -- (3, 7) -- (3, 3) -- (0, 0);
  
\draw (-1, 6.5) node(x) {\Huge x};
\draw (-1, 3.5) node(y) {\Huge y};
\draw (5.0, 5.5) node(h1y) {\Huge $h_1(y)$};
\draw (5.0, 4.5) node(h1x) {\Huge $h_1(x)$};

\draw[arrow] (-0.5, 6.5) -- (0.5, 6.5) -- (2.5, 4.5) -- (3.5, 4.5);
\draw[arrow] (-0.5, 3.5) -- (0.5, 3.5) -- (2.5, 5.5) -- (3.5, 5.5);
\end{tikzpicture}
\end{adjustbox}
        \end{adjustbox}
        \caption{Hash func. 1}
        \label{fig:universal_hashing:hash_function_crossing}
      \end{figure}
    \end{column}
    \begin{column}{0.33\linewidth}
      \begin{figure}[!h]
        \begin{adjustbox}{height=0.8\linewidth}
          \begin{tikzpicture}[
  body/.style={
    draw=none,
    fill={Mittel-Blau!40!white}
  }, arrow/.style={
    ->,
    draw={Mittel-Blau},
    line width=0.5em
  }
]%
\draw[body] (0, 0) -- (0, 10) -- (3, 7) -- (3, 3) -- (0, 0);
  
\draw (-1, 6.5) node(x) {\Huge x};
\draw (-1, 3.5) node(y) {\Huge y};
\draw (5.0, 4.5) node(h2y) {\Huge $h_2(x)$};
\draw (5.0, 3.5) node(h2x) {\Huge $h_2(y)$};

\draw[arrow] (-0.5, 6.5) -- (0.5, 6.5) -- (2.5, 4.5) -- (3.5, 4.5);
\draw[arrow] (-0.5, 3.5) -- (3.5, 3.5);
\end{tikzpicture}
        \end{adjustbox}
        \caption{Hash func. 2}
        \label{fig:universal_hashing:hash_function_normal}
      \end{figure}
    \end{column}
    \begin{column}{0.33\linewidth}
      \begin{figure}[!h]
        \begin{adjustbox}{height=0.8\linewidth}
          \begin{adjustbox}{height=0.8\linewidth}
\tikzstyle{body}=[
  draw=none,
  fill={Mittel-Blau!40!white}
]%
\tikzstyle{arrow_collision}=[
  ->,
  draw={red!10!orange},
  line width=0.5em
]%
\begin{tikzpicture}
\draw[body] (0, 0) -- (0, 10) -- (3, 7) -- (3, 3) -- (0, 0);
  
\draw (-1, 6.5) node(x) {\Huge x};
\draw (-1, 3.5) node(y) {\Huge y};
\draw (6.5, 5.0) node(h3y) {\Huge $h_3(x) = h_3(y)$};

\draw[arrow_collision] (-0.5, 6.5) -- (0.5, 6.5) -- (2.5, 5.0) -- (3.5, 5.0);
\draw[arrow_collision] (-0.5, 3.5) -- (0.5, 3.5) -- (2.5, 5.0) -- (3.5, 5.0);
\end{tikzpicture}
\end{adjustbox}
        \end{adjustbox}
        \caption{Hash func. 2}
        \label{fig:universal_hashing:hash_function_colliding}
      \end{figure}
    \end{column}
  \end{columns}
\end{frame}

%-------------------------------------------------------------------------------

\begin{frame}{Universal Hashing}{Definition}
  \textbf{Definition:}
  \begin{itemize}
    \item
      We call $\mathbb{U}$ the set (universum) of possible keys
    \item
      The size of the hash table $m$
    \item
      Set of hash functions $\mathbb{H} = \{h_1, h_2,\dots, h_n\}$ with
      $h_i: \mathbb{U} \to \{0,\dots, m - 1\}$
  \end{itemize}%
  \begin{figure}[!b]%
    \begin{adjustbox}{height=0.35\linewidth}%
      \begin{tikzpicture}[
  body/.style={
    draw=none,
    fill={Mittel-Blau!40!white}
  }, arrow/.style={
    ->,
    draw={Mittel-Blau},
    line width=0.25em
  }, bucket/.style={
    draw=black,
    fill={Hell-Gruen}
  }, key/.style={
    draw=black,
    fill={yellow!80!white}
  }, background rectangle/.style={
    fill=white}
]%
\fill<1->[white] (-8, 0) rectangle (15, 10);
\draw<3->[body] (0, 0) -- (0, 10) -- (3, 7) -- (3, 3) -- (0, 0);

\draw<1-> (-2.0, 5.0) node[anchor=east](x) {\Huge Key set {\color{Mittel-Blau}$\mathbb{U}$}};
\draw<2-> (5.0, 5.0) node[anchor=west](x) {\Huge {\color{Mittel-Blau}T} (Hashtable)};

%\draw (-2.0, 5.0) node[anchor=east](x) {\Huge $\mathbb{U}$};
%\draw (5.0, 5.0) node[anchor=west](x) {\Huge T};

%\draw (1.5, 10.5) node[color={Mittel-Blau}](x) {\Huge $h_1$};

\foreach \y in {8,...,-8}{
  \draw<1->[key] 
    (-1.0, 0.5*\y + 5.25) -- (-1.5, 0.5*\y + 5.25) --
    (-1.5, 0.5*\y + 4.75) -- (-1.0, 0.5*\y + 4.75) --
    (-1.0, 0.5*\y + 5.25);
}

\foreach \i/\o in {
    4.0/-2.0, 3.5/1.0, 3.0/-3.0, 2.5/2.0, 2.0/3.0, 1.5/3.0, 1.0/1.0,
    0.5/2.0, 0.0/-2.0, -0.5/0.0, -1.0/-1.0, -1.5/-1.0, -2.0/-2.0,
    -2.5/-3.0, -3.0/3.0, -3.5/-1.0, -4.0/2.0
  } {
  \draw<3->[arrow]
    (-0.5, \i + 5.0) -- (0.0, \i + 5.0) --
    (3.0, 0.5*\o + 5.0) -- (3.5, 0.5*\o + 5.0);
}

\foreach \y in {3.0, 2.0, 1.0, 0.0, -1.0, -2.0, -3.0} {
  \draw<2->[bucket]
    (3.75, 0.5*\y + 5.25) -- (4.25, 0.5*\y + 5.25) --
    (4.25, 0.5*\y + 4.75) -- (3.75, 0.5*\y + 4.75) --
    (3.75, 0.5*\y + 5.25);
}
\end{tikzpicture}%
%
    \end{adjustbox}
    \vspace*{-1.0em}%
    \caption{Hash function $h_1$}%
    \label{fig:universal_hashing:hash_function_definition}
  \end{figure}
\end{frame}

%-------------------------------------------------------------------------------

\begin{frame}{Universal Hashing}{Definition}
  \begin{columns}
    \begin{column}{0.5\linewidth}
      \begin{itemize}
        \item
          We choose two random keys $x, y \in \mathbb{U} \mid x \neq 
          y$
        \item
          An average of 3 out of 15 functions produce collisions
      \end{itemize}
    \end{column}
    \begin{column}{0.5\linewidth}
      \begin{figure}[!t]%
        \begin{adjustbox}{width=\linewidth}
          \begin{tikzpicture}[
  body_n/.style={
    draw=Mittel-Blau,
    fill={Mittel-Blau!40!white},
    line width=0.125em
  }, body_c/.style={
    draw=red!80!white,
    fill={red!50!white},
    line width=0.125em
  }, bucket/.style={
    draw=black,
    fill={Hell-Gruen}
  }, key_g/.style={
    draw=black,
    fill={black!20!white}
   }, key_n/.style={
    draw=black,
    fill={yellow!80!white}
  }, arrow/.style={
    draw={Mittel-Gruen!80!white},
    left color={yellow!60!white},
    right color={Hell-Gruen!60!white},
%    draw={Mittel-Gruen!70!Mittel-Blau!80!black!70!white},
%    fill={Hell-Gruen!70!Mittel-Blau!40!white},
    line width=0.125em
  }
]%
\draw (-2.75, 11.0) node[align=center](U_Text) {\Huge Key set};
\draw (-2.75, 10.0) node[align=center](U) {\Huge$\mathbb{U}$};

\draw (5.0, 8.5) node[anchor=west](T_Text) {\Huge Hashtable};
\draw (5.5, 7.5) node[align=center](T) {\Huge T};


%\draw (-2.0, 5.0) node[anchor=east](x) {\Huge $\mathbb{U}$};
%\draw (5.0, 5.0) node[anchor=west](x) {\Huge T};

\foreach \y in {8,...,-8}{
  \draw[key_g]
    (-2.5, 0.5*\y + 5.25) -- (-3.0, 0.5*\y + 5.25) --
    (-3.0, 0.5*\y + 4.75) -- (-2.5, 0.5*\y + 4.75) --
    (-2.5, 0.5*\y + 5.25);
}

\draw (-3.0, 6.5) node[anchor=east](x) {\Huge x};
\draw (-3.0, 3.5) node[anchor=east](y) {\Huge y};
\foreach \y in {3, -3}{
  \draw[key_n]
  (-2.5, 0.5*\y + 5.25) -- (-3.0, 0.5*\y + 5.25) --
  (-3.0, 0.5*\y + 4.75) -- (-2.5, 0.5*\y + 4.75) --
  (-2.5, 0.5*\y + 5.25);
}

\foreach \y [count=\yi] in {n, n, n, n, c, n, c, n, n, n, c, n, n, n, n} {
  \draw[body_\y]
    (1.875-0.25*\yi, 1.875-0.25*\yi) -- (1.875-0.25*\yi, 11.875-0.25*\yi) --
    (4.875-0.25*\yi, 8.875-0.25*\yi) -- (4.875-0.25*\yi, 4.875-0.25*\yi) --
    (1.875-0.25*\yi, 1.875-0.25*\yi);
}

\foreach \y in {3.0, 2.0, 1.0, 0.0, -1.0, -2.0, -3.0} {
  \draw[bucket]
    (5.25, 0.5*\y + 5.25) -- (5.75, 0.5*\y + 5.25) --
    (5.75, 0.5*\y + 4.75) -- (5.25, 0.5*\y + 4.75) --
    (5.25, 0.5*\y + 5.25);
}

\draw[arrow]
  (-1.0, 4.0) -- (3.0, 4.0) -- (3.0, 5.5) -- (4.5, 3.0) --
  (3.0, 0.5) -- (3.0, 2.0) -- (-1.0, 2.0) -- (-1.0, 4.0);
\end{tikzpicture}%%
        \end{adjustbox}
        \caption{Hash universe $\mathbb{H}$}%
        \label{fig:universal_hashing:hash_universe}
      \end{figure}
    \end{column}
  \end{columns}
\end{frame}

%-------------------------------------------------------------------------------

\begin{frame}{Universal Hashing}{Definition}
  \textbf{Definition:}
  $\mathbb{H}$ is {\color{Mittel-Blau}$c$-universal} if
  $\forall x, y \in \mathbb{U} \mid x \neq y:$
  \begin{displaymath}
    \overbrace{
      \frac{
        \vert \{h_i \in \mathbb{H}\!: h_i(x) = h_i(y)\} \vert
      }{
      \underbrace{\vert \mathbb{H} \vert}_\text{Number of hash functions}
      }
    }^\text{
      Number of hash functions that create collisions
    }
    \leq \frac{c}{m}, \hspace*{1.5em} c \in \mathbb{R}
  \end{displaymath}
  \begin{itemize}
    \item
      If $h \in \mathbb{H}$ is chosen randomly then
      $\mathbb{P}(h(x) = h(y)) \leq \frac{c}{m}$
  \end{itemize}
  \begin{block}{Note: Best $c$ but not useful}
    If the hash function chooses the {\color{Mittel-Blau}bucket} 
    randomly then
    $\mathbb{P}(h(x) = h(y)) = \frac{1}{m} \Leftrightarrow c = 1$
  \end{block}
\end{frame}

%-------------------------------------------------------------------------------

\begin{frame}{Universal Hashing}{Definition}
  \begin{columns}
    \begin{column}{0.5\linewidth}
      \begin{itemize}
        \item
          $\mathbb{S}$:
          Key universe
        \item
          $\mathbb{S}_i \subseteq \mathbb{S}$:
          Keys mapping to Bucket $i$ (\enquote{synonyms})
       \item
         Ideal would be
         $\vert \mathbb{S}_i \vert =\dfrac{\vert \mathbb{S} \vert}{m}$
      \end{itemize}
    \end{column}
    \begin{column}{0.5\linewidth}
      \begin{figure}[!h]%
        \begin{adjustbox}{width=\linewidth}%
          \begin{adjustbox}{width=\linewidth}%
\begin{tikzpicture}[
  body/.style={
    draw=none,
    fill={Mittel-Blau!40!white}
  }, arrow/.style={
    ->,
    draw={black!20!white},
    line width=0.25em
  }, arrow_selected/.style={
    ->,
    draw={red!10!orange},
    line width=0.25em
  }, bucket/.style={
    draw=black,
    fill={black!20!white}
  }, bucket_selected/.style={
    draw=black,
    fill={Hell-Gruen}
  }, key/.style={
    draw=black,
    fill={black!20!white}
  }, key_selected/.style={
    draw=black,
    fill={yellow!80!white}
  }
]%
\draw[body] (0, 0) -- (0, 10) -- (3, 7) -- (3, 3) -- (0, 0);

\draw (-1.25, 11.0) node[align=center](U_Text) {\Huge Key set};
\draw (-1.25, 10.0) node[align=center](U) {\Huge$\mathbb{U}$};

\draw (2.0, 8.5) node[anchor=west](T_Text) {\Huge Hashtable};
\draw (4.0, 7.5) node[align=center](T) {\Huge T};

\draw (4.5, 6.0) node[anchor=west](T) {\Huge Bucket $i$};

%\draw (-2.0, 5.0) node[anchor=east](x) {\Huge $\mathbb{U}$};
%\draw (5.0, 5.0) node[anchor=west](x) {\Huge T};

%\draw (1.5, 10.5) node[color={Mittel-Blau}](x) {\Huge $h_1$};

\foreach \y in {8,...,-8}{
  \draw[key]
    (-1.0, 0.5*\y + 5.25) -- (-1.5, 0.5*\y + 5.25) --
    (-1.5, 0.5*\y + 4.75) -- (-1.0, 0.5*\y + 4.75) --
    (-1.0, 0.5*\y + 5.25);
}

\foreach \y [count=\yi] in {5, 1, -8}{
  \draw[key_selected]
    (-1.0, 0.5*\y + 5.25) -- (-1.5, 0.5*\y + 5.25) --
    (-1.5, 0.5*\y + 4.75) -- (-1.0, 0.5*\y + 4.75) --
    (-1.0, 0.5*\y + 5.25);
  \draw (-1.75, 0.5*\y + 5.0) node[anchor=east]
    {\Huge $s_\yi \in \mathbb{S}_i$};
}

\foreach \i/\o in {
    4.0/-2.0, 3.5/1.0, 3.0/-3.0, 2.0/3.0, 1.5/3.0, 1.0/1.0,
    0.0/-2.0, -0.5/0.0, -1.0/-1.0, -1.5/-1.0, -2.0/-2.0,
    -2.5/-3.0, -3.0/3.0, -3.5/-1.0
  } {
  \draw[arrow]
    (-0.5, \i + 5.0) -- (0.0, \i + 5.0) --
    (3.0, 0.5*\o + 5.0) -- (3.5, 0.5*\o + 5.0);
}

\foreach \i/\o in {
    2.5/2.0, 0.5/2.0, -4.0/2.0
  } {
  \draw[arrow_selected]
    (-0.5, \i + 5.0) -- (0.0, \i + 5.0) --
    (3.0, 0.5*\o + 5.0) -- (3.5, 0.5*\o + 5.0);
}

\foreach \y in {3.0, 1.0, 0.0, -1.0, -2.0, -3.0} {
  \draw[bucket]
    (3.75, 0.5*\y + 5.25) -- (4.25, 0.5*\y + 5.25) --
    (4.25, 0.5*\y + 4.75) -- (3.75, 0.5*\y + 4.75) --
    (3.75, 0.5*\y + 5.25);
}

\draw[bucket_selected]
  (3.75, 6.25) -- (4.25, 6.25) -- (4.25, 5.75) -- (3.75, 5.75) -- (3.75, 6.25);
\end{tikzpicture}%
\end{adjustbox}%%
        \end{adjustbox}
        \caption{Hash function $h \in \mathbb{H}$}%
        \label{fig:universal_hashing:hash_function_to_bucket}
      \end{figure}
    \end{column}
  \end{columns}
\end{frame}

%-------------------------------------------------------------------------------

\begin{frame}{Universal Hashing}{Definition}
  \begin{itemize}
    \item
      We choose the keys $\mathbb{S} \subset \mathbb{U}$ and the
      hash function $h \in \mathbb{H}$ randomly
    \item
      Let $\mathbb{S}_i = \{x \in \mathbb{S} | \; h(x) = i\}$ be the 
      keys which
      map to bucket $i$
    \item
      The expected average number of elements to search through per 
      bucket is
      \[\mathbb{E}\left[\vert \mathbb{S}_i \vert\right]
        \leq 1 + c \cdot \frac{\vert \mathbb{S} \vert}{m}\]
    \item
      If the number of buckets $\ll$ number of keys
      ($m \in \Omega(\vert \mathbb{S} \vert)$) then
      $\mathbb{E}\left[\vert \mathbb{S}_i \vert\right] = \mathcal{O}(n)$
  \end{itemize}
\end{frame}