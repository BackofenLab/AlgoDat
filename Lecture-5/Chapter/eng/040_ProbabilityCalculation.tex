\subsection{Probability Calculation}

\begin{frame}{Universal Hashing}{Probability Calculation}
  \begin{columns}
    \begin{column}{0.65\textwidth}
      \begin{itemize}
        \item<2->
          We just discuss the discrete case.
        \item<3->
          Probability space {\color{Mittel-Blau}$\Omega$} with {\color{Mittel-Blau}elementary (simple) events}
        \item<4->
          Events have probabilities .. Condition  {\color{Mittel-Blau}\[\sum_{e \in \Omega} P(e) = 1\]}
        \item<5->
          The probability for a subset of events {\color{Mittel-Blau}$E \subseteq \Omega$} is
          {\color{Mittel-Blau}\[P(E) = \sum_{e \in E} P(e) \mid e \in E\]}
      \end{itemize}
    \end{column}
    \begin{column}{0.35\textwidth}
      \onslide <6->
      \begin{table}[!h]
        \caption{Throwing a dice}
        \label{tab:probabilities:rolling_dice}
        \begin{tabularx}{0.5\linewidth}{c|c}
          {\color{Mittel-Blau}$e$} & {\color{Mittel-Blau}$P(e)$}\\
          \midrule
          1 & $\sfrac{1}{6}$\\
          2 & $\sfrac{1}{6}$\\
          3 & $\sfrac{1}{6}$\\
          4 & $\sfrac{1}{6}$\\
          5 & $\sfrac{1}{6}$\\
          6 & $\sfrac{1}{6}$\\
        \end{tabularx}
      \end{table}
    \end{column}
  \end{columns}
\end{frame}

%-------------------------------------------------------------------------------

\begin{frame}{Universal Hashing}{Probability Calculation}
  \begin{columns}
    \begin{column}{0.65\linewidth}
      \textbf{Example:}
      \begin{itemize}
        \item<2->
          Rolling a dice twice ({\color{Mittel-Blau}$\Omega = \{1,\dots,6\}^2$})
        \item<3->
          Each event {\color{Mittel-Blau}$e_{(i,\,j)} \in \Omega$} has the probability
          {\color{Mittel-Blau}$P(e) = \sfrac{1}{36}$}
        \item<4->
          {\color{Mittel-Blau}$E =$} if both eye numbers even, then {\color{Mittel-Blau}$P(E)=$}
      \end{itemize}
    \end{column}
    \onslide<3->
    \begin{column}{0.35\linewidth}
      \begin{table}[!h]
        \caption{Throwing a dice twice}
        \label{tab:probabilities_rolling_dice_twice}
        \begin{tabularx}{0.8\linewidth}{c|c}
          {\color{Mittel-Blau}$e_{(i,\,j)}$} &{\color{Mittel-Blau} $P(e_{(i,\,j)})$}\\
          \midrule
          $(1, 1)$ & $\sfrac{1}{36}$\\
          $(1, 2)$ & $\sfrac{1}{36}$\\
          $(1, 3)$ & $\sfrac{1}{36}$\\
          $\dots$ & $\dots$\\
          $(6, 5)$ & $\sfrac{1}{36}$\\
          $(6, 6)$ & $\sfrac{1}{36}$\\
        \end{tabularx}
      \end{table}
    \end{column}
  \end{columns}
\end{frame}

%-------------------------------------------------------------------------------

\begin{frame}{Universal Hashing}{Probability Calculation}
  \begin{columns}
    \begin{column}{0.55\linewidth}
      \textbf{Example:}
      \begin{itemize}
        \item
          We define: {\color{Mittel-Blau}$X$} is the result variable of the experiment
        \item
          We can reduce the result set by creating preconditions for our
          result variable {\color{Mittel-Blau}$X$}\\[0.5em]
          Examples:
          \begin{itemize}
            \item
              {\color{Mittel-Blau}$P(X = 2) = $}
            \item
              {\color{Mittel-Blau}$P(X = 4) = $}
          \end{itemize}
      \end{itemize}
    \end{column}
    \begin{column}{0.45\linewidth}
      \begin{table}[!h]
        \caption{Throwing a dice twice}
        \label{tab:probabilities:rolling_dice_twice2}
        \begin{tabularx}{0.95\linewidth}{c|cc}
          {\color{Mittel-Blau}$e_{(i,\,j)}$} & {\color{Mittel-Blau}$P(e_{(i,\,j)})$} & {\color{Mittel-Blau}$X = i + j$}\\
          \midrule
          $(1, 1)$ & $\sfrac{1}{36}$ & 2\\
          $(1, 2)$ & $\sfrac{1}{36}$ & 3\\
          $(1, 3)$ & $\sfrac{1}{36}$ & 4\\
          $\dots$ & $\dots$ & $\dots$\\
          $(6, 5)$ & $\sfrac{1}{36}$ & 11\\
          $(6, 6)$ & $\sfrac{1}{36}$ & 12\\
        \end{tabularx}
      \end{table}
    \end{column}
  \end{columns}
\end{frame}

%-------------------------------------------------------------------------------

\begin{frame}{Universal Hashing}{Probability Calculation}
  \begin{columns}
    \begin{column}{0.55\linewidth}
      \textbf{Example:}
      \begin{itemize}
        \item
          {\color{Mittel-Blau}$E = \{e_{(i,\,j)} \in \Omega
          \mid X \;\mathrm{mod}\; 2 = 0\}$}\\
          {\color{Mittel-Blau}$P(E) =$}
      \end{itemize}
    \end{column}
    \begin{column}{0.45\linewidth}
      \begin{table}[!h]
        \caption{Throwing a dice twice}
        \label{tab:probabilities:rolling_dice_twice3}
        \begin{tabularx}{0.95\linewidth}{c|cc}
          {\color{Mittel-Blau}$e_{(i,\,j)}$} & {\color{Mittel-Blau}$P(e_{(i,\,j)})$} &{\color{Mittel-Blau} $X = i + j$}\\
          \midrule
          $(1, 1)$ & $\sfrac{1}{36}$ & 2\\
          $(1, 2)$ & $\sfrac{1}{36}$ & 3\\
          $(1, 3)$ & $\sfrac{1}{36}$ & 4\\
          $\dots$ & $\dots$ & $\dots$\\
          $(6, 5)$ & $\sfrac{1}{36}$ & 11\\
          $(6, 6)$ & $\sfrac{1}{36}$ & 12\\
        \end{tabularx}
      \end{table}
    \end{column}
  \end{columns}
\end{frame}

%-------------------------------------------------------------------------------

\begin{frame}{Universal Hashing}{Probability Calculation}
  \textbf{Expected value:}
  {\color{Mittel-Blau}\[E[X]
    = \sum_{X \in X} \left(k \cdot P(X = k)\right)\]}
  \begin{itemize}
    \item
      The weighted average of all possible resulting values {\color{Mittel-Blau}$X$}
    \item
      The weight factor is the result value {\color{Mittel-Blau}$X$} itself
  \end{itemize}
\end{frame}

%-------------------------------------------------------------------------------

\begin{frame}{Universal Hashing}{Probability Calculation}
  \vspace*{-1.5em}
  \begin{table}[!h]
    \caption{Throwing a dice once}
    \label{tab:probabilities:value_rolling_dice_once}
    \begin{tabularx}{0.25\linewidth}{c|cc}
      {\color{Mittel-Blau}$X_1 = i$} & {\color{Mittel-Blau}$P(X_1)$}\\
      \midrule
      1 & $\sfrac{1}{6}$\\
      2 & $\sfrac{1}{6}$\\
      3 & $\sfrac{1}{6}$\\
      4 & $\sfrac{1}{6}$\\
      5 & $\sfrac{1}{6}$\\
      6 & $\sfrac{1}{6}$\\
    \end{tabularx}
  \end{table}
  Throwing the dice once:
  {\color{Mittel-Blau}\[E[X_1] = 1 \cdot \frac{1}{6} + 2 \cdot \frac{1}{6}
    + \dots + 6 \cdot \frac{1}{6} = 3.5\]}
\end{frame}

%-------------------------------------------------------------------------------

\begin{frame}{Universal Hashing}{Probability Calculation}
  \vspace*{-1.5em}
  \begin{table}[!h]
    \caption{Throwing a dice twice}
    \label{tab:probabilities:value_rolling_dice_twice}
    \begin{tabularx}{0.275\linewidth}{c|cc}
      {\color{Mittel-Blau}$X_2 = i + j$ }&{\color{Mittel-Blau} $P(X_2)$}\\
      \midrule
      2 & $\sfrac{1}{36}$\\
      3 & $\sfrac{2}{36}$\\
      4 & $\sfrac{3}{36}$\\
      $\dots$ & $\dots$\\
      11 & $\sfrac{2}{36}$\\
      12 & $\sfrac{1}{36}$\\
    \end{tabularx}
  \end{table}
  Throwing the dice twice:
  {\color{Mittel-Blau}\[E[X_2] = 2 \cdot \frac{1}{36} + 3 \cdot \frac{2}{36}
    + \dots + 12 \cdot \frac{1}{36} = 7\]}
\end{frame}

%-------------------------------------------------------------------------------

\begin{frame}{Universal Hashing}{Probability Calculation}
  \textbf{Sum of expected values:}
  For independent (discrete) result variables {\color{Mittel-Blau}$X_1,\dots,X_n$} we can write:
    {\color{Mittel-Blau}\[E\left[X_1+\dots+X_n\right]
      = E[X_1] + \dots + E[X_n]\]}
  \textbf{Example:} Throwing two dice
  \begin{itemize}
    \item
      {\color{Mittel-Blau}$X_a$}: Expected number of eyes dice {\color{Mittel-Blau}$a$}: {\color{Mittel-Blau}$E[X_a] = 3.5$}
    \item
      {\color{Mittel-Blau}$X_b$}: Expected number of eyes dice {\color{Mittel-Blau}$b$}: {\color{Mittel-Blau}$E[X_b] = 3.5$}
    \item
      {\color{Mittel-Blau}$X = X_a + X_b$}: Expected total number of eyes:
      {\color{Mittel-Blau}\[E[X]
        = E[X_a + X_b]
        = E[X_a] + E[X_b] = 3.5 + 3.5 = 7\]}
  \end{itemize}
\end{frame}

%-------------------------------------------------------------------------------

\begin{frame}{Universal Hashing}{Probability Calculation}
  \begin{block}{Corollary:}
    The probability of the event $E$ is $p = P(E)$.
    Let $X$ be the occurences of the event $E$ and $n$ be the number
    of executions.
    {\color{Mittel-Blau}\[E[X] = n \cdot P(E) = n \cdot p\]}
  \end{block}
  \begin{example}[Rolling the dice 60 times:]
    \[E\left[\text{number of 6}\right] = \frac{1}{6} \cdot 60 = 10\]
  \end{example}
\end{frame}

%-------------------------------------------------------------------------------

\begin{frame}{Universal Hashing}{Probability Calculation}
  \begin{proof}[Proof Corollary:]
    Indicator variable: {\color{Mittel-Blau}$X_i$}\\
    \vspace*{-1.5em}
    {\color{Mittel-Blau}
    \begin{align*}
      X_i &=
        \left\lbrace
          \begin{array}{ll}
            1, & \text{if event occurs}\\
            0, & \text{else}
          \end{array}
        \right. \hspace{1.5em}
        \Rightarrow \; X = \sum_{i=1}^{n} X_i
    \end{align*}}
    \vspace*{-1.0em}
    {\color{Mittel-Blau}
    \begin{align*}
      E[X_i]
        &= \; 0 \cdot P(X_i = 0) + 1 \cdot P(X_i = 1)\\
      {} &= \; P(X_i = 1) = p\\[0.5em]
      E[X] &= E\left[\sum_{i=1}^{n} X_i\right]
        = \sum_{i=1}^{n} E[X_i]
        \stackrel{\text{def. $E$-value}}{=}
        \sum_{i=1}^{n} p = n \cdot p
    \end{align*}}
    \qedhere
  \end{proof}
\end{frame}
