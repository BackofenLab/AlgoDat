\subsection{Probability Calculation}

\begin{frame}{Universal Hashing}{Probability Calculation}
  \begin{columns}
    \begin{column}{0.65\textwidth}
      \begin{itemize}
        \item<2->
          We just discuss the discrete case.
        \item<3->
          Probability space {\color{Mittel-Blau}$\Omega$} with {\color{Mittel-Blau}elementary (simple) events}
        \item<4->
          Events have probabilities .. Condition  {\color{Mittel-Blau}\[\sum_{e \in \Omega} P(e) = 1\]}
        \item<5->
          The probability for a subset of events {\color{Mittel-Blau}$E \subseteq \Omega$} is
          {\color{Mittel-Blau}\[P(E) = \sum_{e \in E} P(e) \mid e \in E\]}
      \end{itemize}
    \end{column}
    \begin{column}{0.35\textwidth}
      \onslide <6->
      \begin{table}[!h]
        \caption{Throwing a dice}
        \label{tab:probabilities:rolling_dice}
        \begin{tabularx}{0.5\linewidth}{c|c}
          {\color{Mittel-Blau}$e$} & {\color{Mittel-Blau}$P(e)$}\\
          \midrule
          1 & $\sfrac{1}{6}$\\
          2 & $\sfrac{1}{6}$\\
          3 & $\sfrac{1}{6}$\\
          4 & $\sfrac{1}{6}$\\
          5 & $\sfrac{1}{6}$\\
          6 & $\sfrac{1}{6}$\\
        \end{tabularx}
      \end{table}
    \end{column}
  \end{columns}
\end{frame}

%-------------------------------------------------------------------------------

\begin{frame}{Universal Hashing}{Probability Calculation}
  \begin{columns}
    \begin{column}{0.65\linewidth}
      \textbf{Example:}
      \begin{itemize}
        \item<2->
          Rolling a dice twice ({\color{Mittel-Blau}$\Omega = \{1,\dots,6\}^2$})
        \item<3->
          Each event {\color{Mittel-Blau}$e \in \Omega$} has the probability
          {\color{Mittel-Blau}$P(e) = \sfrac{1}{36}$}
        \item<4->
          {\color{Mittel-Blau}$E =$} if both eye numbers even, then {\color{Mittel-Blau}$P(E)=$}
      \end{itemize}
    \end{column}
    \onslide<3->
    \begin{column}{0.35\linewidth}
      \begin{table}[!h]
        \caption{Throwing a dice twice}
        \label{tab:probabilities_rolling_dice_twice}
        \begin{tabularx}{0.8\linewidth}{c|c}
          {\color{Mittel-Blau}$e$} &{\color{Mittel-Blau} $P(e)$}\\
          \midrule
          $(1, 1)$ & $\sfrac{1}{36}$\\
          $(1, 2)$ & $\sfrac{1}{36}$\\
          $(1, 3)$ & $\sfrac{1}{36}$\\
          $\dots$ & $\dots$\\
          $(6, 5)$ & $\sfrac{1}{36}$\\
          $(6, 6)$ & $\sfrac{1}{36}$\\
        \end{tabularx}
      \end{table}
    \end{column}
  \end{columns}
\end{frame}

%-------------------------------------------------------------------------------

\begin{frame}{Universal Hashing}{Probability Calculation}
  \begin{columns}
    \begin{column}{0.55\linewidth}
      \textbf{Example:}
      \begin{itemize}
      \setlength\itemsep{1em}
      \item <1->
        Random variable 
        \begin{itemize}
        \item <2->
        Assigns a number to the result of an experiment          
        \item <3->
        For example: {\color{Mittel-Blau}$X =$} Sum of eye numbers for rolling twice  
        \item <4->
          {\color{Mittel-Blau}$X = 12$} and {\color{Mittel-Blau}$X \geq 7$} are then just events
        \item <5->
          Example 1: {\color{Mittel-Blau}$P(X = 2) = $}
        \item <6->
          Example 2: {\color{Mittel-Blau}$P(X = 4) = $}  
        \end{itemize}
      \end{itemize}
    \end{column}
    \onslide <4->
    \begin{column}{0.45\linewidth}
      \begin{table}[!h]
        \caption{Throwing a dice twice}
        \label{tab:probabilities:rolling_dice_twice2}
        \begin{tabularx}{0.95\linewidth}{c|cc}
          {\color{Mittel-Blau}$e$} & {\color{Mittel-Blau}$P(e)$} & {\color{Mittel-Blau}$X$}\\
          \midrule
          $(1, 1)$ & $\sfrac{1}{36}$ & 2\\
          $(1, 2)$ & $\sfrac{1}{36}$ & 3\\
          $(1, 3)$ & $\sfrac{1}{36}$ & 4\\
          $\dots$ & $\dots$ & $\dots$\\
          $(6, 5)$ & $\sfrac{1}{36}$ & 11\\
          $(6, 6)$ & $\sfrac{1}{36}$ & 12\\
        \end{tabularx}
      \end{table}
    \end{column}
  \end{columns}
\end{frame}

%-------------------------------------------------------------------------------

%% \begin{frame}{Universal Hashing}{Probability Calculation}
%%   \begin{columns}
%%     \begin{column}{0.55\linewidth}
%%       \textbf{Example:}
%%       \vspace{1em}
%%       \begin{itemize}
%%         \item
%%           {\color{Mittel-Blau}$E = \{e_{(i,\,j)} \in \Omega
%%           \mid X \;\mathrm{mod}\; 2 = 0\}$}\\
%%           {\color{Mittel-Blau}$P(E) =$}
%%       \end{itemize}
%%     \end{column}
%%     \begin{column}{0.45\linewidth}
%%       \begin{table}[!h]
%%         \caption{Throwing a dice twice}
%%         \label{tab:probabilities:rolling_dice_twice3}
%%         \begin{tabularx}{0.95\linewidth}{c|cc}
%%           {\color{Mittel-Blau}$e_{(i,\,j)}$} & {\color{Mittel-Blau}$P(e_{(i,\,j)})$} &{\color{Mittel-Blau} $X = i + j$}\\
%%           \midrule
%%           $(1, 1)$ & $\sfrac{1}{36}$ & 2\\
%%           $(1, 2)$ & $\sfrac{1}{36}$ & 3\\
%%           $(1, 3)$ & $\sfrac{1}{36}$ & 4\\
%%           $\dots$ & $\dots$ & $\dots$\\
%%           $(6, 5)$ & $\sfrac{1}{36}$ & 11\\
%%           $(6, 6)$ & $\sfrac{1}{36}$ & 12\\
%%         \end{tabularx}
%%       \end{table}
%%     \end{column}
%%   \end{columns}
%% \end{frame}

%-------------------------------------------------------------------------------

\begin{frame}{Universal Hashing}{Probability Calculation}
  \onslide<1->
  \textbf{Expected value} is defined as  {\color{Mittel-Blau}$E(X)  = \sum \left(k \cdot P(X = k)\right)$}
  \vspace*{0em}
  \begin{itemize}
    \item <2->
      Intuitive: The weighted average of possible values of {\color{Mittel-Blau}$X$}, where
      the weights are the probabilities of the values
  \end{itemize}
  \vspace*{-1.0em}
  \onslide <3->
 \begin{columns}
   \begin{column}{0.5\linewidth}
     \begin{table}[!h]
       \small{
     \caption{Throwing a dice once}
    \label{tab:probabilities:value_rolling_dice_once}
    \begin{tabularx}{0.25\linewidth}{c|cc}
      {\color{Mittel-Blau}$X$} & {\color{Mittel-Blau}$P(X)$}\\
      \midrule
      1 & $\sfrac{1}{6}$\\
      2 & $\sfrac{1}{6}$\\
      3 & $\sfrac{1}{6}$\\
      4 & $\sfrac{1}{6}$\\
      5 & $\sfrac{1}{6}$\\
      6 & $\sfrac{1}{6}$\\
    \end{tabularx}}
  \end{table}  
   \end{column}
   \begin{column}{0.5\linewidth}
     \begin{table}[!h]
       \small{
    \caption{Throwing a dice twice}
    \label{tab:probabilities:value_rolling_dice_twice}
    \begin{tabularx}{0.275\linewidth}{c|cc}
      {\color{Mittel-Blau}$X$ }&{\color{Mittel-Blau} $P(X)$}\\
      \midrule
      2 & $\sfrac{1}{36}$\\
      3 & $\sfrac{2}{36}$\\
      4 & $\sfrac{3}{36}$\\
      $\dots$ & $\dots$\\
      11 & $\sfrac{2}{36}$\\
      12 & $\sfrac{1}{36}$\\
    \end{tabularx}}
   \end{table}
   \end{column}
 \end{columns}
 \begin{itemize}
 \vspace*{-1.0em}
 \item<4-> Example rolling once: \onslide<5->
  {\color{Mittel-Blau}$E(X) = 1 \cdot \frac{1}{6} + 2 \cdot \frac{1}{6}
    + \dots + 6 \cdot \frac{1}{6} = 3.5$}
 \item<6-> Example rolling twice: \onslide<7->{\color{Mittel-Blau}$E(X) = 2 \cdot \frac{1}{36} + 3 \cdot \frac{2}{36}
    + \dots + 12 \cdot \frac{1}{36} = 7$}
   \end{itemize}
\end{frame}


%% \begin{frame}{Universal Hashing}{Probability Calculation}
%%   \textbf{Expected value:}
%%   {\color{Mittel-Blau}\[E(X)
%%     = \sum \left(k \cdot P(X = k)\right)\]}
%%   \begin{itemize}
%%     \item
%%       The weighted average of all possible resulting values {\color{Mittel-Blau}$X$}
%%     \item
%%       The weight factor is the result value {\color{Mittel-Blau}$X$} itself
%%   \end{itemize}
%% \end{frame}

%% %-------------------------------------------------------------------------------

%% \begin{frame}{Universal Hashing}{Probability Calculation}
%%   \vspace*{-1.5em}
%%   \begin{table}[!h]
%%     \caption{Throwing a dice once}
%%     \label{tab:probabilities:value_rolling_dice_once}
%%     \begin{tabularx}{0.25\linewidth}{c|cc}
%%       {\color{Mittel-Blau}$X$} & {\color{Mittel-Blau}$P(X)$}\\
%%       \midrule
%%       1 & $\sfrac{1}{6}$\\
%%       2 & $\sfrac{1}{6}$\\
%%       3 & $\sfrac{1}{6}$\\
%%       4 & $\sfrac{1}{6}$\\
%%       5 & $\sfrac{1}{6}$\\
%%       6 & $\sfrac{1}{6}$\\
%%     \end{tabularx}
%%   \end{table}
%%   \onslide<1>
%%   Throwing the dice once:
%%   {\color{Mittel-Blau}\[E(X) = 1 \cdot \frac{1}{6} + 2 \cdot \frac{1}{6}
%%     + \dots + 6 \cdot \frac{1}{6} = 3.5\]}
%% \end{frame}

%% %-------------------------------------------------------------------------------

%% \begin{frame}{Universal Hashing}{Probability Calculation}
%%   \vspace*{-1.5em}
%%   \begin{table}[!h]
%%     \caption{Throwing a dice twice}
%%     \label{tab:probabilities:value_rolling_dice_twice}
%%     \begin{tabularx}{0.275\linewidth}{c|cc}
%%       {\color{Mittel-Blau}$X$ }&{\color{Mittel-Blau} $P(X)$}\\
%%       \midrule
%%       2 & $\sfrac{1}{36}$\\
%%       3 & $\sfrac{2}{36}$\\
%%       4 & $\sfrac{3}{36}$\\
%%       $\dots$ & $\dots$\\
%%       11 & $\sfrac{2}{36}$\\
%%       12 & $\sfrac{1}{36}$\\
%%     \end{tabularx}
%%   \end{table}
%%   Throwing the dice twice:
%%   {\color{Mittel-Blau}\[E(X) = 2 \cdot \frac{1}{36} + 3 \cdot \frac{2}{36}
%%     + \dots + 12 \cdot \frac{1}{36} = 7\]}
9%% \end{frame}

%% %-------------------------------------------------------------------------------

\begin{frame}{Universal Hashing}{Probability Calculation}
   \textbf{Sum of expected values:}
   For independent (discrete) result variables {\color{Mittel-Blau}$X_1,\dots,X_n$} we can write:
     {\color{Mittel-Blau}\[E\left(X_1+\dots+X_n\right)
       = E(X_1) + \dots + E(X_n)\]}
   \onslide<2->
   \textbf{Example:} Throwing two dice
   \begin{itemize}
     \item<3->
       {\color{Mittel-Blau}$X_1$}: Expected number of eyes dice {\color{Mittel-Blau}$1$}: {\color{Mittel-Blau}$E(X_1) = 3.5$}
     \item<4->
       {\color{Mittel-Blau}$X_2$}: Expected number of eyes dice {\color{Mittel-Blau}$2$}: {\color{Mittel-Blau}$E(X_2) = 3.5$}
     \item<5->
       {\color{Mittel-Blau}$X = X_1 + X_2$}: Expected total number of eyes:
       {\color{Mittel-Blau}\[E(X)
         = E(X_1 + X_2)
         = E(X_1) + E(X_2) = 3.5 + 3.5 = 7\]}
   \end{itemize}
\end{frame}

%% %-------------------------------------------------------------------------------

\begin{frame}{Universal Hashing}{Probability Calculation}
  \onslide<1->
  \begin{block}{Corollary:}
    The probability of the event $E$ is $p = P(E)$.
    Let $X$ be the occurences of the event $E$ and $n$ be the number
    of executions of the experiment. Then {\color{Mittel-Blau}$E(X) = n \cdot P(E) = n \cdot p$}
  \end{block}
  \onslide<2->
  \begin{example}[Rolling the dice 60 times:]
    \[E\left(\text{occurences of 6}\right) = \frac{1}{6} \cdot 60 = 10\]
  \end{example}
\end{frame}

%-------------------------------------------------------------------------------

\begin{frame}{Universal Hashing}{Probability Calculation}
  \begin{proof}[Proof Corollary:]
    Indicator variable: {\color{Mittel-Blau}$X_i$}\\
    \vspace*{-1.5em}
            {\color{Mittel-Blau}
    \onslide<2->
    \begin{align*}
      X_i &=
        \left\lbrace
          \begin{array}{ll}
            1, & \text{if event occurs}\\
            0, & \text{else}
          \end{array}
          \right. \hspace{1.5em}\\
        \Rightarrow \; X = \sum_{i=1}^{n} X_i
    \end{align*}}
    \vspace*{-1.0em}
            {\color{Mittel-Blau}
    \onslide<3->          
    \begin{align*}
      E(X) &= E\left(\sum_{i=1}^{n} X_i\right)
        = \sum_{i=1}^{n} E(X_i)
        \stackrel{\text{def. $E$-value}}{=}
        \sum_{i=1}^{n} p = n \cdot p
    \end{align*}
    \qedhere
    \onslide<4->
    \vspace*{-1.0em}
    \begin{align*}
        \text{Def. $E$-value: }  E(X_i) &= \; 0 \cdot P(X_i = 0) + 1 \cdot P(X_i = 1) &= \; P(X_i = 1) \\[0.5em]   
    \end{align*}}
  \end{proof}
\end{frame}
