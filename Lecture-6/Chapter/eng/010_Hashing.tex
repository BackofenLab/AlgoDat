\section{Hashing}

\toclesssubsection{Recap}

\begin{frame}{Hashing}{Recap}
  \textbf{Hashing:}
  \begin{itemize}
    \item
      One hash function can not be good for all key sets!\\
      The result set (amount of buckets) is mostly smaller than the input
      universe\\
      $\Rightarrow \vert \mathbb{U} \vert > m:$
      There are always overlapping elements
    \item
      You can use a simple hash function for random key sets.
      The randomness of the keys will create a equivalent distribution onto
      the buckets
      \[\Rightarrow h(x) = x \mod m\]
  \end{itemize}
\end{frame}

%-------------------------------------------------------------------------------

\begin{frame}{Hashing}{Recap}
  \textbf{Hashing:}
  \begin{itemize}
    \item
      For a fixed key set the most hash functions are sufficient but not all of
      them
    \item
      Universal hashing is needed to find a good hash function for each key set
    \item
      Not every hash function will produce good results but the most of the
      functions in the universe
  \end{itemize}
\end{frame}

%-------------------------------------------------------------------------------

\begin{frame}{Hashing}{Recap}
  \textbf{Rehashing:}
  \begin{itemize}
    \item
      It is possible to get bad hash functions with universal hashing, but it
      is unlikely
    \item
      This is determinable by monitoring the bucket size and start a rehash
      if a specific level is reached
  \end{itemize}
  \textbf{How to rehash?}
  \begin{itemize}
    \item
      New hash table with a new random hash function
    \item
      Copy elements into the new table
      \begin{itemize}
        \item
          Expensive but happens not often
        \item
          Average cost is low
        \item
          Look at amortized analysis in the next lecture
      \end{itemize}
  \end{itemize}
\end{frame}

%-------------------------------------------------------------------------------

\subsection{Linked List}

\begin{frame}{Hashing}{Linked List}
  \textbf{Buckets as linked list:}
  \begin{itemize}
    \item
      Each bucket is represented as a linked list
    \item
      Colliding keys are appended to the matching bucket
    \item
      A dynamic number of elements is possible
  \end{itemize}
  \vspace*{-1.0em}
  \begin{table}[!b]
    \caption{Hash table}
    \label{tab:hashing:linked_list:hash_table}
    \begin{tabularx}{0.875\textwidth}{c|l}
      Bucket & Value\\
      \midrule
      0 & \lstinline[
        language=Python,
        style={python-idle-code},
        basicstyle=\small
      ]|None|\\
      1 &
      \lstinline[
        language=Python,
        style={python-idle-code},
        mathescape=true,
        basicstyle=\small
      ]|(3126, "B") $\rightarrow$ (4561, "D") $\rightarrow$ None|\\
      2 &
      \lstinline[
        language=Python,
        style={python-idle-code},
        mathescape=true,
        basicstyle=\small
      ]|(5147, "C") $\rightarrow$ None|\\
      3 &
      \lstinline[
        language=Python,
        style={python-idle-code},
        mathescape=true,
        basicstyle=\small
      ]|(3903, "A") $\rightarrow$ (1683, "D")
        $\rightarrow$ (4818, "E")
        $\rightarrow$ None|\\
      4 & \lstinline[
        language=Python,
        style={python-idle-code},
        basicstyle=\small
      ]|None|\\
    \end{tabularx}
  \end{table}
\end{frame}

%-------------------------------------------------------------------------------

\begin{frame}{Hashing}{Linked Lists}
  \textbf{Buckets as linked list:}
  \begin{itemize}
    \item
      {\color{Mittel-Blau}Best case}: $\mathcal{O}(1)$\\
      If table size and hash function chosen properly
    \item
      {\color{Mittel-Blau}Worst case}: $\mathcal{O}(n)$\\
      The table size is too small for the chosen key set\\
      (eg. 1 bucket)
  \end{itemize}
\end{frame}

%-------------------------------------------------------------------------------

\subsection{Open Addressing}

\begin{frame}{Hashing}{Open Addressing}
  \begin{itemize}
    \item
      For colliding keys we choose a new free entry
    \item
      Static, fixed number of elements
    \item
      An algorithm determines the new position of the entry
    \begin{itemize}
      \item
        \textbf{Insert:}
        We have to generate a sequence of new locations until a free spot is
        found
      \item
        \textbf{Lookup:}
        We have to iterate over all possible locations until the searched
        element or an empty spot is found
    \end{itemize}
  \end{itemize}
\end{frame}

%-------------------------------------------------------------------------------

\begin{frame}{Hashing}{Open Addressing}
  \textbf{Definitions:}
  \begin{itemize}
    \item[\color{Mittel-Blau}$h(s)$]
      Hash function for key {\color{Mittel-Blau}$s$}
    \item[\color{Mittel-Blau}$g(s,j)$]
      Sequence generator choosing a location for key {\color{Mittel-Blau}$s$}
      and the current try ${\color{Mittel-Blau}j} \in \{0,\dots,m-1\}$
    \item
      The sequence of probing is calculated by
      \begin{displaymath}
        {\color{Mittel-Blau}h(s,j)}
        = ({\color{Mittel-Blau}h(s)} - {\color{Mittel-Blau}g(s,j)})
        \mod m \in \{0,\dots,m-1\}
      \end{displaymath}
  \end{itemize}
\end{frame}

%-------------------------------------------------------------------------------

\codeslide{python}{
\begin{frame}{Hashing}{Open Addressing - Python}
  \lstinputlisting[
    language=Python,
    basicstyle=\small,
    tabsize=4,
    style={python-idle-code},
    breaklines=false,
    emph={insert, None},
    emphstyle=\color{blue}
  ]{Lecture/Code/OpenHashing_Insert.py}
\end{frame}

%-------------------------------------------------------------------------------

\begin{frame}{Hashing}{Open Addressing - Python}
  \lstinputlisting[
    language=Python,
    basicstyle=\small,
    tabsize=4,
    style={python-idle-code},
    breaklines=false,
    emph={lookup, None},
    emphstyle=\color{blue}
  ]{Lecture/Code/OpenHashing_Lookup.py}
\end{frame}
}

%-------------------------------------------------------------------------------

\codeslide{java}{
\begin{frame}{Hashing}{Open Addressing - Java}
  \lstinputlisting[
    language=Java,
    basicstyle=\scriptesize,
    tabsize=4,
    style={java-eclipse-code},
    breaklines=false,
    mathescape=true,
    escapechar={@},
    emph={j},
    emphstyle=\color{java_variable}
  ]{Lecture/Code/OpenHashing_Insert.java}
\end{frame}

%-------------------------------------------------------------------------------

\begin{frame}{Hashing}{Open Addressing - Java}
  \lstinputlisting[
    language=Java,
    basicstyle=\scriptesize,
    tabsize=4,
    style={java-eclipse-code},
    breaklines=false,
    mathescape=true,
    escapechar={@},
    emph={j},
    emphstyle=\color{java_variable}
  ]{Lecture/Code/OpenHashing_Lookup.java}
\end{frame}
}

%-------------------------------------------------------------------------------

\codeslide{cpp}{
\begin{frame}{Hashing}{Open Addressing - C++}
  \lstinputlisting[
    language=C++,
    basicstyle=\small,
    tabsize=2,
    style={cpp-eclipse-code},
    breaklines=false,
    escapechar={@}
  ]{Lecture/Code/OpenHashing_Insert.cpp}
\end{frame}

%-------------------------------------------------------------------------------

\begin{frame}{Hashing}{Open Addressing - C++}
  \lstinputlisting[
    language=C++,
    basicstyle=\small,
    tabsize=2,
    style={cpp-eclipse-code},
    breaklines=false,
    morekeywords={nullptr},
    escapechar={@}
  ]{Lecture/Code/OpenHashing_Lookup.cpp}
\end{frame}
}

%-------------------------------------------------------------------------------

\begin{frame}{Hashing}{Open Addressing - Linear Probing}
  \begin{figure}[!h]
    \begin{adjustbox}{width=0.75\linewidth}%
      \begin{adjustbox}{width=0.75\linewidth}%
\begin{tikzpicture}[
  bucket/.style={
    draw={Mittel-Blau},
    fill={white},
    line width=0.075em
  }, bucket_text/.style={
    draw=none,
    color=black
  }, arrow_taken/.style={
    draw={Mittel-Gruen},
    line width=0.125em
  }, arrow_shadow/.style={
    draw={Mittel-Gruen!40!white},
    line width=0.125em
  }, hash_text/.style={
    draw=none,
    color={Mittel-Blau}
  }, hash_text_final/.style={
    draw=none,
    color={Mittel-Blau}
  }
]%
\foreach \x/\xi in
    {\relax/0,${\color{Mittel-Blau}s}$/1,X/2,X/3,X/4,X/5,\relax/6}{
  \draw[bucket]
    (\xi, 0.0) rectangle (\xi + 1.0, 0.5);
  \draw (\xi + 0.5, 0.25) node[bucket_text] {\x};
  \draw (\xi + 0.5, 0.5) node[bucket_text, anchor=south] {\xi};
}

\foreach \type [count=\xi] in
    {shadow,taken,taken,taken,taken}{
  \draw (\xi + 0.45, 1.0) edge[arrow_\type, in=0, out=90] (\xi, 1.4);
  \draw (\xi, 1.4) edge[->, arrow_\type, in=90, out=180] (\xi - 0.45, 1.0);
}

\draw (0.45, 1.0) edge[arrow_shadow, in=0, out=90] (0, 1.4);
\draw (7.0, 1.4) edge[->, arrow_shadow, in=90, out=180] (6.55, 1.0);

\draw (1.5, -0.75) node[hash_text] {$h(s, 4)$};
\draw (5.5, -0.75) node[hash_text_final] {$h(s, 0)$};
\draw (1.5, -0.5) edge[->, arrow_taken, color={Mittel-Blau}] (1.5, -0.1);
\draw (5.5, -0.5) edge[->, arrow_taken] (5.5, -0.1);
\end{tikzpicture}%
\end{adjustbox}%%
    \end{adjustbox}
    \vspace{-1.0em}
    \caption{Linear probing sequence}%
    \label{fig:hashing:open_addressing:linear_probing}%
  \end{figure}
  \vspace{-1.0em}
  \begin{itemize}
    \item
      Check the element with lower index:
      \begin{math}
        {\color{Mittel-Blau}g(s,j)} := {\color{Mittel-Blau}j}
      \end{math}\\
      $\Rightarrow$ Hash function:
      \begin{math}
        {\color{Mittel-Blau}h(s,j)}
        = ({\color{Mittel-Blau}h(s)} - {\color{Mittel-Blau}j}) \mod m
      \end{math}
    \item
      This leads to the following sequence
      \begin{displaymath}
        {\color{Mittel-Blau}h(s)},
        {\color{Mittel-Blau}h(s)}-1,
        {\color{Mittel-Blau}h(s)}-2,
        \dots,
        \underbrace{0, m-1}_\text{clipping},
        m-2,
        \dots,
        {\color{Mittel-Blau}h(s)}+1
      \end{displaymath}
  \end{itemize}
\end{frame}

%-------------------------------------------------------------------------------

\begin{frame}{Hashing}{Open Addressing - Linear Probing}
  \begin{figure}[!h]
    \begin{adjustbox}{width=0.75\linewidth}%
      \begin{adjustbox}{width=0.75\linewidth}%
\begin{tikzpicture}[
  bucket/.style={
    draw={Mittel-Blau},
    fill={white},
    line width=0.075em
  }, bucket_text/.style={
    draw=none,
    color=black
  }, arrow_taken/.style={
    draw={Mittel-Gruen},
    line width=0.125em
  }, arrow_shadow/.style={
    draw={Mittel-Gruen!40!white},
    line width=0.125em
  }, hash_text/.style={
    draw=none,
    color={Mittel-Blau}
  }, hash_text_final/.style={
    draw=none,
    color={Mittel-Blau}
  }
]%
\foreach \x/\xi in
    {\relax/0,${\color{Mittel-Blau}s}$/1,X/2,X/3,X/4,X/5,\relax/6}{
  \draw[bucket]
    (\xi, 0.0) rectangle (\xi + 1.0, 0.5);
  \draw (\xi + 0.5, 0.25) node[bucket_text] {\x};
  \draw (\xi + 0.5, 0.5) node[bucket_text, anchor=south] {\xi};
}

\foreach \type [count=\xi] in
    {shadow,taken,taken,taken,taken}{
  \draw (\xi + 0.45, 1.0) edge[arrow_\type, in=0, out=90] (\xi, 1.4);
  \draw (\xi, 1.4) edge[->, arrow_\type, in=90, out=180] (\xi - 0.45, 1.0);
}

\draw (0.45, 1.0) edge[arrow_shadow, in=0, out=90] (0, 1.4);
\draw (7.0, 1.4) edge[->, arrow_shadow, in=90, out=180] (6.55, 1.0);

\draw (1.5, -0.75) node[hash_text] {$h(s, 4)$};
\draw (5.5, -0.75) node[hash_text_final] {$h(s, 0)$};
\draw (1.5, -0.5) edge[->, arrow_taken, color={Mittel-Blau}] (1.5, -0.1);
\draw (5.5, -0.5) edge[->, arrow_taken] (5.5, -0.1);
\end{tikzpicture}%
\end{adjustbox}%%
    \end{adjustbox}
    \vspace{-0.5em}
    \caption{Linear probing sequence}%
    \label{fig:hashing:open_addressing:linear_probing2}%
  \end{figure}
  \vspace{-1.5em}
  \begin{itemize}
    \item
      Can result in primary clustering
    \item
      Dealing with a hash collision will result in a higher probability of hash
      collisions in close entries
  \end{itemize}
\end{frame}

%-------------------------------------------------------------------------------

\begin{frame}{Hashing}{Open Addressing - Linear Probing}
  \textbf{Example:}
  \begin{itemize}
    \item
      Key universe:
      \begin{math}
        \mathbb{U} = \{12, 53, 5, 15, 2, 19\}
      \end{math}
    \item
      Hash function:
      \begin{math}
        {\color{Mittel-Blau}h(s,j)}
        = ({\color{Mittel-Blau}s} \mod 7 - {\color{Mittel-Blau}j}) \mod 7
      \end{math}
  \end{itemize}
  \hfill\\[0.5em]
  \textbf{Insertion sequence:}
  \begin{itemize}
    \item
      \lstinline[
        language=Python,
        style={python-idle-code},
        mathescape=true,
        basicstyle=\small
      ]|t.insert(12, "A")|$,\;$
      \begin{math}
        {\color{Mittel-Blau}h(12, 0)} = 5
      \end{math}
      \begin{figure}[!h]
        \def\LPEData{
          \relax/0,
          \relax/1,
          \relax/2,
          \relax/3,
          \relax/4,
          {12, {\color{green2}A}}/5,
          \relax/6
        }%
        \def\LPEShowIndex{1}%
        \begin{adjustbox}{width=0.75\linewidth}%
          \begin{tikzpicture}[
  bucket/.style={
    draw={Mittel-Blau},
    fill={white},
    line width=0.075em
  }, bucket_text/.style={
    draw=none,
    color=black
  }
]%
\foreach \x/\xi in \LPEData {
  \draw[bucket]
    (\xi, 0.0) rectangle (\xi + 1.0, 0.5);
  \draw (\xi + 0.5, 0.25) node[bucket_text] {\x};
  
  \ifnum \LPEShowIndex>0
    \draw (\xi + 0.5, 0.5) node[bucket_text, anchor=south] {\xi};
  \fi
}
\end{tikzpicture}%%
        \end{adjustbox}
        \vspace{-0.5em}%
%        \caption{Insertion sequence}%
        \label{fig:hashing:open_addressing:linear_probing_example}%
      \end{figure}
      \hfill\\[1.0em]
    \item
      \lstinline[
        language=Python,
        style={python-idle-code},
        mathescape=true,
        basicstyle=\small
      ]|t.insert(53, "B")|$,\;$
      \begin{math}
        {\color{Mittel-Blau}h(53, 0)} = 4
      \end{math}
      \begin{figure}[!h]
        \def\LPEData{
          \relax/0,
          \relax/1,
          \relax/2,
          \relax/3,
          {53, {\color{green2}B}}/4,
          {12, {\color{green2}A}}/5,
          \relax/6
        }%
        \def\LPEShowIndex{0}%
        \begin{adjustbox}{width=0.75\linewidth}%
          \begin{tikzpicture}[
  bucket/.style={
    draw={Mittel-Blau},
    fill={white},
    line width=0.075em
  }, bucket_text/.style={
    draw=none,
    color=black
  }
]%
\foreach \x/\xi in \LPEData {
  \draw[bucket]
    (\xi, 0.0) rectangle (\xi + 1.0, 0.5);
  \draw (\xi + 0.5, 0.25) node[bucket_text] {\x};
  
  \ifnum \LPEShowIndex>0
    \draw (\xi + 0.5, 0.5) node[bucket_text, anchor=south] {\xi};
  \fi
}
\end{tikzpicture}%%
        \end{adjustbox}
        \vspace{-0.5em}%
        \caption{Insertion sequence on a hash map}%
        \label{fig:hashing:open_addressing:linear_probing_example2}%
      \end{figure}
  \end{itemize}
\end{frame}

%-------------------------------------------------------------------------------

\begin{frame}{Hashing}{Open Addressing - Linear Probing}
  \textbf{Example:}
  \begin{itemize}
    \item
      Hash function:
      \begin{math}
        {\color{Mittel-Blau}h(s,j)}
        = ({\color{Mittel-Blau}s} \mod 7 - {\color{Mittel-Blau}j}) \mod 7
      \end{math}
  \end{itemize}
  \hfill\\[0.5em]
  \textbf{Insertion sequence:}
  \begin{itemize}
    \item
      \lstinline[
        language=Python,
        style={python-idle-code},
        mathescape=true,
        basicstyle=\small
      ]|t.insert(5, "C")|$,\;$
      \begin{math}
        {\color{Mittel-Blau}h(5, 0)} = 5 ,\;
        {\color{Mittel-Blau}h(5, 1)} = 4 ,\;
        {\color{Mittel-Blau}h(5, 2)} = 3
      \end{math}
      \begin{figure}[!h]
        \def\LPEData{
          \relax/0,
          \relax/1,
          \relax/2,
          {5, {\color{green2}C}}/3,
          {53, {\color{green2}B}}/4,
          {12, {\color{green2}A}}/5,
          \relax/6
        }%
        \def\LPEShowIndex{1}%
        \begin{adjustbox}{width=0.75\linewidth}%
          \begin{tikzpicture}[
  bucket/.style={
    draw={Mittel-Blau},
    fill={white},
    line width=0.075em
  }, bucket_text/.style={
    draw=none,
    color=black
  }
]%
\foreach \x/\xi in \LPEData {
  \draw[bucket]
    (\xi, 0.0) rectangle (\xi + 1.0, 0.5);
  \draw (\xi + 0.5, 0.25) node[bucket_text] {\x};
  
  \ifnum \LPEShowIndex>0
    \draw (\xi + 0.5, 0.5) node[bucket_text, anchor=south] {\xi};
  \fi
}
\end{tikzpicture}%%
        \end{adjustbox}
        \vspace{-0.5em}%
%        \caption{Insertion sequence}%
        \label{fig:hashing:open_addressing:linear_probing_example3}%
      \end{figure}
      \hfill\\[1.0em]
    \item
      \lstinline[
        language=Python,
        style={python-idle-code},
        mathescape=true,
        basicstyle=\small
      ]|t.insert(15, "D")|$,\;$
      \begin{math}
        {\color{Mittel-Blau}h(15, 0)} = 1
      \end{math}
      \begin{figure}[!h]
        \def\LPEData{
          \relax/0,
          {15, {\color{green2}D}}/1,
          \relax/2,
          {5, {\color{green2}C}}/3,
          {53, {\color{green2}B}}/4,
          {12, {\color{green2}A}}/5,
          \relax/6
        }%
        \def\LPEShowIndex{0}%
        \begin{adjustbox}{width=0.75\linewidth}%
          \begin{tikzpicture}[
  bucket/.style={
    draw={Mittel-Blau},
    fill={white},
    line width=0.075em
  }, bucket_text/.style={
    draw=none,
    color=black
  }
]%
\foreach \x/\xi in \LPEData {
  \draw[bucket]
    (\xi, 0.0) rectangle (\xi + 1.0, 0.5);
  \draw (\xi + 0.5, 0.25) node[bucket_text] {\x};
  
  \ifnum \LPEShowIndex>0
    \draw (\xi + 0.5, 0.5) node[bucket_text, anchor=south] {\xi};
  \fi
}
\end{tikzpicture}%%
        \end{adjustbox}
        \vspace{-0.5em}%
        \caption{Insertion sequence on a hash map}%
        \label{fig:hashing:open_addressing:linear_probing_example4}%
      \end{figure}
  \end{itemize}
\end{frame}

%-------------------------------------------------------------------------------

\begin{frame}{Hashing}{Open Addressing - Linear Probing}
  \textbf{Example:}
  \begin{itemize}
    \item
      Hash function:
      \begin{math}
        {\color{Mittel-Blau}h(s,j)}
        = ({\color{Mittel-Blau}s} \mod 7 - {\color{Mittel-Blau}j}) \mod 7
      \end{math}
  \end{itemize}
  \hfill\\[0.5em]
  \textbf{Insertion sequence:}
  \begin{itemize}
    \item
      \lstinline[
        language=Python,
        style={python-idle-code},
        mathescape=true,
        basicstyle=\small
      ]|t.insert(2, "E")|$,\;$
      \begin{math}
      {\color{Mittel-Blau}h(2, 0)} = 2
      \end{math}
      \begin{figure}[!h]
        \def\LPEData{
          \relax/0,
          {15, {\color{green2}D}}/1,
          {2, {\color{green2}E}}/2,
          {5, {\color{green2}C}}/3,
          {53, {\color{green2}B}}/4,
          {12, {\color{green2}A}}/5,
          \relax/6
        }%
        \def\LPEShowIndex{1}%
        \begin{adjustbox}{width=0.75\linewidth}%
          \begin{tikzpicture}[
  bucket/.style={
    draw={Mittel-Blau},
    fill={white},
    line width=0.075em
  }, bucket_text/.style={
    draw=none,
    color=black
  }
]%
\foreach \x/\xi in \LPEData {
  \draw[bucket]
    (\xi, 0.0) rectangle (\xi + 1.0, 0.5);
  \draw (\xi + 0.5, 0.25) node[bucket_text] {\x};
  
  \ifnum \LPEShowIndex>0
    \draw (\xi + 0.5, 0.5) node[bucket_text, anchor=south] {\xi};
  \fi
}
\end{tikzpicture}%%
        \end{adjustbox}
        \vspace{-0.5em}%
%        \caption{Insertion sequence on a hash map}%
        \label{fig:hashing:open_addressing:linear_probing_example6}%
      \end{figure}
      \hfill\\[1.0em]
    \item
      \lstinline[
        language=Python,
        style={python-idle-code},
        mathescape=true,
        basicstyle=\small
      ]|t.insert(19, "F")|$,\;$
      \begin{math}
        {\color{Mittel-Blau}h(19, 0)} = 5 ,\;
        {\color{Mittel-Blau}h(19, 1)} = 4 ,\;
      \end{math}\\
      \begin{math}
        \hspace{1.5em}
        {\color{Mittel-Blau}h(19, 2)} = 3 ,\;
        {\color{Mittel-Blau}h(19, 3)} = 2 ,\;
        {\color{Mittel-Blau}h(19, 4)} = 1 ,\;
        {\color{Mittel-Blau}h(19, 5)} = 0
      \end{math}
      \begin{figure}[!h]
        \def\LPEData{
          {19, {\color{green2}F}}/0,
          {15, {\color{green2}D}}/1,
          {2, {\color{green2}E}}/2,
          {5, {\color{green2}C}}/3,
          {53, {\color{green2}B}}/4,
          {12, {\color{green2}A}}/5,
          \relax/6
        }%
        \def\LPEShowIndex{0}%
        \begin{adjustbox}{width=0.75\linewidth}%
          \begin{tikzpicture}[
  bucket/.style={
    draw={Mittel-Blau},
    fill={white},
    line width=0.075em
  }, bucket_text/.style={
    draw=none,
    color=black
  }
]%
\foreach \x/\xi in \LPEData {
  \draw[bucket]
    (\xi, 0.0) rectangle (\xi + 1.0, 0.5);
  \draw (\xi + 0.5, 0.25) node[bucket_text] {\x};
  
  \ifnum \LPEShowIndex>0
    \draw (\xi + 0.5, 0.5) node[bucket_text, anchor=south] {\xi};
  \fi
}
\end{tikzpicture}%%
        \end{adjustbox}
        \vspace{-0.5em}%
        \caption{Insertion sequence on a hash map}%
        \label{fig:hashing:open_addressing:linear_probing_example7}%
      \end{figure}
  \end{itemize}
\end{frame}

%-------------------------------------------------------------------------------

\begin{frame}{Hashing}{Open Addressing - Squared Probing}
  \vspace{-2.0em}
  \begin{figure}[!h]
    \begin{adjustbox}{width=0.7\linewidth}%
      \begin{adjustbox}{width=0.7\linewidth}%
\begin{tikzpicture}[
  bucket/.style={
    draw={Mittel-Blau},
    fill={white},
    line width=0.075em
  }, bucket_text/.style={
    draw=none,
    color=black
  }, arrow_taken/.style={
    draw={Mittel-Gruen},
    line width=0.125em
  }, arrow_shadow/.style={
    draw={Mittel-Gruen!40!white},
    line width=0.125em
  }, hash_text/.style={
    draw=none,
    color={Mittel-Blau}
  }, hash_text_final/.style={
    draw=none,
    color={Mittel-Blau}
  }
]%
\foreach \x/\xi in {
  \relax/0,\relax/1,\relax/2,\relax/3,
  X/4,X/5,X/6,
  \relax/7,\relax/8,
  {\color{Mittel-Blau}s}/9,
  \relax/10,\relax/11
}{
  \draw[bucket] (\xi, 0.0) rectangle (\xi + 1.0, 0.5);
  \draw (\xi + 0.5, 0.25) node[bucket_text] {\x};
  \draw (\xi + 0.5, 0.5) node[bucket_text, anchor=south] {\xi};
}

\foreach \type/\from/\to/\spacea/\spaceb in {
  shadow/9/1/0.05/0.0,
  taken/4/9/-0.05/-0.05,
  taken/6/4/0.05/0.05,
  taken/5/6/0.05/-0.05
}{
  \draw
    (\from + 0.5 + \spacea, 1.0)
    edge[->, arrow_\type, in=90, out=90]
    (\to + 0.5 + \spaceb, 1.0);
}

\draw (9.5, -0.75) node[hash_text] {\large $h(s, 3)$};
\draw (5.5, -0.75) node[hash_text_final] {\large $h(s, 0)$};
\draw (9.5, -0.5) edge[->, arrow_taken, color={Mittel-Blau}] (9.5, -0.1);
\draw (5.5, -0.5) edge[->, arrow_taken] (5.5, -0.1);
\end{tikzpicture}%
\end{adjustbox}%%
    \end{adjustbox}
    \vspace{-1.0em}
    \caption{Squared probing sequence}%
    \label{fig:hashing:open_addressing:squared_probing}%
  \end{figure}
  \vspace{-1.0em}
  \textbf{Squared probing:}
  \begin{itemize}
    \item
      Motivation: Avoid local clustering
      \begin{displaymath}
        {\color{Mittel-Blau}g(s,j)}
        := (-1)^{\color{Mittel-Blau}j}
        \left\lceil \frac{{\color{Mittel-Blau}j}}{2} \right\rceil^2
      \end{displaymath}
    \item
      This leads to the following sequence
      \begin{displaymath}
        {\color{Mittel-Blau}h(s)} ,\;
        {\color{Mittel-Blau}h(s)}+1 ,\;
        {\color{Mittel-Blau}h(s)}-1 ,\;
        {\color{Mittel-Blau}h(s)}+4 ,\;
        {\color{Mittel-Blau}h(s)}-4 ,\;
        {\color{Mittel-Blau}h(s)}+9 ,\;
        {\color{Mittel-Blau}h(s)}-9 ,\;
        \dotsc
      \end{displaymath}
  \end{itemize}
  \vspace{1.0em}
\end{frame}

%-------------------------------------------------------------------------------

\begin{frame}{Hashing}{Open Addressing - Squared Probing}
  \textbf{Squared probing:}
  \begin{displaymath}
    {\color{Mittel-Blau}g(s,j)}
    := (-1)^{\color{Mittel-Blau}j}
    \left\lceil \frac{{\color{Mittel-Blau}j}}{2} \right\rceil^2
  \end{displaymath}
  \vspace{-1.0em}
  \begin{itemize}
    \item
      Alternatively:
      \begin{math}
        {\color{Mittel-Blau}g(s,j)}
        := c_1 \cdot {\color{Mittel-Blau}j}
        + c_2 \cdot {\color{Mittel-Blau}j}^2,
        \hspace{1.5em} c_1, c_2 \in \mathbb{N}
      \end{math}
    \item
      If ${\color{Mittel-Blau}m}$ is a prime number with the form
      \begin{math}
        {\color{Mittel-Blau}m} = 4 \cdot k + 3,
        \hspace{1.5em} k \in \mathbb{N}
      \end{math}
      \begin{itemize}
        \item
          The generated sequence is a permutation of all indices
        \item[$\Rightarrow$]
          All indices are visited but in a different order
      \end{itemize}
    \item
      No more primary clustering
    \item
      \textbf{Secondary clustering:}
      All keys with the same hash value produce the same permutation sequence
  \end{itemize}
\end{frame}

%-------------------------------------------------------------------------------

\begin{frame}{Hashing}{Open Addressing - Uniform Probing}
  \textbf{Uniform Probing:}
  \begin{itemize}
    \item
      Motivation: The function ${\color{Mittel-Blau}g(s,j)}$ uses only the step
      counter ${\color{Mittel-Blau}j}$ for linear and squared probing\\
      $\Rightarrow$ The sequence is independent of the {\color{Mittel-Blau}key}
    \item
      We use the key ${\color{Mittel-Blau}s}$ as factor for the generated
      sequence
    \item
      \textbf{Advantage:}
      Prevents clustering because keys with the same hash value do not produce
      the same diversion sequence
    \item
      \textbf{Disadvantage:}
      Hard to implement
  \end{itemize}
\end{frame}

%-------------------------------------------------------------------------------

\begin{frame}{Hashing}{Open Addressing - Double Hashing}
  \vspace{-2.0em}
  \begin{figure}[!h]
    \begin{adjustbox}{width=0.9\linewidth}%
      \begin{adjustbox}{width=0.9\linewidth}%
\begin{tikzpicture}[
  bucket/.style={
    draw={Mittel-Blau},
    fill={white},
    line width=0.075em
  }, bucket_text/.style={
    draw=none,
    color=black
  }, arrow_taken/.style={
    draw={Mittel-Gruen},
    line width=0.125em
  }, arrow_shadow/.style={
    draw={Mittel-Gruen!40!white},
    line width=0.125em
  }, arrow_double_shadow/.style={
    draw={Mittel-Gruen!40!white},
    densely dotted,
    line width=0.125em
  }, hash_text/.style={
    draw=none,
    color={Mittel-Blau}
  }, hash_text_final/.style={
    draw=none,
    color={Mittel-Blau}
  }
]%
\foreach \x/\xi in {
  \relax/0,
  \relax/1,
  X/2,
  \relax/3,
  X/4,
  X/5,
  X/6,
  \relax/7,
  X/8,
  X/9,
  \relax/10,
  {\color{Mittel-Blau}s}/11,
  \relax/12
}{
  \draw[bucket] (\xi, 0.0) rectangle (\xi + 1.0, 0.5);
  \draw (\xi + 0.5, 0.25) node[bucket_text] {\x};
  \draw (\xi + 0.5, 0.5) node[bucket_text, anchor=south] {\xi};
}

\foreach \type/\from/\to/\spacea/\spaceb in {
  double_shadow/1/4/0.05/-0.05,
  taken/2/5/0.05/-0.05,
  taken/5/8/0.05/-0.05,
  taken/8/11/0.05/-0.05
}{
  \draw
    (\from + 0.55, 1.0)
    edge[arrow_\type, in=180, out=90]
    (0.5*\from + 0.5*\to + 0.5, 2.0);
  \draw
    (0.5*\from + 0.5*\to + 0.5, 2.0)
    edge[->, arrow_\type, in=90, out=0]
     (\to + 0.45, 1.0);
}
\draw
  (11.55, 1.0)
  edge[arrow_shadow, in=180, out=90]
  (13.0, 2.0);
\draw
  (0.0, 2.0)
  edge[->, arrow_shadow, in=90, out=0]
  (1.45, 1.0);

\draw (11.5, -0.75) node[hash_text] {\large $h(s, 3)$};
\draw (2.5, -0.75) node[hash_text_final] {%
\large${\color{Mittel-Blau}h(s, 0)} = {\color{Mittel-Blau}h_1(s)}$%
};%
\draw (11.5, -0.5) edge[->, arrow_taken, color={Mittel-Blau}] (11.5, -0.1);
\draw (2.5, -0.5) edge[->, arrow_taken] (2.5, -0.1);


\draw[bucket, decorate, decoration={brace,amplitude=0.75em}, line width=0.125em]
  (2.5, 2.1) -- (5.5, 2.1)
  node [midway,yshift=1.50em] {\large ${\color{Mittel-Gruen}h_2(s)}$};
\draw[bucket, decorate, decoration={brace,amplitude=0.75em}, line width=0.125em]
  (5.5, 2.1) -- (8.5, 2.1)
  node [midway,yshift=1.50em] {\large ${\color{Mittel-Gruen}h_2(s)}$};
\end{tikzpicture}%
\end{adjustbox}%%
    \end{adjustbox}
    \vspace{-1.0em}
    \caption{Double hashing probing sequence}%
    \label{fig:hashing:open_addressing:double_hashing}%
  \end{figure}
  \vspace{-1.0em}
  \textbf{Double Hashing:}
  \begin{itemize}
    \item
      Motivation: Use the keys as influence for the diversion sequence
    \item
      Use two independent hash functions
      \begin{math}
        {\color{Mittel-Blau}h_1(s)},
        {\color{Mittel-Gruen}h_2(s)}
      \end{math}
    \item
      Hash function:
      \begin{math}
        {\color{Mittel-Blau}h(s,j)}
        = ({\color{Mittel-Blau}h_1(s)}
        + {\color{Mittel-Blau}j} \cdot {\color{Mittel-Gruen}h_2(s)})
        \mod m
      \end{math}
  \end{itemize}
\end{frame}

%-------------------------------------------------------------------------------

\begin{frame}{Hashing}{Open Addressing - Double Hashing}
  \textbf{Double Hashing:}
  \begin{itemize}
    \item
      Hash function:
      \begin{math}
        {\color{Mittel-Blau}h(s,j)}
        = ({\color{Mittel-Blau}h_1(s)}
        + {\color{Mittel-Blau}j} \cdot {\color{Mittel-Gruen}h_2(s)})
        \mod m
      \end{math}
    \item
      Diversion sequence:
      \begin{displaymath}
        {\color{Mittel-Blau}h_1(s)} ,\;
        {\color{Mittel-Blau}h_1(s)} + {\color{Mittel-Gruen}h_2(s)} ,\;
        {\color{Mittel-Blau}h_1(s)} + 2 \cdot {\color{Mittel-Gruen}h_2(s)} ,\;
        {\color{Mittel-Blau}h_1(s)} + 3 \cdot {\color{Mittel-Gruen}h_2(s)} ,\;
        \dotsc
      \end{displaymath}
    \item
      This method is only an approximation of uniform probing,
      but works well in practical use
  \end{itemize}
\end{frame}

%-------------------------------------------------------------------------------

\begin{frame}{Hashing}{Open Addressing - Double Hashing - Example}
  \textbf{Example:}
  \begin{itemize}
    \item
      The efficiency of double hashing is dependent on
      \begin{math}
        {\color{Mittel-Blau}h_1(s)} \neq {\color{Mittel-Gruen}h_2(s)}
      \end{math}
    \item
      The diversion sequence is dependent on the key ${\color{Mittel-Blau}s}$
  \end{itemize}
  \begin{align*}
    {\color{Mittel-Blau}h_1(s)} &= {\color{Mittel-Blau}s} \mod 7\\
    {\color{Mittel-Gruen}h_2(s)}
      &= \left({\color{Mittel-Blau}s} \mod 5\right) + 1\\
    {\color{Mittel-Blau}h(s,j)}
      &= {\color{Mittel-Blau}h_1(s)}
      + {\color{Mittel-Blau}j} \cdot {\color{Mittel-Gruen}h_2(s)}
      \mod 7
  \end{align*}
  \begin{table}[!h]
    \caption{Comparing both hash functions}
    \begin{tabular}{c|cccccc}
      {\color{Mittel-Blau}s} & 10 & 19 & 31 & 22 & 14 & 16\\
      \midrule
      ${\color{Mittel-Blau}h_1(s)}$ & 3 & 5 & 3 & 1 & 0 & 2\\
      ${\color{Mittel-Gruen}h_2(s)}$ & 1 & 5 & 2 & 3 & 5 & 2
    \end{tabular}
  \end{table}
\end{frame}

%-------------------------------------------------------------------------------

\begin{frame}{Hashing}{Open Addressing - Double Hashing - Optimization}
  \vspace{-2.0em}
  \begin{figure}[!h]
    \begin{adjustbox}{width=0.9\linewidth}%
      \begin{tikzpicture}[
  bucket/.style={
    draw={Mittel-Blau},
    fill={white},
    line width=0.075em
  }, bucket_text/.style={
    draw=none,
    color=black
  }, arrow_taken/.style={
    draw={Mittel-Gruen},
    line width=0.125em
  }, arrow_shadow/.style={
    draw={Mittel-Gruen!40!white},
    line width=0.125em
  }, arrow_double_shadow/.style={
    draw={Mittel-Gruen!40!white},
    densely dotted,
    line width=0.125em
  }, hash_text/.style={
    draw=none,
    color={Mittel-Blau}
  }, hash_text_final/.style={
    draw=none,
    color={Mittel-Blau}
  }
]%
\foreach \x/\xi in {
  \relax/0,
  \relax/1,
  ${\color{Mittel-Blau}s_1}$/2,
  \relax/3,
  \relax/4,
  X/5,
  \relax/6,
  \relax/7,
  X/8,
  \relax/9,
  \relax/10,
  X/11,
  \relax/12
}{
  \draw[bucket] (\xi, 0.0) rectangle (\xi + 1.0, 0.5);
  \draw (\xi + 0.5, 0.25) node[bucket_text] {\x};
  \draw (\xi + 0.5, 0.5) node[bucket_text, anchor=south] {\xi};
}

\foreach \type/\from/\to/\spacea/\spaceb\o in {
  taken/2/5/0.05/-0.05/3,
  taken/5/8/0.05/-0.05/4,
  taken/8/11/0.05/-0.05/5
}{
  \onslide<\o->{\draw
    (\from + 0.55, 1.0)
    edge[arrow_\type, in=180, out=90]
    (0.5*\from + 0.5*\to + 0.5, 2.0);
  \draw
    (0.5*\from + 0.5*\to + 0.5, 2.0)
    edge[->, arrow_\type, in=90, out=0]
     (\to + 0.45, 1.0);}
}
\onslide<6->{\draw
  (11.55, 1.0)
  edge[->, arrow_taken, in=180, out=90]
  (13.0, 2.0);}

\draw (2.45, 2.25) node[hash_text] {\large $h(s_1, 0)$};
\onslide<2->{\draw (2.5, -0.75) node[hash_text_final] {%
  \large%
  ${\color{Mittel-Gruen}h(s_2, 0)}$%
};}%
\onslide<3->{\draw (5.5, -0.75) node[hash_text_final] {%
  \large%
  ${\color{Mittel-Gruen}h(s_2, 1)}$%
};}%
\onslide<4->{\draw (8.5, -0.75) node[hash_text_final] {%
  \large%
  ${\color{Mittel-Gruen}h(s_2, 2)}$%
};}%
\onslide<5->{\draw (11.5, -0.75) node[hash_text_final] {%
  \large%
  ${\color{Mittel-Gruen}h(s_2, 2)}$%
};}%
\draw (2.45, 2.0) edge[->, arrow_taken, color={Mittel-Blau}] (2.45, 1.0);
\onslide<2->{\draw (2.5, -0.5) edge[->, arrow_taken] (2.5, -0.1);}
\onslide<3->{\draw (5.5, -0.5) edge[->, arrow_taken] (5.5, -0.1);}
\onslide<4->{\draw (8.5, -0.5) edge[->, arrow_taken] (8.5, -0.1);}
\onslide<5->{\draw (11.5, -0.5) edge[->, arrow_taken] (11.5, -0.1);}
\end{tikzpicture}%%
    \end{adjustbox}
    \vspace{-1.0em}
    \caption{Double hashing}%
    \label{fig:hashing:open_addressing:double_hashing_no_brent}%
  \end{figure}
  \textbf{Double hashing by Brent:}
  \begin{itemize}
    \item
      The order of the insertions has impact on efficiency of a lookup
    \item
      Different keys have different probing orders
  \end{itemize}
\end{frame}

%-------------------------------------------------------------------------------

\begin{frame}{Hashing}{Open Addressing - Double Hashing - Optimization}
  \vspace{-2.0em}
  \begin{figure}[!h]
    \begin{adjustbox}{width=0.9\linewidth}%
      \begin{tikzpicture}[
  bucket/.style={
    draw={Mittel-Blau},
    fill={white},
    line width=0.075em
  }, bucket_text/.style={
    draw=none,
    color=black
  }, arrow_taken/.style={
    draw={Mittel-Gruen},
    line width=0.125em
  }, arrow_shadow/.style={
    draw={Mittel-Gruen!40!white},
    line width=0.125em
  }, arrow_double_shadow/.style={
    draw={Mittel-Gruen!40!white},
    densely dotted,
    line width=0.125em
  }, hash_text/.style={
    draw=none,
    color={Mittel-Blau}
  }, hash_text_final/.style={
    draw=none,
    color={Mittel-Blau}
  }
]%
\foreach \x/\xi in {
  \relax/0,
  \relax/1,
  ${\color{Mittel-Blau}s_1}$/2,
  \relax/3,
  \relax/4,
  X/5,
  \relax/6,
  \relax/7,
  X/8,
  \relax/9,
  \relax/10,
  X/11,
  \relax/12
}{
  \draw[bucket] (\xi, 0.0) rectangle (\xi + 1.0, 0.5);
  \draw (\xi + 0.5, 0.25) node[bucket_text] {\x};
  \draw (\xi + 0.5, 0.5) node[bucket_text, anchor=south] {\xi};
}

\foreach \type/\from/\to/\spacea/\spaceb\o in {
  taken/2/5/0.05/-0.05/3,
  taken/5/8/0.05/-0.05/4,
  taken/8/11/0.05/-0.05/5
}{
  \onslide<\o->{\draw
    (\from + 0.55, 1.0)
    edge[arrow_\type, in=180, out=90]
    (0.5*\from + 0.5*\to + 0.5, 2.0);
  \draw
    (0.5*\from + 0.5*\to + 0.5, 2.0)
    edge[->, arrow_\type, in=90, out=0]
     (\to + 0.45, 1.0);}
}
\onslide<6->{\draw
  (11.55, 1.0)
  edge[->, arrow_taken, in=180, out=90]
  (13.0, 2.0);}

\draw (2.45, 2.25) node[hash_text] {\large $h(s_1, 0)$};
\onslide<2->{\draw (2.5, -0.75) node[hash_text_final] {%
  \large%
  ${\color{Mittel-Gruen}h(s_2, 0)}$%
};}%
\onslide<3->{\draw (5.5, -0.75) node[hash_text_final] {%
  \large%
  ${\color{Mittel-Gruen}h(s_2, 1)}$%
};}%
\onslide<4->{\draw (8.5, -0.75) node[hash_text_final] {%
  \large%
  ${\color{Mittel-Gruen}h(s_2, 2)}$%
};}%
\onslide<5->{\draw (11.5, -0.75) node[hash_text_final] {%
  \large%
  ${\color{Mittel-Gruen}h(s_2, 2)}$%
};}%
\draw (2.45, 2.0) edge[->, arrow_taken, color={Mittel-Blau}] (2.45, 1.0);
\onslide<2->{\draw (2.5, -0.5) edge[->, arrow_taken] (2.5, -0.1);}
\onslide<3->{\draw (5.5, -0.5) edge[->, arrow_taken] (5.5, -0.1);}
\onslide<4->{\draw (8.5, -0.5) edge[->, arrow_taken] (8.5, -0.1);}
\onslide<5->{\draw (11.5, -0.5) edge[->, arrow_taken] (11.5, -0.1);}
\end{tikzpicture}%%
    \end{adjustbox}
    \vspace{-1.0em}
    \caption{Double hashing}%
    \label{fig:hashing:open_addressing:double_hashing_no_brent2}%
  \end{figure}
  \vspace{-1.0em}
  \textbf{Example:}
  \begin{itemize}
    \item
      The key ${\color{Mittel-Blau}s_1}$ is inserted at position
      \begin{math}
        p_1 = {\color{Mittel-Blau}h(s_1, 0)}
      \end{math}
    \item
      The hash function for ${\color{Mittel-Gruen}s_2}$ also results in
      \begin{math}
        p_2 = {\color{Mittel-Gruen}h(s_2, 0)} = p_1
      \end{math}
    \item
      The locations
      \begin{math}
        {\color{Mittel-Gruen}h(s_2, j)}, \;
        {\color{Mittel-Blau}j} \in \{1,\dots,n\}
      \end{math}
      are also occupied
    \item
      If we insert ${\color{Mittel-Gruen}s_2}$ at position
      ${\color{Mittel-Gruen}h(s_2, n + 1)}$ the search will be inefficient
  \end{itemize}
\end{frame}

%-------------------------------------------------------------------------------

\begin{frame}{Hashing}{Open Addressing - Double Hashing - Optimization}
  \vspace{-2.0em}
  \begin{figure}[!h]
    \begin{adjustbox}{width=0.9\linewidth}%
      \begin{tikzpicture}[
  bucket/.style={
    draw={Mittel-Blau},
    fill={white},
    line width=0.075em
  }, bucket_text/.style={
    draw=none,
    color=black
  }, arrow_taken/.style={
    draw={Mittel-Gruen},
    line width=0.125em
  }, arrow_shadow/.style={
    draw={Mittel-Gruen!40!white},
    line width=0.125em
  }, arrow_double_shadow/.style={
    draw={Mittel-Gruen!40!white},
    densely dotted,
    line width=0.125em
  }, hash_text/.style={
    draw=none,
    color={Mittel-Blau}
  }, hash_text_final/.style={
    draw=none,
    color={Mittel-Blau}
  }
]%
\foreach \x/\xi in {
  \relax/0,
  \relax/1,
  ${\color{Mittel-Gruen}s_2}$/2,
  \relax/3,
  $\onslide<3->{\color{Mittel-Blau}s_1}$/4,
  X/5,
  \relax/6,
  \relax/7,
  X/8,
  \relax/9,
  \relax/10,
  X/11,
  \relax/12
}{
  \draw[bucket] (\xi, 0.0) rectangle (\xi + 1.0, 0.5);
  \draw (\xi + 0.5, 0.25) node[bucket_text] {\x};
  \draw (\xi + 0.5, 0.5) node[bucket_text, anchor=south] {\xi};
}

\onslide<3->{\draw
  (2.55, 1.0)
  edge[arrow_taken, in=180, out=90, color={Mittel-Blau}]
  (3.5, 2.0);
\draw
  (3.5, 2.0)
  edge[->, arrow_taken, in=90, out=0, color={Mittel-Blau}]
  (4.45, 1.0);}

\onslide<2->{\draw (2.45, 2.25) node[hash_text] {\large $h(s_1, 0)$};}
\draw (2.5, -0.75) node[hash_text_final] {%
  \large%
  ${\color{Mittel-Gruen}h(s_2, 0)}$%
};%
\onslide<3->{\draw (4.5, -0.75) node[hash_text_final] {%
  \large%
  $h(s_1, 1)$%
};}%
\onslide<2->{\draw (2.45, 2.0) edge[->, arrow_taken, color={Mittel-Blau}] (2.45, 1.0);}
\draw (2.5, -0.5) edge[->, arrow_taken] (2.5, -0.1);
\onslide<3->{\draw (4.5, -0.5) edge[->, arrow_taken, color={Mittel-Blau}] (4.5, -0.1);}
\end{tikzpicture}%%
    \end{adjustbox}
    \vspace{-1.0em}
    \caption{Double hashing by Brent}%
    \label{fig:hashing:open_addressing:double_hashing_brent}%
  \end{figure}
  \vspace{-1.0em}
  \textbf{Idea:}
  \begin{itemize}
    \item
      The key ${\color{Mittel-Blau}s_1}$ is inserted at position
      \begin{math}
        p_1 = {\color{Mittel-Blau}h(s_1, 0)}
      \end{math}
    \item
      The hash function for ${\color{Mittel-Gruen}s_2}$ also results in
      \begin{math}
        p_2 = {\color{Mittel-Gruen}h(s_2, 0)} = p_1
      \end{math}
    \item
      Test if location ${\color{Mittel-Blau}h(s_1, 1)}$ is free
    \item
      If yes, move ${\color{Mittel-Blau}s_1}$ from
      ${\color{Mittel-Blau}h(s_1, 0)}$ to ${\color{Mittel-Blau}h(s_1, 1)}$
      and insert ${\color{Mittel-Gruen}s_2}$ at
      ${\color{Mittel-Gruen}h(s_2, 0)}$
  \end{itemize}
\end{frame}

%-------------------------------------------------------------------------------

\begin{frame}{Hashing}{Open Addressing - Ordered Hashing}
  \textbf{Idea:}
  \begin{itemize}
    \item
      Insert colliding elements with a sorted order
    \item
      We can cancel a unsuccessful search for elements when using
      linear probing or double hashing
  \end{itemize}
  \hfill\\[0.5em]
  \textbf{Implementation:}
  \begin{itemize}
    \item
      Compare both keys if a collision occurs
    \item
      Insert the smaller key at $p_1$
    \item
      Search a position based on the diversion order for the bigger key
  \end{itemize}
\end{frame}

%-------------------------------------------------------------------------------

\begin{frame}{Hashing}{Open Addressing - Ordered Hashing}
  \textbf{Example:}
  \begin{itemize}
    \item
      The key ${\color{Mittel-Blau}12}$ is saved at position
      $p_1 = {\color{Mittel-Blau}h(12, 0)}$
    \item
      We insert the key ${\color{Mittel-Blau}5}$ into the hash map
    \item
      We assume ${\color{Mittel-Blau}h(5,0)}$ results in location $p_1$
    \item
      Because ${\color{Mittel-Blau}5} < {\color{Mittel-Blau}12}$ we insert
      the key ${\color{Mittel-Blau}5}$ at position $p_1$
    \item
      For the key ${\color{Mittel-Blau}12}$ we iterate through the deviation
      sequence
      \begin{displaymath}
        {\color{Mittel-Blau}h(12, 1)} ,\;
        {\color{Mittel-Blau}h(12, 2)} ,\;
        {\color{Mittel-Blau}h(12, 3)} ,\;
        \dots
      \end{displaymath}
  \end{itemize}
\end{frame}

%-------------------------------------------------------------------------------

\begin{frame}{Hashing}{Open Addressing - Robin-Hood Hashing}
  \textbf{Idea:}
  \begin{itemize}
    \item
      We want an equal distibution of search lengths
    \item
      Fitting the length of the probing order for all elements
    \item
      The sum of all search costs will stay the same
  \end{itemize}
  \hfill\\[0.5em]
  \textbf{Implementation:}
  \begin{itemize}
    \item
      If two keys ${\color{Mittel-Blau}s_1}, {\color{Mittel-Blau}s_2}$ collide
      (\begin{math}
        p_1 = {\color{Mittel-Blau}h(s_1, j_1)}
            = {\color{Mittel-Blau}h(s_2, j_2)}
      \end{math})
      we compare the length of the deviation sequence
      (\begin{math}
        {\color{Mittel-Blau}j_1} \;\text{or}\; {\color{Mittel-Blau}j_2}
      \end{math})
    \item
      The key with the bigger search sequence is inserted at $p_1$
    \item
      The other key is assigned a new location based on the deviation sequence
  \end{itemize}
\end{frame}

%-------------------------------------------------------------------------------

\begin{frame}{Hashing}{Open Addressing - Robin-Hood Hashing}
  \textbf{Example:}
  \begin{itemize}
    \item
      The key ${\color{Mittel-Blau}12}$ is saved at position
      $p_1 = {\color{Mittel-Blau}h(12, 7)}$
    \item
      We insert the key ${\color{Mittel-Blau}5}$ into the hash map
    \item
      We assume ${\color{Mittel-Blau}h(5,0)}$ results in location $p_1$
    \item
      Because
      \begin{math}
        {\color{Mittel-Blau}j_1} < {\color{Mittel-Blau}j_2} \;
        ({\color{Mittel-Blau}0} < {\color{Mittel-Blau}7})
      \end{math}
      the key ${\color{Mittel-Blau}12}$ stays at position $p_1$
    \item
      For the key ${\color{Mittel-Blau}5}$ we iterate through the deviation
      sequence
      \begin{displaymath}
        {\color{Mittel-Blau}h(5, 1)} ,\;
        {\color{Mittel-Blau}h(5, 2)} ,\;
        {\color{Mittel-Blau}h(5, 3)} ,\;
        \dots
      \end{displaymath}
  \end{itemize}
\end{frame}

%-------------------------------------------------------------------------------

\begin{frame}{Hashing}{Open Addressing - Implement Insert / Remove}
  \textbf{Problem:}
  \begin{itemize}
    \item
      The key ${\color{Mittel-Blau}s_1}$ is inserted at position $p_1$
    \item
      The key ${\color{Mittel-Blau}s_2}$ returns the same hash value,
      but is inserted at position $p_2$ because of the probing order
    \item
      If ${\color{Mittel-Blau}s_1}$ is removed, it is impossible to find
      ${\color{Mittel-Blau}s_2}$
  \end{itemize}
  \vspace{1.0em}
  \textbf{Solution:}
  \begin{itemize}
    \item
      \textbf{Remove:}
      Elements are marked as removed, but not deleted
    \item
      \textbf{Inserting:}
      Elements marked as removed will we overwritten
  \end{itemize}
\end{frame}

%-------------------------------------------------------------------------------

\subsection{Summary}

\begin{frame}{Hashing}{Open Addressing - Summary Collision Handling}
  \textbf{Bucket as linked list:}
  {\color{Mittel-Blau}(dynamic, number of elements variable)}
  \begin{itemize}
    \item
      Save colliding elements as linked list
  \end{itemize}
  \vspace{1.0em}
  \textbf{Open hashing:}
  {\color{Mittel-Blau}(static, number of elements fixed)}
  \begin{itemize}
    \item
      Determine a probing order, permutation of all hash values
    \item
      Linear, quadratic probing:
      \begin{itemize}
        \item
          Easy to implement
        \item
          Raise the probability of collisions because probing order does
          not depend on the key
      \end{itemize}
  \end{itemize}
\end{frame}

%-------------------------------------------------------------------------------

\begin{frame}{Hashing}{Open Addressing - Summary Collision Handling}
  \textbf{Open hashing:}
  {\color{Mittel-Blau}(static, number of elements fixed)}
  \begin{itemize}
    \item
      Uniform probing, double hashing:
      \begin{itemize}
        \item
          Different probing orders for different keys
        \item
          Avoids clustering of elements
      \end{itemize}
  \end{itemize}
  \vspace{1.0em}
  \textbf{Improving efficiency:}
  {\color{Mittel-Blau}(Brent, Ordered Hashing)}
  \begin{itemize}
    \item
      Improve search efficiency by sorting colliding insertions
      \begin{itemize}
        \item
          Abortion of unsuccessfull search
        \item
          Search sequence length balancing
      \end{itemize}
  \end{itemize}
\end{frame}

%-------------------------------------------------------------------------------

\begin{frame}{Hashing}{Open Addressing - Summary Hashing}
  \textbf{Hashing:}
  \begin{itemize}
    \item
      Efficient fo dictionary operations:\\
      \hspace{1.5em}\texttt{Insert:} $\mathcal{O}(1) \dots \mathcal{O}(n)$\\
      \hspace{1.5em}\texttt{Search:} $\mathcal{O}(1) \dots \mathcal{O}(n)$\\
      \hspace{1.5em}\texttt{Remove:} $\mathcal{O}(1) \dots \mathcal{O}(n)$
    \item
      Direct access of all elements in a hash table $(\mathcal{O}(1))$
    \item
      Using a hash function to find the position (hash value) in the hash table
  \end{itemize}
\end{frame}

%-------------------------------------------------------------------------------

\begin{frame}{Hashing}{Open Addressing - Summary Hashing}
  \textbf{Hashing:}
  \begin{itemize}
    \item
      Same hash values for different keys results in hash collisions
    \item
      Efficiency is affected by:
      \begin{itemize}
        \item
          Hash function(s)
        \item
          Size of hash table
        \item
          Strategy to prevent hash collision
      \end{itemize}
  \end{itemize}
\end{frame}