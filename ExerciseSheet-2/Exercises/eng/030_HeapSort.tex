\exercise{3} \points{10}\\
Implement the \textit{Heapsort} algorithm as described in the first two
lectures.
In order to make the algorithm work \enquote{in-place}
(i.e. the output is in the same array as the input)
you have to implement a max heap where you can
swap the largest element to the end of the unsorted array part
(as seen in the YouTube video from lecture 1).
Write a unit test for each method.  Each test should check at least one
non-trivial input example.  Also test if your algorithm e.g. works with an
even or uneven number of elements. If critical boundary cases
can be checked easily (e.g. empty input case) please check those too.
Generate a \textit{Heapsort} runtime plot T(n) with different input sizes n
as shown in the lecture for\textit{MinSort}.
Choose appropriate sizes to get a meaningful plot with enough data points and
reasonable runtime (at most 1 min).\\\vspace{3em}

Hints: TODO