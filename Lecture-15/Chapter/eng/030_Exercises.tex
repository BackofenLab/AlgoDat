\def\ExerciseSheetExercise{Exercise}
\def\ExerciseSheetPoints{points}

\newcommand{\exercise}[1]{\par\vspace{3mm}{\bf
    \ExerciseSheetExercise{\hspace{1mm}}#1}\hspace{1mm}}
\newcommand{\points}[1]{\hspace{1mm}(#1 \ExerciseSheetPoints)\hspace{1mm}}

%-------------------------------------------------------------------------------

\begin{frame}{Example exam task}
  \exercise{1} Binary Heap \points{5}\\
  In the following binary min-heap, the element 42 is reduced to 3.
  Which operations are necessary to repair the heap?
  Draw the operations as arrows to the linear array,
  and write down the resulting heap as linear array.
  \begin{displaymath}
    \left[ 1, 4, 6, 6, 7, 7, 9, 13, 42, 12, 14, 8, 15, 11, 10, 15,
      14, 43 \right]
  \end{displaymath}
\end{frame}

%-------------------------------------------------------------------------------

\begin{frame}{Example exam task}
  \exercise{2} Binary Heap \points{5}\\
  Give all representations of $\{1, 2, 3, 4\}$ as a binary min-heap.
  Draw the trees and the corresponding linear arrays.
\end{frame}
%-------------------------------------------------------------------------------

\codeslide{python}{
\begin{frame}{Example exam task}
  \exercise{3} Algorithm \points{8}\\
  What is computed in the following algorithm?\\
  What is the runtime complexity of the improved algorithm?\\
  (Hint: divide and conquer does not help here)
  \lstinputlisting[
    language=Python,
    basicstyle=\small,
    tabsize=4,
    style={python-idle-code},
    escapechar={@},
    emph={func},
    emphstyle=\color{blue}
  ]{Code/Exam/example1.py}
\end{frame}
}