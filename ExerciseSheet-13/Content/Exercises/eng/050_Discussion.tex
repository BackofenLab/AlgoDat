\exercise{4} \points{5}\\
4.1) Write a short program \textit{RoutePlannerMain}, to calculate on the Graph 
\emph{bawue\_bayern.graph}
\begin{enumerate} \itemsep-2pt
	\item the shortest path
	\item the fastest traveling time by car (max. speed 130 km/h)
	\item the fastest traveling time by tuned motorbike (max. speed 100 km/h) 
\end{enumerate}
between the Faculty of Engineering (5508637 nodes) and the train main station 
of Nürnberg (4435496 nodes). 
Before that, set the edge-costs by using the methods 
\emph{setArcCostToDistance()} and \emph{setArcCostToTravelTime()} from exercise 
1. Provide the distance in km, the travel time in hours and the compute-time 
(without reading the graph). Moreover, please provide the 3 routes in 
\emph{MapBBCode}-format, to visualize the routes (example see below).

4.2) Find the node in the graph which is most far away from the technical 
faculty in regard of the travel time per car and motorbike. Please provide 
again the distance in km, the travel time in hours, the compute-time and the 
\emph{MapBBCode}.

Example \emph{MapBBCode}:
{\small
	\begin{verbatim}
	[map]
	<lat11>,<lon11> <lat12>,<lon12> ... <lat1n>,<lon1n>(blue|label1);
	<lat21>,<lon21> <lat22>,<lon22> ... <lat2n>,<lon2n>(red|label2);
	<lat31>,<lon31> <lat32>,<lon32> ... <lat3n>,<lon3n>(green|label3);
	[/map]
	\end{verbatim}
	
	Every route exists of a pair of coordinates  \emph{<latitude>,<longitude>} 
	separated by a comma.
	Coordinate pairs are separated by a space character. Routes are separated by 
	an semicolon.
	After the last coordinate pair optional parameters can be given in the form 
	of: \emph{(color|label)}.
	This can be used to provide colors or labels. Mark the route 1 as blue, the 
	route 2 as red and route 3 as green. The generated code can be visualized on 
	\emph{http://share.mapbbcode.org/}. \\[10pt] Please edit the table, which is 
	linked from the homepage, especially the calculated values and the URL 
	pointing to the visualized routes on \emph{http://share.mapbbcode.org/}.