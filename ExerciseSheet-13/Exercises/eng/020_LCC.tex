\exercise{1} \points{5} \textit{Largest Connected Component}\\
On the website you can find the \emph{graph.py} file containing the \texttt{Graph}
class and some already implemented methods (e.g. for reading in the \emph{.graph}
file). Have a look at the already supplied functionalities. Based on these, implement
the method \texttt{compute\_lcc} which calculates all connected components and marks
the nodes in the largest connected component (= those with the highest count of
marked nodes). For this you can utilize the already implemented method
\texttt{compute\_reachable\_nodes} inside the class for the computation of reachable
nodes for a given node. Be aware that you might need to adapt the output of
the method for this exercise.
Also make sure to pass \texttt{False} as second parameter to \texttt{read\_graph\_from\_file} to load the graph with undirected edges.

