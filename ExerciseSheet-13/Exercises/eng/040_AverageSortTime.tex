\exercise{3} \points{5} \textit{Route Planner}\\
4.1) Write a short program \textit{route\_planner.py}, which calculates on the Graph 
\emph{bawue\_bayern.graph} found on the course website 
\begin{enumerate} \itemsep-2pt
	\item the shortest path
	\item the fastest traveling time by car (max. speed 130 km/h)
	\item the fastest traveling time by tuned moped (max. speed 100 km/h) 
\end{enumerate}
between the Faculty of Engineering (node ID 5508637) and the central railway station 
of Nürnberg (node ID 4435496).
To switch between shortest path and fastest traveling time, set the edge costs by using the methods 
\emph{set\_arc\_costs\_to\_distance()} and \emph{set\_arc\_costs\_to\_travel\_time()} 
included in \emph{graph.py}. Provide the distance in km, the travel time in hours 
and minutes, and the computation time (excluding reading in the graph). 

In addition, you can convert the 3 routes into the 
\emph{MapBBCode} format to visualize them (example see below).

4.2) Find the node in the graph which is furthest away from the technical 
faculty in regard of the travel time per car and moped. Please provide 
again the distance in km, the travel time in hours and minutes, and the computation 
time (and optionally the \emph{MapBBCode}).

Example \emph{MapBBCode}:
{\small
	\begin{verbatim}
	[map]
	<lat11>,<lon11> <lat12>,<lon12> ... <lat1n>,<lon1n>(blue|label1);
	<lat21>,<lon21> <lat22>,<lon22> ... <lat2n>,<lon2n>(red|label2);
	<lat31>,<lon31> <lat32>,<lon32> ... <lat3n>,<lon3n>(green|label3)
	[/map]
	E.g.
	[map]47.7811,8.34618 47.7812,8.34682(blue|label1)[/map]
	\end{verbatim}
	
	Every route consists of a pairs of coordinates \emph{<latitude>,<longitude>} 
	with each pair separated by a comma.
	Coordinate pairs are separated by a space character. Routes are separated by 
	a semicolon (not after the last or only one).
	After the last coordinate pair, optional parameters can be given, e.g. in the form 
	of \emph{(color|label)}, which provides colors and labels to the routes. 
	Mark route 1 as blue, route 2 as red and route 3 as green. 
	The generated code can be visualized on \emph{http://share.mapbbcode.org/}.
