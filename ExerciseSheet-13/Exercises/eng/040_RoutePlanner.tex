\exercise{3} \points{5} \textit{Route Planner}\\
% Original openstreetmap nodes:
%     Faculty of Engineering: 335711003
%     Freiburg SC Stadion: 28340550
4.1)
Write a short program \emph{route\_planner.py}, which calculates the route between the Faculty of Engineering (node ID 95466) and the SC Freiburg stadion (node ID 136096) regarding the following ascpects:

\begin{enumerate} \itemsep-2pt
	\item Shortest path (max. speed 300 km/h)
	\item Fastest traveling time by car (max. speed 130 km/h)
	\item Fastest traveling time by moped (max. speed 50 km/h)
\end{enumerate}

The necessary graph data \emph{freiburg.graph} can found on the course website.
Load the data by passing \texttt{True} as second parameter to \texttt{read\_graph\_from\_file} to load the graph with directed edges.
To switch between shortest path and fastest traveling time, set the edge costs by using the methods \texttt{set\_arc\_costs\_to\_distance()} and \texttt{set\_arc\_costs\_to\_travel\_time()} included in \texttt{graph.py}.
Provide the distance in km, the travel time in hours and minutes, and the computation time in seconds and milliseconds (excluding reading in the graph).

In addition, you can convert the 3 routes into the \emph{MapBBCode} format to visualize them (example see below).

4.2)
Find the node in the graph which is furthest away from the technical
faculty in regard of the travel time per car and moped.
Please provide again the distance in km, the travel time in hours and minutes, and the computation time (and \textbf{optionally} the \emph{MapBBCode}).

Example \emph{MapBBCode}:
\begin{center}
  \small\begin{verbatim}
    [map]
      <lat11>,<lon11> <lat12>,<lon12> ... <lat1n>,<lon1n>(blue|label1);
      <lat21>,<lon21> <lat22>,<lon22> ... <lat2n>,<lon2n>(red|label2);
      <lat31>,<lon31> <lat32>,<lon32> ... <lat3n>,<lon3n>(green|label3)
    [/map]

    Example:
    [map]47.7811,8.34618 47.7812,8.34682(blue|Some route)[/map]
  \end{verbatim}
\end{center}

Every route consists of a pairs of coordinates \emph{<latitude>,<longitude>}
with each pair separated by a comma.
Coordinate pairs are separated by a space character. Routes are separated by
a semicolon (not after the last or only one).
After the last coordinate pair, optional parameters can be given, e.g. in the form of \emph{(color|label)}, which provides colors and labels to the routes.
Mark route 1 as blue, route 2 as red and route 3 as green.
The generated code can be visualized on \emph{http://share.mapbbcode.org/}.
