\exercise{1} PriorityQueue \points{20}\\
Implement the class \textit{PriorityQueue} with the following methods
(as described in the lecture):
\begin{itemize}
  \item\textit{insert} \points{6}
  \item\textit{getMin} \points{1}
  \item\textit{deleteMin} \points{6}
  \item\textit{changeKey} \points{6}
  \item\textit{size} \points{1}
\end{itemize}
For the precise specification of these methods
(in particular the border cases)
as well as useful helper methods you can find two pseudo code files as 
suggestions on the course website.
If you do not have any prior experience with C++ templates or Java generics
programming or do not want to use them,
you can also use the string data type in \textit{PriorityQueueItem}
(instead of templates / gGenerics).
As usual write a unit test for each of these methods.
As mentioned before, the best way to do so is to first write the test and to
implement the method afterwards.
It is fine to test all methods together in one test function.
Notice that you can reuse parts of your own code from the second exercise sheet
(HeapSort) or from the code in the provided solutions
(you can find them inside Daphne).
If you do so, please copy the code and do not try to incorporate code from
\textit{uebungsblatt\_02} directly.