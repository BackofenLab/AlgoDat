\exercise{1} PriorityQueue \points{20}\\
Implement the class \textit{PriorityQueueMinHeap} with the following methods
(as described in the lecture):
\begin{itemize}
  \item\textit{insert} \points{6}
  \item\textit{get\_min} (1 point)
  \item\textit{delete\_min} \points{6}
  \item\textit{change\_key} \points{6}
  \item\textit{size} (1 point)
\end{itemize}

You can find a Python3 template file on the website, which also includes some additional hints. 
As usual write unit tests for all important methods, and try to first write some tests 
before implementing the functionalities.

\textbf{Hints:}
\begin{itemize}
\item The \emph{insert()} method creates an item object (\emph{PriorityQueueItem} class included in template), appends it to the priority queue list and calls the \emph{repair\_heap\_up()} method
\item When you swap two items, do not forget to also swap their list indices
\item Mark member variables and methods that you only use inside the class as private, using an underscore before the name (see template file for examples)
\end{itemize}
