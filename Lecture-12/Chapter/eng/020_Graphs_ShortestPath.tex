\begin{frame}{Graphs}{Degrees (Valency)}
  \textbf{Degree of a vertex:} Undirected graph:
  {\color{Mittel-Blau}$G = (V , E)$}
  \begin{itemize}
    \item<2->
      {\color{Mittel-Blau}Degree} of a vertex {\color{Mittel-Blau}$u$}
      is the number of {\color{Mittel-Blau}vertices} adjacent to the
      vertex
      \begin{displaymath}
        \mathrm{deg}({\color{Mittel-Blau}u}) =
        \left\vert \{
          \{v, {\color{Mittel-Blau}u}\}
          : \; \{v, {\color{Mittel-Blau}u}\} \in {\color{Mittel-Blau}E} \} 
        \right\vert
      \end{displaymath}
  \end{itemize}
  \onslide<3->
  \begin{figure}
    \begin{adjustbox}{width=0.325\linewidth}
      \begin{tikzpicture}[
  vertice/.style={
    circle,
    draw=Mittel-Blau,
    color=Mittel-Blau,
    inner sep=0em,
    minimum size=1.75em,
    line width=0.1em,
    font=\large
  }, edge/.style={
    draw=Mittel-Gruen,
    line width=0.2em
  }
]%

\draw (0.0, 0.0) node[vertice] (vert0) {};

\draw[edge] (vert0) -- (30:4em);
\draw[edge] (vert0) -- (0:4em);
\draw[edge] (vert0) -- (330:4em);

\draw[edge] (vert0) -- (180:4em);
\end{tikzpicture}%
    \end{adjustbox}
    \caption{Vertex with degree of 4}
    \label{fig:graph:degree_undirected}
  \end{figure}
\end{frame}

%-------------------------------------------------------------------------------

\begin{frame}{Graphs}{Degrees (Valency)}
  \textbf{Degree of a vertex:} Directed graph:
  {\color{Mittel-Blau}$G = (V , E)$}
  \begin{itemize}
    \item<2->
     {\color{Mittel-Blau}Indegree} of a vertex {\color{Mittel-Blau}$u$}
     is the number of {\color{Mittel-Blau}edge heads} adjacent to the
     vertex
     \begin{displaymath}
       \mathrm{deg}^{+}({\color{Mittel-Blau}u}) =
       \left\vert \{
         (v, {\color{Mittel-Blau}u})
         : \; (v, {\color{Mittel-Blau}u}) \in {\color{Mittel-Blau}E} \} 
       \right\vert
      \end{displaymath}
    \item<3->
      {\color{Mittel-Blau}Outdegree} of a vertex {\color{Mittel-Blau}$u$}
      is the number of {\color{Mittel-Blau}edge tails} adjacent to the
      vertex
      \begin{displaymath}
        \mathrm{deg}^{-}({\color{Mittel-Blau}u}) =
        \left\vert \{
          ({\color{Mittel-Blau}u}, v)
          : \; ({\color{Mittel-Blau}u}, v) \in {\color{Mittel-Blau}E} \}
        \right\vert
      \end{displaymath}
  \end{itemize}
  \onslide<4->
  \begin{figure}
    \begin{adjustbox}{width=0.325\linewidth}
      \begin{tikzpicture}[
  vertice/.style={
    circle,
    draw=Mittel-Blau,
    color=Mittel-Blau,
    inner sep=0em,
    minimum size=1.75em,
    line width=0.1em,
    font=\large
  }, edge/.style={
    draw=Mittel-Gruen,
    line width=0.2em
  }, edge_arrow/.style={
    edge,
    ->
  }
]%

\draw (0.0, 0.0) node[vertice] (vert0) {};

\draw[edge_arrow] (vert0) -- (20:4em);
\draw[edge_arrow] (vert0) -- (340:4em);

\draw[edge_arrow, <-] (vert0) -- (150:4em);
\draw[edge_arrow, <-] (vert0) -- (180:4em);
\draw[edge_arrow, <-] (vert0) -- (210:4em);
\end{tikzpicture}%
    \end{adjustbox}
    \caption{Vertex with in- / outdegree of 3 / 2}
    \label{fig:graph:degree_directed}
  \end{figure}
\end{frame}

%-------------------------------------------------------------------------------

\begin{frame}{Graphs}{Paths}
  \textbf{Paths in a graph:}
  {\color{Mittel-Blau}$G = (V , E)$}
  \begin{itemize}
    \item<2->
      A path of {\color{Mittel-Blau}$G$} is a sequence of edges
      \begin{math}
        {\color{Mittel-Blau}u_1},
        {\color{Mittel-Blau}u_2},
        \dots,
        {\color{Mittel-Blau}u_i} \in {\color{Mittel-Blau}V}
      \end{math}
      with
      \begin{itemize}
        \item<3->
          Undirected graph:
          \begin{math}
            \{{\color{Mittel-Blau}u_1}, {\color{Mittel-Blau}u_2}\},
            \{{\color{Mittel-Blau}u_2}, {\color{Mittel-Blau}u_3}\},
            \dots,
            \{{\color{Mittel-Blau}u_{i-1}}, {\color{Mittel-Blau}u_i}\} \in
            {\color{Mittel-Blau}E}
          \end{math}
        \item<4->
          Directed graph:
          \begin{math}
          ({\color{Mittel-Blau}u_1}, {\color{Mittel-Blau}u_2}),
          ({\color{Mittel-Blau}u_2}, {\color{Mittel-Blau}u_3}),
          \dots,
          ({\color{Mittel-Blau}u_{i-1}}, {\color{Mittel-Blau}u_i}) \in
          {\color{Mittel-Blau}E}
          \end{math}
      \end{itemize}
  \end{itemize}
  \onslide<5->
  \begin{columns}
    \begin{column}[b]{0.5\linewidth}
      \begin{figure}
        \begin{adjustbox}{width=0.95\linewidth}
          \begin{tikzpicture}[
  vertice/.style={
    circle,
    draw=Mittel-Blau,
    color=Mittel-Blau,
    inner sep=0em,
    minimum size=1.75em,
    line width=0.1em,
    font=\large
  }, edge/.style={
    draw=Mittel-Gruen,
    line width=0.2em
  }, edge_highlight/.style={
    edge,
    preaction={
      draw,yellow,-,% Draw yellow without any arrow head
      double=yellow,
      double distance=2\pgflinewidth,
    }
  }
]%
\draw (-2.0, -1.0) node[vertice] (vert0) {0};
\draw (1.5, -0.5) node[vertice] (vert1) {1};
\draw (4.0, -1.5) node[vertice] (vert2) {2};
\draw (1.5, -2.5) node[vertice] (vert3) {3};
\draw (4.5, 1.5) node[vertice] (vert4) {4};

\draw[edge] (vert0) -- (vert1);
\draw[edge_highlight] (vert0) -- (vert3);

\draw[edge] (vert1) -- (vert2);
\draw[edge] (vert1) -- (vert4);

\draw[edge_highlight] (vert2) -- (vert4);
\draw[edge_highlight] (vert2) -- (vert3);
\draw[edge, out=0, in=270, looseness=5]
  (vert2.east) to (vert2.south);

\draw[edge] (vert3) -- (vert1);
\end{tikzpicture}%
        \end{adjustbox}
        \caption{{\color{Mittel-Blau}Undirected path} of length 3 \newline
          ${\color{Mittel-Blau}P} = (0, 3, 2, 4)$}
        \label{fig:graphs:undirected_path}
      \end{figure}
    \end{column}
    \begin{column}[b]{0.5\linewidth}
      \begin{figure}
        \begin{adjustbox}{width=0.95\linewidth}
          \begin{tikzpicture}[
  vertice/.style={
    circle,
    draw=Mittel-Blau,
    color=Mittel-Blau,
    inner sep=0em,
    minimum size=1.75em,
    line width=0.1em,
    font=\large
  }, edge/.style={
    draw=Mittel-Gruen,
    line width=0.2em
  }, edge_arrow/.style={
    edge,
    ->
  }, edge_highlight/.style={
    edge_arrow,
    preaction={
      draw,yellow,-,% Draw yellow without any arrow head
      double=yellow,
      double distance=2\pgflinewidth,
    }
  }
]%
\draw (-2.0, -1.0) node[vertice] (vert0) {0};
\draw (1.5, -0.5) node[vertice] (vert1) {1};
\draw (4.0, -1.5) node[vertice] (vert2) {2};
\draw (1.5, -2.5) node[vertice] (vert3) {3};
\draw (4.5, 1.5) node[vertice] (vert4) {4};

\draw[edge_arrow] (vert0) -- (vert1);
\draw[edge_highlight] (vert0) -- (vert3);

\draw[edge_arrow] (vert1) -- (vert2);
\draw[edge_highlight, bend left=10] (vert1) to (vert4);

\draw[edge_arrow] (vert2) -- (vert4);
\draw[edge_arrow] (vert2) -- (vert3);
\draw[edge_arrow, out=0, in=270, looseness=5]
  (vert2.east) to (vert2.south);

\draw[edge_arrow, edge_highlight] (vert3) -- (vert1);
\draw[edge_arrow, bend left=10] (vert4) to (vert1);
\end{tikzpicture}%
        \end{adjustbox}
        \caption{{\color{Mittel-Blau}Directed path} of length 3 \newline
          ${\color{Mittel-Blau}P} = (0, 3, 1, 4)$}
        \label{fig:graphs:directed_path}
      \end{figure}
    \end{column}
  \end{columns}
\end{frame}

%-------------------------------------------------------------------------------

\begin{frame}{Graphs}{Paths}
  \textbf{Paths in a graph:}
  {\color{Mittel-Blau}$G = (V , E)$}
  \begin{itemize}
    \item<2->
      The {\color{Mittel-Blau}length of a path} is:
      (also costs of a path)
      \begin{itemize}
        \item<3->
          Without weights:
          {\color{Mittel-Blau}number of edges} taken
        \item<4->
          With weights:
          {\color{Mittel-Blau}sum of weigths of edges} taken
      \end{itemize}
  \end{itemize}
  \onslide<5->
  \begin{columns}
    \begin{column}[b]{0.5\linewidth}
      \begin{figure}
        \begin{adjustbox}{width=0.95\linewidth}
          \begin{tikzpicture}[
  vertice/.style={
    circle,
    draw=Mittel-Blau,
    color=Mittel-Blau,
    inner sep=0em,
    minimum size=1.75em,
    line width=0.1em,
    font=\large
  }, edge/.style={
    draw=Mittel-Gruen,
    line width=0.2em
  }, edge_arrow/.style={
    edge,
    ->
  }, edge_highlight/.style={
    edge_arrow,
    preaction={
      draw,yellow,-,% Draw yellow without any arrow head
      double=yellow,
      double distance=2\pgflinewidth,
    }
  }
]%
\draw (-2.0, -1.0) node[vertice] (vert0) {0};
\draw (1.5, -0.5) node[vertice] (vert1) {1};
\draw (4.0, -1.5) node[vertice] (vert2) {2};
\draw (1.5, -2.5) node[vertice] (vert3) {3};
\draw (4.5, 1.5) node[vertice] (vert4) {4};

\draw[edge_arrow] (vert0) -- (vert1);
\draw[edge_highlight] (vert0) -- (vert3);

\draw[edge_arrow] (vert1) -- (vert2);
\draw[edge_highlight, bend left=10] (vert1) to (vert4);

\draw[edge_arrow] (vert2) -- (vert4);
\draw[edge_arrow] (vert2) -- (vert3);
\draw[edge_arrow, out=0, in=270, looseness=5]
  (vert2.east) to (vert2.south);

\draw[edge_arrow, edge_highlight] (vert3) -- (vert1);
\draw[edge_arrow, bend left=10] (vert4) to (vert1);
\end{tikzpicture}%
        \end{adjustbox}
        \caption{{\color{Mittel-Blau}Directed path} of length 3 \newline
          ${\color{Mittel-Blau}P} = (0, 3, 1, 4)$}
        \label{fig:graphs:directed_path_length}
      \end{figure}
    \end{column}
    \begin{column}[b]{0.5\linewidth}
      \begin{figure}
        \begin{adjustbox}{width=0.8\linewidth}
          \begin{tikzpicture}[
  vertice/.style={
    circle,
    draw=Mittel-Blau,
    color=Mittel-Blau,
    inner sep=0em,
    minimum size=1.75em,
    line width=0.1em,
    font=\large
  }, edge/.style={
    draw=Mittel-Gruen,
    line width=0.2em
  }, edge_arrow/.style={
    edge,
    ->
  }, edge_cost/.style={
    midway,
    color=Hell-Gruen,
    font=\Large
  }, edge_highlight/.style={
    edge_arrow,
    preaction={
      draw,yellow,-,% Draw yellow without any arrow head
      double=yellow,
      double distance=2\pgflinewidth,
    }
  }
]%
\draw (-0.5, 0.5) node[vertice] (vert0) {0};
\draw (4.0, 1.0) node[vertice] (vert1) {1};
\draw (4.0, -2.0) node[vertice] (vert2) {2};
\draw (-1.0, -1.5) node[vertice] (vert3) {3};

\draw[edge_arrow] (vert0) to node[edge_cost, left] {3} (vert3);

\draw[edge_arrow] (vert1) to node[edge_cost, right] {9} (vert2);

\draw[edge_highlight] (vert3) to node[edge_cost, above] {7} (vert1);

% Overlay
\draw[edge_arrow] (vert0) to node[edge_cost, above] {2} (vert1);

\draw[edge_highlight, bend right=5]
(vert2) to node[edge_cost, above] {-2} (vert3);

\draw[edge_arrow, bend right=10] (vert3) to node[edge_cost, below] {3} (vert2);
\end{tikzpicture}%
        \end{adjustbox}
        \caption{{\color{Mittel-Blau}Weighted path} with cost 6 \newline
          ${\color{Mittel-Blau}P} = (2, 3, 1)$}
        \label{fig:graphs:weighted_path_length}
      \end{figure}
    \end{column}
  \end{columns}
\end{frame}

%-------------------------------------------------------------------------------

\begin{frame}{Graphs}{Paths}
  \textbf{Shortest path in a graph:}
  {\color{Mittel-Blau}$G = (V , E)$}
  \begin{itemize}
    \item<2->
      The {\color{Mittel-Blau}shortest path} between two vertices
      ${\color{Mittel-Blau}u}, {\color{Mittel-Blau}v}$ is the path
      \begin{math}
        {\color{Mittel-Blau}P}
          = ({\color{Mittel-Blau}u}, \dots, {\color{Mittel-Blau}v})
      \end{math}
      with the shortest length ${\color{Mittel-Blau}d(u,v)}$ or lowest costs
  \end{itemize}
  \onslide<3->
  \begin{figure}
    \begin{adjustbox}{width=0.75\linewidth}
      \begin{tikzpicture}[
  vertice/.style={
    circle,
    draw=Mittel-Blau,
    color=Mittel-Blau,
    inner sep=0em,
    minimum size=1.75em,
    line width=0.1em,
    font=\large
  }, edge/.style={
    draw=Mittel-Gruen,
    line width=0.2em
  }, edge_arrow/.style={
    edge,
    ->
  }, edge_cost/.style={
    midway,
    color=Hell-Gruen,
    font=\Large
  }, edge_highlight/.style={
    edge_arrow,
    preaction={
      draw,yellow,-,% Draw yellow without any arrow head
      double=yellow,
      double distance=2\pgflinewidth,
    }
  }
]%
\draw (-1.0, 0.0) node[vertice] (vert0) {0};
\draw (2.5, 2.0) node[vertice] (vert1) {1};
\draw (4.0, -2.0) node[vertice] (vert2) {2};
\draw (4.0, 0.0) node[vertice] (vert3) {3};
\draw (5.5, 2.0) node[vertice] (vert4) {4};
\draw (9.0, 0.0) node[vertice] (vert5) {5};

\draw[edge_arrow] (vert0) to node[edge_cost, above] {1} (vert1);
\draw[edge_arrow] (vert0) to node[edge_cost, above] {5} (vert3);
\draw[edge_arrow] (vert0) to node[edge_cost, above] {10} (vert2);

\draw[edge_arrow] (vert1) to node[edge_cost, above] {1} (vert4);
\draw[edge_arrow] (vert1) to node[edge_cost, above] {3} (vert3);

\draw[edge_arrow] (vert2) to node[edge_cost, above] {7} (vert5);

\draw[edge_arrow, bend right=15]
  (vert3) to node[edge_cost, left] {3} (vert2);

\draw[edge_arrow] (vert4) to node[edge_cost, above] {1} (vert3);
\draw[edge_arrow] (vert4) to node[edge_cost, above] {4} (vert5);

\draw[edge_arrow] (vert5) to node[edge_cost, above] {2} (vert3);

% Overlay
\draw[edge_arrow, bend right=15] (vert2) to node[edge_cost, right] {3} (vert3);
\end{tikzpicture}%

    \end{adjustbox}
    \caption{{\color{Mittel-Blau}Shortest path} from 0 to 2 with cost / distance
      {\color{Mittel-Blau}$d(0,2) = 6$}
      ${\color{Mittel-Blau}P}~=~(0, 1, 4, 3, 2)$}
    \label{fig:graphs:shortest_path}
  \end{figure}
\end{frame}

%-------------------------------------------------------------------------------

\begin{frame}{Graphs}{Paths}
  \textbf{Diameter of a graph:}
  {\color{Mittel-Blau}$G = (V , E)$}
  \begin{itemize}
    \item<2->
      The {\color{Mittel-Blau}diameter} of a graph is the length / the costs of
      the {\color{Mittel-Blau}longest shortest path}
  \end{itemize}
  \vspace{-0.5em}
  \hfill\begin{math}
    \displaystyle
    {\color{Mittel-Blau}d} =
      \max\limits_{{\color{Mittel-Blau}u}, {\color{Mittel-Blau}v}
        \in {\color{Mittel-Blau}V}}
      {\color{Mittel-Blau}d(u, \color{Mittel-Blau}v)}
  \end{math}
  \onslide<3->
  \begin{figure}
    \begin{adjustbox}{width=0.75\linewidth}
      \begin{tikzpicture}[
  vertice/.style={
    circle,
    draw=Mittel-Blau,
    color=Mittel-Blau,
    inner sep=0em,
    minimum size=1.75em,
    line width=0.1em,
    font=\large
  }, edge/.style={
    draw=Mittel-Gruen,
    line width=0.2em
  }, edge_arrow/.style={
    edge,
    ->
  }, edge_cost/.style={
    midway,
    color=Hell-Gruen,
    font=\Large
  }, edge_highlight/.style={
    edge_arrow,
    preaction={
      draw,yellow,-,% Draw yellow without any arrow head
      double=yellow,
      double distance=2\pgflinewidth,
    }
  }
]%
\draw (-1.0, 0.0) node[vertice] (vert0) {0};
\draw (2.5, 2.0) node[vertice] (vert1) {1};
\draw (4.0, -2.0) node[vertice] (vert2) {2};
\draw (4.0, 0.0) node[vertice] (vert3) {3};
\draw (5.5, 2.0) node[vertice] (vert4) {4};
\draw (9.0, 0.0) node[vertice] (vert5) {5};

\draw[edge_arrow] (vert0) to node[edge_cost, above] {1} (vert1);
\draw[edge_arrow] (vert0) to node[edge_cost, above] {5} (vert3);
\draw[edge_arrow] (vert0) to node[edge_cost, above] {10} (vert2);

\draw[edge_arrow] (vert1) to node[edge_cost, above] {1} (vert4);
\draw[edge_arrow] (vert1) to node[edge_cost, above] {3} (vert3);

\draw[edge_arrow] (vert2) to node[edge_cost, above] {7} (vert5);

\draw[edge_arrow, bend right=15]
  (vert3) to node[edge_cost, left] {3} (vert2);

\draw[edge_arrow] (vert4) to node[edge_cost, above] {1} (vert3);
\draw[edge_arrow] (vert4) to node[edge_cost, above] {4} (vert5);

\draw[edge_arrow] (vert5) to node[edge_cost, above] {2} (vert3);

% Overlay
\draw[edge_arrow, bend right=15] (vert2) to node[edge_cost, right] {3} (vert3);
\end{tikzpicture}%

    \end{adjustbox}
    \caption{{\color{Mittel-Blau}Diameter} of graph is
      ${\color{Mittel-Blau}d} = 10, \; {\color{Mittel-Blau}P} = (3,2,5)$}
    \label{fig:graphs:diameter}
  \end{figure}
\end{frame}

%-------------------------------------------------------------------------------

\begin{frame}{Graphs}{Connected Components}
  \textbf{Connected components:}
  {\color{Mittel-Blau}$G = (V , E)$}
  \begin{itemize}
    \item<2->
      Undirected graph:
      \begin{itemize}
        \item<3->
          All connected components are a partition of {\color{Mittel-Blau}$V$}
          \begin{displaymath}
            {\color{Mittel-Blau}V}
              = {\color{Mittel-Blau}V_1} \cup \dots \cup 
              {\color{Mittel-Blau}V_k}
          \end{displaymath}
        \item<4->
          Two vertices ${\color{Mittel-Blau}u}, {\color{Mittel-Blau}v}$
          are in the same connected component if a path between
          {\color{Mittel-Blau}$u$} and {\color{Mittel-Blau}$v$} exists
      \end{itemize}
  \end{itemize}
  \onslide<5->
  \begin{figure}
    \begin{adjustbox}{width=0.65\linewidth}
      \begin{tikzpicture}[
  vertice/.style={
    circle,
    draw=Mittel-Blau,
    color=Mittel-Blau,
    inner sep=0em,
    minimum size=1.75em,
    line width=0.1em,
    font=\large
  }, edge/.style={
    draw=Mittel-Gruen,
    line width=0.2em
  }
]%
\draw (0.0, 0.0) node[vertice] (vert0) {0};
\draw (2.0, 1.0) node[vertice] (vert1) {1};
\draw (4.0, -1.0) node[vertice] (vert2) {2};
\draw (1.0, -1.5) node[vertice] (vert3) {3};
\draw (-1.0, -1.75) node[vertice] (vert4) {4};

\draw (4.5, 0.75) node[vertice] (vert5) {5};
\draw (6.0, 1.25) node[vertice] (vert6) {6};
\draw (6.0, -1.25) node[vertice] (vert7) {7};

\draw (7.0, 0.75) node[vertice] (vert8) {8};
\draw (8.0, 1.0) node[vertice] (vert9) {9};

\draw (-2.0, 1.0) node[vertice] (vert10) {10};

\draw[edge] (vert0) -- (vert1);
\draw[edge] (vert1) -- (vert2);
\draw[edge] (vert1) -- (vert3);
\draw[edge] (vert2) -- (vert3);
\draw[edge] (vert3) -- (vert4);

\draw[edge] (vert5) -- (vert7);
\draw[edge] (vert6) -- (vert7);

\draw[edge] (vert8) -- (vert9);
\end{tikzpicture}%
    \end{adjustbox}
    \caption{Four connected components}
    \label{fig:graph:connected_components}
  \end{figure}
\end{frame}

%-------------------------------------------------------------------------------

\begin{frame}{Graphs}{Connected Components}
  \textbf{Connected components:}
  {\color{Mittel-Blau}$G = (V , E)$}
  \begin{itemize}
    \item<2->
      Directed graph:
      \begin{itemize}
        \item<3->
          Named {\color{Mittel-Blau}strongly connected components}
        \item<4->
          Direction of edge has to be regarded
        \item<5->
          Not part of this lecture
      \end{itemize}
  \end{itemize}
\end{frame}

%-------------------------------------------------------------------------------

\begin{frame}{Graphs}{Connected Components - Graph Exploration}
  \textbf{Graph Exploration:} (Informal definition)
  \begin{itemize}
    \item<2->
      Let {\color{Mittel-Blau}$G = (V , E)$} be a graph and
      ${\color{Mittel-Blau}s} \in {\color{Mittel-Blau}V}$ a start vertex
    \item<3->
      We visit each reachable vertex connected to {\color{Mittel-Blau}$s$}
    \item<4->
      {\color{Mittel-Blau}Breadth-first search}: Searching in order of the
      smallest distance to {\color{Mittel-Blau}$s$}
    \item<5->
      {\color{Mittel-Blau}Depth-first search}: Searching in order of the
      largest distance to {\color{Mittel-Blau}$s$}
    \item<6->
      Is often used as subroutine of other algorithms
      \begin{itemize}
        \item<7->
          Searching of connected components
        \item<8->
          Flood fill in drawing programms
      \end{itemize}
  \end{itemize}
\end{frame}

%-------------------------------------------------------------------------------

\begin{frame}{Graphs}{Connected Components - Breadth-First Search}
  \textbf{Idea:}
  \begin{enumerate}
    \item<2->
      We start with all vertices unmarked and
      {\color{Mittel-Blau}mark visited vertices}
    \item<3->
      Mark the start vertex {\color{Mittel-Blau}$s$}
      ({\color{Mittel-Blau}level 0})
    \item<4->
      Mark all unmarked {\color{Mittel-Blau}connected vertices}
      ({\color{Mittel-Blau}level 1})
    \item<5->
      Mark all unmarked {\color{Mittel-Blau}vertices connected} to a
      {\color{Mittel-Blau}level 1}-vertex
      ({\color{Mittel-Blau}level 2})
    \item<6->
      Mark iterative reachable vertices for all level
    \item<7->
      All connected nodes are now marked and in the same
      {\color{Mittel-Blau}connected component} as the start vertex
      {\color{Mittel-Blau}$s$}
  \end{enumerate}
\end{frame}

%-------------------------------------------------------------------------------

\begin{frame}{Graphs}{Connected Components - Breadth-First Search}
  \begin{itemize}
    \item<2->
      The marked vertices create a \enquote{spanning tree} containing all
      reachable nodes
  \end{itemize}
  \onslide<3->
  \begin{figure}
    \begin{adjustbox}{width=0.95\linewidth}
      \begin{tikzpicture}[
  vertex/.style={
    circle,
    draw=Mittel-Blau,
    color=Mittel-Blau,
    inner sep=0em,
    minimum size=1.75em,
    line width=0.1em,
    font=\large
  }, edge/.style={
    draw=Mittel-Gruen,
    line width=0.2em
  }, edge_arrow/.style={
    edge,
    ->
  }, edge_level_1/.style={
    edge_arrow,
    preaction={
      draw,
      -,
      opacity=0.5,
      orange!50!yellow,
      double=orange!50!yellow,
      double distance=2\pgflinewidth,
    }
  }, edge_level_2/.style={
    edge_arrow,
    preaction={
      draw,
      -,
      opacity=0.5,
      Hell-Gruen,
      double=Hell-Gruen,
      double distance=2\pgflinewidth,
    }
  }, edge_level_3/.style={
    edge_arrow,
    preaction={
      draw,
      -,
      opacity=0.5,
      cyan,
      double=cyan,
      double distance=2\pgflinewidth,
    }
  }, vertex_level_0/.style={
    vertex,
    fill=red!60!white
  }, vertex_level_1/.style={
    vertex,
    fill=orange!50!yellow
  }, vertex_level_2/.style={
    vertex,
    fill=Hell-Gruen
  }, vertex_level_3/.style={
    vertex,
    fill=cyan!50!white
  }, edge_label/.style={
    sloped,
    anchor=south,
    auto=false
  }
]%
% Draw vertices
\draw (0.0, 0.0) node[vertex_level_0] (vert0) {};
\draw (2.5, 1.0) node[vertex_level_1] (vert1) {};
\draw (2.0, -2.0) node[vertex_level_1] (vert2) {};
\draw (9.0, 0.0) node[vertex_level_1] (vert3) {};
\draw (5.0, 2.0) node[vertex_level_2] (vert4) {};
\draw (11.0, -2.0) node[vertex_level_2] (vert5) {};
\draw (7.0, -3.0) node[vertex_level_3] (vert6) {};
\draw (7.5, 1.5) node[vertex] (vert7) {};
\draw (9.5, 2.0) node[vertex] (vert8) {};

% Draw edges
\draw[edge_level_1] (vert0) -- (vert2);
\draw[edge_level_1] (vert0) -- (vert3);
\draw[edge_level_1] (vert0) -- (vert1);

\draw[edge_level_2] (vert1) -- (vert4);
\draw[edge_level_2] (vert3) -- (vert5);

\draw[edge_level_3] (vert4) -- (vert6);

\draw[edge_arrow] (vert2) to node[edge_label] {forward edge} (vert4);
\draw[edge_arrow]
  (vert3) to node[edge_label, shift={(-3em,0em)}] {cross edge} (vert2);
\draw[edge_arrow] (vert6) to node[edge_label, below] {back edge} (vert3);

\draw[edge_arrow]
  (vert7) to node[above, shift={(0em, 1.5em)}, align=left]
  {Not reachable from\\start-node {\color{Mittel-Blau}$s$}} (vert8);

% Draw labels
\foreach \i/\color in {0/red!85!black,1/orange!50!yellow,2/Hell-Gruen,3/cyan}{
\draw (-3.0, 3.0 - \i*0.75)
  node[vertex_level_\i, minimum size=1.25em,
    label={[anchor=west,color=\color]right:\LARGE level \i}
  ] {};
}
\end{tikzpicture}%
    \end{adjustbox}
    \caption{spanning tree of a breadth-first search}
    \label{fig:graph:breadth_first_search_spanning_tree}
  \end{figure}
\end{frame}

%-------------------------------------------------------------------------------

\begin{frame}{Graphs}{Connected Components - Depth-First Search}
  \textbf{Idea:}
  \begin{enumerate}
    \item<2->
      We start with all vertices unmarked and
      {\color{Mittel-Blau}mark visited vertices}
    \item<3->
      Mark the start vertex {\color{Mittel-Blau}$s$}
    \item<4->
      Pick an unmarked {\color{Mittel-Blau}connected vertex} and start a
      {\color{Mittel-Blau}recursive depth-first search} with the vertex as
      start vertex\\
      (continue on step 2)
    \item<5->
      If no unmarked connected vertex exists go one vertex back\\
      (reduce the recursion level by one)
  \end{enumerate}
\end{frame}

%-------------------------------------------------------------------------------

\begin{frame}{Graphs}{Connected Components - Depth-First Search}
  \textbf{Depth-first search:}
  \begin{itemize}
    \item<2->
      Search starts with {\color{Mittel-Blau}long paths} (searching with depth)
    \item<3->
      Marks like {\color{Mittel-Blau}breadth-first search} all connected
      vertices
    \item<4->
      If the graph is acyclic we get a {\color{Mittel-Blau}topological sorting}
      \begin{itemize}
        \item<5->
          Each newly visited vertex gets marked by an increasing number
        \item<6->
          The numbers increase with path from the start vertex
      \end{itemize}
  \end{itemize}
\end{frame}

%-------------------------------------------------------------------------------

\begin{frame}{Graphs}{Connected Components - Depth-First Search}
  \begin{itemize}
    \item<2->
      The marked vertices create a different spanning tree containing all
      reachable nodes
  \end{itemize}
  \onslide<3->
  \begin{figure}
    \begin{adjustbox}{width=0.95\linewidth}
      \begin{tikzpicture}[
  vertex/.style={
    circle,
    color=Mittel-Blau,
    draw=Mittel-Blau,
    inner sep=0em,
    minimum size=1.75em,
    line width=0.1em,
    font=\large
  }, edge/.style={
    draw=Mittel-Gruen,
    line width=0.2em
  }, edge_arrow/.style={
    edge,
    ->
  }, edge_level_1/.style={
    edge_arrow,
    preaction={
      draw,
      -,
      opacity=0.5,
      orange!50!yellow,
      double=orange!50!yellow,
      double distance=2\pgflinewidth,
    }
  }, edge_level_2/.style={
    edge_arrow,
    preaction={
      draw,
      -,
      opacity=0.5,
      Hell-Gruen,
      double=Hell-Gruen,
      double distance=2\pgflinewidth,
    }
  }, edge_level_3/.style={
    edge_arrow,
    preaction={
      draw,
      -,
      opacity=0.5,
      cyan,
      double=cyan,
      double distance=2\pgflinewidth,
    }
  }, vertex_level_0/.style={
    vertex,
    fill=red!60!white
  }, vertex_level_1/.style={
    vertex,
    fill=orange!50!yellow
  }, vertex_level_2/.style={
    vertex,
    fill=Hell-Gruen
  }, vertex_level_3/.style={
    vertex,
    fill=cyan!50!white
  }, edge_label/.style={
    sloped,
    anchor=south,
    auto=false
  }
]%
% Draw vertices
\draw (0.0, 0.0) node[vertex_level_0] (vert0) {0};
\draw (2.5, 1.0) node[vertex_level_3] (vert1) {6};
\draw (2.0, -2.5) node[vertex_level_1] (vert2) {2};
\draw (9.0, 0.0) node[vertex_level_1] (vert3) {1};
\draw (5.0, 2.0) node[vertex_level_1] (vert4) {3};
\draw (11.0, -2.0) node[vertex_level_2] (vert5) {5};
\draw (7.0, -3.0) node[vertex_level_1] (vert6) {4};
\draw (7.5, 1.5) node[vertex] (vert7) {};
\draw (9.5, 2.0) node[vertex] (vert8) {};

% Layer below
\draw[edge_level_1] (vert3) -- (vert2);
\draw[edge_level_3] (vert0) -- (vert1);

% Draw edges
\draw[edge_level_1] (vert0) -- (vert3);

\draw[edge_arrow] (vert1) to node[edge_label] {back edge} (vert4);
\draw[edge_level_2] (vert3) -- (vert5);

\draw[edge_level_1] (vert4) -- (vert6);

\draw[edge_level_1] (vert2) -- (vert4);
\draw[edge_arrow] (vert6) to node[edge_label, below] {back edge} (vert3);

\draw[edge_arrow]
  (vert7) to node[above, shift={(0em, 1.5em)}, align=left]
  {Not reachable from\\start-node {\color{Mittel-Blau}$s$}} (vert8);

% Layer above
\draw[edge_arrow] (vert0) to node[edge_label] {forward edge} (vert2);

% Draw labels
\foreach \i/\text/\color in {
  0/start-node/red!85!black,%
  1/path 1/orange!50!yellow,%
  2/path 2/Hell-Gruen,%
  3/path 3/cyan%
}{
\draw (-3.0, 3.0 - \i*0.75)
  node[vertex_level_\i, minimum size=1.25em,
    label={[anchor=west,color=\color]right:\LARGE \text}
  ] {};
}
\end{tikzpicture}%
    \end{adjustbox}
    \caption{spanning tree of a depth-first search}
    \label{fig:graph:depth_first_search_spanning_tree}
  \end{figure}
\end{frame}

%-------------------------------------------------------------------------------

\begin{frame}{Graphs}{Connected Components - Breadth-/Depth-First Search}
  \textbf{Runtime complexity:}
  \begin{itemize}
    \item<2->
      Constant costs for each visited vertex and edge
    \item<3->
      Let {\color{Mittel-Blau}$V'$} and {\color{Mittel-Blau}$E'$} be the
      reachable vertices and edges
    \item<4->
      All vertices of {\color{Mittel-Blau}$V'$} are in the same connected
      component as our start vertex {\color{Mittel-Blau}$s$}
    \item<5->
      We get a runtime complexity of
      \begin{math}
        \Theta(\vert {\color{Mittel-Blau}V'} \vert
          + \vert {\color{Mittel-Blau}E'} \vert)
      \end{math}
    \item<6->
      This can only be improved by a constant factor
  \end{itemize}
\end{frame}
