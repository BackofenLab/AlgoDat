\subsection{Application example}

\begin{frame}{Application example}{Image processing}
  \begin{itemize}
    \item<2->
      Connected component labeling
    \item<3->
      Counting of objects in an image
  \end{itemize}
  \onslide<4->
  \begin{figure}[!h]
    \begin{columns}
      \begin{column}{0.45\linewidth}
        \includegraphics[width=\linewidth]%
          {Images/ImageProcessing/particles_source.png}
      \end{column}
      \begin{column}{0.05\linewidth}
        \begin{centering}
          \color{Mittel-Blau}$\Rightarrow$
        \end{centering}
      \end{column}
      \begin{column}{0.45\linewidth}
        \includegraphics[width=\linewidth]%
          {Images/ImageProcessing/particles_labeled.png}
      \end{column}
    \end{columns}
  \end{figure}
\end{frame}

%-------------------------------------------------------------------------------

\begin{frame}{Application example}{Image processing}
  \textbf{What is object, what is background?}\\
  \begin{center}
    \begin{adjustbox}{width=0.8\linewidth}
      \begin{tikzpicture}[
  zoom_line/.style={
    Mittel-Gruen,
    line width=0.3em
  },
  image/.style={
    inner sep=0
  }
]%
  \node[image, anchor=south west] (small_image) at (0,0) {%
      \includegraphics[width=\textwidth]%
        {Images/ImageProcessing/particles_source.png}%
    };
  \node[image, anchor=west, xshift=5em] (big_image) at (small_image.east) {%
      \includegraphics[width=2.5\textwidth, clip, trim=0 100 100 0]%
        {Images/ImageProcessing/particles_excel_data.png}%
    };
  \begin{scope}[x={(small_image.south east)}, y={(small_image.north west)}]
    \node (rect) at (0.0375, 0.825) [
      draw,
      rectangle,
      minimum size=30,
      zoom_line
    ] {};
    %\draw[red,ultra thick,rounded corners]
    %  (-0.025, 0.7625) rectangle (0.1, 0.88125);
    %\draw[help lines,xstep=.1,ystep=.1] (0,0) grid (1,1);
    %\foreach \x in {0,1,...,9} { \node [anchor=north] at (\x/10,0) {0.\x}; }
    %\foreach \y in {0,1,...,9} { \node [anchor=east] at (0,\y/10) {0.\y}; }
  \end{scope}

  \draw[zoom_line] (rect.north west) to (big_image.north west);
  \draw[zoom_line] (rect.south west) to (big_image.south west);
\end{tikzpicture}
    \end{adjustbox}
  \end{center}
\end{frame}

%-------------------------------------------------------------------------------

\begin{frame}{Application example}{Image processing}
  \textbf{Convert to black and white using threshold:}\\
  \codeslide{python}{%
    \lstinline[language=Python, style={python-idle-code}]%
      {value = 255 if value > 100 else 0}%
  }%
  \codeslide{java}{%
    \lstinline[language=Java, style={java-eclipse-code}]%
      {value = value > 100 ? 255 : 0}%
  }%
  \codeslide{cpp}{%
    \lstinline[language=C++, style={cpp-eclipse-code}]%
      {value = value > 100 ? 255 : 0}%
  }%
  \begin{center}
    \begin{adjustbox}{width=0.8\linewidth}
      \begin{tikzpicture}[
  zoom_line/.style={
    Mittel-Gruen,
    line width=0.3em
  },
  image/.style={
    inner sep=0
  }
]%
  \node[image, anchor=south west] (small_image) at (0,0) {%
      \includegraphics[width=\textwidth]%
        {Images/ImageProcessing/particles_black_white.png}%
    };
  \node[image, anchor=west, xshift=5em] (big_image) at (small_image.east) {%
      \includegraphics[width=2.5\textwidth, clip, trim=0 200 200 0]%
        {Images/ImageProcessing/particles_black_white_excel_data.png}%
    };
  \begin{scope}[x={(small_image.south east)}, y={(small_image.north west)}]
    \node (rect) at (0.0375, 0.825) [
      draw,
      rectangle,
      minimum size=30,
      zoom_line
    ] {};
    %\draw[red,ultra thick,rounded corners]
    %  (-0.025, 0.7625) rectangle (0.1, 0.88125);
    %\draw[help lines,xstep=.1,ystep=.1] (0,0) grid (1,1);
    %\foreach \x in {0,1,...,9} { \node [anchor=north] at (\x/10,0) {0.\x}; }
    %\foreach \y in {0,1,...,9} { \node [anchor=east] at (0,\y/10) {0.\y}; }
  \end{scope}

  \draw[zoom_line] (rect.north west) to (big_image.north west);
  \draw[zoom_line] (rect.south west) to (big_image.south west);
\end{tikzpicture}
    \end{adjustbox}
  \end{center}
\end{frame}

%-------------------------------------------------------------------------------

\begin{frame}{Application example}{Image processing}
  \textbf{Interpret image as graph:}
  \begin{itemize}
    \item<2->
      Each white pixel is a node
    \item<3->
      Edges between adjacent pixels (normally 4 or 8 neighbors)
    \item<4->
      Edges are not saved externally,
      algorithm works directly on array
    \item<5->
      Breadth- / depth-first search find all connected components (particles)
  \end{itemize}
\end{frame}

%-------------------------------------------------------------------------------

\begin{frame}{Application example}{Image processing}
  \textbf{Find connected components:}\\
  \vspace{0.5em}
  \onslide<2->
  \begin{columns}[T]
    \begin{column}{0.6\linewidth}
      \includegraphics[width=\linewidth]%
        {Images/ImageProcessing/particles_excel_search.png}
    \end{column}
    \begin{column}{0.4\linewidth}
      \begin{itemize}
        \item<3->
          Search pixel-by-pixel for non-zero intensity
        \item<4->
          Label found pixel as component~1
      \end{itemize}
    \end{column}
  \end{columns}
\end{frame}

%-------------------------------------------------------------------------------

\begin{frame}{Application example}{Image processing}
  \textbf{Find connected components:}\\
  \vspace{0.5em}
  \begin{columns}[T]
    \begin{column}{0.6\linewidth}
      \includegraphics[width=\linewidth]%
        {Images/ImageProcessing/particles_excel_search2.png}
    \end{column}
    \begin{column}{0.4\linewidth}
      \begin{itemize}
        \item
          {\color{gray}Search pixel-by-pixel for non-zero intensity}
        \item
          {\color{gray}Label found pixel as component~1}
        \item
          Check neighbors of all new labeled pixels
      \end{itemize}
    \end{column}
  \end{columns}
\end{frame}

%-------------------------------------------------------------------------------

\begin{frame}{Application example}{Image processing}
  \textbf{Find connected components:}\\
  \vspace{0.5em}
  \begin{columns}[T]
    \begin{column}{0.6\linewidth}
      \includegraphics[width=\linewidth]%
        {Images/ImageProcessing/particles_excel_search3.png}
    \end{column}
    \begin{column}{0.4\linewidth}
      \begin{itemize}
        \item
          {\color{gray}Search pixel-by-pixel for non-zero intensity}
        \item
          {\color{gray}Label found pixel as component~1}
        \item
          {\color{gray}Check neighbors of all new labeled pixels}
        \item
          Label non-zero pixels as component~1
      \end{itemize}
    \end{column}
  \end{columns}
\end{frame}

%-------------------------------------------------------------------------------

\begin{frame}{Application example}{Image processing}
  \textbf{Find connected components:}\\
  \vspace{0.5em}
  \begin{columns}[T]
    \begin{column}{0.6\linewidth}
      \includegraphics[width=\linewidth]%
        {Images/ImageProcessing/particles_excel_search4.png}
    \end{column}
    \begin{column}{0.4\linewidth}
      \begin{itemize}
        \item
          {\color{gray}Search pixel-by-pixel for non-zero intensity}
        \item
          {\color{gray}Label found pixel as component~1}
        \item
          Check neighbors of all new labeled pixels
        \item
          {\color{gray}Label non-zero pixels as component~1}
      \end{itemize}
    \end{column}
  \end{columns}
\end{frame}

%-------------------------------------------------------------------------------

\begin{frame}{Application example}{Image processing}
  \textbf{Find connected components:}\\
  \vspace{0.5em}
  \begin{columns}[T]
    \begin{column}{0.6\linewidth}
      \includegraphics[width=\linewidth]%
        {Images/ImageProcessing/particles_excel_search5.png}
    \end{column}
    \begin{column}{0.4\linewidth}
      \begin{itemize}
        \item
          {\color{gray}Search pixel-by-pixel for non-zero intensity}
        \item
          {\color{gray}Label found pixel as component~1}
        \item
          {\color{gray}Check neighbors of all new labeled pixels}
        \item
          Label non-zero pixels as component~1
      \end{itemize}
    \end{column}
  \end{columns}
\end{frame}

%-------------------------------------------------------------------------------

\begin{frame}{Application example}{Image processing}
  \textbf{Find connected components:}\\
  \vspace{0.5em}
  \begin{columns}[T]
    \begin{column}{0.6\linewidth}
      \includegraphics[width=\linewidth]%
        {Images/ImageProcessing/particles_excel_search6.png}
    \end{column}
    \begin{column}{0.4\linewidth}
      \begin{itemize}
        \item
          {\color{gray}Search pixel-by-pixel for non-zero intensity}
        \item
          {\color{gray}Label found pixel as component~1}
        \item
          Check neighbors of all new labeled pixels
        \item
          {\color{gray}Label non-zero pixels as component~1}
      \end{itemize}
    \end{column}
  \end{columns}
\end{frame}

%-------------------------------------------------------------------------------

\begin{frame}{Application example}{Image processing}
  \textbf{Find connected components:}\\
  \vspace{0.5em}
  \begin{columns}[T]
    \begin{column}{0.6\linewidth}
      \includegraphics[width=\linewidth]%
        {Images/ImageProcessing/particles_excel_search7.png}
    \end{column}
    \begin{column}{0.4\linewidth}
      \begin{itemize}
        \item
          {\color{gray}Search pixel-by-pixel for non-zero intensity}
        \item
          {\color{gray}Label found pixel as component~1}
        \item
          {\color{gray}Check neighbors of all new labeled pixels}
        \item
          Label non-zero pixels as component~1
      \end{itemize}
    \end{column}
  \end{columns}
\end{frame}

%-------------------------------------------------------------------------------

\begin{frame}{Application example}{Image processing}
  \textbf{Find connected components:}\\
  \vspace{0.5em}
  \begin{columns}[T]
    \begin{column}{0.6\linewidth}
      \includegraphics[width=\linewidth]%
        {Images/ImageProcessing/particles_excel_search8.png}
    \end{column}
    \begin{column}{0.4\linewidth}
      \begin{itemize}
        \item
          {\color{gray}Search pixel-by-pixel for non-zero intensity}
        \item
          {\color{gray}Label found pixel as component~1}
        \item
          Check neighbors of all new labeled pixels
        \item
          {\color{gray}Label non-zero pixels as component~1}
      \end{itemize}
    \end{column}
  \end{columns}
\end{frame}

%-------------------------------------------------------------------------------

\begin{frame}{Application example}{Image processing}
  \textbf{Find connected components:}\\
  \vspace{0.5em}
  \begin{columns}[T]
    \begin{column}{0.6\linewidth}
      \includegraphics[width=\linewidth]%
        {Images/ImageProcessing/particles_excel_search9.png}
    \end{column}
    \begin{column}{0.4\linewidth}
      \begin{itemize}
        \item
          {\color{gray}Search pixel-by-pixel for non-zero intensity}
        \item
          {\color{gray}Label found pixel as component~1}
        \item
          {\color{gray}Check neighbors of all new labeled pixels}
        \item
          Label non-zero pixels as component~1
      \end{itemize}
    \end{column}
  \end{columns}
\end{frame}

%-------------------------------------------------------------------------------

\begin{frame}{Application example}{Image processing}
  \textbf{Find connected components:}\\
  \vspace{0.5em}
  \begin{columns}[T]
    \begin{column}{0.6\linewidth}
      \includegraphics[width=\linewidth]%
        {Images/ImageProcessing/particles_excel_search10.png}
    \end{column}
    \begin{column}{0.4\linewidth}
      \begin{itemize}
        \item
          {\color{gray}Search pixel-by-pixel for non-zero intensity}
        \item
          {\color{gray}Label found pixel as component~1}
        \item
          Check neighbors of all new labeled pixels
        \item
          {\color{gray}Label non-zero pixels as component~1}
      \end{itemize}
    \end{column}
  \end{columns}
\end{frame}

%-------------------------------------------------------------------------------

\begin{frame}{Application example}{Image processing}
  \textbf{Find connected components:}\\
  \vspace{0.5em}
  \begin{columns}[T]
    \begin{column}{0.6\linewidth}
      \includegraphics[width=\linewidth]%
        {Images/ImageProcessing/particles_excel_search11.png}
    \end{column}
    \begin{column}{0.4\linewidth}
      \begin{itemize}
        \item
          {\color{gray}Search pixel-by-pixel for non-zero intensity}
        \item
          {\color{gray}Label found pixel as component~1}
        \item
          {\color{gray}Check neighbors of all new labeled pixels}
        \item
          Label non-zero pixels as component~1
      \end{itemize}
    \end{column}
  \end{columns}
\end{frame}

%-------------------------------------------------------------------------------

\begin{frame}{Application example}{Image processing}
  \textbf{Find connected components:}\\
  \vspace{0.5em}
  \begin{columns}[T]
    \begin{column}{0.6\linewidth}
      \includegraphics[width=\linewidth]%
        {Images/ImageProcessing/particles_excel_search12.png}
    \end{column}
    \begin{column}{0.4\linewidth}
      \begin{itemize}
        \item
          Search pixel-by-pixel for non-zero intensity
        \item
          Label found pixel as {\color{Mittel-Blau}component~2}
        \item
          $\cdots$
      \end{itemize}
    \end{column}
  \end{columns}
\end{frame}

%-------------------------------------------------------------------------------

\begin{frame}{Application example}{Image processing}
  \textbf{Result of connected component labeling:}
  \begin{figure}
    \begin{columns}
      \begin{column}{0.45\linewidth}
        \includegraphics[width=\linewidth, clip, trim=0 75 50 0]%
          {Images/ImageProcessing/particles_final_source.png}
      \end{column}
      \begin{column}{0.05\linewidth}
        \begin{centering}
          \color{Mittel-Blau}$\Rightarrow$
        \end{centering}
      \end{column}
      \begin{column}{0.45\linewidth}
        \includegraphics[width=\linewidth, clip, trim=0 75 50 0]%
          {Images/ImageProcessing/particles_final.png}
      \end{column}
    \end{columns}
    \caption{Result: particle indices instead of intensities}
  \end{figure}
\end{frame}
