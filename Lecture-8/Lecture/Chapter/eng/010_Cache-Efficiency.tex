\section{Cache Efficiency}


\begin{frame}{Cache Efficiency}{Background}
	\begin{itemize}
		\item
			Until now we counted the answer of operations as a reference for the 
			runtime of the algorithm/program
		\item
			On the next slide we see an example, were this is not possible
	\end{itemize}
\end{frame}

%-------------------------------------------------------------------------------

\begin{frame}{Cache Efficiency}{Example}
	\begin{itemize}
		\item
			We add up the elements of a field of size n
		\begin{itemize}
			\item
				in natural order: $a[1] + a[2] + a[3] + a[4] + a[5]$
			\item
				in random order: $a[2] + a[5] + a[1] + a[3] + a[4]$
		\end{itemize}
	\end{itemize}
\end{frame}

%-------------------------------------------------------------------------------

\begin{frame}{Cache Efficiency}{Example program}

TODO: Phyton program -> look at slide 5 \vspace{2em}

\end{frame}

%-------------------------------------------------------------------------------

\begin{frame}{Cache Efficiency}{Example program 2}

TODO: Phyton program -> look at slide 6 \vspace{2em}

\end{frame}

%-------------------------------------------------------------------------------

\begin{frame}{Cache Efficiency}{Example program 3}

TODO: Phyton program -> look at slide 7 \vspace{2em}

\end{frame}

%-------------------------------------------------------------------------------

\begin{frame}{Cache Efficiency}{Example program 4}

TODO: Phyton program -> look at slide 8 \vspace{2em}

\end{frame}

%-------------------------------------------------------------------------------

\begin{frame}{Cache Efficiency}{Output of the Example program}

TODO: Table -> look at slide 9 \vspace{2em}

	\begin{itemize}
		\item
			The number of operations are for both identical but the runtime are a lot 
			different
		\item
			the cost for the cache efficiency are much different
	\end{itemize}
\end{frame}

%-------------------------------------------------------------------------------

\begin{frame}{Cache Efficiency}{CPU Cache}
	
	TODO: Graphics -> look at slide 10 \vspace{2em}
	
	\textbf{Layout:}
	\begin{itemize}
		\item
			Accessing one byte of the main memory costs around \SI{100}{\nano \second}
		\item
			Accessing one byte of L1-Cache costs around \SI{1}{\nano \second}
		\item
			For accessing one or more bytes in the main memory, we load a whole block 
			of around 100 bytes in the cache, so it is not necessary to access the 
			main memory anymore for data from the block
	\end{itemize}
\end{frame}

%-------------------------------------------------------------------------------

\begin{frame}{Cache Efficiency}{CPU Chache}

TODO: Graphics -> look at slide 10 \vspace{2em}

	\begin{itemize}
		\item
			The Cache has space for a lot of these blocks
		\begin{itemize}
			\item
				a normal L1-cache has \SI{100}{\kilo \byte}
		\end{itemize}
		\item
			if the cache is full, one block will be deleted
		\begin{itemize}
			\item
				for example the least recently used (LRU) block
		\end{itemize}
	\end{itemize}
\end{frame}

%-------------------------------------------------------------------------------

\begin{frame}{Cache Efficiency}{Block-operations terminology}
	
	TODO: Graphics -> look at slide 12 \vspace{2em}
	
	\textbf{Terminology:}
	\begin{itemize}
		\item
			We have a fast and a slow cache
		\item
			The slow cache is in blocks of size B separated
		\item
			The fast cache has size M (Space for M/B blocks)
	\end{itemize}
\end{frame}

%-------------------------------------------------------------------------------

\begin{frame}{Cache Efficiency}{Block-operations terminology}
	\textbf{Terminology:}
	\begin{itemize}
		\item
			Is the needed block not in the fast cache, the block is loaded in the 
			fast cache
		\item
			The program can choose which blocks are held in the fast cache
		\item
			We count the number of block operations
		\item
			In praxis we also count the number of operations in the fast memory
	\end{itemize}
\end{frame}

%-------------------------------------------------------------------------------

\begin{frame}{Cache Efficiency}{Block-operations locality}
	
	TODO: Graphics -> look at slide 13 \vspace{2em}
	
	\textbf{For accessing the cache B times}
	\begin{itemize}
		\item
			best case: 1 block operation $\rightarrow$ good locality
		\item
			worst case: B block operations $\rightarrow$ bad locality
	\end{itemize}
\end{frame}

%-------------------------------------------------------------------------------

\begin{frame}{Cache Efficiency}{Block-operations locality}
	\textbf{Variations}\\
	varies only by small constant factor in the number of block operations
	\begin{itemize}
		\item
		exact separation in blocks of the slow cache 
		\item
		weather the memory unit is 1 byte or 4 byte or 8 byte
	\end{itemize}
	\textbf{We note}\\
	\begin{itemize}
		\item
		it only gets interesting when the input size is larger than M, otherwise 
		we could load the whole input in the fast cache
	\end{itemize}
\end{frame}

%-------------------------------------------------------------------------------

\begin{frame}{Cache Efficiency}{Block-operations in praxis}
	\textbf{Typical values}(for Server)
	\begin{itemize}
		\item
			CPU Cache: B = 128 Bytes, M = $6 \cdot \SI{32}{\kilo \byte}$ (L1),\\
			 $6 \cdot \SI{256}{\kilo \byte}$ (L2)
		\item
			Disk Cache: B = $\SI{64}{\kilo \byte}$, M = $\SI{64}{\giga \byte}$
		\begin{itemize}
			\item
				The most operating systems uses all unused main memory as disk cache
		\end{itemize}
	\end{itemize}
	\textbf{terminology}
	\begin{itemize}
		\item
			Block operations in the CPU cache is called cache misses
		\item
			Block operations in disk cache is called IOs
		\begin{itemize}
			\item
				IO oder I/O = Input/Output
		\end{itemize}
		\item
			We talk about cache efficiency and IO efficiency
	\end{itemize}
\end{frame}

%-------------------------------------------------------------------------------

\begin{frame}{Cache Efficiency}{Block-operations Example}
	\textbf{Example 1:}\\
	ArraySumMain program
	\begin{itemize}
		\item
			If we iterate over the elements in the order 1, 2, 3, ... is the number 
			of block operations: ceil(n/B)
	\end{itemize}
	
	TODO: Graphics -> look at slide 15 \vspace{2em}
	
	\begin{itemize}
		\item
			If we iterate over the elements in a random order is the number block 
			operations in the worst case: n
	\end{itemize}
		
		TODO: Graphics -> look at slide 15 \vspace{2em}
		
	\begin{itemize}
		\item
			This is a theoretical factor of B difference and this is the main reason 
			for the runtime difference
	\end{itemize}
\end{frame}

%-------------------------------------------------------------------------------

\begin{frame}{Cache Efficiency}{Block-operations Example}
	In praxis is the factor << B 
	\begin{itemize}
		\item
			even in a random order, we need to access 4 nearby bytes (one int) for 
			one element
		\item
			Also, if not n >> M, the next element is sometimes randomly already in  
			the cache memory
	\end{itemize}
\end{frame}

%-------------------------------------------------------------------------------

\begin{frame}{Cache Efficiency}{Example 2 QuickSort}
	\textbf{Example 2:}\\
	sort with QuickSort
	\begin{itemize}
		\item
			How QuickSort works
		\item
			Strategy: divide and concur -> later
		\begin{itemize}
			\item
				Divide the input field in two parts, so that all elements in the left 
				part $\leq$ all in the right part
			\item
				To Divide we choose any element ("Pivot" -Element), for example the 
				first one or the last one
			\item
				Ideally have both parts the same size
			\item
				we sort then both parts recursive
		\end{itemize}
	\end{itemize}
	
	TODO: Graphics -> look at slide 17 \vspace{2em}
	
\end{frame}

%-------------------------------------------------------------------------------

\begin{frame}{Cache Efficiency}{Quicksort algorithm}
	
	TODO: look at slide 18 \vspace{2em}
	
\end{frame}

%-------------------------------------------------------------------------------

\begin{frame}{Cache Efficiency}{number of operations in Quicksort}
	\begin{itemize}
		\item
			T(n) is the runtime for the input size n
		\item
			fields are separated perfectly in the middle
		\item
			n is a power of two and recursion $k = \log_2 n$
	\end{itemize}
	
	TODO: Math -> look at slide 19 \vspace{2em}
	
\end{frame}

%-------------------------------------------------------------------------------

\begin{frame}{Cache Efficiency}{number of block operations in Quicksort}

TODO: Graphics -> look at slide 20 \vspace{2em}

	\begin{itemize}
		\item
			Let IO(n) the number of block operations for the\\
			input size n
		\item
			The assumption is the same as before, but recursion depth $k = \log_2 
			(n/B)$ why ?
	\end{itemize}
	
	TODO: Math -> look at slide 20 \vspace{2em}
	
\end{frame}