\section{Divide and Conquer}
\toclesssubsection{Introduction}

%Maximum subtotal
%\begin{itemize}
%  \item
%  Approach to a problem
%  \item
%  Divide and concur approach
%  \item
%  Runtime
%\end{itemize}
%\item
%Recursion equation
%\begin{itemize}
%  \item
%  Substitution method
%  \item
%  Recursion tree method
%  \item
%  Master theorem

%-------------------------------------------------------------------------------

\begin{frame}{Divide and Conquer}{Introduction}
  \textbf{Concept:}
  \begin{itemize}
    \item
      {\color{Mittel-Blau}Divide} the problem into smaller subproblems
    \item
      {\color{Mittel-Blau}Conquer} the subproblems through recusive solving.\\
      If subproblems are small enough solve them directly
    \item
      {\color{Mittel-Blau}Connect} all solutions to as total solution of the
      problem
  \end{itemize}
  \vspace{1.0em}
  \begin{itemize}
    \item
      {\color{Mittel-Blau}Recursive} application of the algorithm for smaller
      subproblems
    \item
      {\color{Mittel-Blau}Direct} solving of small subproblems
  \end{itemize}
\end{frame}

%-------------------------------------------------------------------------------

\codeslide{python}{
\begin{frame}{Divide and Conquer}{Introduction - Python}
  \vspace{-0.5em}
  \lstinputlisting[
    language=Python,
    basicstyle=\small,
    tabsize=4,
    style={python-idle-code},
    escapechar={@},
    emph={solve},
    emphstyle=\color{blue}
  ]{Code/DivideAndConquer/DivideAndConquer_Concept.py}
\end{frame}
}

%TODO: Implement for Java / C++

%-------------------------------------------------------------------------------

\begin{frame}{Divide and Conquer}{Features}
  \begin{enumerate}
    \item
      Can help with conceptual hard problems
      \begin{itemize}
        \item
          {\color{Mittel-Blau}Solution} of the trivial problems has to be known
        \item
          {\color{Mittel-Blau}Dividing} in subproblems has to be possible
        \item
          {\color{Mittel-Blau}Combination} of solutions has to be possible
      \end{itemize}
    \item
      Suitable for parallel processing
      \begin{itemize}
        \item
          Subproblems are {\color{Mittel-Blau}independent} of each other
        \item
          Only needed input for each subproblem has to be known
      \end{itemize}
  \end{enumerate}
\end{frame}

%-------------------------------------------------------------------------------

\begin{frame}{Divide and Conquer}{Features}
  \begin{enumerate}
    \setcounter{enumi}{2}
    \item
      Realization of {\color{Mittel-Blau}efficient solutions}
      \begin{itemize}
        \item
          If trivial solution is $\in \mathcal{O}(1)$
        \item
          And separation / combination of subproblems is $\in \mathcal{O}(n)$
        \item
          And the number of subproblems is limited
        \item
          The runtime is $\in \mathcal{O}(n \, \log n)$
      \end{itemize}
  \end{enumerate}
\end{frame}

%-------------------------------------------------------------------------------

\begin{frame}{Divide and Conquer}{Implementation}
  \textbf{Trivial solution:}
  \begin{itemize}
    \item
      Small directly solvable subproblems are smart and easy
    \item
      Otherwise the efficiency will be improved if relative big subproblems
      can be solved directly
    \item
      Recursion depth should be small (stack / memory overhead)
  \end{itemize}
\end{frame}

%-------------------------------------------------------------------------------

\begin{frame}{Divide and Conquer}{Implementation}
  \textbf{Division of problems:}
  \begin{itemize}
    \item
      Choosing the number of subproblems and the concrete allocation can be
      demanding
  \end{itemize}
  \vspace{1em}
  \textbf{Combination of solutions:}
  \begin{itemize}
    \item
      Typically conceptional demanding
  \end{itemize}
\end{frame}

%-------------------------------------------------------------------------------

\begin{frame}{Divide and Conquer}{Example - Maximum Subtotal}
  \textbf{Input:}
  \begin{itemize}
    \item
      Progression $X \in \mathbb{G}^n$ of $n$ whole numbers
  \end{itemize}
  \textbf{Output:}
  \begin{itemize}
    \item
      Maximum sum of related subsequence and its index boundary
  \end{itemize}
\end{frame}

%-------------------------------------------------------------------------------

\begin{frame}{Divide and Conquer}{Example - Maximum Subtotal}
  \begin{example}[Maximum Subtotal]
    \vspace{-1em}
    \begin{table}[!t]
      \caption{Input values}
      \begin{tabular}{c|c|c|c|c|c|c|c|c|c|c}
        Index & 0 & 1 & 2 & 3 & 4 & 5 & 6 & 7 & 8 & 9\\
        \midrule
        Value & 31 & -41 & 59 & 26 & -53 & 58 & 97 & -93 & -23 & 84
      \end{tabular}
      \label{tab:divide_and_conquer:max_subtotal_example_values}
    \end{table}
    \vspace{6em}
    %TODO: Hand-Drawings here (Free space) or no free space?
    \textbf{Output:} Sum: 187, Start: 2, End: 6
  \end{example}
\end{frame}

%-------------------------------------------------------------------------------

\begin{frame}{Divide and Conquer}{Example - Maximum Subtotal}
  \textbf{Application:}
  \begin{itemize}
    \item
      Maximum profit of buying and selling shares
  \end{itemize}
  
  TODO: Graphics -> look at slide 31 \vspace{2em}
  
\end{frame}

%-------------------------------------------------------------------------------

\begin{frame}{Divide and Conquer}{Brute Force}
  
  TODO: Code -> look at slide 32 \vspace{2em}
  
\end{frame}

%-------------------------------------------------------------------------------

\begin{frame}{Divide and Conquer}{Python implementation}
  
  TODO: Python Code -> look at slide 33 \vspace{2em}
  
\end{frame}

%-------------------------------------------------------------------------------

\begin{frame}{Divide and Conquer}{Runtime upper bound}
  
  TODO: Code -> look at slide 34 \vspace{2em}
  
\end{frame}

%-------------------------------------------------------------------------------

\begin{frame}{Divide and Conquer}{Runtime lower bound}

TODO: Table -> look at slide 35 \vspace{2em}

  \begin{itemize}
    \item
      There are a minimum of $\dfrac{n}{3}$ values for u, for $\dfrac{n}{3}$ 
      values passed through o, for which again run through $\dfrac{n}{3}$ 
      addition steps in the inner loop
    \item
      We need a minimum of $T(n) = \left(\dfrac{n}{3}\right)^3 \in \Omega 
      (n^3)$ steps
    \item
      $T(n) \in \mathcal{O}(n^3)$ and $T(n) \in \Omega(n^3)$ follows $T(n) \in 
      \Theta (n^3)$
    \item
      it is hard to solve the problem worse ...
  \end{itemize}
\end{frame}

%-------------------------------------------------------------------------------

\begin{frame}{Divide and Conquer}{More efficient alternative}
  ... but its easier to solve it better
  \begin{itemize}
    \item
      Until now:\\
      For every u and for every o we calculate $S_{u,o} = X(u) + X(u+1) + \dots 
      + X(o)$
    \item
      More efficient:\\
      incremental actualization instead of pass through loops\\
      from $S_{u,o+1} = X(u) + X(u+1) + \dots + X(o) + X(o+1)$\\
      and $S_{u,o} = X(u) + X(u+1) + \dots + X(o)$\\
      follows $S_{u,o+1} = S_{u,o} + X(o+1)$ $\rightarrow$ $\mathcal{O}(1)$ 
      instead of $\mathcal{O}(n)$
  \end{itemize}
\end{frame}

%-------------------------------------------------------------------------------

\begin{frame}{Divide and Conquer}{Better solution}
  
  TODO: Code -> look at slide 37 \vspace{2em}
  
\end{frame}

%-------------------------------------------------------------------------------

\begin{frame}{Divide and Conquer}{Python Code}
  
  TODO: Python Code -> look at slide 38 \vspace{2em}
  
\end{frame}

%-------------------------------------------------------------------------------

\begin{frame}{Divide and Conquer}{Idea}
  \begin{itemize}
    \item
    Solve the problem for the left and the right half of X
    \item
      Put the parts of the solution together to the overall solution
  \end{itemize}
  
  TODO: Graphics -> look at slide 39 \vspace{2em}
  
  \begin{itemize}
    \item
      The Maximum is either in the left half (A) or in the right half (B)
    \item
      The Maximum can also be on the border of both half's (C) 
    \item
      For C we have to determine rmax and lmax
    \item
      The overall solution is a maximum of A, B, C
  \end{itemize}
\end{frame}

%-------------------------------------------------------------------------------

\begin{frame}{Divide and Conquer}{Concept}
  \begin{itemize}
    \item
      Small problems will be solved directly: n=1 $\Rightarrow$ Max = X(0)
    \item
      Big problems will be divided in two subproblems and solved recursively. 
      Parts of the solution A and B will be delivered.
  \end{itemize}
  
  TODO: Graphics -> look at slide 40 \vspace{2em}
  
  \begin{itemize}
    \item
      To determine parts of the solution C, we will determine rmax and lmax for 
      the subproblem.
    \item
      The overall solution results in the maximum of A, B, C
  \end{itemize}
\end{frame}

%-------------------------------------------------------------------------------

\begin{frame}{Divide and Conquer}{Implementation}
  
  TODO: Code -> look at slide 42 \vspace{2em}
  
\end{frame}

%-------------------------------------------------------------------------------

\begin{frame}{Divide and Conquer}{Python Code}
  
  TODO: Python Code -> look at slide 43 \vspace{2em}
  
\end{frame}