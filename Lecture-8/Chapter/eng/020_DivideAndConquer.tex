\section{Divide and Conquer}
\toclesssubsection{Introduction}
\label{chap:divide_and_conquer}

%-------------------------------------------------------------------------------

\begin{frame}{Divide and Conquer}{Introduction}
  \textbf{Concept:}
  \begin{itemize}
    \item
      {\color{Mittel-Blau}Divide} the problem into smaller subproblems
    \item
      {\color{Mittel-Blau}Conquer} the subproblems through recusive solving.\\
      If subproblems are small enough solve them directly
    \item
      {\color{Mittel-Blau}Connect} all solutions to as total solution of the
      problem
  \end{itemize}
  \vspace{1.0em}
  \begin{itemize}
    \item
      {\color{Mittel-Blau}Recursive} application of the algorithm for smaller
      subproblems
    \item
      {\color{Mittel-Blau}Direct} solving of small subproblems
  \end{itemize}
\end{frame}

%-------------------------------------------------------------------------------

\codeslide{python}{
\begin{frame}{Divide and Conquer}{Introduction - Python}
  \vspace{-0.5em}
  \lstinputlisting[
    language=Python,
    basicstyle=\small,
    tabsize=4,
    style={python-idle-code},
    escapechar={@},
    emph={solve},
    emphstyle=\color{blue}
  ]{Code/DivideAndConquer/DivideAndConquer_Concept.py}
\end{frame}
}

%TODO: Implement for Java / C++

%-------------------------------------------------------------------------------

\begin{frame}{Divide and Conquer}{Features}
  \begin{enumerate}
    \item
      Can help with conceptual hard problems
      \begin{itemize}
        \item
          {\color{Mittel-Blau}Solution} of the trivial problems has to be known
        \item
          {\color{Mittel-Blau}Dividing} in subproblems has to be possible
        \item
          {\color{Mittel-Blau}Combination} of solutions has to be possible
      \end{itemize}
    \item
      Suitable for parallel processing
      \begin{itemize}
        \item
          Subproblems are {\color{Mittel-Blau}independent} of each other
        \item
          Only needed input for each subproblem has to be known
      \end{itemize}
  \end{enumerate}
\end{frame}

%-------------------------------------------------------------------------------

\begin{frame}{Divide and Conquer}{Features}
  \begin{enumerate}
    \setcounter{enumi}{2}
    \item
      Realization of {\color{Mittel-Blau}efficient solutions}
      \begin{itemize}
        \item
          If trivial solution is $\in \mathcal{O}(1)$
        \item
          And separation / combination of subproblems is $\in \mathcal{O}(n)$
        \item
          And the number of subproblems is limited
        \item
          The runtime is $\in \mathcal{O}(n \, \log n)$
      \end{itemize}
  \end{enumerate}
\end{frame}

%-------------------------------------------------------------------------------

\begin{frame}{Divide and Conquer}{Implementation}
  \textbf{Trivial solution:}
  \begin{itemize}
    \item
      Small directly solvable subproblems are smart and easy
    \item
      Otherwise the efficiency will be improved if relative big subproblems
      can be solved directly
    \item
      Recursion depth should be small (stack / memory overhead)
  \end{itemize}
\end{frame}

%-------------------------------------------------------------------------------

\begin{frame}{Divide and Conquer}{Implementation}
  \textbf{Division of problems:}
  \begin{itemize}
    \item
      Choosing the number of subproblems and the concrete allocation can be
      demanding
  \end{itemize}
  \vspace{1em}
  \textbf{Combination of solutions:}
  \begin{itemize}
    \item
      Typically conceptional demanding
  \end{itemize}
\end{frame}

%-------------------------------------------------------------------------------

\begin{frame}{Divide and Conquer}{Example - Maximum Subtotal}
  \textbf{Input:}
  \begin{itemize}
    \item
      Progression $X \in \mathbb{G}^n$ of $n$ whole numbers
  \end{itemize}
  \textbf{Output:}
  \begin{itemize}
    \item
      Maximum sum of related subsequence and its index boundary
  \end{itemize}
\end{frame}

%-------------------------------------------------------------------------------

\begin{frame}{Divide and Conquer}{Example - Maximum Subtotal}
  \begin{example}[Maximum Subtotal]
    \vspace{-1em}
    \begin{table}[!t]
      \caption{Input values}
      \begin{tabular}{c|c|c|c|c|c|c|c|c|c|c}
        Index & 0 & 1 & 2 & 3 & 4 & 5 & 6 & 7 & 8 & 9\\
        \midrule
        Value & 31 & -41 & 59 & 26 & -53 & 58 & 97 & -93 & -23 & 84
      \end{tabular}
      \label{tab:divide_and_conquer:max_subtotal_example_values}
    \end{table}
    \vspace{6em}
    %TODO: Hand-Drawings here (Free space) or no free space?
    \textbf{Output:} Sum: 187, Start: 2, End: 6
  \end{example}
\end{frame}

%-------------------------------------------------------------------------------

\begin{frame}{Divide and Conquer}{Example - Maximum Subtotal}
  \textbf{Application:}
  \begin{itemize}
    \item
      Maximum profit of buying and selling shares
  \end{itemize}
  \begin{figure}
    \begin{adjustbox}{width=\linewidth}
      \begin{tikzpicture}[
  graph/.style={
    ultra thick,
    color=Mittel-Gruen
  },
  graph_fill/.style={
    fill=Hell-Gruen,
    fill opacity=0.20
  }
]
\begin{axis}[
  width=\linewidth,
  height=0.5\linewidth,
%  xlabel={time},
  ylabel={value},
  grid=major,
  y tick label style={
    /pgf/number format/fixed,
    /pgf/number format/fixed zerofill,
    /pgf/number format/precision=1
  },
  scaled y ticks=false,
  restrict y to domain*=5:20,
  xmin=0,
  xmax=100,
  ymin=5,
  ymax=20
]
\path[name path=axis] (axis cs:0,5) -- (axis cs:100,5);
\addplot[name path=f, graph, domain=0:100, samples=175, smooth]
  (\x, {
    % Offset
    10
    % Pull the left edge up
    + 5.0 / ln(0.01*\x*\x + 2.0)
    % Some fluctuations
    + 5.0 * sin(5.0*\x) * sin(4.0*\x + 0.2) * sin(4.0*\x)
    % Gauss noise generation (via Box-Muller)
    + 0.5
      * sqrt(-2 * ln(0.5 * rand + 0.5)) * cos(rand)
      * (sin(51.0*\x + 1.0) + cos(29.0*\x + 2.0))
  });
\addplot[graph_fill] fill between[of=f and axis, soft clip={domain=0:100}];
\end{axis}
\end{tikzpicture}
    \end{adjustbox}
    \caption{Share value over time}
    \label{fig:divide_and_conquer:shares_value}
  \end{figure}
\end{frame}

%-------------------------------------------------------------------------------

\codeslide{python}{
\begin{frame}{Divide and Conquer}{Example - Maximum Subtotal - Python}
  \vspace{-0.5em}
  \lstinputlisting[
    language=Python,
    basicstyle=\small,
    tabsize=4,
    style={python-idle-code},
    escapechar={@},
    emph={maxSubArray},
    emphstyle=\color{blue}
  ]{Code/DivideAndConquer/MaxSubTotal_Naive.py}
\end{frame}

%-------------------------------------------------------------------------------

\begin{frame}{Divide and Conquer}{Example - Maximum Subtotal - Python}
  \vspace{-0.5em}
  \lstinputlisting[
    language=Python,
    basicstyle=\small,
    tabsize=4,
    style={python-idle-code},
    escapechar={@},
    emph={maxSubArray},
    emphstyle=\color{blue}
  ]{Code/DivideAndConquer/MaxSubTotal_Naive_Runtime.py}
\end{frame}
}

%TODO: Implement for Java / C++

%-------------------------------------------------------------------------------

\begin{frame}{Divide and Conquer}{Example - Maximum Subtotal}
  \textbf{Upper bound:}
  \begin{itemize}
    \item
      Three interleaved loops
    \item
      Each loop with runtime {\color{Mittel-Blau}$\mathcal{O}(n)$}
    \item
      Algorithm runtime of {\color{Mittel-Blau}$\mathcal{O}(n^3)$}
  \end{itemize}
\end{frame}

%-------------------------------------------------------------------------------

\begin{frame}{Divide and Conquer}{Example - Maximum Subtotal - Runtime}
  \textbf{Lower bound:}
  \begin{itemize}
    \item
      We iterate at least $\frac{n}{3}$ values for $i$
    \item
      For each $i$ we iterate at least $\frac{n}{3}$ values for $j$
    \item
      For each $j$ we have at least $\frac{n}{3}$ additions
    \item
      We need at least $T(n) = (\frac{n}{3})^3 \in \Omega(n)$ steps
  \end{itemize}
  \begin{table}
    \caption{Operations}
    \label{fig:divide_and_conquer:max_sub_total_operations}
    \begin{tabular}{c|c|c}
      $i$ & Additions & $j$\\
      \midrule
      $\frac{n}{3} \in \mathcal{O}(n)$ &
      $\frac{n}{3} \in \mathcal{O}(n)$ &
      $\frac{n}{3} \in \mathcal{O}(n)$\\
    \end{tabular}
  \end{table}
\end{frame}

%-------------------------------------------------------------------------------

\begin{frame}{Divide and Conquer}{Example - Maximum Subtotal - Runtime}
  \textbf{Runtime:}
  \begin{itemize}
    \item
      With $T(n) \in {\color{Mittel-Blau}\mathcal{O}(n^3)}$ and
      $T(n) \in {\color{Mittel-Blau}\Omega(n^3)}$ we know:
      \begin{displaymath}
        \color{Mittel-Blau}
        T(n) \in \Theta(n^3)
      \end{displaymath}
    \item
      It is hard to solve the problem worse $\ldots$
  \end{itemize}
\end{frame}

%-------------------------------------------------------------------------------

\begin{frame}{Divide and Conquer}{Example - Maximum Subtotal - Runtime}
  \textbf{Current approach:}
  \begin{itemize}
    \item
      Calculating the sum for range from $i$ to $j$ with loop
      \begin{displaymath}
        S_{i,\,j} = X[i] + X[i+1] + \dots + X[j]
      \end{displaymath}
   \end{itemize}
   \textbf{Better approach:}
   \begin{itemize}
    \item
      Incremental sum instead of loop
      \begin{eqnarray*}
        S_{i,\,j+1} &=& X[i] + X[i+1] + \dots + X[j] + X[j+1]\\
        S_{i,\,j+1} &=& S_{i,\,j} + X[j+1]
        ~{\color{Mittel-Blau}\in \mathcal{O}(1)}
      \end{eqnarray*}
  \end{itemize}
\end{frame}

%-------------------------------------------------------------------------------

\codeslide{python}{
\begin{frame}{Divide and Conquer}{Example - Maximum Subtotal - Python}
  \vspace{-0.5em}
  \lstinputlisting[
    language=Python,
    basicstyle=\small,
    tabsize=4,
    style={python-idle-code},
    escapechar={@},
    emph={maxSubArray},
    emphstyle=\color{blue}
  ]{Code/DivideAndConquer/MaxSubTotal.py}
\end{frame}
}

%TODO: Implement for Java / C++

%-------------------------------------------------------------------------------

\begin{frame}{Divide and Conquer}{Example - Maximum Subtotal}
  \textbf{Divide and Conquer:}
  \begin{itemize}
    \item
      Solve the left / right half of the problem {\color{Mittel-Blau}recursive}
    \item
      Small problems are solved directly: $n = 1 \Rightarrow \max = X[0]$
    \item
      Combine both solutions into a total solution
  \end{itemize}
  \begin{figure}
    \begin{adjustbox}{width=\linewidth}
      \begin{tikzpicture}[
  list/.style={
    draw=black
  }, sum/.style={
    list,
    fill=Hell-Gruen
  }, label_sum/.style={
    color=black,
    font=\huge
  }
]%
% upper list
\draw[list] (0, 0) rectangle ++(16, 1);

\draw[sum] (1.5, 0) rectangle ++(2, 1);
\draw[sum] (6, 0) rectangle ++(5, 1);
\draw[sum] (13, 0) rectangle ++(2.5, 1);

\node[label_sum, anchor=south] at (2.5, 0) {A};
\node[label_sum, anchor=south] at (14.25, 0) {B};
\node[label_sum, anchor=south] at (7, 0) {rmax};
\node[label_sum, anchor=south] at (9.5, 0) {lmax};

\draw[list] (8, 1.5) -- (8, -0.5);

\draw[
  decorate,
  decoration={brace, raise=0.25em, amplitude=0.5em},
  thick
] (6, 1) -- node[midway, yshift=2em, label_sum] {C} (11, 1);
\end{tikzpicture}
    \end{adjustbox}
    \caption{Dividing the problem into two sub-problems}
    \label{fig:divide_and_conquer:max_sub_total_divide}
  \end{figure}
\end{frame}

%-------------------------------------------------------------------------------

\begin{frame}{Divide and Conquer}{Example - Maximum Subtotal}
  \textbf{Divide and Conquer:}
  \begin{itemize}
    \item
      The maximum is located in the {\color{Mittel-Blau}left~half~($A$)}
      or the {\color{Mittel-Blau}right~half~($B$)}
    \item
      The maximum can {\color{Mittel-Blau}extend into the second half ($C$)}
    \item
      To solve $C$ we have to calculate $rmax$ and $lmax$
    \item
      The total solution is the {\color{Mittel-Blau}maximum of $A$, $B$ and $C$}
  \end{itemize}
  \begin{figure}
    \begin{adjustbox}{width=\linewidth}
      \begin{tikzpicture}[
  list/.style={
    draw=black
  }, sum/.style={
    list,
    fill=Hell-Gruen
  }, label_sum/.style={
    color=black,
    font=\huge
  }
]%
% upper list
\draw[list] (0, 0) rectangle ++(16, 1);

\draw[sum] (1.5, 0) rectangle ++(2, 1);
\draw[sum] (6, 0) rectangle ++(5, 1);
\draw[sum] (13, 0) rectangle ++(2.5, 1);

\node[label_sum, anchor=south] at (2.5, 0) {A};
\node[label_sum, anchor=south] at (14.25, 0) {B};
\node[label_sum, anchor=south] at (7, 0) {rmax};
\node[label_sum, anchor=south] at (9.5, 0) {lmax};

\draw[list] (8, 1.5) -- (8, -0.5);

\draw[
  decorate,
  decoration={brace, raise=0.25em, amplitude=0.5em},
  thick
] (6, 1) -- node[midway, yshift=2em, label_sum] {C} (11, 1);
\end{tikzpicture}
    \end{adjustbox}
    \caption{Dividing the problem into two sub-problems}
    \label{fig:divide_and_conquer:max_sub_total_divide2}
  \end{figure}
\end{frame}

%-------------------------------------------------------------------------------


\codeslide{python}{
\begin{frame}{Divide and Conquer}{Example - Maximum Subtotal - Python}
  \vspace{-0.5em}
  \lstinputlisting[
    language=Python,
    basicstyle=\small,
    tabsize=4,
    style={python-idle-code},
    escapechar={@},
    emph={maxSubArray},
    emphstyle=\color{blue}
  ]{Code/DivideAndConquer/MaxSubTotal_DivideAndConquer.py}
\end{frame}
}

%TODO: Implement for Java / C++