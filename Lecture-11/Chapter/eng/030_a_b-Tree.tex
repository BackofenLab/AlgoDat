\subsection{(a,b)-Trees}
\subsubsection{Introduction}

\begin{frame}{(a,b)-Trees}{Introduction}
  \textbf{(a,b)-Tree:}
  \begin{itemize}
    \item
      Also known as \textbf{b-tree} (b for \enquote{balanced})
    \item
      Used in data bases and file systems
  \end{itemize}
  \textbf{Idea:}
  \begin{itemize}
    \item
      Save a varying number of elements per node
    \item
      So we have space for elements on an \texttt{\color{Mittel-Blau}insert}
      and balance operation
  \end{itemize}
\end{frame}

%-------------------------------------------------------------------------------

\begin{frame}{(a,b)-Trees}{Introduction}
  \textbf{(a,b)-Tree:}
  \begin{itemize}
    \item
      All leaves have the same depth
    \item
      Each inner node has {\color{Mittel-Blau}$\geq a$} and
      {\color{Mittel-Blau}$\leq b$} nodes\\
      (Only the root node may have less nodes)
    \item
      Each node with $n$ children is named \enquote{node of degree $n$} and
      holds $n-1$ sorted elements
    \item
      Subtrees hang \enquote{between} the elements
    \item
      We require: {\color{Mittel-Blau}$a \geq 2$} and
      {\color{Mittel-Blau}$b \geq 2\,a - 1$}
  \end{itemize}
  \begin{figure}
    \begin{adjustbox}{width=0.3\linewidth}
      \begin{tikzpicture}[
  node/.style={
    rectangle,
    color=black,
    draw=Mittel-Blau,
    fill=Hell-Blau,
    line width=0.15em,
    minimum size=2.0em,
    inner sep=0em
  }, path_start/.style={
%    -{Computer Modern Rightarrow[length=0.5em]},
    -{Stealth[length=1.25em]},
    line width=0.25em,
    color=Mittel-Gruen
  }, path/.style={
    path_start,
%    {Circle[scale=0.5,round]}-{Computer Modern Rightarrow[length=0.5em]},
    {Circle[scale=0.5,round]}-{Stealth[length=1.25em]},
    shorten <=-0.35em
  }, level/.style={
    sibling distance=0.0em,
    level distance=3.0em
  }]
\node [node, xshift=-2em] {2}
child [path, xshift=-3em, edge from parent path={
  (\tikzparentnode.south west) -- (\tikzchildnode.north)}
] { };

\node [node] {10}
child [path, xshift=-2em, edge from parent path={
  (\tikzparentnode.south west) -- (\tikzchildnode.north)}
] { };

\node [node, xshift=2em] {18}
child [path, xshift=0em, edge from parent path={
  (\tikzparentnode.south west) -- (\tikzchildnode.north)}
] { }
child [path, xshift=3em, edge from parent path={
  (\tikzparentnode.south east) -- (\tikzchildnode.north)}
] { };
\end{tikzpicture}
    \end{adjustbox}
    \caption{Node of degree 3}
    \label{fig:a_b_tree:node_introduction}
  \end{figure}
\end{frame}

%-------------------------------------------------------------------------------

\begin{frame}{(a,b)-Trees}{Introduction}
  \textbf{(2,4)-Tree:}
  \begin{itemize}
    \item
      (2,4)-tree with depth of 3
    \item
      Each node has between 2 and 4 children (1 to 3 elements)
  \end{itemize}
  \begin{figure}
    \begin{adjustbox}{width=0.75\linewidth}
      \begin{tikzpicture}[
  node/.style={
    rectangle,
    color=black,
    draw=Mittel-Blau,
    fill=Hell-Blau,
    line width=0.15em,
    minimum size=2.0em,
    inner sep=0em
  }, path_start/.style={
%    -{Computer Modern Rightarrow[length=0.5em]},
    -{Stealth[length=1.25em]},
    line width=0.25em,
    color=Mittel-Gruen
  }, path/.style={
    path_start,
%    {Circle[scale=0.5,round]}-{Computer Modern Rightarrow[length=0.5em]},
    {Circle[scale=0.5,round]}-{Stealth[length=1.25em]},
    shorten <=-0.35em
  }, level/.style={
    sibling distance=0.0em,
    level distance=5.0em
  }]
\node [node] {23}
child [path, xshift=-5em, edge from parent path={
  (\tikzparentnode.south west) -- (node1.north)}
] {
  node [node, xshift=-2em] {2}
  child [path, xshift=-7.5em, edge from parent path={
    (\tikzparentnode.south west) -- (\tikzchildnode.north)}
  ] {
    node [node] {1}
  }
  node [node] (node1) {10}
  child [path, xshift=-4.5em, edge from parent path={
    (\tikzparentnode.south west) -- (node2.north)}
  ] {
    node [node, xshift=-2em] {3}
    node [node] (node2) {5}
    node [node, xshift=2em] {9}
  }
  node [node, xshift=2em] {18}
  child [path, xshift=-1.5em, edge from parent path={
    (\tikzparentnode.south west) -- (\tikzchildnode.north)}
  ] {
    node [node] {15}
  }
  child [path, xshift=1.5em, edge from parent path={
    (\tikzparentnode.south east) -- (node4.north east)}
  ] {
    node [node] (node4) {20}
    node [node, xshift=2em] {22}
  }
}
child [path, xshift=6em, edge from parent path={
  (\tikzparentnode.south east) -- (node5.north east)}
] {
  node [node, xshift=-1em] (node5) {25}
  child [path, xshift=-1.5em, edge from parent path={
    (\tikzparentnode.south west) -- (\tikzchildnode.north)}
  ] {
    node [node] {24}
  }
  node [node, xshift=1em] {33}
  child [path, xshift=-0.5em, edge from parent path={
    (\tikzparentnode.south west) -- (\tikzchildnode.north)}
  ] {
    node [node] {27}
  }
  child [path, xshift=2.5em, edge from parent path={
    (\tikzparentnode.south east) -- (node6.north east)}
  ] {
    node [node] (node6) {37}
    node [node, xshift=2em] {42}
  }
};
\end{tikzpicture}
    \end{adjustbox}
    \caption{Example of an (2,4)-tree}
    \label{fig:a_b_tree:tree_introduction}
  \end{figure}
\end{frame}

%-------------------------------------------------------------------------------

\begin{frame}{(a,b)-Trees}{Introduction}
  \textbf{Not an (2,4)-Tree:}
  \begin{itemize}
    \item
      Invalid sorting
    \item
      Degree of node too large / too small
    \item
      Leaves on different levels
  \end{itemize}
  \begin{figure}
    \begin{adjustbox}{width=0.5\linewidth}
      \begin{tikzpicture}[
  node/.style={
    rectangle,
    color=black,
    draw=Mittel-Blau,
    fill=Hell-Blau,
    line width=0.15em,
    minimum size=2.0em,
    inner sep=0em
  }, path_start/.style={
%    -{Computer Modern Rightarrow[length=0.5em]},
    -{Stealth[length=1.25em]},
    line width=0.25em,
    color=Mittel-Gruen
  }, path/.style={
    path_start,
%    {Circle[scale=0.5,round]}-{Computer Modern Rightarrow[length=0.5em]},
    {Circle[scale=0.5,round]}-{Stealth[length=1.25em]},
    shorten <=-0.35em
  }, level/.style={
    sibling distance=0.0em,
    level distance=5.0em
  }]
\node [node] {23}
child [path, xshift=-5em, edge from parent path={
  (\tikzparentnode.south west) -- (node1.north east)}
] {
  node [node, xshift=-2em] {2}
  child [path, xshift=-5.5em, edge from parent path={
    (\tikzparentnode.south west) -- (\tikzchildnode.north)}
  ] {
    node [node] {1}
    child [path, xshift=-1.5em, edge from parent path={
      (\tikzparentnode.south west) -- (\tikzchildnode.north)}
    ] {
      node [node] {0}
    }
    child [path, xshift=1.5em, edge from parent path={
      (\tikzparentnode.south east) -- (\tikzchildnode.north)}
    ] {
      node [node] {3}
    }
  }
  node [node] (node1) {18}
  child [path, xshift=-4.5em, edge from parent path={
    (\tikzparentnode.south west) -- (node2.north east)}
  ] {
    node [node] (node2) {5}
    node [node, xshift=2em] {9}
  }
  node [node, xshift=2em] {10}
  child [path, xshift=-1.5em, edge from parent path={
    (\tikzparentnode.south west) -- (\tikzchildnode.north)}
  ] {
    node [node] {15}
  }
  node [node, xshift=4em] {24}
  child [path, xshift=-0.5em, edge from parent path={
    (\tikzparentnode.south west) -- (node4.north east)}
  ] {
    node [node] (node4) {20}
    node [node, xshift=2em] {22}
  }
  child [path, xshift=4.5em, edge from parent path={
    (\tikzparentnode.south east) -- (\tikzchildnode.north)}
  ] {
    node [node] {25}
  }
}
child [path, xshift=6em, edge from parent path={
  (\tikzparentnode.south east) -- (\tikzchildnode.north)}
] {
  node [node] (node5) {33}
  child [path, xshift=0.5em, edge from parent path={
    (\tikzparentnode.south west) -- (\tikzchildnode.north)}
  ] {
    node [node] {27}
  }
};
\end{tikzpicture}
    \end{adjustbox}
    \caption{\textbf{Not} an (2,4)-tree}
    \label{fig:a_b_tree:tree_invalid_introduction}
  \end{figure}
\end{frame}

%-------------------------------------------------------------------------------

\begin{frame}{(a,b)-Trees}{Implementation - Lookup}
  \textbf{Searching an element:} (\texttt{\color{Mittel-Blau}lookup})
  \begin{itemize}
    \item
      The same algorithm as in {\color{Mittel-Blau}BinarySearchTree}
    \item
      Searching from the root downwards
    \item
      The keys at each node set the path
  \end{itemize}
\end{frame}

%-------------------------------------------------------------------------------

\begin{frame}{(a,b)-Trees}{Implementation - Insert}
  \textbf{Inserting an element:} (\texttt{\color{Mittel-Blau}insert})
  \begin{itemize}
    \item
      Search the position to insert the key into
    \item
      This position will always be an leaf
    \item
      Insert the element into the tree
    \item
      If the degree is higher than {\color{Mittel-Blau}$b+1$}
      we split the node
  \end{itemize}
\end{frame}

%-------------------------------------------------------------------------------

\begin{frame}{(a,b)-Trees}{Implementation - Insert}
  \textbf{Inserting an element:} (\texttt{\color{Mittel-Blau}insert})
  \begin{itemize}
    \item
      If the degree is higher than {\color{Mittel-Blau}$b+1$}
      we split the node
    \begin{itemize}
      \item
        This results in a node with
        {\color{Mittel-Blau}$\mathrm{ceil}\left(\frac{b-1}{2}\right)$} and
        {\color{Mittel-Blau}$\mathrm{floor}\left(\frac{b-1}{2}\right)$}
        elements and a element for the parent node
      \item
        Thats why we have the limit {\color{Mittel-Blau}$b \geq 2\,a - 1$}
    \end{itemize}
  \end{itemize}
  \begin{figure}
    \begin{columns}
      \begin{column}{0.35\linewidth}
        \begin{adjustbox}{width=\linewidth}
          \begin{tikzpicture}[
  node/.style={
    rectangle,
    color=black,
    draw=Mittel-Blau,
    fill=Hell-Blau,
    line width=0.15em,
    minimum size=2.0em,
    inner sep=0em
  }, path_start/.style={
%    -{Computer Modern Rightarrow[length=0.5em]},
    -{Stealth[length=1.25em]},
    line width=0.25em,
    color=Mittel-Gruen
  }, path/.style={
    path_start,
%    {Circle[scale=0.5,round]}-{Computer Modern Rightarrow[length=0.5em]},
    {Circle[scale=0.5,round]}-{Stealth[length=1.25em]},
    shorten <=-0.35em
  }, level/.style={
    sibling distance=0.0em,
    level distance=5.0em
  }, background/.style={
    fill=black!20!white,
    color=black!20!white
  }, full/.style={
    fill=orange!40!yellow,
    draw=red!20!orange
  }]
\node [node, xshift=-1em, background] { }
child [path, -, xshift=-5em, yshift=2em, background, edge from parent path={
  (\tikzparentnode.south west) -- (\tikzchildnode.north)}
] { };
\node [node, xshift=1em, background] { }
child [path, xshift=-1em, edge from parent path={
  (\tikzparentnode.south west) -- (node1.north east)}
] {
  node [node, full, xshift=-3em] {2}
  node [node, full, xshift=-1em] (node1) {10}
  node [node, full, xshift=1em] {15}
  node [node, full, xshift=3em] {24}
}
child [path, -, xshift=5em, yshift=2em, background, edge from parent path={
  (\tikzparentnode.south east) -- (\tikzchildnode.north)}
] { };
\end{tikzpicture}
        \end{adjustbox}
      \end{column}
      \begin{column}{0.1\linewidth}
        \begin{center}
          $\Rightarrow$
        \end{center}
      \end{column}
      \begin{column}{0.35\linewidth}
        \begin{adjustbox}{width=\linewidth}
          \begin{tikzpicture}[
  node/.style={
    rectangle,
    color=black,
    draw=Mittel-Blau,
    fill=Hell-Blau,
    line width=0.15em,
    minimum size=2.0em,
    inner sep=0em
  }, path_start/.style={
%    -{Computer Modern Rightarrow[length=0.5em]},
    -{Stealth[length=1.25em]},
    line width=0.25em,
    color=Mittel-Gruen
  }, path/.style={
    path_start,
%    {Circle[scale=0.5,round]}-{Computer Modern Rightarrow[length=0.5em]},
    {Circle[scale=0.5,round]}-{Stealth[length=1.25em]},
    shorten <=-0.35em
  }, level/.style={
    sibling distance=0.0em,
    level distance=5.0em
  }, background/.style={
    fill=black!20!white,
    color=black!20!white
  }, full/.style={
    fill=orange!40!yellow,
    draw=orange
  }]
\node [node, xshift=-2em, background] { }
child [path, -, xshift=-5em, yshift=2em, background, edge from parent path={
  (\tikzparentnode.south west) -- (\tikzchildnode.north)}
] { };
\node [node, xshift=2em, background] { }
child [path, -, xshift=5em, yshift=2em, background, edge from parent path={
  (\tikzparentnode.south east) -- (\tikzchildnode.north)}
] { };
\node [node] {15}
child [path, xshift=-2em, edge from parent path={
  (\tikzparentnode.south west) -- (node1.north east)}
] {
  node [node, xshift=-1em] (node1) {2}
  node [node, xshift=1em] {10}
}
child [path, xshift=1em, edge from parent path={
  (\tikzparentnode.south east) -- (\tikzchildnode.north)}
] {
  node [node, xshift=1em] {24}
};
\end{tikzpicture}
        \end{adjustbox}
      \end{column}
    \end{columns}
    \caption{Splitting a node}
    \label{fig:a_b_tree:insert_node_split}
  \end{figure}
\end{frame}

%-------------------------------------------------------------------------------

\begin{frame}{(a,b)-Trees}{Implementation - Insert}
  \textbf{Inserting an element:} (\texttt{\color{Mittel-Blau}insert})
  \begin{itemize}
    \item
      If the degree is higher than {\color{Mittel-Blau}$b+1$}
      we split the node
    \item
      Now the parent node can be of a higher degree than
      {\color{Mittel-Blau}$b+1$}
    \item
      We {\color{Mittel-Blau}split} the parent nodes the same way
    \item
      If the node to split is the root we split it and create a new root node\\
      (The tree is now one level higher)
  \end{itemize}
\end{frame}

%-------------------------------------------------------------------------------

\begin{frame}{(a,b)-Trees}{Implementation - Remove}
  \textbf{Removing an element:} (\texttt{\color{Mittel-Blau}remove})
  \begin{itemize}
    \item
      Search the element in $\mathcal{O}(\log n)$ time
    \item
      \textbf{Case 1:}
      The element is contained by a leaf, remove it
    \item
      \textbf{Case 2:}
      The element is contained by an inner node
      \begin{itemize}
        \item
          Search the {\color{Mittel-Blau}successor} in the right subtree
        \item
          The {\color{Mittel-Blau}successor} is always contained by a leaf
        \item
          Replace the element with its {\color{Mittel-Blau}successor} and
          delete the {\color{Mittel-Blau}successor} from the leaf
      \end{itemize}
    \item
      \textbf{Case 1+2:}
      The leaf might be too small (degree of {\color{Mittel-Blau}$a-1$})\\
      $\Rightarrow$ We {\color{Mittel-Blau}rebalance} the tree
  \end{itemize}
\end{frame}

%-------------------------------------------------------------------------------

\begin{frame}{(a,b)-Trees}{Implementation - Remove}
  \textbf{Removing an element:} (\texttt{\color{Mittel-Blau}remove})
  \begin{itemize}
    \item
      \textbf{Case 1+2:}
      The leaf might be too small (degree of {\color{Mittel-Blau}$a-1$})\\
      $\Rightarrow$ We {\color{Mittel-Blau}rebalance} the tree
      \vspace{0.5em}
      \begin{itemize}
        \item
        \textbf{Case a:}
        If the left or right neighbour node has a degree greater than
        {\color{Mittel-Blau}a} we \textbf{borrow} one element from this node
      \end{itemize}
      \begin{figure}
        \begin{columns}
          \begin{column}{0.45\linewidth}
            \begin{adjustbox}{width=\linewidth}
              \begin{tikzpicture}[
  node/.style={
    rectangle,
    color=black,
    draw=Mittel-Blau,
    fill=Hell-Blau,
    line width=0.15em,
    minimum size=2.0em,
    inner sep=0em
  }, path_start/.style={
%    -{Computer Modern Rightarrow[length=0.5em]},
    -{Stealth[length=1.25em]},
    line width=0.25em,
    color=Mittel-Gruen
  }, path/.style={
    path_start,
%    {Circle[scale=0.5,round]}-{Computer Modern Rightarrow[length=0.5em]},
    {Circle[scale=0.5,round]}-{Stealth[length=1.25em]},
    shorten <=-0.35em
  }, level/.style={
    sibling distance=0.0em,
    level distance=5.0em
  }, background/.style={
    fill=black!20!white,
    color=black!20!white
  }, full/.style={
    fill=orange!40!yellow,
    draw=orange
  }]
\node [node, xshift=-2em, background] { }
child [path, -, xshift=-5em, yshift=2em, background, edge from parent path={
  (\tikzparentnode.south west) -- (\tikzchildnode.north)}
] { };
\node [node, xshift=2em, background] { }
child [path, -, xshift=5em, yshift=2em, background, edge from parent path={
  (\tikzparentnode.south east) -- (\tikzchildnode.north)}
] { };
\node [node] {15}
child [path, xshift=-2em, edge from parent path={
  (\tikzparentnode.south west) -- (node1.north)}
] {
  node [node, xshift=-2em] {2}
  node [node, xshift=0em] (node1) {7}
  node [node, xshift=2em] {10}
}
child [path, xshift=2em, edge from parent path={
  (\tikzparentnode.south east) -- (\tikzchildnode.north)}
] {
  node [node, xshift=1em, minimum width=0.5em] {}
};
\end{tikzpicture}

            \end{adjustbox}
          \end{column}
          \begin{column}{0.1\linewidth}
            \begin{center}
              $\Rightarrow$
            \end{center}
          \end{column}
          \begin{column}{0.45\linewidth}
            \begin{adjustbox}{width=\linewidth}
              \begin{tikzpicture}[
  node/.style={
    rectangle,
    color=black,
    draw=Mittel-Blau,
    fill=Hell-Blau,
    line width=0.15em,
    minimum size=2.0em,
    inner sep=0em
  }, path_start/.style={
%    -{Computer Modern Rightarrow[length=0.5em]},
    -{Stealth[length=1.25em]},
    line width=0.25em,
    color=Mittel-Gruen
  }, path/.style={
    path_start,
%    {Circle[scale=0.5,round]}-{Computer Modern Rightarrow[length=0.5em]},
    {Circle[scale=0.5,round]}-{Stealth[length=1.25em]},
    shorten <=-0.35em
  }, level/.style={
    sibling distance=0.0em,
    level distance=5.0em
  }, background/.style={
    fill=black!20!white,
    color=black!20!white
  }, full/.style={
    fill=orange!40!yellow,
    draw=orange
  }]
\node [node, xshift=-2em, background] { }
child [path, -, xshift=-5em, yshift=2em, background, edge from parent path={
  (\tikzparentnode.south west) -- (\tikzchildnode.north)}
] { };
\node [node, xshift=2em, background] { }
child [path, -, xshift=5em, yshift=2em, background, edge from parent path={
  (\tikzparentnode.south east) -- (\tikzchildnode.north)}
] { };
\node [node] {10}
child [path, xshift=-2em, edge from parent path={
  (\tikzparentnode.south west) -- (node1.north east)}
] {
  node [node, xshift=-1em] (node1) {2}
  node [node, xshift=1em] {7}
}
child [path, xshift=2em, edge from parent path={
  (\tikzparentnode.south east) -- (\tikzchildnode.north)}
] {
  node [node] {15}
};
\end{tikzpicture}
            \end{adjustbox}
          \end{column}
        \end{columns}
        \caption{Borrowing an element}
        \label{fig:a_b_tree:move}
      \end{figure}
  \end{itemize}
\end{frame}

%-------------------------------------------------------------------------------

\begin{frame}{(a,b)-Trees}{Implementation - Remove}
  \textbf{Removing an element:} (\texttt{\color{Mittel-Blau}remove})
  \begin{itemize}
    \item
      \textbf{Case 1+2:}
      The leaf might be too small (degree of {\color{Mittel-Blau}$a-1$})\\
      $\Rightarrow$ We {\color{Mittel-Blau}rebalance} the tree
      \vspace{0.5em}
      \begin{itemize}
        \item
          \textbf{Case b:}
          We {\color{Mittel-Blau}combine} the node with its right or left
          neighbour
      \end{itemize}
      \begin{figure}
        \begin{columns}
          \begin{column}{0.45\linewidth}
            \begin{adjustbox}{width=\linewidth}
              \begin{tikzpicture}[
  node/.style={
    rectangle,
    color=black,
    draw=Mittel-Blau,
    fill=Hell-Blau,
    line width=0.15em,
    minimum size=2.0em,
    inner sep=0em
  }, path_start/.style={
%    -{Computer Modern Rightarrow[length=0.5em]},
    -{Stealth[length=1.25em]},
    line width=0.25em,
    color=Mittel-Gruen
  }, path/.style={
    path_start,
%    {Circle[scale=0.5,round]}-{Computer Modern Rightarrow[length=0.5em]},
    {Circle[scale=0.5,round]}-{Stealth[length=1.25em]},
    shorten <=-0.35em
  }, level/.style={
    sibling distance=0.0em,
    level distance=5.0em
  }, background/.style={
    fill=black!20!white,
    color=black!20!white
  }, full/.style={
    fill=orange!40!yellow,
    draw=orange
  }]
\node [node, xshift=-2em, background] { }
child [path, -, xshift=-5em, yshift=2em, background, edge from parent path={
  (\tikzparentnode.south west) -- (\tikzchildnode.north)}
] { };
\node [node, xshift=2em, background] { }
child [path, -, xshift=5em, yshift=2em, background, edge from parent path={
  (\tikzparentnode.south east) -- (\tikzchildnode.north)}
] { };
\node [node] {23}
child [path, xshift=-2em, edge from parent path={
  (\tikzparentnode.south west) -- (\tikzchildnode.north)}
] {
  node [node] {17}
}
child [path, xshift=2em, edge from parent path={
  (\tikzparentnode.south east) -- (\tikzchildnode.north)}
] {
  node [node, xshift=1em, minimum width=0.5em] {}
};
\end{tikzpicture}
            \end{adjustbox}
          \end{column}
          \begin{column}{0.1\linewidth}
            \begin{center}
              $\Rightarrow$
            \end{center}
          \end{column}
          \begin{column}{0.45\linewidth}
            \begin{adjustbox}{width=0.85\linewidth}
              \begin{tikzpicture}[
  node/.style={
    rectangle,
    color=black,
    draw=Mittel-Blau,
    fill=Hell-Blau,
    line width=0.15em,
    minimum size=2.0em,
    inner sep=0em
  }, path_start/.style={
%    -{Computer Modern Rightarrow[length=0.5em]},
    -{Stealth[length=1.25em]},
    line width=0.25em,
    color=Mittel-Gruen
  }, path/.style={
    path_start,
%    {Circle[scale=0.5,round]}-{Computer Modern Rightarrow[length=0.5em]},
    {Circle[scale=0.5,round]}-{Stealth[length=1.25em]},
    shorten <=-0.35em
  }, level/.style={
    sibling distance=0.0em,
    level distance=5.0em
  }, background/.style={
    fill=black!20!white,
    color=black!20!white
  }, full/.style={
    fill=orange!40!yellow,
    draw=red!20!orange
  }]
\node [node, xshift=-1em, background] { }
child [path, -, xshift=-5em, yshift=2em, background, edge from parent path={
  (\tikzparentnode.south west) -- (\tikzchildnode.north)}
] { };
\node [node, xshift=1em, background] { }
child [path, xshift=-1em, edge from parent path={
  (\tikzparentnode.south west) -- (node1.north east)}
] {
  node [node, xshift=-1em] (node1) {17}
  node [node, xshift=1em] {23}
}
child [path, -, xshift=5em, yshift=2em, background, edge from parent path={
  (\tikzparentnode.south east) -- (\tikzchildnode.north)}
] { };
\end{tikzpicture}
            \end{adjustbox}
          \end{column}
        \end{columns}
        \caption{Combining two nodes}
        \label{fig:a_b_tree:merge}
      \end{figure}
  \end{itemize}
\end{frame}

%-------------------------------------------------------------------------------

\begin{frame}{(a,b)-Trees}{Implementation - Remove}
  \textbf{Removing an element:} (\texttt{\color{Mittel-Blau}remove})
  \begin{itemize}
    \item
      Now the parent node can be of a lower degree than
      {\color{Mittel-Blau}$a$}
    \item
      We {\color{Mittel-Blau}combine} parent nodes the same way
    \item
      If the root has only one child left we take the child as new root\\
      (The tree is now one level smaller)
  \end{itemize}
\end{frame}

%-------------------------------------------------------------------------------

\subsubsection{Runtime Complexity}

\begin{frame}{(a,b)-Trees}{Runtime Complexity}
  \textbf{Runtime complexity of \texttt{\color{Mittel-Blau}lookup},
    \texttt{\color{Mittel-Blau}insert} and \texttt{\color{Mittel-Blau}remove}:}
  \begin{itemize}
    \item
      All operations in {\color{Mittel-Blau}$\mathcal{O}(d)$} with
      {\color{Mittel-Blau}$d$} being the depth of the tree
    \item
      Each node (except the root) has more than {\color{Mittel-Blau}$a$}
      children\\
      \begin{math}
        \Rightarrow
        \color{Mittel-Blau}
        n \geq a^{d-1}
      \end{math}
      and
      \begin{math}
        \color{Mittel-Blau}
        d \leq 1 + \log_a n = \mathcal{O}(log_a n)
      \end{math}
    \item
      If we look closer:
      \begin{itemize}
        \item
          \texttt{\color{Mittel-Blau}lookup} always takes
          \begin{math}
            \color{Mittel-Blau}
            \Theta(d)
          \end{math}
        \item
          \texttt{\color{Mittel-Blau}insert} and
          \texttt{\color{Mittel-Blau}remove}
          often require only $\mathcal{O}(1)$ time
        \item
          Only in the {\color{Mittel-Blau}worst case} we have to
          {\color{Mittel-Blau}split} or
          {\color{Mittel-Blau}combine} all nodes on a path up to the root
      \end{itemize}
  \end{itemize}
\end{frame}

%-------------------------------------------------------------------------------

\begin{frame}{(a,b)-Trees}{Runtime Complexity}
  \textbf{Runtime complexity of \texttt{\color{Mittel-Blau}lookup},
    \texttt{\color{Mittel-Blau}insert} and \texttt{\color{Mittel-Blau}remove}:}
  \begin{itemize}
    \item
      If we look closer:
      \begin{itemize}
        \item
          \texttt{\color{Mittel-Blau}insert} and
          \texttt{\color{Mittel-Blau}remove}
          often require only $\mathcal{O}(1)$ time
        \item
          Only in the {\color{Mittel-Blau}worst case} we have to
          {\color{Mittel-Blau}split} or
          {\color{Mittel-Blau}combine} all nodes on a path up to the root
      \end{itemize}
    \item
      We need {\color{Mittel-Blau}$b \geq 2 \, a$} instead of
      {\color{Mittel-Blau}$b \geq 2 \, a - 1$}
    \item
      Here is a negative example of a (2,3)-tree
  \end{itemize}
\end{frame}

%-------------------------------------------------------------------------------

\begin{frame}{(a,b)-Trees}{Runtime Complexity - Negative Example (2,3)-Tree}
  \textbf{(2,3)-Tree:}
  \begin{itemize}
    \item
      Before executing \texttt{\color{Mittel-Blau}delete(11)}
  \end{itemize}
  \begin{figure}
    \begin{adjustbox}{width=0.6\linewidth}
      \begin{tikzpicture}[
  node/.style={
    rectangle,
    color=black,
    draw=Mittel-Blau,
    fill=Hell-Blau,
    line width=0.15em,
    minimum size=2.0em,
    inner sep=0em
  }, path_start/.style={
%    -{Computer Modern Rightarrow[length=0.5em]},
    -{Stealth[length=1.25em]},
    line width=0.25em,
    color=Mittel-Gruen
  }, path/.style={
    path_start,
%    {Circle[scale=0.5,round]}-{Computer Modern Rightarrow[length=0.5em]},
    {Circle[scale=0.5,round]}-{Stealth[length=1.25em]},
    shorten <=-0.35em
  }, level/.style={
    sibling distance=0.0em,
    level distance=5.0em
  }]
\node [node] {8}
child [path, xshift=-6em, edge from parent path={
  (\tikzparentnode.south west) -- (\tikzchildnode.north)}
] {
  node [node, xshift=0em] {4}
  child [path, xshift=-3.0em, edge from parent path={
    (\tikzparentnode.south west) -- (\tikzchildnode.north)}
  ] {
    node [node] {2}
    child [path, xshift=-1.5em, edge from parent path={
      (\tikzparentnode.south west) -- (\tikzchildnode.north)}
    ] { node [node] {1} }
    child [path, xshift=1.5em, edge from parent path={
      (\tikzparentnode.south east) -- (\tikzchildnode.north)}
    ] { node [node] {3} }
  }
  child [path, xshift=3.0em, edge from parent path={
    (\tikzparentnode.south east) -- (\tikzchildnode.north)}
  ] {
    node [node] {6}
    child [path, xshift=-1.5em, edge from parent path={
      (\tikzparentnode.south west) -- (\tikzchildnode.north)}
    ] { node [node] {5} }
    child [path, xshift=1.5em, edge from parent path={
      (\tikzparentnode.south east) -- (\tikzchildnode.north)}
    ] { node [node] {7} }
  }
}
child [path, xshift=6em, edge from parent path={
  (\tikzparentnode.south east) -- (\tikzchildnode.north)}
] {
  node [node, xshift=0em] {12}
  child [path, xshift=-3.0em, edge from parent path={
    (\tikzparentnode.south west) -- (\tikzchildnode.north)}
  ] {
    node [node] {10}
    child [path, xshift=-1.5em, edge from parent path={
      (\tikzparentnode.south west) -- (\tikzchildnode.north)}
    ] { node [node] {9} }
    child [path, xshift=1.5em, edge from parent path={
      (\tikzparentnode.south east) -- (\tikzchildnode.north)}
    ] { node [node] {11} }
  }
  child [path, xshift=3.0em, edge from parent path={
    (\tikzparentnode.south east) -- (\tikzchildnode.north)}
  ] {
    node [node] {14}
    child [path, xshift=-1.5em, edge from parent path={
      (\tikzparentnode.south west) -- (\tikzchildnode.north)}
    ] { node [node] {13} }
    child [path, xshift=1.5em, edge from parent path={
      (\tikzparentnode.south east) -- (\tikzchildnode.north)}
    ] { node [node] {15} }
  }
};
\end{tikzpicture}
    \end{adjustbox}
    \label{fig:a_b_tree:2_3_tree_1}
    \caption{Normal (2,3)-Tree}
  \end{figure}
\end{frame}

%-------------------------------------------------------------------------------

\begin{frame}{(a,b)-Trees}{Runtime Complexity - Negative Example (2,3)-Tree}
  \textbf{(2,3)-Tree:}
  \begin{itemize}
    \item
      Executing \texttt{\color{Mittel-Blau}delete(11)}
  \end{itemize}
  \begin{figure}
    \begin{adjustbox}{width=0.6\linewidth}
      \begin{tikzpicture}[
  node/.style={
    rectangle,
    color=black,
    draw=Mittel-Blau,
    fill=Hell-Blau,
    line width=0.15em,
    minimum size=2.0em,
    inner sep=0em
  }, path_start/.style={
%    -{Computer Modern Rightarrow[length=0.5em]},
    -{Stealth[length=1.25em]},
    line width=0.25em,
    color=Mittel-Gruen
  }, path/.style={
    path_start,
%    {Circle[scale=0.5,round]}-{Computer Modern Rightarrow[length=0.5em]},
    {Circle[scale=0.5,round]}-{Stealth[length=1.25em]},
    shorten <=-0.35em
  }, level/.style={
    sibling distance=0.0em,
    level distance=5.0em
  }]
\node [node] {8}
child [path, xshift=-6em, edge from parent path={
  (\tikzparentnode.south west) -- (\tikzchildnode.north)}
] {
  node [node, xshift=0em] {4}
  child [path, xshift=-3.0em, edge from parent path={
    (\tikzparentnode.south west) -- (\tikzchildnode.north)}
  ] {
    node [node] {2}
    child [path, xshift=-1.5em, edge from parent path={
      (\tikzparentnode.south west) -- (\tikzchildnode.north)}
    ] { node [node] {1} }
    child [path, xshift=1.5em, edge from parent path={
      (\tikzparentnode.south east) -- (\tikzchildnode.north)}
    ] { node [node] {3} }
  }
  child [path, xshift=3.0em, edge from parent path={
    (\tikzparentnode.south east) -- (\tikzchildnode.north)}
  ] {
    node [node] {6}
    child [path, xshift=-1.5em, edge from parent path={
      (\tikzparentnode.south west) -- (\tikzchildnode.north)}
    ] { node [node] {5} }
    child [path, xshift=1.5em, edge from parent path={
      (\tikzparentnode.south east) -- (\tikzchildnode.north)}
    ] { node [node] {7} }
  }
}
child [path, xshift=6em, edge from parent path={
  (\tikzparentnode.south east) -- (\tikzchildnode.north)}
] {
  node [node, xshift=0em] {12}
  child [path, xshift=-3.0em, edge from parent path={
    (\tikzparentnode.south west) -- (\tikzchildnode.north)}
  ] {
    node [node] {10}
    child [path, xshift=-1.5em, edge from parent path={
      (\tikzparentnode.south west) -- (\tikzchildnode.north)}
    ] { node [node] {9} }
    child [path, xshift=1.5em, edge from parent path={
      (\tikzparentnode.south east) -- (\tikzchildnode.north)}
    ] { node [node, minimum width=0.5em] {} }
  }
  child [path, xshift=3.0em, edge from parent path={
    (\tikzparentnode.south east) -- (\tikzchildnode.north)}
  ] {
    node [node] {14}
    child [path, xshift=-1.5em, edge from parent path={
      (\tikzparentnode.south west) -- (\tikzchildnode.north)}
    ] { node [node] {13} }
    child [path, xshift=1.5em, edge from parent path={
      (\tikzparentnode.south east) -- (\tikzchildnode.north)}
    ] { node [node] {15} }
  }
};
\end{tikzpicture}
    \end{adjustbox}
    \label{fig:a_b_tree:2_3_tree_2}
    \caption{(2,3)-Tree - Delete step 1}
  \end{figure}
\end{frame}

%-------------------------------------------------------------------------------

\begin{frame}{(a,b)-Trees}{Runtime Complexity - Negative Example (2,3)-Tree}
  \textbf{(2,3)-Tree:}
  \begin{itemize}
    \item
      Executing \texttt{\color{Mittel-Blau}delete(11)}
  \end{itemize}
  \begin{figure}
    \begin{adjustbox}{width=0.6\linewidth}
      \begin{tikzpicture}[
  node/.style={
    rectangle,
    color=black,
    draw=Mittel-Blau,
    fill=Hell-Blau,
    line width=0.15em,
    minimum size=2.0em,
    inner sep=0em
  }, path_start/.style={
%    -{Computer Modern Rightarrow[length=0.5em]},
    -{Stealth[length=1.25em]},
    line width=0.25em,
    color=Mittel-Gruen
  }, path/.style={
    path_start,
%    {Circle[scale=0.5,round]}-{Computer Modern Rightarrow[length=0.5em]},
    {Circle[scale=0.5,round]}-{Stealth[length=1.25em]},
    shorten <=-0.35em
  }, level/.style={
    sibling distance=0.0em,
    level distance=5.0em
  }]
\node [node] {8}
child [path, xshift=-6em, edge from parent path={
  (\tikzparentnode.south west) -- (\tikzchildnode.north)}
] {
  node [node, xshift=0em] {4}
  child [path, xshift=-3.0em, edge from parent path={
    (\tikzparentnode.south west) -- (\tikzchildnode.north)}
  ] {
    node [node] {2}
    child [path, xshift=-1.5em, edge from parent path={
      (\tikzparentnode.south west) -- (\tikzchildnode.north)}
    ] { node [node] {1} }
    child [path, xshift=1.5em, edge from parent path={
      (\tikzparentnode.south east) -- (\tikzchildnode.north)}
    ] { node [node] {3} }
  }
  child [path, xshift=3.0em, edge from parent path={
    (\tikzparentnode.south east) -- (\tikzchildnode.north)}
  ] {
    node [node] {6}
    child [path, xshift=-1.5em, edge from parent path={
      (\tikzparentnode.south west) -- (\tikzchildnode.north)}
    ] { node [node] {5} }
    child [path, xshift=1.5em, edge from parent path={
      (\tikzparentnode.south east) -- (\tikzchildnode.north)}
    ] { node [node] {7} }
  }
}
child [path, xshift=6em, edge from parent path={
  (\tikzparentnode.south east) -- (\tikzchildnode.north)}
] {
  node [node, xshift=0em] {12}
  child [path, xshift=-3.0em, edge from parent path={
    (\tikzparentnode.south west) -- (\tikzchildnode.north)}
  ] {
    node [node, minimum width=0.5em] {}
    child [path, edge from parent path={
      (\tikzparentnode.south west) -- (node1.north east)}
    ] {
      node [node, xshift=-1.0em] (node1) {9}
      node [node, xshift=1.0em] {10}
    }
  }
  child [path, xshift=3.0em, edge from parent path={
    (\tikzparentnode.south east) -- (\tikzchildnode.north)}
  ] {
    node [node] {14}
    child [path, xshift=-1.5em, edge from parent path={
      (\tikzparentnode.south west) -- (\tikzchildnode.north)}
    ] { node [node] {13} }
    child [path, xshift=1.5em, edge from parent path={
      (\tikzparentnode.south east) -- (\tikzchildnode.north)}
    ] { node [node] {15} }
  }
};
\end{tikzpicture}
    \end{adjustbox}
    \label{fig:a_b_tree:2_3_tree_3}
    \caption{(2,3)-Tree - Delete step 2}
  \end{figure}
\end{frame}

%-------------------------------------------------------------------------------

\begin{frame}{(a,b)-Trees}{Runtime Complexity - Negative Example (2,3)-Tree}
  \textbf{(2,3)-Tree:}
  \begin{itemize}
    \item
      Executing \texttt{\color{Mittel-Blau}delete(11)}
  \end{itemize}
  \begin{figure}
    \begin{adjustbox}{width=0.6\linewidth}
      \begin{tikzpicture}[
  node/.style={
    rectangle,
    color=black,
    draw=Mittel-Blau,
    fill=Hell-Blau,
    line width=0.15em,
    minimum size=2.0em,
    inner sep=0em
  }, path_start/.style={
%    -{Computer Modern Rightarrow[length=0.5em]},
    -{Stealth[length=1.25em]},
    line width=0.25em,
    color=Mittel-Gruen
  }, path/.style={
    path_start,
%    {Circle[scale=0.5,round]}-{Computer Modern Rightarrow[length=0.5em]},
    {Circle[scale=0.5,round]}-{Stealth[length=1.25em]},
    shorten <=-0.35em
  }, level/.style={
    sibling distance=0.0em,
    level distance=5.0em
  }]
\node [node] {8}
child [path, xshift=-6em, edge from parent path={
  (\tikzparentnode.south west) -- (\tikzchildnode.north)}
] {
  node [node, xshift=0em] {4}
  child [path, xshift=-3.0em, edge from parent path={
    (\tikzparentnode.south west) -- (\tikzchildnode.north)}
  ] {
    node [node] {2}
    child [path, xshift=-1.5em, edge from parent path={
      (\tikzparentnode.south west) -- (\tikzchildnode.north)}
    ] { node [node] {1} }
    child [path, xshift=1.5em, edge from parent path={
      (\tikzparentnode.south east) -- (\tikzchildnode.north)}
    ] { node [node] {3} }
  }
  child [path, xshift=3.0em, edge from parent path={
    (\tikzparentnode.south east) -- (\tikzchildnode.north)}
  ] {
    node [node] {6}
    child [path, xshift=-1.5em, edge from parent path={
      (\tikzparentnode.south west) -- (\tikzchildnode.north)}
    ] { node [node] {5} }
    child [path, xshift=1.5em, edge from parent path={
      (\tikzparentnode.south east) -- (\tikzchildnode.north)}
    ] { node [node] {7} }
  }
}
child [path, xshift=6em, edge from parent path={
  (\tikzparentnode.south east) -- (\tikzchildnode.north)}
] {
  node [node, minimum width=0.5em] {}
  child [path, edge from parent path={
    (\tikzparentnode.south west) -- (node2.north east)}
  ] {
    node [node, xshift=-1.0em] (node2) {12}
    child [path, xshift=-2.5em, edge from parent path={
      (\tikzparentnode.south west) -- (node1.north east)}
    ] {
      node [node, xshift=-1.0em] (node1) {9}
      node [node, xshift=1.0em] {10}
    }
    node [node, xshift=1.0em] {14}
    child [path, xshift=-0.5em, edge from parent path={
      (\tikzparentnode.south west) -- (\tikzchildnode.north)}
    ] { node [node] {13} }
    child [path, xshift=2.5em, edge from parent path={
      (\tikzparentnode.south east) -- (\tikzchildnode.north)}
    ] { node [node] {15} }
  }
};
\end{tikzpicture}
    \end{adjustbox}
    \label{fig:a_b_tree:2_3_tree_4}
    \caption{(2,3)-Tree - Delete step 3}
  \end{figure}
\end{frame}

%-------------------------------------------------------------------------------

\begin{frame}{(a,b)-Trees}{Runtime Complexity - Negative Example (2,3)-Tree}
  \textbf{(2,3)-Tree:}
  \begin{itemize}
    \item
      Executed \texttt{\color{Mittel-Blau}delete(11)}
  \end{itemize}
  \begin{figure}
    \begin{adjustbox}{width=0.6\linewidth}
      \begin{tikzpicture}[
  node/.style={
    rectangle,
    color=black,
    draw=Mittel-Blau,
    fill=Hell-Blau,
    line width=0.15em,
    minimum size=2.0em,
    inner sep=0em
  }, path_start/.style={
%    -{Computer Modern Rightarrow[length=0.5em]},
    -{Stealth[length=1.25em]},
    line width=0.25em,
    color=Mittel-Gruen
  }, path/.style={
    path_start,
%    {Circle[scale=0.5,round]}-{Computer Modern Rightarrow[length=0.5em]},
    {Circle[scale=0.5,round]}-{Stealth[length=1.25em]},
    shorten <=-0.35em
  }, level/.style={
    sibling distance=0.0em,
    level distance=5.0em
  }]
\node [node, xshift=-1em] {4}
child [path, xshift=-7.0em, edge from parent path={
  (\tikzparentnode.south west) -- (\tikzchildnode.north)}
] {
  node [node] {2}
  child [path, xshift=-1.5em, edge from parent path={
    (\tikzparentnode.south west) -- (\tikzchildnode.north)}
  ] { node [node] {1} }
  child [path, xshift=1.5em, edge from parent path={
    (\tikzparentnode.south east) -- (\tikzchildnode.north)}
  ] { node [node] {3} }
}
node [node, xshift=1em] {8}
child [path, xshift=-3.0em, edge from parent path={
  (\tikzparentnode.south west) -- (\tikzchildnode.north)}
] {
  node [node] {6}
  child [path, xshift=-1.5em, edge from parent path={
    (\tikzparentnode.south west) -- (\tikzchildnode.north)}
  ] { node [node] {5} }
  child [path, xshift=1.5em, edge from parent path={
    (\tikzparentnode.south east) -- (\tikzchildnode.north)}
  ] { node [node] {7} }
}
child [path, xshift=6.0em, edge from parent path={
  (\tikzparentnode.south east) -- (node2.north east)}
] {
  node [node, xshift=-1.0em] (node2) {12}
  child [path, xshift=-2.5em, edge from parent path={
    (\tikzparentnode.south west) -- (node1.north east)}
  ] {
    node [node, xshift=-1.0em] (node1) {9}
    node [node, xshift=1.0em] {10}
  }
  node [node, xshift=1.0em] {14}
  child [path, xshift=-0.5em, edge from parent path={
    (\tikzparentnode.south west) -- (\tikzchildnode.north)}
  ] { node [node] {13} }
  child [path, xshift=2.5em, edge from parent path={
    (\tikzparentnode.south east) -- (\tikzchildnode.north)}
  ] { node [node] {15} }
};
\end{tikzpicture}
    \end{adjustbox}
    \label{fig:a_b_tree:2_3_tree_5}
    \caption{(2,3)-Tree - Delete step 4}
  \end{figure}
\end{frame}

%-------------------------------------------------------------------------------

\begin{frame}{(a,b)-Trees}{Runtime Complexity - Negative Example (2,3)-Tree}
  \textbf{(2,3)-Tree:}
  \begin{itemize}
    \item
      Executing \texttt{\color{Mittel-Blau}insert(11)}
  \end{itemize}
  \begin{figure}
    \begin{adjustbox}{width=0.7\linewidth}
      \begin{tikzpicture}[
  node/.style={
    rectangle,
    color=black,
    draw=Mittel-Blau,
    fill=Hell-Blau,
    line width=0.15em,
    minimum size=2.0em,
    inner sep=0em
  }, path_start/.style={
%    -{Computer Modern Rightarrow[length=0.5em]},
    -{Stealth[length=1.25em]},
    line width=0.25em,
    color=Mittel-Gruen
  }, path/.style={
    path_start,
%    {Circle[scale=0.5,round]}-{Computer Modern Rightarrow[length=0.5em]},
    {Circle[scale=0.5,round]}-{Stealth[length=1.25em]},
    shorten <=-0.35em
  }, level/.style={
    sibling distance=0.0em,
    level distance=5.0em
  }, full/.style={
    fill=orange!40!yellow,
    draw=orange
  }]
\node [node, xshift=-1em] {4}
child [path, xshift=-7.0em, edge from parent path={
  (\tikzparentnode.south west) -- (\tikzchildnode.north)}
] {
  node [node] {2}
  child [path, xshift=-1.5em, edge from parent path={
    (\tikzparentnode.south west) -- (\tikzchildnode.north)}
  ] { node [node] {1} }
  child [path, xshift=1.5em, edge from parent path={
    (\tikzparentnode.south east) -- (\tikzchildnode.north)}
  ] { node [node] {3} }
}
node [node, xshift=1em] {8}
child [path, xshift=-3.0em, edge from parent path={
  (\tikzparentnode.south west) -- (\tikzchildnode.north)}
] {
  node [node] {6}
  child [path, xshift=-1.5em, edge from parent path={
    (\tikzparentnode.south west) -- (\tikzchildnode.north)}
  ] { node [node] {5} }
  child [path, xshift=1.5em, edge from parent path={
    (\tikzparentnode.south east) -- (\tikzchildnode.north)}
  ] { node [node] {7} }
}
child [path, xshift=7.0em, edge from parent path={
  (\tikzparentnode.south east) -- (node2.north east)}
] {
  node [node, xshift=-1.0em] (node2) {12}
  child [path, xshift=-2.5em, edge from parent path={
    (\tikzparentnode.south west) -- (node1.north)}
  ] {
    node [node, full, xshift=-2.0em] {9}
    node [node, full, xshift=0.0em] (node1) {10}
    node [node, full, xshift=2.0em] {11}
  }
  node [node, xshift=1.0em] {14}
  child [path, xshift=0.5em, edge from parent path={
    (\tikzparentnode.south west) -- (\tikzchildnode.north)}
  ] { node [node] {13} }
  child [path, xshift=3.5em, edge from parent path={
    (\tikzparentnode.south east) -- (\tikzchildnode.north)}
  ] { node [node] {15} }
};
\end{tikzpicture}
    \end{adjustbox}
    \label{fig:a_b_tree:2_3_tree_6}
    \caption{(2,3)-Tree - Insert step 1}
  \end{figure}
\end{frame}

%-------------------------------------------------------------------------------

\begin{frame}{(a,b)-Trees}{Runtime Complexity - Negative Example (2,3)-Tree}
  \textbf{(2,3)-Tree:}
  \begin{itemize}
    \item
      Executing \texttt{\color{Mittel-Blau}insert(11)}
  \end{itemize}
  \begin{figure}
    \begin{adjustbox}{width=0.7\linewidth}
      \begin{tikzpicture}[
  node/.style={
    rectangle,
    color=black,
    draw=Mittel-Blau,
    fill=Hell-Blau,
    line width=0.15em,
    minimum size=2.0em,
    inner sep=0em
  }, path_start/.style={
%    -{Computer Modern Rightarrow[length=0.5em]},
    -{Stealth[length=1.25em]},
    line width=0.25em,
    color=Mittel-Gruen
  }, path/.style={
    path_start,
%    {Circle[scale=0.5,round]}-{Computer Modern Rightarrow[length=0.5em]},
    {Circle[scale=0.5,round]}-{Stealth[length=1.25em]},
    shorten <=-0.35em
  }, level/.style={
    sibling distance=0.0em,
    level distance=5.0em
  }, full/.style={
    fill=orange!40!yellow,
    draw=orange
  }]
\node [node, xshift=-1em] {4}
child [path, xshift=-7.0em, edge from parent path={
  (\tikzparentnode.south west) -- (\tikzchildnode.north)}
] {
  node [node] {2}
  child [path, xshift=-1.5em, edge from parent path={
    (\tikzparentnode.south west) -- (\tikzchildnode.north)}
  ] { node [node] {1} }
  child [path, xshift=1.5em, edge from parent path={
    (\tikzparentnode.south east) -- (\tikzchildnode.north)}
  ] { node [node] {3} }
}
node [node, xshift=1em] {8}
child [path, xshift=-3.0em, edge from parent path={
  (\tikzparentnode.south west) -- (\tikzchildnode.north)}
] {
  node [node] {6}
  child [path, xshift=-1.5em, edge from parent path={
    (\tikzparentnode.south west) -- (\tikzchildnode.north)}
  ] { node [node] {5} }
  child [path, xshift=1.5em, edge from parent path={
    (\tikzparentnode.south east) -- (\tikzchildnode.north)}
  ] { node [node] {7} }
}
child [path, xshift=5.0em, edge from parent path={
  (\tikzparentnode.south east) -- (node2.north)}
] {
  node [node, full, xshift=-2.0em] {10}
  child [path, xshift=-1.5em, edge from parent path={
    (\tikzparentnode.south west) -- (\tikzchildnode.north)}
  ] { node [node] {9} }
  node [node, full, xshift=0.0em] (node2) {12}
  child [path, xshift=-0.5em, edge from parent path={
    (\tikzparentnode.south west) -- (\tikzchildnode.north)}
  ] { node [node] {11} }
  node [node, full, xshift=2.0em] {14}
  child [path, xshift=0.5em, edge from parent path={
    (\tikzparentnode.south west) -- (\tikzchildnode.north)}
  ] { node [node] {13} }
  child [path, xshift=3.5em, edge from parent path={
    (\tikzparentnode.south east) -- (\tikzchildnode.north)}
  ] { node [node] {15} }
};
\end{tikzpicture}
    \end{adjustbox}
    \label{fig:a_b_tree:2_3_tree_7}
    \caption{(2,3)-Tree - Insert step 2}
  \end{figure}
\end{frame}

%-------------------------------------------------------------------------------

\begin{frame}{(a,b)-Trees}{Runtime Complexity - Negative Example (2,3)-Tree}
  \textbf{(2,3)-Tree:}
  \begin{itemize}
    \item
      Executing \texttt{\color{Mittel-Blau}insert(11)}
  \end{itemize}
  \begin{figure}
    \begin{adjustbox}{width=0.7\linewidth}
      \begin{tikzpicture}[
  node/.style={
    rectangle,
    color=black,
    draw=Mittel-Blau,
    fill=Hell-Blau,
    line width=0.15em,
    minimum size=2.0em,
    inner sep=0em
  }, path_start/.style={
%    -{Computer Modern Rightarrow[length=0.5em]},
    -{Stealth[length=1.25em]},
    line width=0.25em,
    color=Mittel-Gruen
  }, path/.style={
    path_start,
%    {Circle[scale=0.5,round]}-{Computer Modern Rightarrow[length=0.5em]},
    {Circle[scale=0.5,round]}-{Stealth[length=1.25em]},
    shorten <=-0.35em
  }, level/.style={
    sibling distance=0.0em,
    level distance=5.0em
  }, full/.style={
    fill=orange!40!yellow,
    draw=orange
  }]
\node [node, full, xshift=-2em] {4}
child [path, xshift=-7.0em, edge from parent path={
  (\tikzparentnode.south west) -- (\tikzchildnode.north)}
] {
  node [node] {2}
  child [path, xshift=-1.5em, edge from parent path={
    (\tikzparentnode.south west) -- (\tikzchildnode.north)}
  ] { node [node] {1} }
  child [path, xshift=1.5em, edge from parent path={
    (\tikzparentnode.south east) -- (\tikzchildnode.north)}
  ] { node [node] {3} }
}
node [node, full, xshift=0em] {8}
child [path, xshift=-3.0em, edge from parent path={
  (\tikzparentnode.south west) -- (\tikzchildnode.north)}
] {
  node [node] {6}
  child [path, xshift=-1.5em, edge from parent path={
    (\tikzparentnode.south west) -- (\tikzchildnode.north)}
  ] { node [node] {5} }
  child [path, xshift=1.5em, edge from parent path={
    (\tikzparentnode.south east) -- (\tikzchildnode.north)}
  ] { node [node] {7} }
}
node [node, full, xshift=2em] {12}
child [path, xshift=3.0em, edge from parent path={
  (\tikzparentnode.south west) -- (\tikzchildnode.north)}
] {
  node [node, xshift=-2.0em] {10}
  child [path, xshift=-1.5em, edge from parent path={
    (\tikzparentnode.south west) -- (\tikzchildnode.north)}
  ] { node [node] {9} }
  child [path, xshift=1.5em, edge from parent path={
    (\tikzparentnode.south east) -- (\tikzchildnode.north)}
  ] { node [node] {11} }
}
child [path, xshift=9.0em, edge from parent path={
  (\tikzparentnode.south east) -- (\tikzchildnode.north)}
] {
  node [node, xshift=-2.0em] {14}
  child [path, xshift=-1.5em, edge from parent path={
    (\tikzparentnode.south west) -- (\tikzchildnode.north)}
  ] { node [node] {13} }
  child [path, xshift=1.5em, edge from parent path={
    (\tikzparentnode.south east) -- (\tikzchildnode.north)}
  ] { node [node] {15} }
};
\end{tikzpicture}
    \end{adjustbox}
    \label{fig:a_b_tree:2_3_tree_8}
    \caption{(2,3)-Tree - Insert step 3}
  \end{figure}
\end{frame}

%-------------------------------------------------------------------------------

\begin{frame}{(a,b)-Trees}{Runtime Complexity - Negative Example (2,3)-Tree}
  \textbf{(2,3)-Tree:}
  \begin{itemize}
    \item
      Executed \texttt{\color{Mittel-Blau}insert(11)}
  \end{itemize}
  \begin{figure}
    \begin{adjustbox}{width=0.6\linewidth}
      \begin{tikzpicture}[
  node/.style={
    rectangle,
    color=black,
    draw=Mittel-Blau,
    fill=Hell-Blau,
    line width=0.15em,
    minimum size=2.0em,
    inner sep=0em
  }, path_start/.style={
%    -{Computer Modern Rightarrow[length=0.5em]},
    -{Stealth[length=1.25em]},
    line width=0.25em,
    color=Mittel-Gruen
  }, path/.style={
    path_start,
%    {Circle[scale=0.5,round]}-{Computer Modern Rightarrow[length=0.5em]},
    {Circle[scale=0.5,round]}-{Stealth[length=1.25em]},
    shorten <=-0.35em
  }, level/.style={
    sibling distance=0.0em,
    level distance=5.0em
  }]
\node [node] {8}
child [path, xshift=-6em, edge from parent path={
  (\tikzparentnode.south west) -- (\tikzchildnode.north)}
] {
  node [node, xshift=0em] {4}
  child [path, xshift=-3.0em, edge from parent path={
    (\tikzparentnode.south west) -- (\tikzchildnode.north)}
  ] {
    node [node] {2}
    child [path, xshift=-1.5em, edge from parent path={
      (\tikzparentnode.south west) -- (\tikzchildnode.north)}
    ] { node [node] {1} }
    child [path, xshift=1.5em, edge from parent path={
      (\tikzparentnode.south east) -- (\tikzchildnode.north)}
    ] { node [node] {3} }
  }
  child [path, xshift=3.0em, edge from parent path={
    (\tikzparentnode.south east) -- (\tikzchildnode.north)}
  ] {
    node [node] {6}
    child [path, xshift=-1.5em, edge from parent path={
      (\tikzparentnode.south west) -- (\tikzchildnode.north)}
    ] { node [node] {5} }
    child [path, xshift=1.5em, edge from parent path={
      (\tikzparentnode.south east) -- (\tikzchildnode.north)}
    ] { node [node] {7} }
  }
}
child [path, xshift=6em, edge from parent path={
  (\tikzparentnode.south east) -- (\tikzchildnode.north)}
] {
  node [node, xshift=0em] {12}
  child [path, xshift=-3.0em, edge from parent path={
    (\tikzparentnode.south west) -- (\tikzchildnode.north)}
  ] {
    node [node] {10}
    child [path, xshift=-1.5em, edge from parent path={
      (\tikzparentnode.south west) -- (\tikzchildnode.north)}
    ] { node [node] {9} }
    child [path, xshift=1.5em, edge from parent path={
      (\tikzparentnode.south east) -- (\tikzchildnode.north)}
    ] { node [node] {11} }
  }
  child [path, xshift=3.0em, edge from parent path={
    (\tikzparentnode.south east) -- (\tikzchildnode.north)}
  ] {
    node [node] {14}
    child [path, xshift=-1.5em, edge from parent path={
      (\tikzparentnode.south west) -- (\tikzchildnode.north)}
    ] { node [node] {13} }
    child [path, xshift=1.5em, edge from parent path={
      (\tikzparentnode.south east) -- (\tikzchildnode.north)}
    ] { node [node] {15} }
  }
};
\end{tikzpicture}
    \end{adjustbox}
    \label{fig:a_b_tree:2_3_tree_9}
    \caption{(2,3)-Tree - Insert step 4}
  \end{figure}
\end{frame}

%-------------------------------------------------------------------------------

\begin{frame}{(a,b)-Trees}{Runtime Complexity - Negative Example (2,3)-Tree}
  \begin{columns}
    \begin{column}{0.6\linewidth}
      \textbf{(2,3)-Tree:}
      \begin{itemize}
        \item
          We are exactly where we started
        \item
          If {\color{Mittel-Blau}$b=2\,a-1$} then we can create a sequence of
          \texttt{\color{Mittel-Blau}insert} and
          \texttt{\color{Mittel-Blau}remove} operatios wehere each operation
          costs {\color{Mittel-Blau}$\mathcal{O}(\log n)$}
        \item
          We need {\color{Mittel-Blau}$b \geq 2 \, a$} instead of
          {\color{Mittel-Blau}$b \geq 2 \, a - 1$}
      \end{itemize}
    \end{column}
    \begin{column}{0.4\linewidth}
      \begin{figure}
        \begin{adjustbox}{width=\linewidth}
          \begin{tikzpicture}[
  node/.style={
    rectangle,
    color=black,
    draw=Mittel-Blau,
    fill=Hell-Blau,
    line width=0.15em,
    minimum size=2.0em,
    inner sep=0em
  }, path_start/.style={
%    -{Computer Modern Rightarrow[length=0.5em]},
    -{Stealth[length=1.25em]},
    line width=0.25em,
    color=Mittel-Gruen
  }, path/.style={
    path_start,
%    {Circle[scale=0.5,round]}-{Computer Modern Rightarrow[length=0.5em]},
    {Circle[scale=0.5,round]}-{Stealth[length=1.25em]},
    shorten <=-0.35em
  }, level/.style={
    sibling distance=0.0em,
    level distance=5.0em
  }]
\node [node] {8}
child [path, xshift=-6em, edge from parent path={
  (\tikzparentnode.south west) -- (\tikzchildnode.north)}
] {
  node [node, xshift=0em] {4}
  child [path, xshift=-3.0em, edge from parent path={
    (\tikzparentnode.south west) -- (\tikzchildnode.north)}
  ] {
    node [node] {2}
    child [path, xshift=-1.5em, edge from parent path={
      (\tikzparentnode.south west) -- (\tikzchildnode.north)}
    ] { node [node] {1} }
    child [path, xshift=1.5em, edge from parent path={
      (\tikzparentnode.south east) -- (\tikzchildnode.north)}
    ] { node [node] {3} }
  }
  child [path, xshift=3.0em, edge from parent path={
    (\tikzparentnode.south east) -- (\tikzchildnode.north)}
  ] {
    node [node] {6}
    child [path, xshift=-1.5em, edge from parent path={
      (\tikzparentnode.south west) -- (\tikzchildnode.north)}
    ] { node [node] {5} }
    child [path, xshift=1.5em, edge from parent path={
      (\tikzparentnode.south east) -- (\tikzchildnode.north)}
    ] { node [node] {7} }
  }
}
child [path, xshift=6em, edge from parent path={
  (\tikzparentnode.south east) -- (\tikzchildnode.north)}
] {
  node [node, xshift=0em] {12}
  child [path, xshift=-3.0em, edge from parent path={
    (\tikzparentnode.south west) -- (\tikzchildnode.north)}
  ] {
    node [node] {10}
    child [path, xshift=-1.5em, edge from parent path={
      (\tikzparentnode.south west) -- (\tikzchildnode.north)}
    ] { node [node] {9} }
    child [path, xshift=1.5em, edge from parent path={
      (\tikzparentnode.south east) -- (\tikzchildnode.north)}
    ] { node [node] {11} }
  }
  child [path, xshift=3.0em, edge from parent path={
    (\tikzparentnode.south east) -- (\tikzchildnode.north)}
  ] {
    node [node] {14}
    child [path, xshift=-1.5em, edge from parent path={
      (\tikzparentnode.south west) -- (\tikzchildnode.north)}
    ] { node [node] {13} }
    child [path, xshift=1.5em, edge from parent path={
      (\tikzparentnode.south east) -- (\tikzchildnode.north)}
    ] { node [node] {15} }
  }
};
\end{tikzpicture}
        \end{adjustbox}
        \vspace{-1.5em}
        \label{fig:a_b_tree:2_3_tree_10}
        \caption{(2,3)-Tree}
      \end{figure}
    \end{column}
  \end{columns}
\end{frame}

%-------------------------------------------------------------------------------

\begin{frame}{(a,b)-Trees}{Runtime Complexity - (2,4)-Tree}
  \textbf{(2,4)-Tree:}
  \begin{itemize}
    \item
      If all nodes have {\color{Mittel-Blau}2 children} we have to
      {\color{Mittel-Blau}combine} the nodes up to the root on a
      \texttt{\color{Mittel-Blau}remove} operation
    \item
      If all nodes have {\color{Mittel-Blau}4 children} we have to
      {\color{Mittel-Blau}split} the nodes up to the root on a
      \texttt{\color{Mittel-Blau}insert} operation
    \item
      If all nodes have {\color{Mittel-Blau}3 children} it takes some time
      to reach one of the previous two states
    \item[$\Rightarrow$]
      \textbf{Nodes of degree 3 are harmless}\\
      Neither an \texttt{\color{Mittel-Blau}insert} nor a
      \texttt{\color{Mittel-Blau}remove} operation trigger rebalancing
      operations
  \end{itemize}
\end{frame}

%-------------------------------------------------------------------------------

\begin{frame}{(a,b)-Trees}{Runtime Complexity - (2,4)-Tree}
  \textbf{(2,4)-Tree:}
  \begin{itemize}
    \item
      \textbf{Idea:}
      \begin{itemize}
        \item
          After an expenive operation the tree is in a stable state
        \item
          It takes some time until the next expensive operation occurs
      \end{itemize}
    \item
      Like with our dynamic fields:
      \begin{itemize}
        \item
          {\color{Mittel-Blau}reallocation} is expensive but it takes some time
          until the next expensive operation occurs
        \item
          If we {\color{Mittel-Blau}overallocate} clever we have an amortized
          runtime of {\color{Mittel-Blau}$\mathcal{O}(1)$}
      \end{itemize}
  \end{itemize}
\end{frame}

%-------------------------------------------------------------------------------

\begin{frame}{(a,b)-Trees}{Runtime Complexity - (2,4)-Tree}
  \textbf{Terminology:}
  \begin{itemize}
    \item
      We analyze a sequence of {\color{Mittel-Blau}$n$} operations
    \item
      Let {\color{Mittel-Blau}$\Phi_i$} be the potential of the tree after
      the {\color{Mittel-Blau}$t$}-\textit{th} operation
    \item
      The potential {\color{Mittel-Blau}$\Phi_i$} is the number of nodes
      with {\color{Mittel-Blau}degree 3}
    \item
      We start with an empty tree: ${\color{Mittel-Blau}\Phi_0} = 0$
  \end{itemize}
\end{frame}

%-------------------------------------------------------------------------------

\begin{frame}{(a,b)-Trees}{Runtime Complexity - (2,4)-Tree}
  \textbf{Example:}
  \begin{itemize}
    \item
      Nodes of {\color{Mittel-Blau}degree 3} are highlighted
  \end{itemize}
  \begin{figure}[!h]
    \begin{adjustbox}{width=0.8\linewidth}
      \begin{tikzpicture}[
  node/.style={
    rectangle,
    color=black,
    draw=Mittel-Blau,
    fill=Hell-Blau,
    line width=0.15em,
    minimum size=2.0em,
    inner sep=0em
  }, path_start/.style={
%    -{Computer Modern Rightarrow[length=0.5em]},
    -{Stealth[length=1.25em]},
    line width=0.25em,
    color=Mittel-Gruen
  }, path/.style={
    path_start,
%    {Circle[scale=0.5,round]}-{Computer Modern Rightarrow[length=0.5em]},
    {Circle[scale=0.5,round]}-{Stealth[length=1.25em]},
    shorten <=-0.35em
  }, level/.style={
    sibling distance=0.0em,
    level distance=5.0em
  }, border/.style={
    rectangle,
    draw=orange!40!yellow,
    minimum height=2.0em,
    minimum width=4.0em,
    line width=1.0em
  }]
\node [node] {23}
child [path, xshift=-5em, edge from parent path={
  (\tikzparentnode.south west) -- (node1.north)}
] {
  node [node, xshift=-2em] {2}
  child [path, xshift=-7.5em, edge from parent path={
    (\tikzparentnode.south west) -- (node0.north east)}
  ] {
    node [border, xshift=-1.0em] {}
    node [node, xshift=-2em] (node0) {0}
    node [node] {1}
  }
  node [node] (node1) {10}
  child [path, xshift=-4.5em, edge from parent path={
    (\tikzparentnode.south west) -- (node2.north)}
  ] {
    node [node, xshift=-2em] {3}
    node [node] (node2) {5}
    node [node, xshift=2em] {9}
  }
  node [node, xshift=2em] {18}
  child [path, xshift=-1.5em, edge from parent path={
    (\tikzparentnode.south west) -- (\tikzchildnode.north)}
  ] {
    node [node] {15}
  }
  child [path, xshift=1.5em, edge from parent path={
    (\tikzparentnode.south east) -- (node4.north east)}
  ] {
    node [border, xshift=1.0em] {}
    node [node] (node4) {20}
    node [node, xshift=2em] {22}
  }
}
child [path, xshift=6em, edge from parent path={
  (\tikzparentnode.south east) -- (node5.north east)}
] {
  node [border, xshift=0.0em] {}
  node [node, xshift=-1em] (node5) {25}
  child [path, xshift=-1.5em, edge from parent path={
    (\tikzparentnode.south west) -- (\tikzchildnode.north)}
  ] {
    node [node] {24}
  }
  node [node, xshift=1em] {33}
  child [path, xshift=-0.5em, edge from parent path={
    (\tikzparentnode.south west) -- (\tikzchildnode.north)}
  ] {
    node [node] {27}
  }
  child [path, xshift=2.5em, edge from parent path={
    (\tikzparentnode.south east) -- (node6.north east)}
  ] {
    node [border, xshift=1.0em] {}
    node [node] (node6) {37}
    node [node, xshift=2em] {42}
  }
};
\end{tikzpicture}
    \end{adjustbox}
    \caption{Tree with potential ${\color{Mittel-Blau}\Phi} = 4$}
    \label{fig:a_b_tree:potential_introduction}
  \end{figure}
\end{frame}

%-------------------------------------------------------------------------------

\begin{frame}{(a,b)-Trees}{Runtime Complexity - (2,4)-Tree}
  \textbf{Terminology:}
  \begin{itemize}
    \item
      Let {\color{Mittel-Blau}$c_i$} be the costs of operation
      {\color{Mittel-Blau}$i$}
    \item
      The costs for operation {\color{Mittel-Blau}$i$} are coupled to the
      difference of the potential levels
      \begin{displaymath}
        {\color{Mittel-Blau}c_i} \leq A \cdot \underbrace{(
          {\color{Mittel-Blau}\Phi_i} -
          {\color{Mittel-Blau}\Phi_{i-1}}
        )} + B, \hspace{1.5em} A,B \in \mathbb{R}
      \end{displaymath}\\
      \vspace{-0.75em}
      Number of harmless ({\color{Mittel-Blau}degree 3}) nodes at operation
      {\color{Mittel-Blau}$i$}
  \end{itemize}
\end{frame}

%-------------------------------------------------------------------------------

\begin{frame}{(a,b)-Trees}{Runtime Complexity - (2,4)-Tree}
  \textbf{We will show:}
  \begin{itemize}
    \item
      Each operation can at maximum destroys one harmmless node
    \item
      Each operation producing costs creates a new harmless node
    \item
      The sum of all costs upto operation {\color{Mittel-Blau}$i$} will be
      {\color{Mittel-Blau}$\mathcal{O}(i)$}
    \item
      With that each operation has an amortitzed cost of
      {\color{Mittel-Blau}$\mathcal{O}(1)$}
  \end{itemize}
\end{frame}

%-------------------------------------------------------------------------------

\begin{frame}{(a,b)-Trees}{Runtime Complexity - (2,4)-Tree}
  \textbf{Case 1:}
  {\color{Mittel-Blau}$i$}-\textit{th} operation is an
  \texttt{\color{Mittel-Blau}insert} operation on a full node
  \begin{itemize}
    \item
      Each splitted node creates a node of {\color{Mittel-Blau}degree 3}
    \item
      The parent node receives an element from the splitted node
    \item
      If the parent node is also full we have to split it too
  \end{itemize}
  \begin{figure}
    \begin{columns}
      \begin{column}{0.35\linewidth}
        \begin{adjustbox}{width=\linewidth}
          \begin{tikzpicture}[
  node/.style={
    rectangle,
    color=black,
    draw=Mittel-Blau,
    fill=Hell-Blau,
    line width=0.15em,
    minimum size=2.0em,
    inner sep=0em
  }, path_start/.style={
%    -{Computer Modern Rightarrow[length=0.5em]},
    -{Stealth[length=1.25em]},
    line width=0.25em,
    color=Mittel-Gruen
  }, path/.style={
    path_start,
%    {Circle[scale=0.5,round]}-{Computer Modern Rightarrow[length=0.5em]},
    {Circle[scale=0.5,round]}-{Stealth[length=1.25em]},
    shorten <=-0.35em
  }, level/.style={
    sibling distance=0.0em,
    level distance=5.0em
  }, background/.style={
    fill=black!20!white,
    color=black!20!white
  }, full/.style={
    fill=orange!40!yellow,
    draw=red!20!orange
  }]
\node [node, xshift=-1em, background] { }
child [path, -, xshift=-5em, yshift=2em, background, edge from parent path={
  (\tikzparentnode.south west) -- (\tikzchildnode.north)}
] { };
\node [node, xshift=1em, background] { }
child [path, xshift=-1em, edge from parent path={
  (\tikzparentnode.south west) -- (node1.north east)}
] {
  node [node, full, xshift=-3em] {2}
  node [node, full, xshift=-1em] (node1) {10}
  node [node, full, xshift=1em] {15}
  node [node, full, xshift=3em] {24}
}
child [path, -, xshift=5em, yshift=2em, background, edge from parent path={
  (\tikzparentnode.south east) -- (\tikzchildnode.north)}
] { };
\end{tikzpicture}
        \end{adjustbox}
      \end{column}
      \begin{column}{0.1\linewidth}
        \begin{center}
          $\Rightarrow$
        \end{center}
      \end{column}
      \begin{column}{0.35\linewidth}
        \begin{adjustbox}{width=\linewidth}
          \begin{tikzpicture}[
  node/.style={
    rectangle,
    color=black,
    draw=Mittel-Blau,
    fill=Hell-Blau,
    line width=0.15em,
    minimum size=2.0em,
    inner sep=0em
  }, path_start/.style={
%    -{Computer Modern Rightarrow[length=0.5em]},
    -{Stealth[length=1.25em]},
    line width=0.25em,
    color=Mittel-Gruen
  }, path/.style={
    path_start,
%    {Circle[scale=0.5,round]}-{Computer Modern Rightarrow[length=0.5em]},
    {Circle[scale=0.5,round]}-{Stealth[length=1.25em]},
    shorten <=-0.35em
  }, level/.style={
    sibling distance=0.0em,
    level distance=5.0em
  }, background/.style={
    fill=black!20!white,
    color=black!20!white
  }, full/.style={
    fill=orange!40!yellow,
    draw=orange
  }, border/.style={
    rectangle,
    draw=orange!40!yellow,
    minimum height=2.0em,
    minimum width=4.0em,
    line width=1.0em
  }]
\node [node, xshift=-2em, background] { }
child [path, -, xshift=-5em, yshift=2em, background, edge from parent path={
  (\tikzparentnode.south west) -- (\tikzchildnode.north)}
] { };
\node [node, xshift=2em, background] { }
child [path, -, xshift=5em, yshift=2em, background, edge from parent path={
  (\tikzparentnode.south east) -- (\tikzchildnode.north)}
] { };
\node [node] {15}
child [path, xshift=-2em, edge from parent path={
  (\tikzparentnode.south west) -- (node1.north east)}
] {
  node [border, xshift=0.0em] {}
  node [node, xshift=-1em] (node1) {2}
  node [node, xshift=1em] {10}
}
child [path, xshift=1em, edge from parent path={
  (\tikzparentnode.south east) -- (\tikzchildnode.north)}
] {
  node [node, xshift=1em] {24}
};
\end{tikzpicture}
        \end{adjustbox}
      \end{column}
    \end{columns}
    \caption{Splitting a node on \texttt{\color{Mittel-Blau}insert}}
    \label{fig:a_b_tree:node_split_potential}
  \end{figure}
\end{frame}

%-------------------------------------------------------------------------------

\begin{frame}{(a,b)-Trees}{Runtime Complexity - (2,4)-Tree}
  \textbf{Case 1:}
  {\color{Mittel-Blau}$i$}-\textit{th} operation is an
  \texttt{\color{Mittel-Blau}insert} operation on a full node
  \begin{itemize}
    \item
      Let {\color{Mittel-Blau}$m$} be the number of nodes split
    \item
      The potential rises by {\color{Mittel-Blau}$m$}
    \item
      If the \enquote{stop-node} is of {\color{Mittel-Blau}degree 3} then the
      potential goes down by one
  \end{itemize}
  \begin{align*}
    {\color{Mittel-Blau}\Phi_i}
      &\geq {\color{Mittel-Blau}\Phi_{i-1}} + {\color{Mittel-Blau}m} - 1\\
    \Rightarrow {\color{Mittel-Blau}m}
      &\leq {\color{Mittel-Blau}\Phi_i} - {\color{Mittel-Blau}\Phi_{i-1}} + 1
  \end{align*}
  Costs:
  \begin{math}
    {\color{Mittel-Blau}c_i} \leq A \cdot {\color{Mittel-Blau}m} + B
  \end{math}
  \begin{align*}
    \Rightarrow {\color{Mittel-Blau}c_i}
      &\leq A \cdot \left(
        {\color{Mittel-Blau}\Phi_i} - {\color{Mittel-Blau}\Phi_{i-1}} + 1
      \right) + B\\
    {\color{Mittel-Blau}c_i}
      &\leq A \cdot \left(
      {\color{Mittel-Blau}\Phi_i} - {\color{Mittel-Blau}\Phi_{i-1}}
      \right) + \underbrace{A + B}_{B'}\\
  \end{align*}
\end{frame}

%-------------------------------------------------------------------------------

\begin{frame}{(a,b)-Trees}{Runtime Complexity - (2,4)-Tree}
  \textbf{Case 2:}
  {\color{Mittel-Blau}$i$}-\textit{th} operation is an
  \texttt{\color{Mittel-Blau}remove} operation
  \begin{itemize}
    \item
      \textbf{Case 2.1:} Inner node
      \begin{itemize}
        \item
         Searching the successor in a tree is
          \begin{math}
           {\color{Mittel-Blau}\mathcal{O}(d)} =
           {\color{Mittel-Blau}\mathcal{O}(\log n)}
          \end{math}
        \item
          Normally the tree is coupled with a doubly linked list\\
          $\Rightarrow$ We can find the succcessor in
          {\color{Mittel-Blau}$\mathcal{O}(1)$}
    \end{itemize}
  \end{itemize}
  \vspace{0.5em}
  \begin{figure}
    \begin{adjustbox}{width=0.75\linewidth}
      \begin{adjustbox}{width=\linewidth}%
\begin{tikzpicture}[
  element/.style={
    draw={Mittel-Blau},
    fill={white},
    line width=0.075em
  }, element_head/.style={
    draw={Mittel-Blau},
    fill={white},
    pattern=north east lines,
    pattern color={Mittel-Blau},
    line width=0.075em
  }, element_link/.style={
    draw={Mittel-Blau},
    fill={Hell-Blau},
    line width=0.075em
  }, arrow/.style={
    draw={Mittel-Blau},
    line width=0.125em
  }
]%
% Draw elements
\foreach \x in {4,3,...,0}{
  \ifnum \x=3 \else
    \draw[element_link] (1.5*\x, 0.5) rectangle (1.5*\x + 1.0, 0.75);
    \draw[element_link] (1.5*\x, 1.0) rectangle (1.5*\x + 1.0, 0.75);
    \draw[element] (1.5*\x, 0.0) rectangle (1.5*\x + 1.0, 0.5);
    
    \ifnum \x<4
      \draw (1.5*\x + 0.5, 0.25) node {\footnotesize \x};
    \else
      \draw (1.5*\x + 0.5, 0.25) node {\footnotesize $n$};
    \fi
    
    \draw[->, arrow] (1.5*\x + 0.5, 1.5) -- (1.5*\x + 0.5, 1.0);
  \fi
  
  \ifnum \x<4
    \draw[->, arrow] (1.5*\x + 1.0, 0.875) -- (1.5*\x + 1.5, 0.875);
    \draw[<-, arrow] (1.5*\x + 1.0, 0.625) -- (1.5*\x + 1.5, 0.625);
  \fi
}

% Pointer
\draw[->, arrow] (7.0, 0.875) -- (7.5, 0.875) -- (7.5, -0.5) --
  (-0.5, -0.5) -- (-0.5, 0.875) -- (0.0, 0.875);
\draw[<-, arrow] (7.0, 0.625) -- (7.25, 0.625) -- (7.25, -0.25) --
  (-0.25, -0.25) -- (-0.25, 0.625) -- (0.0, 0.625);

\draw (5.0, 0.5) node {\footnotesize ...};

% Draw navigation structure
\draw[element] (0.0, 1.5) rectangle (7.0, 2.25);
\draw (3.5, 1.825) node {\footnotesize tree navigation structre};
\end{tikzpicture}%
\end{adjustbox}%4
    \end{adjustbox}
    \caption{Tree with doubly linked list}
    \label{fig:a_b_tree:doubly_linked_list}
  \end{figure}
\end{frame}

%-------------------------------------------------------------------------------

\begin{frame}{(a,b)-Trees}{Runtime Complexity - (2,4)-Tree}
  \textbf{Case 2:}
  {\color{Mittel-Blau}$i$}-\textit{th} operation is an
  \texttt{\color{Mittel-Blau}remove} operation
  \begin{itemize}
    \item
      \textbf{Case 2.1:} Borrowing a node
      \begin{itemize}
        \item
          Creates no additional operations
        \item
          Case 2.1.1: Potential rises by one
      \end{itemize}
  \end{itemize}
  \begin{figure}
    \begin{columns}
      \begin{column}{0.35\linewidth}
        \begin{adjustbox}{width=\linewidth}
          \begin{tikzpicture}[
  node/.style={
    rectangle,
    color=black,
    draw=Mittel-Blau,
    fill=Hell-Blau,
    line width=0.15em,
    minimum size=2.0em,
    inner sep=0em
  }, path_start/.style={
%    -{Computer Modern Rightarrow[length=0.5em]},
    -{Stealth[length=1.25em]},
    line width=0.25em,
    color=Mittel-Gruen
  }, path/.style={
    path_start,
%    {Circle[scale=0.5,round]}-{Computer Modern Rightarrow[length=0.5em]},
    {Circle[scale=0.5,round]}-{Stealth[length=1.25em]},
    shorten <=-0.35em
  }, level/.style={
    sibling distance=0.0em,
    level distance=5.0em
  }, background/.style={
    fill=black!20!white,
    color=black!20!white
  }, full/.style={
    fill=orange!40!yellow,
    draw=orange
  }]
\node [node, xshift=-2em, background] { }
child [path, -, xshift=-5em, yshift=2em, background, edge from parent path={
  (\tikzparentnode.south west) -- (\tikzchildnode.north)}
] { };
\node [node, xshift=2em, background] { }
child [path, -, xshift=5em, yshift=2em, background, edge from parent path={
  (\tikzparentnode.south east) -- (\tikzchildnode.north)}
] { };
\node [node] {15}
child [path, xshift=-2em, edge from parent path={
  (\tikzparentnode.south west) -- (node1.north)}
] {
  node [node, xshift=-2em] {2}
  node [node, xshift=0em] (node1) {7}
  node [node, xshift=2em] {10}
}
child [path, xshift=2em, edge from parent path={
  (\tikzparentnode.south east) -- (\tikzchildnode.north)}
] {
  node [node, xshift=1em, minimum width=0.5em] {}
};
\end{tikzpicture}

        \end{adjustbox}
      \end{column}
      \begin{column}{0.1\linewidth}
        \begin{center}
          $\Rightarrow$
        \end{center}
      \end{column}
      \begin{column}{0.35\linewidth}
        \begin{adjustbox}{width=\linewidth}
          \begin{tikzpicture}[
  node/.style={
    rectangle,
    color=black,
    draw=Mittel-Blau,
    fill=Hell-Blau,
    line width=0.15em,
    minimum size=2.0em,
    inner sep=0em
  }, path_start/.style={
%    -{Computer Modern Rightarrow[length=0.5em]},
    -{Stealth[length=1.25em]},
    line width=0.25em,
    color=Mittel-Gruen
  }, path/.style={
    path_start,
%    {Circle[scale=0.5,round]}-{Computer Modern Rightarrow[length=0.5em]},
    {Circle[scale=0.5,round]}-{Stealth[length=1.25em]},
    shorten <=-0.35em
  }, level/.style={
    sibling distance=0.0em,
    level distance=5.0em
  }, background/.style={
    fill=black!20!white,
    color=black!20!white
  }, full/.style={
    fill=orange!40!yellow,
    draw=orange
  }, border/.style={
    rectangle,
    draw=orange!40!yellow,
    minimum height=2.0em,
    minimum width=4.0em,
    line width=1.0em
  }]
\node [node, xshift=-2em, background] { }
child [path, -, xshift=-5em, yshift=2em, background, edge from parent path={
  (\tikzparentnode.south west) -- (\tikzchildnode.north)}
] { };
\node [node, xshift=2em, background] { }
child [path, -, xshift=5em, yshift=2em, background, edge from parent path={
  (\tikzparentnode.south east) -- (\tikzchildnode.north)}
] { };
\node [node] {10}
child [path, xshift=-2em, edge from parent path={
  (\tikzparentnode.south west) -- (node1.north east)}
] {
  node [border, xshift=0.0em] {}
  node [node, xshift=-1em] (node1) {2}
  node [node, xshift=1em] {7}
}
child [path, xshift=2em, edge from parent path={
  (\tikzparentnode.south east) -- (\tikzchildnode.north)}
] {
  node [node] {15}
};
\end{tikzpicture}
        \end{adjustbox}
      \end{column}
    \end{columns}
    \caption{Borrowing an element case 2.1.1}
    \label{fig:a_b_tree:move_potential_1}
  \end{figure}
\end{frame}

%-------------------------------------------------------------------------------

\begin{frame}{(a,b)-Trees}{Runtime Complexity - (2,4)-Tree}
  \textbf{Case 2:}
  {\color{Mittel-Blau}$i$}-\textit{th} operation is an
  \texttt{\color{Mittel-Blau}remove} operation
  \begin{itemize}
    \item
    \textbf{Case 2.1:} Borrowing a node
    \begin{itemize}
      \item
      Creates no additional operations
      \item
      Case 2.1.2: Potential lowers by one
    \end{itemize}
  \end{itemize}
  \begin{figure}
    \begin{columns}
      \begin{column}{0.35\linewidth}
        \begin{adjustbox}{width=\linewidth}
          \begin{tikzpicture}[
  node/.style={
    rectangle,
    color=black,
    draw=Mittel-Blau,
    fill=Hell-Blau,
    line width=0.15em,
    minimum size=2.0em,
    inner sep=0em
  }, path_start/.style={
%    -{Computer Modern Rightarrow[length=0.5em]},
    -{Stealth[length=1.25em]},
    line width=0.25em,
    color=Mittel-Gruen
  }, path/.style={
    path_start,
%    {Circle[scale=0.5,round]}-{Computer Modern Rightarrow[length=0.5em]},
    {Circle[scale=0.5,round]}-{Stealth[length=1.25em]},
    shorten <=-0.35em
  }, level/.style={
    sibling distance=0.0em,
    level distance=5.0em
  }, background/.style={
    fill=black!20!white,
    color=black!20!white
  }, full/.style={
    fill=orange!40!yellow,
    draw=orange
  }, border/.style={
    rectangle,
    draw=orange!40!yellow,
    minimum height=2.0em,
    minimum width=4.0em,
    line width=1.0em
  }]
\node [node, xshift=-2em, background] { }
child [path, -, xshift=-5em, yshift=2em, background, edge from parent path={
  (\tikzparentnode.south west) -- (\tikzchildnode.north)}
] { };
\node [node, xshift=2em, background] { }
child [path, -, xshift=5em, yshift=2em, background, edge from parent path={
  (\tikzparentnode.south east) -- (\tikzchildnode.north)}
] { };
\node [node] {17}
child [path, xshift=-2em, edge from parent path={
  (\tikzparentnode.south west) -- (node1.north east)}
] {
  node [border, xshift=0.0em] {}
  node [node, xshift=-1em] (node1) {5}
  node [node, xshift=1em] {9}
}
child [path, xshift=1em, edge from parent path={
  (\tikzparentnode.south east) -- (\tikzchildnode.north)}
] {
  node [node, xshift=1em, minimum width=0.5em] {}
};
\end{tikzpicture}
        \end{adjustbox}
      \end{column}
      \begin{column}{0.1\linewidth}
        \begin{center}
          $\Rightarrow$
        \end{center}
      \end{column}
      \begin{column}{0.35\linewidth}
        \begin{adjustbox}{width=\linewidth}
          \begin{tikzpicture}[
  node/.style={
    rectangle,
    color=black,
    draw=Mittel-Blau,
    fill=Hell-Blau,
    line width=0.15em,
    minimum size=2.0em,
    inner sep=0em
  }, path_start/.style={
%    -{Computer Modern Rightarrow[length=0.5em]},
    -{Stealth[length=1.25em]},
    line width=0.25em,
    color=Mittel-Gruen
  }, path/.style={
    path_start,
%    {Circle[scale=0.5,round]}-{Computer Modern Rightarrow[length=0.5em]},
    {Circle[scale=0.5,round]}-{Stealth[length=1.25em]},
    shorten <=-0.35em
  }, level/.style={
    sibling distance=0.0em,
    level distance=5.0em
  }, background/.style={
    fill=black!20!white,
    color=black!20!white
  }, full/.style={
    fill=orange!40!yellow,
    draw=orange
  }]
\node [node, xshift=-2em, background] { }
child [path, -, xshift=-5em, yshift=2em, background, edge from parent path={
  (\tikzparentnode.south west) -- (\tikzchildnode.north)}
] { };
\node [node, xshift=2em, background] { }
child [path, -, xshift=5em, yshift=2em, background, edge from parent path={
  (\tikzparentnode.south east) -- (\tikzchildnode.north)}
] { };
\node [node] {9}
child [path, xshift=-1.5em, edge from parent path={
  (\tikzparentnode.south west) -- (\tikzchildnode.north)}
] {
  node [node] {5}
}
child [path, xshift=1.5em, edge from parent path={
  (\tikzparentnode.south east) -- (\tikzchildnode.north)}
] {
  node [node] {17}
};
\end{tikzpicture}
        \end{adjustbox}
      \end{column}
    \end{columns}
    \caption{Borrowing an element case 2.1.2}
    \label{fig:a_b_tree:move_potential_2}
  \end{figure}
\end{frame}

%-------------------------------------------------------------------------------

\begin{frame}{(a,b)-Trees}{Runtime Complexity - (2,4)-Tree}
  \textbf{Case 2:}
  {\color{Mittel-Blau}$i$}-\textit{th} operation is an
  \texttt{\color{Mittel-Blau}remove} operation
  \begin{itemize}
    \item
      \textbf{Case 2.2:} Merging a node
    \begin{itemize}
      \item
        Potential rises by one
      \item
        Parent node has one element less after the operation
      \item
        This operation propagates up to the root until a node of degree
        {\color{Mittel-Blau}$>a$} is found
      \item
        The chain also stops if a node of {\color{Mittel-Blau}degree 2}
        is met which can borrow an element from one of its neighbours
    \end{itemize}
  \end{itemize}
  \begin{figure}
    \begin{columns}
      \begin{column}{0.35\linewidth}
        \begin{adjustbox}{width=\linewidth}
          \begin{tikzpicture}[
  node/.style={
    rectangle,
    color=black,
    draw=Mittel-Blau,
    fill=Hell-Blau,
    line width=0.15em,
    minimum size=2.0em,
    inner sep=0em
  }, path_start/.style={
%    -{Computer Modern Rightarrow[length=0.5em]},
    -{Stealth[length=1.25em]},
    line width=0.25em,
    color=Mittel-Gruen
  }, path/.style={
    path_start,
%    {Circle[scale=0.5,round]}-{Computer Modern Rightarrow[length=0.5em]},
    {Circle[scale=0.5,round]}-{Stealth[length=1.25em]},
    shorten <=-0.35em
  }, level/.style={
    sibling distance=0.0em,
    level distance=5.0em
  }, background/.style={
    fill=black!20!white,
    color=black!20!white
  }, full/.style={
    fill=orange!40!yellow,
    draw=orange
  }]
\node [node, xshift=-2em, background] { }
child [path, -, xshift=-5em, yshift=2em, background, edge from parent path={
  (\tikzparentnode.south west) -- (\tikzchildnode.north)}
] { };
\node [node, xshift=2em, background] { }
child [path, -, xshift=5em, yshift=2em, background, edge from parent path={
  (\tikzparentnode.south east) -- (\tikzchildnode.north)}
] { };
\node [node] {23}
child [path, xshift=-2em, edge from parent path={
  (\tikzparentnode.south west) -- (\tikzchildnode.north)}
] {
  node [node] {17}
}
child [path, xshift=2em, edge from parent path={
  (\tikzparentnode.south east) -- (\tikzchildnode.north)}
] {
  node [node, xshift=1em, minimum width=0.5em] {}
};
\end{tikzpicture}
        \end{adjustbox}
      \end{column}
      \begin{column}{0.1\linewidth}
        \begin{center}
          $\Rightarrow$
        \end{center}
      \end{column}
      \begin{column}{0.35\linewidth}
        \begin{adjustbox}{width=0.85\linewidth}
          \begin{tikzpicture}[
  node/.style={
    rectangle,
    color=black,
    draw=Mittel-Blau,
    fill=Hell-Blau,
    line width=0.15em,
    minimum size=2.0em,
    inner sep=0em
  }, path_start/.style={
%    -{Computer Modern Rightarrow[length=0.5em]},
    -{Stealth[length=1.25em]},
    line width=0.25em,
    color=Mittel-Gruen
  }, path/.style={
    path_start,
%    {Circle[scale=0.5,round]}-{Computer Modern Rightarrow[length=0.5em]},
    {Circle[scale=0.5,round]}-{Stealth[length=1.25em]},
    shorten <=-0.35em
  }, level/.style={
    sibling distance=0.0em,
    level distance=5.0em
  }, background/.style={
    fill=black!20!white,
    color=black!20!white
  }, full/.style={
    fill=orange!40!yellow,
    draw=red!20!orange
  }, border/.style={
    rectangle,
    draw=orange!40!yellow,
    minimum height=2.0em,
    minimum width=4.0em,
    line width=1.0em
  }]
\node [node, xshift=-1em, background] { }
child [path, -, xshift=-5em, yshift=2em, background, edge from parent path={
  (\tikzparentnode.south west) -- (\tikzchildnode.north)}
] { };
\node [node, xshift=1em, background] { }
child [path, xshift=-1em, edge from parent path={
  (\tikzparentnode.south west) -- (node1.north east)}
] {
  node [border, xshift=0.0em] {}
  node [node, xshift=-1em] (node1) {17}
  node [node, xshift=1em] {23}
}
child [path, -, xshift=5em, yshift=2em, background, edge from parent path={
  (\tikzparentnode.south east) -- (\tikzchildnode.north)}
] { };
\end{tikzpicture}
        \end{adjustbox}
      \end{column}
    \end{columns}
    \caption{Merging two nodes}
    \label{fig:a_b_tree:merge_potential}
  \end{figure}
\end{frame}

%-------------------------------------------------------------------------------

\begin{frame}{(a,b)-Trees}{Runtime Complexity - (2,4)-Tree}
  \textbf{Case 2:}
  {\color{Mittel-Blau}$i$}-\textit{th} operation is an
  \texttt{\color{Mittel-Blau}remove} operation
  \begin{itemize}
    \item
    \textbf{Case 2.2:} Merging a node
    \begin{itemize}
      \item
        Let {\color{Mittel-Blau}$m$} be the number of nodes merged
      \item
        The potential rises by {\color{Mittel-Blau}$m$}
      \item
        If the \enquote{stop-node} is of {\color{Mittel-Blau}degree 2} then the
        potential eventually goes down by one
      \item
        Same costs as \texttt{\color{Mittel-Blau}insert}
    \end{itemize}
  \end{itemize}
\end{frame}

%-------------------------------------------------------------------------------

\begin{frame}{(a,b)-Trees}{Runtime Complexity - (2,4)-Tree - Lemma}
  \begin{block}{Lemma:}
    \begin{itemize}
      \item
        We know:
        \begin{displaymath}
        {\color{Mittel-Blau}c_i} \leq A \cdot \left(
        {\color{Mittel-Blau}\Phi_i} - {\color{Mittel-Blau}\Phi_{i-1}}
        \right) + B, \hspace{1.5em}A,B \in \mathbb{R}
        \end{displaymath}
      \item
        With that we can conclude:
        \begin{displaymath}
          \sum_{i=0}^{n} {\color{Mittel-Blau}c_i} \in \mathcal{O}(n)
        \end{displaymath}
    \end{itemize}
  \end{block}
\end{frame}

%-------------------------------------------------------------------------------

\begin{frame}{(a,b)-Trees}{Runtime Complexity - (2,4)-Tree - Lemma - Proof}
  \textbf{Proof:}
  \begin{align*}
    \sum_{i=0}^{n} {\color{Mittel-Blau}c_i}
      &\leq \underbrace{A \cdot \left(
        {\color{Mittel-Blau}\Phi_1} - {\color{Mittel-Blau}\Phi_0}
      \right) + B}_{\leq {\color{Mittel-Blau}c_1}} +
      \underbrace{A \cdot \left(
        {\color{Mittel-Blau}\Phi_2} - {\color{Mittel-Blau}\Phi_1}
      \right) + B}_{\leq {\color{Mittel-Blau}c_1}} + \dots +
      \underbrace{A \cdot \left(
        {\color{Mittel-Blau}\Phi_n} - {\color{Mittel-Blau}\Phi_{n-1}}
      \right) + B}_{\leq {\color{Mittel-Blau}c_n}}\\
    &\hspace{1.0em}= A \cdot \left(
      {\color{Mittel-Blau}\Phi_n} - {\color{Mittel-Blau}\Phi_0}
    \right) + B \cdot n \hspace{1.5em} \mid \text{telescoping series}\\
    &\hspace{1.0em}= A \cdot {\color{Mittel-Blau}\Phi_n} + B \cdot n
      \hspace{4.5em} \mid \text{we start with an empty tree}\\
    &\hspace{1.0em}< A \cdot n + B \cdot n \in \mathcal{O}(n)
      \hspace{1.85em} \mid
      \text{number of {\color{Mittel-Blau}degree 3} nodes}\\
    &\hspace{12.5em} \text{< number of nodes}
  \end{align*}
\end{frame}