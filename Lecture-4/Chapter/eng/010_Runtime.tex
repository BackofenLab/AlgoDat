%%% ===================================================================
%%% RCS Header:
%%% $RCSfile: local.el,v $
%%% $Revision: 1.27 $
%%% $Date: 2014/11/07 15:36:00 $
%%% $Author: backofen $
%%% $Locker:  $
%%% ===================================================================

\section{Runtime Complexity}

\begin{frame}{Runtime Complexity}
  \begin{itemize}
    \item
      The runtime does not entirely depend on the size of the problem, but also
      on the type of input
    \item
      This results in:
    \begin{itemize}
      \item
        {\color{Mittel-Blau}Best runtime:}\\
        Lowest possible runtime complexity for an input of size $n$
      \item
        {\color{Mittel-Blau}Worst runtime:}\\
        Highest possible runtime complexity for an input of size $n$
      \item
        {\color{Mittel-Blau}Average / Expected runtime:}\\
        The average of all runtime complexities for an input of size $n$
    \end{itemize}
  \end{itemize}
\end{frame}

%-------------------------------------------------------------------------------

\begin{frame}{Runtime Complexity}{Example 1 - Conditions}
  \begin{itemize}
    \item
      Input: Field $a$ with $n$ elements
      $a[i] \in \mathbb{N}, \; 0 \leq a[i] \leq n, \; 0 \leq i < n$
    \item
      Output: Field $a$ with $n$ elements $a[0] \neq 1$
  \end{itemize}
  \begin{tabularx}{\textwidth}{P{11em}O{4em}O{1.75em}O{4.5em}O{1.75em}L}
    if a[0] == 0: & $\mathcal{O}(1)$ & {} & {} & {} & {}\\
    \hhline{~*{1}{-}}
    $\hspace*{1.5em}$ a[0] = 2 & $\mathcal{O}(1)$ &
    \multirow{-2}{*}{$\left.%
      \begin{array}{@{}c@{}}\\[1em]\end{array}%
      \right\rbrace$} &%
    \multirow{-2}{*}{$\mathcal{O}(1)$} & {} & {}\\
    else: & {} & {} & {} & {} & {}\\
    $\hspace*{1.5em}$ for i in range(0, n): & $\mathcal{O}(n)$ &
    {} & {} & {} & {}\\
    \hhline{~*{1}{-}}
    $\hspace*{3em}$ a[i] = 2 & $\mathcal{O}(1)$ &
    \multirow{-2}{*}{$\left.
      \begin{array}{@{}c@{}}\\[1em]\end{array}
      \right\rbrace$} &
    \multirow{-2}{*}{
      $\begin{array}{@{}c@{}}
      \mathcal{O}(n) \cdot \mathcal{O}(1)\\
      = \mathcal{O}(n)
      \end{array}$
    } &%
    \multirow{-5}{*}{$\left.
      \begin{array}{@{}c@{}}\\[4.5em]\end{array}
      \right\rbrace$} &%
    \hspace*{-0.5em}%
    \multirow{-5}{*}{%
      $\mathcal{O}(?)$
    }%
  \end{tabularx}
  \begin{itemize}
    \item
      {\color{Mittel-Blau}Best runtime:}
      $\mathcal{O}(1) + \mathcal{O}(1) = \mathcal{O}(1)$
    \item
      {\color{Mittel-Blau}Worst runtime:}
      $\mathcal{O}(1) + \mathcal{O}(n) = \mathcal{O}(n)$
  \end{itemize}
\end{frame}

%-------------------------------------------------------------------------------

\begin{frame}{Runtime Complexity}{Example 1 - Average Runtime}
  \begin{itemize}
    \item
      The {\color{Mittel-Blau}average runtime} is determined by the average
      runtime for all input instances of size $n$
    \item
      Every element of $a$ can have $n$ different values\\
      \begin{math}
        \hspace{1.5em}\Rightarrow n \cdot \ldots \cdot n = n^n
      \end{math}
      different input instances of size $n$
      \begin{itemize}
        \item
          \lstinline[
            language=Python,
            style={python-idle-code},
            basicstyle=\small
            ]|a[i] == 1|
          in $n^{n-1}$ instances
        \item
          \lstinline[
            language=Python,
            style={python-idle-code},
            basicstyle=\small
          ]|a[i] != 1|
          in $n^n - n^{n-1} = n^{n-1} \cdot (n-1)$ instances
      \end{itemize}
    \item
      Sum of all runtime complexities:
      \begin{displaymath}
        \underbrace{n^{n-1} \cdot \mathcal{O}(1)\vphantom{()}}_{
          \lstinline[
            language=Python,
            style={python-idle-code},
            basicstyle=\small
          ]|a[i] == 1|
        } + \underbrace{n^{n-1} \cdot (n-1) \cdot \mathcal{O}(n)}_{
          \lstinline[
            language=Python,
            style={python-idle-code},
            basicstyle=\small
          ]|a[i] != 1|
        }
      \end{displaymath}
    \item
      {\color{Mittel-Blau}Average runtime}:
      \begin{displaymath}
        \frac{n^{n-1} + n^{n-1} \cdot (n-1) \cdot n}{n^n}
        = \frac{1}{n} + n - 1
        \in \mathcal{O}(n)
      \end{displaymath}
  \end{itemize}
\end{frame}

%-------------------------------------------------------------------------------

\begin{frame}{Runtime Complexity}{Example 2 - Binary Addition}
  \begin{itemize}
    \item
      Input: $n$ digit dual number $a$
    \item
      Output: $n$ digit dual number $a + 1$
    \item
      Runtime of the algorithm is determined by the number of bits getting
      changed (steps)
      \begin{enumerate}
        \item "0" $\to$ "1"
        \item "1" $\to$ "0"
      \end{enumerate}
    \item
      {\color{Mittel-Blau}Best runtime:}
      $1\,\text{step} = \mathcal{O}(1)$
    \item
      {\color{Mittel-Blau}Worst runtime:}
      $n\,\text{steps} = \mathcal{O}(n)$
  \end{itemize}
  \vspace{-1.0em}%
  \begin{table}[!h]%
    \caption{Binary addition}%
    \label{tab:runtime:binary_addition}%
    \begin{tabular}{crrc}%
      Digits ($n$) & Input & Output & Steps\\
      \midrule
      10 & 1011100100 & 1011100101 & 1\\
      4 & 1011 & 1100 & 3\\
      8 & 11111111 & 00000000 & 8
    \end{tabular}
  \end{table}
\end{frame}

%-------------------------------------------------------------------------------

\begin{frame}{Runtime Complexity}{Example 2 - Average Steps}
  \vspace{-3em}
  \begin{columns}
    \begin{column}[t]{0.5\linewidth}
      \begin{table}[!t]%
        \caption{Binary addition with $n=1$}%
        \label{tab:runtime:binary_addition_one}%
        \begin{tabular}{rrc}%
          Input & Output & Steps\\
          \midrule
          0 & 1 & 1\\
          1 & 0 & 1
        \end{tabular}
      \end{table}
    \end{column}
    \begin{column}[t]{0.5\linewidth}
      \begin{table}[!t]%
        \caption{Binary addition with $n=2$}%
        \label{tab:runtime:binary_addition_two}%
        \begin{tabular}{rrc}%
          Input & Output & Steps\\
          \midrule
          00 & 01 & 1\\
          01 & 10 & 2\\
          10 & 11 & 1\\
          11 & 00 & 2
        \end{tabular}
      \end{table}
    \end{column}
  \end{columns}
  \begin{columns}
    \begin{column}{0.5\linewidth}
      \begin{align*}
        \overline{\text{steps}} &= \frac{1+1}{2} = 1\\[0.5em]
        {} &= 2 - \frac{1}{1} = {\color{Mittel-Blau}2 - \frac{1}{2^{n-1}}}
      \end{align*}
    \end{column}
    \begin{column}{0.5\linewidth}
      \begin{align*}
        \overline{\text{steps}} &= \frac{1+2+1+2}{4} = \frac{3}{2}\\[0.5em]
        {} &= 2 - \frac{1}{2} = {\color{Mittel-Blau}2 - \frac{1}{2^{n-1}}}
      \end{align*}
    \end{column}
  \end{columns}
\end{frame}

%-------------------------------------------------------------------------------

\begin{frame}{Runtime Complexity}{Example 2 - Average Steps}
  \vspace{-3em}
  \begin{columns}
    \begin{column}{0.5\linewidth}
      \begin{table}[!h]%
        \caption{Binary addition with $n=3$}%
        \label{tab:runtime:binary_addition_three}%
        \begin{tabular}{rrc}%
          Input & Output & Steps\\
          \midrule
          000 & 001 & 1\\
          001 & 010 & 2\\
          010 & 011 & 1\\
          011 & 100 & 3\\
          \midrule
          100 & 101 & 1\\
          101 & 110 & 2\\
          110 & 111 & 1\\
          111 & 000 & 3
        \end{tabular}
      \end{table}
    \end{column}
    \begin{column}{0.5\linewidth}
      \begin{align*}
        \overline{\text{steps}}
          &= \frac{1+2+1+3+1+2+1+3}{8} = \frac{7}{4}\\[0.5em]
        {} &= 2 - \frac{1}{4} = {\color{Mittel-Blau}2 - \frac{1}{2^{n-1}}}
      \end{align*}\\
      $\Rightarrow$ {\color{Mittel-Blau}Average runtime}:
      \begin{math}
        \displaystyle\hspace*{1.5em}
        2 - \frac{1}{2^{n-1}} \in \mathcal{O}(1)
        \end{math}
    \end{column}
  \end{columns}
\end{frame}

%-------------------------------------------------------------------------------

\begin{frame}{Runtime Complexity}{Example 2 - Average Steps}
  \vspace{-2.0em}
  \begin{table}[!h]%
    \caption{Case analysis for instances of size $n$}%
    \label{tab:runtime:binary_addition_case_analysis}%
    \vspace{-0.5em}%
    \begin{tabular}{cccc}
      Input & Output & Instances & Steps\\
      \midrule
      $\_\_\_ \ldots \_\_\_0$ & $\_\_\_ \ldots \_\_\_1$ &$2^{n-1}$ & 1\\
      $\_\_\_ \ldots \_\_01$ & $\_\_\_ \ldots \_\_10$ & $2^{n-2}$ & 2\\
      $\_\_\_ \ldots \_011$ & $\_\_\_ \ldots \_100$ & $2^{n-3}$ & 3\\
      $\vdots$ & $\vdots$ & $\vdots$ & $\vdots$\\
      $\_01 \ldots 1111$ & $\_10 \ldots 0000$ & $2^{1}$ & n-1\\
      $011 \ldots 1111$ & $100 \ldots 0000$ & $2^0$ & n\\
      $111 \ldots 1111$ & $000 \ldots 0000$ & 1 & n\\
    \end{tabular}
  \end{table}
  \vspace{-0.5em}
  {\color{Mittel-Blau}Average steps}:
  \vspace{-1.0em}
  \begin{displaymath}
    \dfrac{
      1 \cdot 2^{n-1} + 2 \cdot 2^{n-2}+\dots+
      (n-1) \cdot 2^1 + n \cdot 2^0 + n \cdot 1
    }{2^{n-1}+2^{n-2}+\dots+2^{1}+2^{0}+1}
    = \frac{
      \sum\limits_{i=1}^n \left(i \cdot 2^{n-i}\right) + n
    }{
      \sum\limits_{i=0}^{n-1} 2^i + 1
    }
  \end{displaymath}
\end{frame}

%-------------------------------------------------------------------------------

\begin{frame}{Runtime Complexity}{Example 2 - Average Steps}
  \begin{itemize}
    \item
      Denominator:
      \begin{displaymath}
        \sum_{i=0}^{n-1} 2^i + 1
        \stackrel{\begin{array}{c}
          \text{geometric}\\
          \text{series}
        \end{array}}{=}
        2^n - 1 + 1 = 2^n
      \end{displaymath}
    \item
      Counter:
  \end{itemize}
  \begin{alignat*}{4}
    & \sum_{i=1}^n \left(i \cdot 2^{n-i}\right) + n
      \stackrel{a = 2\,a - a}{=}
        2 \, \sum_{i=1}^n \left(i \cdot 2^{n-i}\right)
        - \sum_{i=1}^n \left(i \cdot 2^{n-i}\right) + n\\
    & \hspace*{1.5em} = 1 \cdot 2^n + 2 \cdot 2^{n-1} + 3 \cdot n^{n-2} + \dots
        + (n-1) \cdot 2^2 + n \cdot 2^1\\
    & \hspace{4.5em}- 1 \cdot 2^{n-1} - 2 \cdot 2^{n-2} - \dots
        - (n-2) \cdot 2^2 - (n-1) \cdot 2^1 - n \cdot 2^0 + n\\
    & \hspace*{1.5em} = \underbrace{2^n + 2^{n-1} + \dots
        +2^1 + 2^0}_{2^{n+1}-1} - 1
      = 2^{n+1} - 2
  \end{alignat*}
\end{frame}

%-------------------------------------------------------------------------------

\begin{frame}{Runtime Complexity}{Example 2 - Average Steps}
  {\color{Mittel-Blau}Average steps}:
  \begin{displaymath}
    \overline{steps}
    = \frac{
      \sum\limits_{i=1}^n \left(i \cdot 2^{n-i}\right) + n
    }{\sum\limits_{i=0}^{n-1} 2^i + 1}
    = \frac{2^{n+1} - 2}{2^n}
    = 2 - \frac{1}{2^{n-1}}
  \end{displaymath}
  \begin{displaymath}
    \lim_{n \to \infty} \left(2 - \frac{1}{2^{n-1}}\right) = 2
    \in \mathcal{O}(1)
  \end{displaymath}
\end{frame}
<<<<<<< HEAD
%%% ===================================================================
%%% This should be at the END of the file !!!!!!
%%%
%%% Local Variables: 
%%% mode: latex 
%%% TeX-master: "~/TeX/TeXinput/Scripts/Algo-Data-EMS/Rolf-2016/AlgoDat/Lecture-4/Lecture.tex" 
%%% End: 
%%% ===================================================================
=======
>>>>>>> 6e8bc94a34e9ade9277389a55d60932246d4b1da
