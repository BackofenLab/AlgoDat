\subsubsection{Associative Array}

\begin{frame}<beamer>{\LectureToC}
  \tableofcontents[currentsection,currentsubsection]
\end{frame}

\begin{frame}{Associative Arrays}{Practical Example - Associative Array}
  \textbf{Idea using associative arrays:}
  \begin{itemize}
    \item
      Take the continent as {\color{Mittel-Blau}key}
    \item
      Use a counter (occurences) or a list with all countries found belonging
      to this continent as {\color{Mittel-Blau}value}
  \end{itemize}
  \textbf{Advantage:}
  \begin{itemize}
    \item
      Runtime $\mathcal{O}(n)$, implied we can find an element in 
      $\mathcal{O}(1)$ like in an normal array
  \end{itemize}
\end{frame}

\codeslide{python}{
\begin{frame}{Associative Arrays}{Python}
  \textbf{Python:}
  \lstinputlisting[
    language=Python,
    basicstyle=\small,
    tabsize=4,
    style={python-idle-code},
    escapechar={@}
  ]{Code/AssociativeArray.py}
\end{frame}
}

%-------------------------------------------------------------------------------

\codeslide{java}{
\begin{frame}{Associative Arrays}{Java}
  \vspace{-0.5em}
  \textbf{Java:}
  \vspace{-0.25em}
  \lstinputlisting[
    language=Java,
    basicstyle=\small,
    tabsize=4,
    style={java-eclipse-code},
    breaklines=false,
    escapechar={@},
    emph={countries},
    emphstyle=\color{java_variable}
  ]{Code/AssociativeArray.java}
\end{frame}
}

%-------------------------------------------------------------------------------

\codeslide{cpp}{
\begin{frame}{Associative Arrays}{C++}
  \textbf{C++:}
  \lstinputlisting[
    language=C++,
    basicstyle=\small,
    tabsize=4,
    style={cpp-eclipse-code},
    breaklines=false,
    morekeywords={endl},
    escapechar={@}
  ]{Code/AssociativeArray.cpp}
\end{frame}
}

%-------------------------------------------------------------------------------

\begin{frame}{Associative Arrays}{Efficiency}
  \textbf{Efficiency:}
  \begin{itemize}
    \item
      Depends on implementation
    \item
      Two typical implementations:
      \begin{itemize}
        \item
          \textbf{Hashing:}
          Calculates a checksum of the key and uses as key of a normal array\\
          \texttt{search}: $\mathcal{O}(1) \dots \mathcal{O}(n)$\\
          \texttt{insert}: $\mathcal{O}(1) \dots \mathcal{O}(n)$\\
          \texttt{delete}: $\mathcal{O}(1) \dots \mathcal{O}(n)$
        \item
          \textbf{(Binary-)Tree:}
          Creates a sorted (binary) tree\\
          \texttt{search}: $\mathcal{O}(\log n) \dots \mathcal{O}(n)$\\
          \texttt{insert}: $\mathcal{O}(\log n) \dots \mathcal{O}(n)$\\
          \texttt{delete}: $\mathcal{O}(\log n) \dots \mathcal{O}(n)$
      \end{itemize}
  \end{itemize}
\end{frame}

%-------------------------------------------------------------------------------

\begin{frame}{Associative Arrays}{Efficiency}
  \begin{table}[!h]
    \caption{Map implementions of programming languages}
    \begin{tabular}{c|cc}
      {} & Hashing & (Binary-)Tree\\
      \midrule
      Python & all dictionaries & {}\\
      Java & \texttt{java.util.HashMap} & \texttt{java.util.TreeMap}\\
      C++11/14 & \texttt{std::unordered\_map} & \texttt{std::map}\\
      C++98 & \texttt{\_\_gnu\_cxx::hash\_map} & \texttt{std::map}
    \end{tabular}
  \end{table}
\end{frame}

%-------------------------------------------------------------------------------

\codeslide{cpp}{
\begin{frame}{Associative Arrays}{C++}
  \textbf{\enquote{Feature} of C++ maps:}
  \lstinputlisting[
    language=C++,
    basicstyle=\small,
    tabsize=4,
    style={cpp-eclipse-code},
    breaklines=false,
    morekeywords={endl},
    escapechar={@}
  ]{Code/AssociativeArrayFeature.cpp}
\end{frame}
}