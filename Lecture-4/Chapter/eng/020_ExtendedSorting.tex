\section{Associative Arrays}

\toclesssubsection{Introduction}

\begin{frame}{Associative Arrays}{Introduction}
  \textbf{Normal array:}\\
  \begin{displaymath}
    A = [0 \Rightarrow A_0, \; 1 \Rightarrow A_1, \;
      2 \Rightarrow A_2, \; 3 \Rightarrow A_3, \; \dotsc]
  \end{displaymath}
  \begin{itemize}
    \item
      Searching elements by {\color{Mittel-Blau}index}\\
    \item
      Lookup of element with index $"3"$:\\
      $\hspace{1.5em}\Rightarrow A[3] = A_3$
  \end{itemize}
\end{frame}

%-------------------------------------------------------------------------------

\begin{frame}{Associative Arrays}{Introduction}
  \begin{itemize}
  \item In practice: all major programming project require associative arrays
  \item In our lecture: example of countries with associated information
  \end{itemize}
  \textbf{Associative array:}\\
  \begin{displaymath}
    A = \left[\begin{array}{l}
      "Europa" \Rightarrow A_0, \; "Amerika" \Rightarrow A_1,\\
      "Asien" \Rightarrow A_2, \; "Afrika" \Rightarrow A_3,\\
      \dotsc
    \end{array}\right]
  \end{displaymath}
  \begin{itemize}
    \item
      Searching elements by {\color{Mittel-Blau}key}\\
    \item
      The keys can be of any type with unique elements
    \item
      Lookup of element with key $"Afrika"$:\\
      $\hspace{1.5em}\Rightarrow A["Afrika"] = A_3$
  \end{itemize}
\end{frame}

%-------------------------------------------------------------------------------

\subsection{Practical Example}

\begin{frame}{Associative Arrays}{Practical Example}
  \vspace{-2.0em}
  \begin{columns}
    \begin{column}{0.9\textwidth}
      \begin{table}[!t]
        \label{tab:extended_sorting:country_query}
        \caption{Country data query from \url{http://geonames.org}}
        \begin{tabular}{ccccc}
          ISO & ISO3 & Country & Continent & $\dots$\\
          \midrule
          AD & AND & Andorra & EU & $\dots$\\
          AE & ARE & United Arab Emirates & AS & $\dots$\\
          AF & AFG & Afghanistan & AS & $\dots$\\
          AG & ATG & Antigua and Barbuda & NA & $\dots$\\
          AI & AIA & Anguilla & NA & $\dots$\\
          AL & ALB & Albania & EU & $\dots$\\
          AM & ARM & Armenia & AS & $\dots$\\
          AO & AGO & Angola & AF & $\dots$\\
          AQ & ATA & Antarctica & AN & $\dots$\\
          $\vdots$ & $\vdots$ & $\vdots$ & $\vdots$ & $\ddots$
        \end{tabular}
      \end{table}
    \end{column}
    \begin{column}{0.1\textwidth}
      \vspace{-6.0em}
      \qrcode[height=6em]{http://www.geonames.org/}\\
    \end{column}
  \end{columns}
\end{frame}

%-------------------------------------------------------------------------------

\begin{frame}{Associative Arrays}{Practical Example}
  \textbf{Task:}
  How many countries belong to each continent?
  \begin{itemize}
    \item
      We are interested in column 3 (Country) and 4 (Continent)
    \item
      There are two typical ways to solve this:
      \begin{itemize}
        \item
          Using sorting
        \item
          Using an associative array
      \end{itemize}
  \end{itemize}
\end{frame}

%-------------------------------------------------------------------------------

\subsubsection{Sorting}

\begin{frame}<beamer>{\LectureToC}
  \tableofcontents[currentsection,currentsubsection]
\end{frame}

\begin{frame}{Associative Arrays}{Practical Example}
  \textbf{Idea using sorting:}
  \begin{itemize}
    \item
      We sort the table by Continent, so that all countries
      from the same continent are grouped in one block
    \item
      We count the size of the blocks\\[0.5em]
      \textbf{Disadvantage:}
      \begin{itemize}
        \item
          Runtime of $\Theta (n \log n)$
        \item
          We have to iterate the array twice (sort and then count)
      \end{itemize}
      \textbf{Advantage:}
      \begin{itemize}
        \item
          Easy to implement (even with simple linux / unix commands)
      \end{itemize}
  \end{itemize}
\end{frame}

%-------------------------------------------------------------------------------

\begin{frame}{Associative Arrays}{Practical Example -
    Sorting With Linux / Unix Commands}
  \textbf{Input:}
  \begin{itemize}
    \item
      The data is saved as tab seperated text
      (\href{http://download.geonames.org/export/dump/countryInfo.txt}{countryInfo.txt})
    \item
      Comments begin with a hash sign \#
  \end{itemize}
  \hfill\\[1.0em]
  \textbf{Commands:}
  \begin{itemize}
    \item
      \textbf{grep}: Selects a specific set of lines (filter by ...)\\
      \texttt{grep -v '\^{}\#' countryInfo.txt}\\
      $\hspace{1.5em}$\texttt{-v}: not\\
      $\hspace{1.5em}$\texttt{\^{}\#}: \# at start of line
    \item
      \textbf{cut}: Selects specific columns of each line (tab separated)\\
      \texttt{cut -f5,9}\\
      $\hspace{1.5em}$\texttt{-f5,9}: columns 5 and 9
      (= columns 3+4 of shown Table~\ref{tab:extended_sorting:country_query})\\
  \end{itemize}
\end{frame}

%-------------------------------------------------------------------------------

\begin{frame}{Associative Arrays}{Practical Example -
    Sorting With Linux / Unix Commands}
  \textbf{Commands:}
  \begin{itemize}
    \item
    \textbf{sort}: Sorts lines by a key\\
      \texttt{sort -t '\hspace{1.5em}' -k2,2}\\
      $\hspace{1.5em}$\texttt{-t '\hspace{1.5em}'}: Separator: Tab
      (Insert with \texttt{CTRL-V TAB})\\
      $\hspace{1.5em}$\texttt{-k2,2}: Key from column 2 to 2
    \item
      \textbf{uniq}: Finds or counts unique keys\\
      \texttt{uniq -c}\\
      $\hspace{1.5em}$\texttt{-c}: count occurences of keys\\
    \item
      \textbf{head}: Displays a provided number of lines\\
      \texttt{head -n30}\\
      $\hspace{1.5em}$\texttt{-n30}: Displays the first 30 lines
    \item
      \textbf{less}: Displays the file page wise
  \end{itemize}
\end{frame}

%-------------------------------------------------------------------------------

\begin{frame}{Associative Arrays}{Practical Example -
    Sorting With Linux / Unix Commands}
  \textbf{Sort countries by continent:}\\
  \texttt{grep -v '\^{}\#' countryInfo.txt | cut -f5,9 \textbackslash}\\
    $\hspace{1.5em}$\texttt{| sort -t '$\hspace{1.5em}$' -k2,2 | less}
  \vspace{-2.5em}
  \begin{columns}
    \begin{column}[t]{0.5\linewidth}
      \begin{table}[!h]
        \caption{Resulting data}
        \label{tab:extended_sorting:country_sorted}
        \begin{tabular}{cc}
          Algeria & AF\\
          Angola & AF\\
          Benin & AF\\
          Botswana & AF\\
          Burkina Faso & AF\\
          Burundi & AF\\
          Cameroon & AF\\
          Cape Verde & AF\\
          $\vdots$ & $\vdots$
        \end{tabular}
      \end{table}
    \end{column}
    \begin{column}[t]{0.5\linewidth}%
      \begin{figure}[!h]%
        \caption{Data pipeline}%
        \label{fig:extended_sorting:country_sort_pipeline}%
        \def\SBData{less/1,sort/1,cut/1,grep/0}%
        \begin{adjustbox}{height=0.8\linewidth}%
\begin{tikzpicture}[%
  box/.style={
    draw={Mittel-Blau},
    fill={Hell-Blau},
    line width=0.05em
  }, box_text/.style={
    draw=none,
    color=black
  }, arrow/.style={
  draw={Mittel-Gruen},
    fill={Hell-Gruen},
    line width=0.05em
  }
]%
\foreach \text/\arrow [count=\y] in \SBData {
  \ifnum \arrow>0
    \draw[arrow]
      (0.25, \y + 1.0) -- (0.75, \y + 1.0) -- (0.75, \y + 0.75) --
      (1.0, \y + 0.75) -- (0.5, \y + 0.5) -- (0.0, \y + 0.75) --
      (0.25, \y + 0.75) -- cycle;
  \fi
}
\foreach \text/\arrow [count=\y] in \SBData {
  \draw[box] (0.0, \y) rectangle (1.0, \y + 0.5);
  \draw (0.5, \y + 0.25) node[box_text] {\text};
}
\end{tikzpicture}%
\end{adjustbox}%%
      \end{figure}
    \end{column}
  \end{columns}
\end{frame}

%-------------------------------------------------------------------------------

\begin{frame}{Associative Arrays}{Practical Example -
    Sorting With Linux / Unix Commands}
  \textbf{Count countries per continent:}\\
  \texttt{grep -v '\^{}\#' countryInfo.txt | cut -f9 \textbackslash}\\
  $\hspace{1.5em}$\texttt{| sort | uniq -c | sort -nr}
  \vspace{-2.5em}
  \begin{columns}
    \begin{column}[t]{0.5\linewidth}
      \begin{table}[!h]
        \caption{Resulting data}
        \label{tab:extended_sorting:country_counted}
        \begin{tabular}{cc}
          58 & AF\\
          54 & EU\\
          52 & AS\\
          42 & NA\\
          27 & OC\\
          14 & SA\\
          5 & AN
        \end{tabular}
      \end{table}
    \end{column}
    \begin{column}[t]{0.5\linewidth}%
      \begin{figure}[!h]%
        \caption{Data pipeline}%
        \label{fig:extended_sorting:country_count_pipeline}%
        \def\SBData{sort/1,uniq/1,sort/1,cut/1,grep/0}%
        \begin{adjustbox}{height=0.8\linewidth}%
\begin{tikzpicture}[%
  box/.style={
    draw={Mittel-Blau},
    fill={Hell-Blau},
    line width=0.05em
  }, box_text/.style={
    draw=none,
    color=black
  }, arrow/.style={
  draw={Mittel-Gruen},
    fill={Hell-Gruen},
    line width=0.05em
  }
]%
\foreach \text/\arrow [count=\y] in \SBData {
  \ifnum \arrow>0
    \draw[arrow]
      (0.25, \y + 1.0) -- (0.75, \y + 1.0) -- (0.75, \y + 0.75) --
      (1.0, \y + 0.75) -- (0.5, \y + 0.5) -- (0.0, \y + 0.75) --
      (0.25, \y + 0.75) -- cycle;
  \fi
}
\foreach \text/\arrow [count=\y] in \SBData {
  \draw[box] (0.0, \y) rectangle (1.0, \y + 0.5);
  \draw (0.5, \y + 0.25) node[box_text] {\text};
}
\end{tikzpicture}%
\end{adjustbox}%%
      \end{figure}
    \end{column}
  \end{columns}
\end{frame}